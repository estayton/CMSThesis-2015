% $Log: abstract.tex,v $
% Revision 1.1  93/05/14  14:56:25  starflt
% Initial revision
% 
% Revision 1.1  90/05/04  10:41:01  lwvanels
% Initial revision
% 
%
%% The text of your abstract and nothing else (other than comments) goes here.
%% It will be single-spaced and the rest of the text that is supposed to go on
%% the abstract page will be generated by the abstractpage environment.  This
%% file should be \input (not \include 'd) from cover.tex.

% 327 words, we are OK ( < 350)
In this work Erik Stayton examines dominant and alternative ideas about ground
vehicle automation, and concludes that current and imagined automation
technology is far more hybrid than 
is often recognized, presenting different questions about necessary or
appropriate roles for human beings.

Automated cars, popularly rendered as ``driverless'' or
``self-driving'' cars, are a major sector of technological
development in artificial intelligence and present a variety of
questions for design, policy, and the culture at large. This work addresses the
dominant narratives and ideologies around self-driving 
vehicles and their historical antecedents, examining both the media's representation
of self-driving vehicles and 
the sources of the idea, common both among the media and many
self-driving vehicle researchers, that complete vehicle autonomy is
the most valuable future vision, or even the only one  worth
discussing and investigating. 
This popular story has important social stakes (including
surveillance, responsibility, and access), embedded in the
technologies and fields involved in visions of full
automation (machine vision, mapping, algorithmic ethics), which bear investigating
for the possible futures of 
automation that they present.
However, other potential narratives for looking at automation exist,
representing lenses from
literature in the fields of human supervisory control and
joint-cognitive systems design. These fields---compared with
that of AI---provide a very
different read on what automation means and where it is headed in the
future, which leads to the possibility of different futures, with
different stakes and trade-offs. Finally, this work examines what cultural
understandings need to change to 
make this (cyborg) picture more broadly comprehensible, and suggest potential
impacts for policy and future technological development. It argues that
a broader appreciation for 
our hybrid engagements with machines, and recognition that automation
alone does not solve any social problems, can alter public
opinion and policy in productive ways, away from focus on
``autonomous'' robots divorced from human agency, and toward
system-level joint human-machine designs that address social needs. 
