% $Log: abstract.tex,v $
% Revision 1.1  93/05/14  14:56:25  starflt
% Initial revision
% 
% Revision 1.1  90/05/04  10:41:01  lwvanels
% Initial revision
% 
%
%% The text of your abstract and nothing else (other than comments) goes here.
%% It will be single-spaced and the rest of the text that is supposed to go on
%% the abstract page will be generated by the abstractpage environment.  This
%% file should be \input (not \include 'd) from cover.tex.
In this thesis I examine dominant and alternative models of ground
vehicle automation, and conclude that current and imagined automation
technology is far more hybrid than 
is often recognized, presenting different questions about necessary or
appropriate roles for human beings.

Automated vehicles, popularly rendered as ``driverless'' or
``self-driving'' vehicles, are a major sector of technological
development and public interest presenting a great variety of
questions for design, policy, and the culture at large. I address the
dominant narratives around self-driving 
vehicles and their historical antecedents, examining both the media's representation
of self-driving vehicles and 
the sources of the idea, common both among the media and among
self-driving vehicle researchers, that complete vehicle autonomy is
the only future vision worth discussing and investigating.
This popular story has important social stakes, embedded in the
technologies involved in 
automation and the fields with which visions of full automation are
intertwined, which bear investigating for the possible futures of
automation that they present.
However, other potential narratives for automation exist, and are
represented by both historical examples from other regimes of
automation, and
literature in the field of human supervisory control and
joint-cognitive systems design. These fields---compared with
artificial intelligence---provide a very
different read on what automation means and where it is headed in the
future, which leads to the possibility of different futures, with
different stakes and trade-offs. Finally, I examine what cultural
understandings need to change to 
make this picture more broadly comprehensible, and suggest potential
impacts for policy and future technological development. I argue that
a broader appreciation for 
our hybrid engagements with machines, and recognition that automation
alone does not solve any social problems, can alter public
opinion and policy in productive ways, away from focus on
``autonomous'' robots divorced from human agency, and toward
system-level design concerns. 
