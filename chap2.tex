\chapter{The Stakes of our Stories}

3. Stakes of stories/tech (swap 4?)
-data collection
-mapping
-ML/machine vision
-functionalism vs. Understanding
-statistics and risk
-planning, policy, cities





2)Other problems:
--Göde Both and reluctance to use Machine Learning (1 p)
---brittle, unpredictable, difficult to ``tune''
---airlines/Mil worried about the same thing: need better VV\&T, models
for testing (3 p)
--Mapping and route issues
---Google car needs detailed 3D maps in order to operate (1 p)

 More likely, certain areas will be
mapped and restricted, or separate divided public rapid transit
systems will operate on roads that can be carefully monitored.i

While
Mountain View, California may be mapped early, rural West Virginia or
Northern Maine may not be mapped as soon or as frequently.
Inequalities may be increased if routes frequented by upper-middle
class professional commuters, most likely to own new autonomous cars,
are mapped first, while roads around low-income communities are
ignored. 

Prior knowledge of speed limits should make the car's
behavior more reliable and predictable in all conditions, even if
speed limit signs are missing or obscured—consider how often human
drivers, when faced with an absence of signs, base their behaviors on
supposition or prior knowledge. When Google's car was certified for
testing in Nevada, Google was allowed to pre-select the route the car
would take, so that they could build the comprehensive model the
system requires beforehand.ii The system would likely not have been
capable of passing a test in which the examiner could have added
detours on the fly. And though Google claims to have driven more than
700,000 miles with their cars, those are not 700,000 unique miles. A
limited, thoroughly pre-mapped route has been driven many times to
achieve those numbers.iii


---also rain, snow, weather (1 p)
--Sensors issues and inability to distinguish crushed newspaper from
rock (1 p)
---requires a lot of other information, more than simple physical
volume
---object identification work in machine vision is slowly progressing,
but then we hit the above problems of VV\&T for such models (2 p)

>traffic-light detection and problems of infrastructure??


2.5??) Worth bringing in questions of terminology, ``intelligence''
etc. and the necessary caution in ascribing them to tech; we can look
and see that the ``intelligence'' in action here is really very
different than what we usually mean when we use that term
(human-centric)
--so does knowing this reality change what we want to see? (couple of
p)


%% 2.1) data gathering and mapping
%% --this approach presupposes large amounts of data collection (maps),
%% and opens the way to more information about the passengers (2 p)
%% --see Google's patents in particular which talk about sensing things
%% like the number of passengers (4 p)
\subsection{Data Gathering and Monitoring} 

%%   $$$$$$$$      NOTE      $$$$$$$$$
%%
%%
%%
%%      More HISTORY, Less CRITIQUE
%%                (here)
%%
%%
%% ``this assumes xyz'' rather than ``but really abc''


Autonomous vehicles are and will continue to be networked
technologies. This connectedness brings with it great possibilities
for coordinating traffic and improving city planning, as well as great
risks to privacy and security, whether devices are networked with each
other or simply connected to central servers. 

With what networks, and for what reasons, will autonomous vehicles be
connected? Current driverless car concepts depend on networked
information for vehicle guidance. While, historically, certain
guidance systems have been insulated from communications---inertial
guidance systems for intercontinental ballistic
missiles are a particular example of this,\cite{mackenzie}---and
Google's vehicles use inertial navigation devices\cite{knightfurther}
alongside other sources of position data, current navigation systems
depend on global positioning satellites. Google's approach to
driverless car development currently necessitates accurate
and expensive\footnote{An autonomous vehicle researcher at MIT
  quoted prices in the range of $70,000 to $100,000 per device for the
GPS alone, while noting that those would of course come down with
greater production volume.} differential GPS receivers.

But a number of other developments already suggest that autonomous
vehicles will be connected to more than just global positioning
networks. GM's OnStar service already connects equipped vehicles to central servers for
purposes of safety, security, and convenience. The system has the
ability to automatically alert the authorities in case of an accident
or theft. It also provides vehicle diagnostics to the owner's tablet
or smartphone, and the OnStar app also allows the owner to configure settings, lock and
unlock doors, and remotely operate the lights and horn.\cite{onstar}
While these vehicles are not (yet) autonomous, OnStar's present
capabilities are representative of features that will become more common in highly
connected and computerized vehicles, including autonomous ones.

Additionally, vehicles that can receive information from each other and from the
roadway are a top priority for the NHTSA, and have been on the
research agenda for decades.\cite[p. 11]{wetmore} These concepts would
collect and use data to streamline traffic flow and provide
information to reduce delays and accidents. But there is potential to
do far more with the available information,
and data collected by the vehicle to make possible its own functioning
could be made to serve other purposes. Information about the vehicle
and its surroundings, including the locations of cars and pedestrians,
precise GPS coordinates of the vehicle itself, and the vehicle's speed
and acceleration, not only represent important knowledge for
path-finding by the vehicle itself, but new sources of potential
revenue for the groups in position to collect them. Uber, which
through its GPS-enabled ride-hiring applications still collects only a
fraction of the data that would be available through a self-driving
vehicle, has agreed to share its ride data with municipalities for purposes of
city planning.\cite{uberJardin} Though this data is ostesibly being
shared for the public good, it also serves private ends.

Google has envisioned vehicles that can determine their number of
occupants, and use facial-recognition or other biometric systems to
identify them. According to one patent,\cite{predictPatent} these vehicles could prevent
unauthorized persons from putting a child in a car, prevent convicted
sex offenders from operating their vehicles within the
legally-required distances of schools and playgrounds, or prevent a
car's doors from being opened (even from the inside) by a child unless
an authorized adult is present. These are only visions, and patents
are notorious for trying to cover as many possible angles of a
technology even if they are not intended to be applied in practice.
But these suggestions represent a perspective on safety and societal
order that posits technological surveillance and enforcement as an
appropriate preventative measure against criminal behavior. Whether or not protecting
against these threats is an appropriate use of this information is a
matter for societal judgment, but such proposals, if enacted, would
require these vehicles to have unprecedented levels of very sensitive
knowledge about people and their lives: biometrics, criminal
histories, family and trust networks. 

Data ideology tells us that we can understand ourselves better through
these data, that we can use them to determine patterns we never knew were
there. Out of this ideology come publications like Uber's blog, which
describes customer ``insights'' gained through their ride data, which
are ostensibly interesting to the public. ``Look here,'' they seem to
say, ``we can tell you about \emph{you}.'' But it important to
recognize that the ideologies of data collection---for corporate
profit and public use---are deeply intertwined, and strongly
influenced by the possibility of using machine learning techniques and
statistical analyses to find and exploit patterns.


\subsection{Maps and Mapping}

Other types of data collection are also implicated in the current
vision of the driverless car project. In order to drive with us,
autonomous systems will have to understand,
for at least a practical sense of “understanding,” traffic rules and
their accompanying signs, signals, lanes, and customs. This is, at its
core, a highly complex problem of interpretation and representation
for machines. Human understanding is built
through years of experience: it is through existing as a human being
in a particular cultural context that we know to drive on roads but
not on sidewalks, and how to tell the difference. But we do not have
this luxury in training machines.

Here, the connections of autonomous systems to information networks
again become important. The vehicles in the
DARPA Grand Challenge did not navigate “on their own”: they used GPS
to follow a path laid out for them in advance, using their autonomy
only to avoid obstacles like rocks and ruts.\footnote{John Leonard,
  discussion with the author, December 3, 2014} And though successful
road tests have been accomplished without navigational assistance,
using only visual stimuli (such as the EUREKA PROMETHEUS project
in the 1990s)\cite{???}, modern systems are tending to use more external
stimuli, rather than less, in an attempt to increase safety. Even as
advanced as it is, Google's autonomous vehicle technology requires
hyper-detailed 3D maps in order to operate properly on public
roadways.\cite{???} These maps are generated by vehicles outfitted with
special sensor arrays, like the LIDAR Google uses for their autonomous
vehicles, which drive a route and collect data which can be used to
reconstruct the model.\cite{???}

Pre-made maps are used so the vehicle knows where stoplights, signs,
and curbs are, reducing the computational load on the machine in the
crowded visual landscape of driving, and allowing it to focus on
elements of the environment that are changing rather than those likely
to be static.\cite{???} Mapping claims a unique capability to represent the real, objectively
and diagrammatically, but also requires that world to remain largely
static, at least on the order of how long it takes to update the map
for a particular region. The necessary level of continual mapping is a massive task if the
vehicles must be usable everywhere. The United States alone contains almost
8.5 million road miles\footnote{Data as of 2008,
  http://blog.cubitplanning.com/2010/02/road-miles-by-state/\cite{???}},
and it took years for Google Streetview to acquire the level of
coverage it currently has. The utopian discourse of driverless cars
implicitly suggests that such vehicles will be available everywhere,
and are the solution to nationwide transit problems. But widespread
egalitarian access to maps-based devices depends upon a rapid,
widespread mapping initiative.

The seemingly universalizing forces of maps and computer programming
have a tendency to help hide issues of geographical and cultural
specificity, which are rendered invisible in the utopian narrative.
But though engineering practice holds that these issues are
conquerable, they should be anything but invisible. Programmed devices
must must know about speed limits, about traffic
lights, about rules of the road that were never designed for
autonomous systems. These devices must respond to human caprices and
be adapted to longstanding, ingrained laws and habits. They must
include historical knowledge, rooted in the legal and social histories
of roadways, which may differ between cities and states, and certainly
between countries across the world. Local customs and behaviors differ, and even if maps are
available, the same vehicle programming may not work for Los Angeles,
Boston, and the rural Midwest, let alone Singapore, Mumbai or Cairo.
The map, for all of its objective standardization, still represents
real places subject to cultural histories and vulnerable to
socio-economic dynamics. These social and regional issues are often
ignored in the driverless vehicle narrative, but nevertheless stand to
be critical to the manner in which these technologies could come to
enter everyday life.



\subsection{Machine Vision}
%% 2.2) machine vision
%% --pull from the Spectator paper (4 p)
%% INCL. machine learning
Interpretation of the world around us is a task that seems
particularly easy for human beings, but particularly difficult for
machines. The invention of the photocell, early a tool for workplace
monitoring and surveillance, provided a simple channel through which
electrical systems could respond to the amount of light reaching
them.\cite{???} Though the photocell can easily provide a computer system
with access to brightness information over time, perceiving detail and
depth, identifying shapes, and interpreting expression and motion are
all capabilities of human vision that require more sophisticated
technologies to reproduce. DARPA's 1983 Strategic Computing Initiative
included image interpretation as one of its main
focus areas.\cite{???} But it is only relatively recently that real-time video
processing, needed for camera-based navigation, became feasible for computer
systems small enough to fit in a standard automobile.\cite{???} And machine
vision problems, including object recognition and scene
interpretation, continue to be difficult, even with increased
processing power and new algorithms. 

As an engineering discipline, computer vision takes a decidedly
practical and reductionist view of what it means to see. The goal is
generally not to achieve creative interpretation or aesthetic
valuation, but to differentiate free space from things a robot should
not run into.\cite{???} But this so-called objective focus still encodes
certain subjective judgments about objects (including people),
behaviors, and intent. And while computer vision is having success
with object detection, there is a wide variety of human knowledge
about objects and scenes that is missing in current computer models.\cite{???}

Though vision has not always been the sensory mode that dominated
autonomous vehicle research, vision is a particularly attractive sense to
use, as it is integral to how humans drive. In an attempt to build
autonomous vehicles that can operate without infrastructural changes,
research has moved away from tracks and cables toward vision-guided
systems. New approaches were pioneered by Ernst Dickmanns at
University Bundswerhr in Munich, with the vision-guided VaMoRs van,
and continued via the EUREKA PROMETHEUS project in 1987, in which
Dickmanns and Daimler-Benz built cars guided by analog video
cameras.\cite{???} Like the earlier VITA project by Daimler that used an
analog video-camera signal processed through a framegrabber, these
cars digitized analog video at relatively low resolutions. The
features the systems searched for, including lane markings and other
cars, are geometrically distinct and visible even in small images.\cite{???}

<Somewhere, mention weather issues for vision>

Vision-guided systems, now using digital video cameras and
off-the-shelf consumer hardware, have the benefit of being inexpensive
and insensitive to interference from other nearly devices (unlike
sonar, for example, which becomes problematic in crowded
situations\footnote{John Leonard, discussion with the author, December
3, 2014.}). Some commercial systems, such as that developed for Mercedes-Benz's
self-driving S-class, which is slowly finding its way into consumer
vehicles, are primarily guided based on such visual sensors.\cite{???} To
these sensors, recent research has added roof-mounted LIDAR arrays.
LIDAR, short for Light Detection and Ranging, is a distance
sensor, which is applied in vehicles to scan the environment with a
rotating array of laser beams to create a detailed 360-degree
representation of objects and their distances. This technology
solves some of the difficulties of image interpretation by default, as
it can provide highly-sensitive information about free space and
obstacles. Shape-detection algorithms can then be used (in addition to
vision-based data) to classify obstacles as different types of
objects: pedestrians, bicyclists, cars, and trucks.\cite{???}

Pedestrian detection algorithms search for person-like shapes, where
“person-like” is determined by, for example, training a classifier
using thousands of images previously labeled (by people) as being images of humans, so
the system can learn the features that correlate with a person being
in a particular region of an image. These detected categories allow the system to make
statistical predictions about likely types of behavior: according to
one of Google's patent applications, bicyclists are likely to be more
erratic than trucks, and should be treated
accordingly.\cite{predictPatent} These sorts of predictions are
something that human drivers do consistently, and are therefore also
likely important to how autonomous vehicles may drive.\footnote{The
  DARPA Urban Challenge crash, the first crash between two autonomous
  cars, provides an important lesson on the vagaries of object
  detection: the classification threshold between moving and
  stationary, set too high, allowed one vehicle to interpret the other
  as stationary, leaving no room for unexpected behavior.\cite{???}} 

<Mention: one notable researcher I spoke to drives around searching
for situations he believes will cause trouble for current vehicles,
trying to record them. They include many situations with police
directing traffic, especially when combined with sun glare, or
occlusions of the sight lines that would make predictions of oncoming
objects difficult.>

And because autonomous cars see---putatively “as we see”---their sight can
be leveraged as visual evidence of their operation. Computer vision systems that identify
pedestrians can be shown to do so, via the detection boxes that act as
diagnostic tools for researchers and direct representations of
internal system information. The new technologies of vehicle automation thereby produce through
their operation new forms of evidence, which can be presented through
electronic information media.


%% 2.2) functionalism over understanding
%% --for better or worse these devices are not humanlike in their
%% understanding
%% ----AI winters & the rise of commercial ML AI with much smaller goals
%% (see Wired articles, etc) (3 p)
%% --the ideology that drives them is an engineering mindset: ``just make
%% it work'' (2 p)
%% --pulling the quote from my USC talk: also interesting to note
%% regarding 1.2 that it seems to expect a humanoid robotic chauffeur in
%% a normal car! (1 p)
%% -and of course interviews with people (Ryan Chin; Walker Smith etc.
%% where they matter to this type of vehicle)
\subsection{Functionalism and Utilitarianism} 
But for better or worse, self-driving cars will not be human-like in
understanding, even while they can detect and identify pedestrians.
A wide range of software companies and startups have entered the AI
industry, but these companies are not primarily focused on
general-purpose AI. Though certain ventures still hold out the dream
of doing so, several AI winters have shown that the creation of
general intelligence is very difficult, and is by no means around the
corner. Those working in the field are well aware of
this.\cite{???-articlerebuttingmusk} So much current work is
fundamentally utilitarian, building systems with clear goals, metrics
for success, and market segments.

The utilitarian model of AI makes good sense for a number of reasons.
First, much can be achieved with current technologies. Rather than
focusing on a long-term project, which would carry much greater risks
and rewards which are further off, achieving short term goals is an
attractive prospect. It can not only attract money but can be
profitable sooner. Though short term commercial viability is not
necessarily applicable for this vision of the self-driving vehicle, it
is possible to have working prototypes on the road, generating
interest and publicity even in controlled conditions. Taking
journalists for test drives drums up interest, even if the technology
is not ready for full-scale deployment. Make no mistake: the current
vehicles are capable of recognizing pedestrians, but any claim that
they ``know'' or ``understand'' is a tenuous misuse of words which
could no longer credibly mean what they generally do. 

Second, humanlike characteristics may not be helpful for building
specific applications. One would likely not want one's self-driving car
to be preoccupied or emotional.\cite{???-wiredFutureofAI} It may be that, for utilitarian
purposes like driving, many of the characteristics of the human are
detrimental, their elimination helpful and intentional. Much of the
discussion around why autonomous vehicles are necessary centers on
just such qualities: distractability, sleepiness, lapses in
concentration.\cite{???} We would not wish to emulate such
characteristics in robotic systems. However, these are human
capabilities, but also only caricatures of the human. People possess a
variety of other capabilities which might be helpful to certain AI
systems. As AI researcher Doug Lenat wrote in 1997:
\begin{quote}
Before we let robotic chauffeurs drive around our streets, I'd want the
  automated driver to have a general common sense about the value of a
cat versus a child versus a car bumper, about children chasing balls
into the streets, about young dogs being more likely to dart in front
of cars than old dogs (which, in turn, are more likely to bolt than
elm trees are), about death being a very undesirable thing.\cite[p.
  122]{ekbia}\end{quote} 

Current approaches, however, assume this kind of deep understanding is
unnecessary, both for the technical creation of such vehicles as well
as their public acceptance. It is an interesting read on the changing
times to notice that Lenat's statement seems to suggest humanoid
robotic drivers operating regular cars. As well as moving toward
functionalist systems, the industry has moved toward embedded systems
within devices, systems that make no pretenses to be humanoid, but
instead revel in appliance-hood. Whether this shift in the form of AI
systems makes customers more or less nervous about computer-driven
vehicles is an empirical question\footnote{CAN I FIND RESEARCH ON
  THIS}. But it certainly suggests a desire to make the systems more
invisible---and actually has important implications for human-computer
interaction, as we will see in the next section. Though it may seem
obvious, it is vital to remember that the devices we discuss here may
have the properties Lenat mentions only in the way that Mario is
``self-aware,'' possessing programmed constraints that cause it to
avoid people, and perhaps expect children to behave erratically. 


%% 2.3) what standard of safety? 
%% --better than an average person? or better than the best people? (1 p)
%% --human problems with using projected statistics to define policy:
%% does it make people feel better to know a system is statistically
%% safer when they are uncomfortable with it? see cars/planes for example
%% (3 p)
%% -->perspective of caution? or  ``losing lives every day'' that
%% could be saved?
%% ----how much risk is legitimate?? (2 p)
%% --airlines as a place where we see the different dimensions of
%% statistical safety vs. passenger's feelings of safety (3 p)
%% ---drawing from PARC/CAST documents
\subsection{Safety and Statistical Risk}

But regardless of understanding, what drives the self-driving car
narrative is safety; specifically, the poor safety record of human
drivers and the potential for machines to do much better, free from
human frailties of distraction and fatigue. How bad are human
drivers, really? Conventional stories of human drivers paint us as
plainly terrible, prone to road rage and drowsyness and generally
doing anything except paying attention\footnote{Numerous popular
  articles take this position\cite{???}}. However, the statistics tell
a slightly different story. Official numbers show that the number of
motor vehicle accidents has remained in the range of 10--11 million
per year for most of the late 2000s. The death rate seems to have
decreased overall in this time, settling somewhere below 1.5 deaths
per 100 million vehicle miles, presumably due to a combination of
better safety features (especially since 1990) and other
factors.\cite{???} The total number of vehicle-related deaths is much
lower, at a still significant 35,000 deaths per year. But this alone
does not tell the story. At around 1.5 deaths per 100 million vehicle
miles, or about 1 death per 67 million miles, humans seem relatively
competent in a statistical sense. The average American, who might
drive 1 million miles\cite{???} is unlikely to be
involved in a fatal crash in his or her lifetime. Looking at non-fatal
accidents as well, humans get involved in about one accident per
286,000 vehicle miles. Part of why the autonomous vehicle problem is
such a difficult one is that these numbers are relatively high.

Theoretically, computers can do better. But especially careful human
drivers can also clearly beat the human average. But how safe do
autonomous vehicles need to be in order to be allowed on our roads?
Safer than the average human? or safer than the very best drivers?
Such questions have real impact when it comes to how devices are
designed and when they become commercially viable. The autonomous
vehicle enterprise seems to call for using such projected statistics
to define policy. One common narrative is that undue caution in the
rollout of autonomous vehicles will directly ``cost'' lives, since
people are killed by human drivers every day.\cite{???-blog} However,
people are also accustomed to the current automobile death rate, and
any autonomous vehicle crashes are likely to attract deep
scrutiny\footnote{As Jim Womack has pointed out, there is no good side
to change as a regulator. Regulators are not congratulated when things
go right, only criticized when things go wrong. So some measure of
tentativeness is almost certainly justified to the regulatory mind.
(Discussion with the author, December 3, 2014)} as
to whether a human could have prevented the accident.\cite{???}

These are questions of policy, but also questions of human acceptance.
This ideology posits lowering accident rates above all else, leaving
no space for human squeamishness about technology and responsibility. 
The statistical argument suggests that the death rate is all that
matters, but humans are notoriously bad at understanding and
responding to statistics. Does it make us feel better---more
comfortable, more likely to get into autonomous taxis and spend our
money on autonomous cars---to know that they are statistically safer
than the average driver? 

Airplanes are one type of vehicle for which these different dimensions of
safety---statistical safety compared with perceptions or feelings of
safety---have already become visible. Despite the comparative statistical safety of flying,
people tend to be more afraid of getting on an airliner than getting in their
cars.\cite{???} While this may have to do with a number of factors,
including that aircraft do not remain on the ground during operation,
it also represents a situation in which passengers give over their
agency to pilots performing a job they do not understand and could not
take over in an emergency.\cite{???-canIFindAnythingOnThis} SIMILAR
ISSUES WITH TRAINS?\cite{??} The
PARC/CAST working group had to account for these issues in their recent
recommendations about airplane safety. They judged that if air traffic
were to continue to increase with XXXX rates of crashes, by XXXX it
would reach one fatal crash per day, which would be intolerable to
airline customers, even though the overall accident rate would be no
larger.\cite{???-PARCCAST} Though aircraft would be as safe as they
had ever been, people, on seeing the sheer number and scale of
accidents, would be unlikely to choose to fly. While the issues facing
aircraft and trust are not precisely the same as those for cars, this
example shows that actual safety rates are only
one component of interest when considering how consumers react to
modes of transportation. Perceived safety---due to accident scale,
publicity, trust, or other factors---may be very different. But
while research into how to get human beings to trust robotic drivers
is being done\cite{???}, the voices pushing for autonomous cars sooner
rather than later would suggest that the statistics are all that
matters. 


\subsection{Planning, Policy, Cities}

Both:
Developers, >>>>they are trying to change society
>>>>Utopian vision of big cities freed of cars: no parked cars \& only
driverless cabs
>>>>>but if you look closer so they have a strong male bias: envision
businessmen taking cabs and students getting drunk and can't drive
home by themselves
>>>>>do the yhave an idea of what it means for people who have
children? they don't think about it

Michigan paper about ``may not be safer''
%''Of course, the researchers are trying to correct what they regard as
%excessive technological optimism. Still, is it entirely fair of them
%to compare robocars only to the best drivers? Most accidents are
%caused by the worst ones, and it is beginning to become clear that
%those are the people that a robot could outperform with one clanky
%arm tied behind its back.'' (Phillip E. Ross, ``Has Robocar Safety
%Been Hyped?'' 20 Jan 2015: http://spectrum.ieee.org/cars-that-think/transportation/safety/has-robocar-safety-been-hyped)
UMTRI-2015-2.pdf: road safety w/ self-driving vehicles, things being
considered
-p6

