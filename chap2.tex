\chapter{The Stakes of Our Stories}
\label{chap:2}

%% 3. Stakes of stories/tech (swap 4?)
%% -data collection
%% -mapping
%% -ML/machine vision
%% -functionalism vs. Understanding
%% -statistics and risk
%% -planning, policy, cities



%% >traffic-light detection and problems of infrastructure??


%% 2.5??) Worth bringing in questions of terminology, ``intelligence''
%% etc. and the necessary caution in ascribing them to tech; we can look
%% and see that the ``intelligence'' in action here is really very
%% different than what we usually mean when we use that term
%% (human-centric)
%% --so does knowing this reality change what we want to see? (couple of
%% p)


%% Note: SOMEWHERE you should talk about the double-speak; freeing up
%% the roads for car lovers . . . doesn't make sense (chap3 handles it)


%TODO: EDITS
%Starr \& Bowker, standards? resonant??

\section{Science Fictions, Designed Dreams}

Technologies, with rare exceptions,\footnote{One could argue that
  vulcanized rubber for example, as an accidental discovery, does not
  fit this pattern.} are imagined before they are made, through design
fictions and/or science fiction. These future
visions serve a number of purposes: for example, they inspire scientists and
engineers,\footnote{In fact I think it would be exceedingly rare to
  find an engineer today who was not influenced by science fiction. My
  interest in AI has been driven by \emph{Star Trek} and \emph{2001: A
  Space Odyssey}, the latter perhaps more morbidly than the former. A
  colleague of mine cites Neal Stephenson's
  \emph{The Diamond Age} as key to shaping his interest in
  computational linguistics and human-machine interfaces. Wernher von
  Braun and other architects of the space program were deeply
  influenced by Jules Verne, and some even wrote science fiction
  themselves \cite[p. 2]{marketingMoon}.} they serve as design
studies for the possible shapes of technology, and they act as
playgrounds to investigate the cultural impacts of technologies.
Though created in different contexts, with potentially 
different levels of scholarly care, these two types of fiction are not
neatly delineated, and are unified as sources of insight into what
might be possible with technology. ``Design fiction is the cousin of
science fiction,'' as Julian Bleecker puts it, and represents a hybrid
practice that attempts to negotiate between facts and wild, playful
imaginings to bring light to the multiplicity of possible
futures \cite[p. 8]{bleecker}. It would be wrong to categorize science
fiction as categorically less relevant, however: science fiction
films
often involve scientific consultants with real
technical knowledge related to the technologies they are responsible for
helping bring to life on screen. These depictions become what David
Kirby calls ``diegetic prototypes'' which are used to ``demonstrate to
large public audiences a technology's need, benevolence and
viability'' \cite[p. 43]{kirbyFuture}. Kirby connects prototypes (primary
``driver[s] of technological innovation''---in
Suchman's terms ``performative artefacts'' \cite[p. 45]{kirbyFuture})
to these diegetic prototypes, 
by recognizing both their similar rhetorical roles as well as the
ability of diegetic prototypes to mobilize funding for real-life
prototypes \cite[p. 44-47]{kirbyFuture}. Science fiction, then, has deep
and compelling relationships with real engineers and engineering via
its depiction of aspirational (or cautionary) futures.

%%COnsider citing the Suchman article itself in Kirby

But science fiction is also a perennial source of popular metaphors and ideas
about technological change, in part perhaps because the visions are so
compelling, but also because they are so easily available. While these images
do not issue forth from a vacuum, as we have seen---coming from a potent combination
of actual engineering and scientific developments mixed with the
imagination---science-fiction images represent easily-digestible
cultural reference 
points for technological stories in the popular press. From Asimov to
\emph{Terminator}, 
these sorts of pictures are endemic to discussions of artificial
intelligence in the press, and are often mobilized specifically to
illustrate the possible forms of automated vehicles. Stories and ideas
about what autonomous cars will be are
influenced by \emph{Minority Report} \cite{fromHollywood}, \emph{KITT}
\cite{wadeKITT}, and \emph{Total
  Recall} \cite{pasdirtzSolution}. Countless articles begin by describing
self-driving vehicles as a Hollywood or science-fiction
staple \cite{scifiToReality}. \emph{Minority Report} is of particular
importance in this context, not only because it is so often referenced
but because its depiction of automated vehicles was developed in
consultation with Harald Belker, an automotive designer responsible
for both numerous Hollywood collaborations as well as real-world
products \cite{melansonMinority}. But popularly referenced fictions are
not all so contemporary. \emph{The Jetsons}, though
the show did not actually foreground driverless vehicles, also gets
recalled as a source of inspiration due to the role of flying cars,
which trigger similar visions of wild 
futures made possible through technology \cite{JetsonsAge}. 

%TODO how do we dedup this?

Many, indeed most, articles seem ignorant of the actual research
history that inspires and is inspired by successive generations of
science and design fictions. Some articles, especially from more
scholarly establishments, bring up 
the pioneering work done by Mercedes and Ernst
Dickmanns in Germany in the 1990s \cite{HCRIDriverless}. But few connect
autonomous vehicles today as part of a long research history going
back to the 1960s, and an even longer set of engineering dreams going
back at least to the 1920s. Such narratives would as
valid, in their own way, as connections to science fiction, but I
suggest that some of the ``newsworthy'' shock value of the technology would be lost
by linking it to a deep history of research and effort. Connecting it
instead to the dreams and aspirations of Hollywood cinema makes it
more novel, exciting, even magical; and the few references to past
technology projects are relegated to their symbolic value only, used
to express a sense of impatience and technological inevitability: ``Isn't
it about time?'' This hype-mindset fits the business
goals of many news outlets interested primarily in page-views.

The few historically-specific 
dreams that do filter through into the narratives of autonomous
technology \cite{CBSPetersen} 
include the 1939 World's Fair, discussed
in the introduction, at which auto companies
previewed a city of the future. Its centerpiece, GM's Futurama
exhibit, presented a bird's-eye view of a future city with multi-lane elevated
expressways filled with largely automated vehicles. GM's Firebird concepts
from the 1950s\footnote{Rarely mentioned are other contemporary research
  initiatives, such as the work of
  Vladimir Zworykin at RCA. His 
  concept, inspired by ``railroad block signals,'' used circuits
  embedded in the road to magnetically sense vehicle speed and
  location. GM's working test system based on their ``Unicontrol''
  joystick instead located much of the sensing and control within the car
itself \cite[p. 8]{wetmore}. But this work too is not often
remembered.} are also occasionally remembered due to their publicity
and 
subsequent cultural presence, \cite{walshVs} despite that the vehicles actually
contained no autonomous technology \cite[p. 7]{wetmore}. They were
themselves engineering fictions. DARPA's investments in automated
driving are recalled more frequently, but largely stripped of their
military valences and instead seen as the start of a vibrant civilian
automated car project \cite{economistSR}.

These historical moments are powerful because in their projections into the
popular discourse, they cease to answer to logic and
sense. They draw upon emotions, hopes, and aspirations. Driverless cars
are in part works of theater, objects of technological spectacle which
are not yet truly real, despite their existence as physical objects
which can be photographed and experienced.\footnote{And in this way
  they are similar to concept cars of other types, which are designed
  to embody company ideals, brand languages, and generally act as
  advertising as much as design studies for actual future vehicles.}
Such real-world contact is 
yet limited to a lucky few. Google hosts special press events to allow
select journalists
to ride in their automated vehicles. Their capabilities are touted,
advertised, but their media picture is still tightly controlled. Not subsumed
to the mundane, the quotidian,\footnote{Certain articles from people
  who have actually gone for a ride in these vehicles stress the
  boredom and mundanity that comes with operating this
  technology \cite{rode500}. But these do not represent the
  preponderance of coverage. It also should be noted that boredom
  always comes with the potential of its opposite. Operating automated
systems can involve, as Sheridan describes, ``hours of boredom
punctuated by moments of terror'' \cite[p. 339]{sheridan}.} they still
possess that magic that 
makes science fiction connections obvious and compelling. But this
does not mean such connections are unproblematic. Far from it. By
their alignment of these vehicles to science fiction, popular
narratives emphasize images of the technology as it has already been
envisioned, and obscure the multitude of potential implementations
that can emerge from engineering practice in the moment. Self-driving
cars threaten to become natural and obvious, in a particular form,
through their associations with existing, known and culturally
assimilated media portrayals.

Close cousins to science fiction representations, the statements of
researchers themselves are also responsible for 
shaping conventional narratives of autonomous vehicle
development---through the making of their own design fictions, or
technocultural dreams that motivate their research.
Since automated vehicles have the potential to reshape many areas of life, it should
be no surprise that the cross-section of researchers interested in the
technology is broad indeed. Some are vehicle engineers (e.g. John Leonard), some of whom have been
interviewed by the press or had their research papers picked up by news outlets.
Others are mathematicians and
computer scientists (e.g. Paolo Santi) working on other problems related to
autonomous vehicle systems. And yet others are designers (e.g. Carlo
Ratti of the SENSEable City Lab), urban
planners, or traffic safety experts interested in how the next
generation of vehicle systems will
affect our cities. The breadth of people involved means a breadth of
technical competencies. And many (though not all) of these contributions buy into the
implicit teleological progress narrative, whether or not they expect
driverless cars to be feasible in the market soon.

One particular strain of ideas comes from those interested in the
social benefits that would accrue if and when cars could be automated.
Daniela Rus of MIT's CSAIL speaks of ``data-driven mobility'' and the
idea of using the affordances of automated vehicles to improve traffic
patterns.\footnote{Speaking at The Road Ahead Conference, hosted by
  the MIT SENSEable City Lab, Cambridge MA, November 21, 2014.} Taking
data from phones, road cameras, satellites, and other 
sources already provides the potential of forecasting congestion, from
this perspective,
provided we ``put it all in the cloud'' so that it can be collated and
queried at scale. While this approach allows incentivizing individuals to drive at
different times or use different routes, widespread use of autonomous vehicles would add
the possibility of moving vehicles around based on predicted or
current demand, and therefore potentially reduce the number of
vehicles needed to move people in a city. Researchers at MIT and Stanford, as part of the
SMART\footnote{Singapore-MIT Alliance for Research and Technology} initiative, have
performed mathematical modeling work on this topic, and found that a
significantly smaller number of cars could supply all the mobility
needs of Singapore if they were automated and could respond to demand.
Specifically, their models suggest that 300,000 vehicles could serve
all mobility needs during peak times with waits of less than 15
minutes, and even a fleet of 200,000 vehicles could reduce wait times
to about 3 minutes during off-peak hours \cite{frazzoliSingapore}. This
compares to 779,890 passenger vehicles actually owned in Singapore,
according to 2011 numbers.

Along similar
lines, Christian Zulberti of the ENEL Foundation\footnote{A non-profit
  energy and sustainable innovation research group.} envisions using
autonomous vehicles to remove traffic lights in
cities.\footnote{Speaking at The Road Ahead Conference, hosted by
  the MIT SENSEable City Lab, Cambridge MA, November 21, 2014.} And Paolo
Santi of the MIT SENSEable City lab talks of replacing traffic lights
with a slot-based system, using cars that can communicate with each
other and with infrastructure to double the capacity of the
roadway.\footnote{Speaking at The Road Ahead Conference, hosted by
  the MIT SENSEable City Lab, Cambridge MA, November 21, 2014.} Each vehicle would request a slot to pass through the
intersection, and would be instructed to proceed according to the
orchestrations of a central control system---a fruition of
Vladimir Zworykin's work at RCA from the 1950s. Singapore is engaging in
controlled vehicle tests to work on just such ideas, to improve
efficiency with vehicle-to-vehicle communications and predictive traffic
patterns.\footnote{Lam Wee Shan, speaking at The Road Ahead Conference, hosted by
  the MIT SENSEable City Lab, Cambridge MA, November 21, 2014.
  Singapore seems to have a unique position in shaping this research,
  due to its small size, high population density, existing vehicle
  restrictions, and willingness to 
  leverage itself as a technology test-bed.} As we
might expect, these
design concepts and stories have telling connections to each other, and
to a web of other fictions, meshing well with the automated vehicle
infrastructures envisioned by science fictions such as \emph{Minority Report}.

Returning to reality, however, these concepts assume particular
communicative competencies. While 
some would seem to require fully automated vehicles, other
visions are amenable to a variety of different types of automation.
But all assume varying levels of persistent, networked surveillance of
vehicle positions, and require vehicle-to-vehicle or
vehicle-to-infrastructure communication channels to send and receive
data and instructions. Several of these ideas also seem to present
particularly classed views of who driverless technology would serve.
An intersection that requires an electronic device to request a slot
to cross will not gracefully tolerate the poor, or anyone who
does not own a smart car or smart phone, whatever his or her age,
race, class, or situation. Utopian visions of doubled capacity, when
described (as this was) as halving the wait-time to get through an
intersection, forget the complex dynamics of human behavior and our
tendency toward homeostasis of inconvenience: city planning has found
time and time again over the last 50 years that increasing the
capacity of roadways does not result in less congestion, only more
people driving \cite[p. 219]{marshallFuture}. One researcher I talked
to put the problem in terms of Jevon's paradox: at some point,
efficiency improvements will saturate, ``so the only true mechanism
[of reducing congestion],
at least the one that's been proven, through research, is pricing, is
through taxation.''

Nevertheless, these
sorts of envisioned futures support a get-out-of-the-way rhetoric. As
one researcher suggested to me in an interview:
``privacy-nuts'' and ``soccer-moms'' will be disturbed by the
technology, but we should press forward anyway. Such dismissive and
derisive categorizations, applied nonchalantly, suggest that anyone
getting in the way of progress is either a luddite or a fearmonger who
does not understand the technology. 

Another prominent design fiction regarding autonomous vehicles focuses
on personal mobility, especially for the elderly or disabled. Ryan
Chin, a notable researcher in urban mobility systems at the MIT Media
Lab, has worked on these systems as part of the Smart Cities
initiative. He suggests that in a world with an aging population, and
especially in the United States given our often lacking public
transportation infrastructure, personal mobility for a growing
population unable to drive themselves should be one of the primary
goals of future development.\footnote{Discussion with the author,
  August 20, 2014.} Mobility, as an aspect of personal agency, is very important both
for one's ability to live and provide for oneself, as well as for
fostering independence and feelings of self-worth. As a side benefit,
one researcher noted, providing automated mobility solutions should
help persuade more senior citizens who can no longer drive safely to
give up their licenses voluntarily, rather than risk a serious
accident.\footnote{This is especially true for those who do not have a
  family member they can count on to assist them.}

This primarily automotive vision is not the only model of urban
mobility through autonomous systems, however, and is fraught with
certain problems. John Leonard, an MIT researcher who was involved
with the DARPA Urban Challenge in 2007, has gone on record as a critic
of replacing cars with driverless vehicles on existing infrastructure.
Not only do weather conditions and negotiations with pedestrians and
other cars present potential issues, interpreting the gestures of
police officers and dealing with other atypical road events also make
this a troubled vision. One alternative is the PRT, or Public Rapid
Transit, model, which is already being deployed in certain
geographically restricted areas. A PRT researcher I spoke to
enumerated the benefits to the approach: separate tracks with
controlled access greatly simplify issues of collision detection and
negotiation with other vehicles, and reduce risks of injury from collisions if
they occur. Of course, controlled access requires new infrastructure,
and seems to call to mind the famous failure of the French Aramis PRT
project, so comprehensively covered by Latour in \emph{Aramis, or the
  Love of Technology} \cite{Aramis}. 

What gives us confidence in these visions? Why don't they seem
outlandish, like some of the works of science-fiction with which they have
been grouped here? Some of this has to do with their manner of
presentation, lent the gravitas of ``science'' by emanating from
scientists rather than from the
illusionary powers of Hollywood. 







We find ourselves in an era of seemingly continual technological
change that is strangely most noticeable in the most mundane parts of
our lives---how we shop, how we communicate, how we find
partners---almost as if ``designed by a bored researcher who kept one
thumb permanently on the fast-forward button'' \cite[p.
  7]{Neuromancer1984}. No longer does the future seem to be
defined by flying cars and jetpacks. Instead, it is defined by
information, and its collection and use. The stakes of autonomous
vehicles are thereby deeply intertwined with the
stakes of other networked information technologies. And it is in these
current models that we should start the search for why competing
narratives of self-driving car development matter. To illustrate these
issues, I will
take as assumed the general form of the autonomous vehicle as present
in Google's development work: though numerous other models exist, they
are the subject of another section. Here, I seek to take this design
to its logical conclusion and interrogate the social,
cultural, and informational stakes implicit in it.


\section{Data Gathering and Monitoring} 

 In May of 2014, the European
Union's Court of Justice 
ruled in the landmark Costeja decision that since Google is processing
``personal data,'' and acting as a ``data controller,'' it may be
compelled to remove links to pages containing personal information
from its search results \cite{ICO}. The EU ruling, in deference to European
tradition and contrary to that of the United States, places an
individual's rights to privacy above the ability of users to access
information online.\footnote{Specifically, ``whilst it is true that the
data subject's rights also override, as a general rule, that interest
of internet users, this balance may however depend, in specific cases,
on the nature of the information in question and its sensitivity for
the data subject's private life and on the interest of the public in
having that information'' \cite{COJCosteja}.} There is growing recognition that publicly
available data can be highly sensitive and that it may be beneficial
to allow individuals certain legal rights to control their own
electronic reputations, at least in particular circumstances. Costeja
opens the floodgates---from previously limited, targeted removals
through court cases and copyright law---to the possibility of free and
open public removal of ``public'' information.\footnote{This removal,
  however, is controlled and curated by Google itself, which has not
  waited for guidance on how to proceed but has forged forward on its
  own, as a way to set the standards by which information removal can
  be justified \cite{powlesChaparro}.} But much of our
information is even harder to control.

Many current data-driven business models\footnote{See for example
  Google, which has as its fundamental revenue stream proceeds from
  advertising, which is sold by virtue of it being more accurate or
  targeted than other channels can provide. Facebook's new ad network
  competes in a similar space. Many of these companies provide
  services for ``free,'' where in reality the services are paid for by
the user in the data they generate and/or the advertisements they
view.} are fundamentally united in that increases in
functionality are predicated on invasions of or encroachments on what
we used to think was private, and represent increasingly invasive data
collection and sharing at a massive scale---often this information is
used internally to improve services, but it may also be aggregated and
sold to third parties, and in either event may be stolen or leaked by
disgruntled employees or thieves. This
is particularly important as a precursor of things to
come in the autonomous vehicle space, as such vehicles will allow
large amounts of data to be collected and shared with other entities. 

Clearly, new technologies will require new information, new types of
sensing, in order to operate; but it may be that not all the
information that is collected should be shared outside the immediate
context of its use. What is sensed, and what can be appropriately
transmitted back to servers for processing and storage, is of
paramount importance: this is a question of privacy in context \cite{nissenbaum}.
Following Nissenbaum, privacy must always be seen in the context of
particular users and a particular use \cite[p. 2]{nissenbaum}. It is not that data about our
commuting routes, for example, should never be collected, but that collected data
should not be sent uncritically to any third party without our
knowledge or consent, a violation of our information norms \cite[p.
  3]{nissenbaum}. 
Automated vehicles are and will continue to be networked
technologies. This connectedness brings with it possibilities
for coordinating traffic and improving city planning, as well as significant
risks to privacy and security, whether devices are networked with each
other or simply connected to central servers. There
may be legitimate uses for certain types of sensitive information, but
while providing it to municipal governments specifically to assist in
city planning may be legitimate, selling it to advertisers to help
them design more effective billboards may not be. Privacy issues
involving motor vehicles are likely to become much more complicated as
vehicles become able to record more, and potentially ``know'' more,
about their passengers.


%% 2.1) data gathering and mapping
%% --this approach presupposes large amounts of data collection (maps),
%% and opens the way to more information about the passengers (2 p)
%% --see Google's patents in particular which talk about sensing things
%% like the number of passengers (4 p)

With what networks, and for what reasons, will autonomous vehicles be
connected? Current driverless car concepts depend on networked
information for vehicle guidance. While, historically, certain
guidance systems have been insulated from communications---inertial
guidance systems for intercontinental ballistic
missiles are a particular example of this \cite{mackenzie}---and
Google's vehicles use inertial navigation devices \cite{knightFurther}
alongside other sources of position data, current navigation systems
depend on global positioning satellites. Google's approach to
driverless car development currently necessitates accurate
and expensive\footnote{An autonomous vehicle researcher at MIT
  quoted prices in the range of \$70,000 to \$100,000 per device for the
GPS alone, while noting that those would of course come down with
greater production volume.} differential GPS receivers.

But a number of other developments already suggest that autonomous
vehicles will be connected to more than just global positioning
networks. GM's OnStar service already connects equipped vehicles to central servers for
purposes of safety, security, and convenience. The system has the
ability to automatically alert the authorities in case of an accident
or theft. It also provides vehicle diagnostics to the owner's tablet
or smartphone, and the OnStar mobile app also allows the owner to
configure settings, lock and 
unlock doors, and remotely operate the lights and horn \cite{onstar}.
While these vehicles are not (yet) autonomous, OnStar's present
capabilities are representative of features that will become more common in highly
connected and computerized vehicles, including autonomous ones.

Additionally, vehicles that can receive information from each other and from the
roadway are a top priority for the NHTSA, and have been on the
research agenda for decades \cite[p. 11]{wetmore}. These concepts would
collect and use data to streamline traffic flow and provide
information to reduce delays and accidents. But there is potential to
do far more with the available information,
and data collected by the vehicle to make possible its own functioning
could be made to serve other purposes. Information about the vehicle
and its surroundings, including the locations of cars and pedestrians,
precise GPS coordinates of the vehicle itself, and the vehicle's speed
and acceleration, not only represent important knowledge for
path-finding by the vehicle itself, but new sources of potential
revenue for the groups in position to collect them. Uber, which
through its GPS-enabled ride-hiring application still collects only a
fraction of the data that would be available through a self-driving
vehicle, has agreed to share its ride data with municipalities for purposes of
city planning \cite{uberJardin}. Though this data is ostensibly being
shared for the public good, it also serves private ends:  to curry favor with
authorities that might otherwise attempt to shut the service down. The
ride data can also be used to provide internal predictions to help
Uber increase revenue by directing drivers to the right locations at
the right times.\footnote{One potential reason for municipalities to
  collect such data is dynamic road pricing. This is not fundamentally
a bad idea if done in a way that is as non-discriminatory as possible,
and as one of my informants noted in an interview, may be 
the only reliable way to reduce congestion. However, putting another
corporate entity in between road users and the road is not necessarily
a good idea. It strikes me, since I am writing this shortly after the
FCC's reclassification of Internet service under common carriage laws,
that if we find ourselves in a battle over ``road neutrality'' in a
decade or two, it will have a decided air of irony to it. Pervasive
tracking and self-driving allow for all manner of road pricing
schemes, some of which will be unduly discriminatory.} And it would
not be far-fetched, in the current 
information landscape, to see such information sold to third-party advertisers.

% consider 1-2 sentences on risks of connected devices;
% http://blog.kaspersky.com/internet-of-crappy-things/

These sorts of security and privacy issues are not unique to
autonomous vehicles, nor even to networked vehicles. What Roger Clarke
calls ``dataveillance'' is already possible: electronic tolls
allow for a measure of tracking \cite[p. 25]{nissenbaum}. Networked
traffic cameras are 
already being used to amass large databases of information about
(non-networked) cars and their travel patterns by reading passing
license plates \cite[p. 26]{nissenbaum}. With sufficient coverage of
cities and regions, 
these databases can capture the movements of large segments of the
population, and therefore include potentially sensitive information
about people's movements. Despite this, their collection and use
remains largely unregulated.\footnote{Groups such as the EFF and ACLU
  are attempting to do something about this \cite{kayyaliEFF}.}
California police are coming under 
criticism from civil rights groups about their mass collection of
vehicle position information through police license plate cameras \cite{maassCivil}.
Clearly, these types of contested data collection are already
possible. But networked vehicles, with their arrays of sensors,
provide more avenues of data collection, and therefore stand to
increase our present-day problems with mass surveillance and personal
privacy.

Increasing surveillance by government and law enforcement is not the
only threat posed by the potential for ever-more-fully networked
vehicles. Networked technologies do not have a great security record,
and there is no reason to believe automated cars will be any
different, barring greater scrutiny and
regulation.\footnote{Commercial networked devices routinely lack basic
security measures. And for a large number of other systems, security
is defeated by being out-of-date or by using default passwords that
were never changed. See for example \cite{zetter}.}
Networked capabilities within vehicles are being used to
implement vehicle kill-switches, capable of remotely disabling
vehicles on the whim of the controlling organization---for example, if
vehicle rental or loan payments are not completed on time. The
potential for this is quite troubling: not only could drivers
potentially find their cars impossible to start at an arbitrary time,
with serious implications for safety, the existence of these devices
presents opportunities for hackers to cause nuisance or
harm \cite{goodman}.
Existing vulnerabilities in web services and dealership practice
allowed Ramos-Lopez to hijack 100 cars from a Texas dealership and
either disable them or set
the horns
going \cite{poulsenHacker}. A
high-school student at a security camp in July of 2014 was apparently
able to hack into an automobile with \$15 worth of components, and
gain access to lights, horns, and even the remote start
feature \cite{bigelow14}.
The
potential for this kind of access gets more frightening when one
considers the effects an integrated, network-connected vehicle that is capable of
moving and navigating on its own might have on the reach of a hacker or negligent employee.

Additionally, Google has envisioned vehicles that can determine their number of
occupants, and use facial-recognition or other biometric systems to
identify them. According to one patent \cite{predictPatent}, these vehicles could prevent
unauthorized persons from putting a child in a car, prevent convicted
sex offenders from operating their vehicles within the
legally-required distances of schools and playgrounds, or prevent a
car's doors from being opened (even from the inside) by a child unless
an authorized adult is present. These are only visions, and patents
are notorious for trying to cover as many possible angles of a
technology even if they are not intended to be applied in practice.
But these suggestions represent a perspective on safety and societal
order that posits technological surveillance and enforcement as an
appropriate preventative measure against criminal behavior. Whether or not protecting
against these threats is an appropriate use of this information is a
matter for societal judgment, but such proposals, if enacted, would
require these vehicles to have unprecedented levels of very sensitive
knowledge about people and their lives: biometrics, criminal
histories, family and trust networks. 

Prevailing data ideology tells us that we can understand ourselves better through
collected data, that we can use the data to determine patterns we never knew were
there. Out of this ideology come publications like Uber's
blog,\footnote{See blog.uber.com. One particularly noteworthy post,
  from March of 2012, was about customers' ``Rides of Glory''---in
  other words, one-night-stands. This post was later taken down by the
Uber team, but articles about it remain \cite{gigaomHarris}. Another
post about the connection between Uber rides and areas of
crime---notably prostitution---was also removed \cite{venturebeatObrien}.} which
describes customer ``insights'' gained through their ride data, which
are ostensibly interesting to the public. ``Look here,'' they seem to
say, ``we can tell you about \emph{you}.'' But it is important to
recognize that the ideologies of data collection---for corporate
profit and public use---are deeply intertwined, and strongly
influenced by the possibility of using machine-learning techniques and
statistical analyses to find and exploit patterns. Automated vehicles
are part of this general culture, and, without significant effort to
resist erosions of privacy and data protection, stand to open our lives
to greater scrutiny in terms of the means of and reasons for our
personal mobility.

%%More on the problems of data ideology? or not the place for it?


\section{Maps and Mapping}

The information that may be collected and processed by automated
vehicles is not only about human inhabitants of the environment: other
types of data collection are also implicated in current
visions of the driverless car project. In order to drive with us,
autonomous systems will have to understand,
for at least a practical sense of ``understanding,'' traffic rules and
their accompanying signs, signals, lanes, and customs. This is, at its
core, a highly complex problem of interpretation and representation
for machines. Human understanding is built
through years of experience: it is through existing as a human being
in a particular cultural context that we know to drive on roads but
not on sidewalks, and how to tell the difference. But we do not have
this luxury in training machines. %%TODO2 consider this last sentence

Localization has been a fundamental issue for robotics since the
beginning of the field. In order to figure out how to act, a robot
needs to know where it is; and in general, robotic environments can be
expected to change rapidly with motion, and to vary significantly from
place-to-place \cite[p. 4]{SLAMbook}. Motions cannot, for the most part,
be pre-planned:  they must continually be reevaluated in new and
emerging situations, as the robot moves through the environment. As a
key part of this puzzle, the robot must be able to determine its own
relationship to some target location it needs to get to: this
``target-robot relation'' is an ``unavoidable''\footnote{It is
  possible to create robots that navigate without any modeling of the
environment, as Fern\'{a}ndez-Madrigal et al.\ point out. However, this
severely limits the range of behavior that is possible. Without
reference to the environment, errors slowly accrue, and a system that
navigates in a known space without localization will find itself
getting further and further off-track.} component of successful
navigation \cite[p. 5]{SLAMbook}. This relation can be expressed
in different forms---either quantitative representations such as maps,
or logical, prepositional statements---and these representations carry
their own techniques of interpretation. 

But there is more to navigation than pure localization. While
localization is the usage of known elements in the environment to
estimate a robot's position, those elements themselves must be known
in order to perform this process. Mapping is the complementary
process, the estimation of ``\emph{unknown} spatial relations that
exist between environment elements'' in order to allow for subsequent
navigation \cite[p. 5]{SLAMbook}. We face a somewhat tricky
precedence problem:  in order to build maps with
autonomous systems, the systems must know where they are; in order to know
where they are, they must have maps to measure
against \cite[p. 6]{SLAMbook}.

Consider the history of AI we already reviewed.
Localization was part of the cause of the poor performance of the
robot Shakey: world modeling through a symbolic approach was slow,
which was rendered even more problematic by the relative lack of
computing power at the time \cite[Afterword]{mccorduck}.
Though one way to determine robot position in the world is through
simultaneous localization and mapping (SLAM)---a serious research area in
robotics, as it is ideal for regions that are impractical to map
beforehand, such as ones that are always changing---it is easier to localize by comparing
measurements to a known map. However, that implies that the map is
pre-made, and therefore sets limits on the rate at which the
environment can change and still allow the robot to operate. By
contrast, Brook's subsumption architecture involved an
attempt to avoid the localization problem by pure sensory response to
the environment; but building a robot that exhibits complex and
predictable behaviors in complex environments through this approach
is exceedingly difficult, as evidenced by Brooks's shift to hybrid
modeling approaches. But new sensors providing new kinds of data, and
new algorithms and techniques to cope with uncertainty---such as those
of probabilistic robotics \cite{thrunProb}---have allowed
significant progress in the field. Today, localization and
mapping problems alone are both considered ``satisfactorily solved in practical
situations''---though SLAM techniques
are still somewhat less reliable or developed---given sufficient
computational resources and environmental data \cite[p.
  5-6]{SLAMbook}.

%%Fernández-Madrigal, Juan-Antonio, and José Luis Blanco Claraco.
%%2013. Simultaneous Localization and Mapping for Mobile Robots :
%%Introduction and Methods. Hershey, Pa: IGI Global, 2013. eBook
%%Collection (EBSCOhost), EBSCOhost (accessed February 21, 2015).

Here, the connections of autonomous systems to information networks
again become important. The vehicles in the
DARPA Grand Challenge did not navigate ``on their own'': they used
inertial GPS
to follow a path laid out for them in advance, applying local autonomy
only to avoid obstacles.\footnote{John Leonard,
  discussion with the author, December 3, 2014} And though successful
road tests have been accomplished without navigational assistance,
depending only on visual stimuli (such as the EUREKA PROMETHEUS project
in the 1990s, which navigated solely with cameras and therefore could
only follow the road in an automated fashion, rather than navigate to
given GPS coordinates) \cite{ulmerVITA}, modern systems are tending to use more external
stimuli, rather than fewer, in an attempt to increase safety and
vehicle capability. Even as
advanced as it is, Google's autonomous vehicle technology requires
hyper-detailed 3D maps in order to operate properly on public
roadways \cite{gomesObstacles}. These maps are generated by human-piloted vehicles outfitted with
special sensor arrays, like the LIDAR Google uses for their autonomous
vehicles, which drive a route and collect data that can be used to
reconstruct the model used for future drives.\footnote{Nhai Cao
  (Global Product Line Manager at TomTom), presentation at The Road
  Ahead Forum on Future Cities 2014, Cambridge, MA, MIT, November 21, 2014.} 

Google's vehicle does not operate in a SLAM mode.
Pre-made maps inform the vehicle where stoplights, signs,
and curbs are, reducing the computational load on the machine in the
crowded visual landscape of driving, and allowing it to focus on
elements of the environment that are changing rather than those likely
to be static \cite{gomesObstacles}. For example, prior knowledge of
speed limits should make the car's 
behavior more reliable and predictable in all conditions, even if
speed limit signs are missing or obscured---consider how often human
drivers, when faced with an absence of signs, base their behaviors on
supposition or prior knowledge. When Google's car was certified for
testing in Nevada, Google was allowed to pre-select the route the car
would take, so that they could build the comprehensive model the
system requires beforehand \cite{harrisNevada}. The system would likely not have been
capable of passing a test in which the examiner could have added
detours on the fly. And though Google claims to have driven more than
700,000 miles with their cars, those are not 700,000 unique miles. A
limited, thoroughly pre-mapped route has been driven many times to
achieve those numbers \cite{gomesCircles}.


%%TODO2 Karen Kaplan and MAPS
Mapping claims a unique capability to
represent the real, objectively
and diagrammatically, but also requires that world to remain largely
static, at least on the order of how long it takes to update the map
for a particular region. The necessary level of continual mapping is a massive task if the
vehicles must be usable everywhere. The United States alone contains almost
8.5 million road miles,\footnote{Data as of 2008 \cite{carneyMiles}.}
and it took years for Google Street View to acquire the level of
coverage it currently has. The maps required for driverless car
localization are a significantly more difficult project in terms of
amount of data, reliability of data, and therefore frequency of
updates. The utopian discourse of driverless cars
implicitly suggests that such vehicles will be available everywhere,
and are the solution to nationwide transit problems. But widespread
egalitarian access to maps-based devices depends upon a rapid,
widespread mapping initiative. The success of such an initiative is
dependent not only on a large amount of human effort and capital
investment, but a series of decisions about what regions should take priority.

TomTom \cite{tomtommaps} and Nokia \cite{ubergizmo} both claim to be attempting such
a mapping project,
but it will not happen immediately or all-at-once. More likely,
certain areas will be 
mapped and restricted, or separate public-rapid-transit
systems will operate on divided roads that can be carefully
monitored.\footnote{This point came up in a discussion with a
  scientist with experience at a UK-based firm working on public
  rapid transit, or PRT, systems.} While
Mountain View, California may be mapped early, rural West Virginia or
Northern Maine may not be mapped as soon or as
frequently.\footnote{And it is worth noting that 60 percent of road
  fatalities occur on rural roads, and that the rate of fatalities per
mile is twice that as for urban roads \cite[p. 11]{broviakCars}.}
Inequalities may be increased if routes frequented by upper-middle
class professional commuters, most likely to own new autonomous cars,
are mapped first, while roads around low-income communities are
ignored. Such decisions are easy to imagine being justified by
commercial exigencies---limited funds, potential markets---but would
cut directly against the utopian narratives of driverless cars.

Furthermore, the seemingly universalizing forces of maps and computer programming
have a tendency to hide issues of geographical and cultural
specificity, which are rendered invisible in these utopian narratives.
But though engineering practice holds that these issues are
conquerable, they should be anything but invisible. More than
abstract problems of localization and mapping, automated vehicles
present the problem of having to exist in an environment that is
highly complicated, varied, and cultured. Maps alone are insufficient
for anything but the most simplistic view of vehicle operations.
Programmed devices
must know about speed limits, about traffic
lights, about rules of the road that were never designed for
autonomous systems. These devices must respond to human caprices and
be adapted to longstanding, ingrained laws and habits. They must
include historical knowledge, rooted in the legal and social histories
of roadways, which may differ between cities and states, and certainly
between countries across the world. Local customs and behaviors vary,
and even if maps are 
available, the same vehicle programming may not work for Los Angeles,
Boston, and the rural Midwest, let alone Singapore, Mumbai, and Cairo.
The map, for all of its objective standardization, still represents
real places subject to cultural histories and vulnerable to
socio-economic dynamics. These social and regional issues are often
ignored in the driverless vehicle narrative, but nevertheless stand to
be critical to the manner in which these technologies could
enter everyday life.



\section{Machine Vision}
%% 2.2) machine vision
%% --pull from the Spectator paper (4 p)
%% INCL. machine learning
For all this, however, vehicle localization and sensing depends upon
visual interpretation of the vehicle's surroundings. Interpretation of
the world around us is a task that seems
particularly easy for human beings, but particularly difficult for
machines. The invention of the photocell, early a tool for workplace
monitoring and surveillance, provided a simple channel through which
electrical systems could respond to the amount of light reaching
them \cite[p. 44]{faxed} \cite[p. 361]{nyeElectrifying}. Though the photocell can easily
provide a computer system 
with access to brightness information over time, perceiving detail and
depth, identifying shapes, and interpreting expression and motion are
all capabilities of human vision that require more sophisticated
technologies to reproduce. DARPA's 1983 Strategic Computing Initiative
included image interpretation as one of its main
focus areas \cite[Afterword]{mccorduck}. But it is only relatively recently that real-time video
processing, needed for camera-based navigation, became feasible for computer
systems small enough to fit into a standard automobile \cite{vamors-p}. And machine
vision problems, including object recognition and scene
interpretation, continue to be difficult, even with increased
processing power and new algorithms. 

As an engineering discipline, computer vision takes a decidedly
practical and reductionist view of what it means to see. The goal is
generally not to achieve creative interpretation or aesthetic
valuation, but to identify particular objects in a scene, to recognize
faces, or to differentiate free space from things a robot should
not run into\footnote{SRI's Shakey robot, for example, made navigation
  plans via rudimentary computer vision tools, specifically sonar range-finding.}. But this so-called objective focus still encodes
certain subjective judgments about objects (including people),
behaviors, and intent. Consider, for instance, the issue of object
detection. While it may be an objective question whether or not a
particular object (e.g. a wrench) is physically present and visible in
a scene, it is not
necessarily so self-evident which objects are noteworthy or important
to detect (e.g. does the worm gear on an adjustable wrench count as a
separate object in addition
to the wrench itself?)---these choices depend on applications, and the judgments of
designers as to what is worth measuring. And while computer vision is having success
with object detection, a wide variety of human knowledge
about objects and scenes is missing in current computer
models, including propositional understandings (``what would happen if
. . ?''), projections about occlusions (``what is behind that?''), and
connections to other sensory modes (``what would that feel like if I
touched it?'') \cite{gomesJordan}.

Regardless, though vision has not always been the sensory mode that dominated
autonomous vehicle research, it is a particularly attractive sense to
use as it is integral to how humans drive. In an attempt to build
autonomous vehicles that can operate without infrastructural changes,
research has moved away from automated systems controlled and
constrained via tracks and cables toward vision-guided
systems. New approaches pioneered by Ernst Dickmanns at
University Bundswerhr in Munich, with the vision-guided VaMoRs van,
were continued via the EUREKA PROMETHEUS project in 1987, in which
Dickmanns and Daimler-Benz built cars guided by analog video
cameras \cite{ulmerVITA-II}. Like the earlier VITA project by Daimler that used an
analog video-camera signal processed through a framegrabber, these
cars (VITA-II and VaMoRs-P) digitized analog video at relatively low resolutions. The
features the systems searched for, including lane markings and other
cars, are geometrically distinct and visible even in small
images \cite{ulmerVITA}. That these vehicles were designed primarily for
highway operation under constant human supervision made this strategy
acceptable, since pedestrian interactions were likely to be
rare.\footnote{The documentation of the VITA-II mentions pedestrian
  interaction management as future work necessary for changing the
  working domain from the highway to urban and country environments \cite{ulmerVITA-II}.}

%%Can cite SivakSchoettle p. 3 on weather; and they cite Lavrinc 2014 there

The apparent primacy of computer vision, however, only holds for fair
weather. Rain, sleet, and 
snow interfere with vision-guided systems, and currently prevent them
from operating safely, as numerous more skeptical news articles are
happy to note \cite{knightFurther} \cite{gomesObstacles}.  While the
human eye is generally very good at
pattern recognition and filtering out noise, getting computers to
appropriately recognize all 
the necessary road objects already taxes
cutting-edge techniques. Precipitation not only makes roads slippery
and less safe for all drivers, it reduces visibility (which decreases
the distances at which objects can be detected, and therefore also the
time the system has to react) as well as potentially decreasing image
contrast and presenting interference to the image---due to drops or
streaks on the glass through which images are taken
\cite{rainADAS}. These issues also 
affect people (though our brains can filter out raindrops on the
windshield rather than interpreting them as obstacles), but human
drivers would, at least ideally, choose to drive more slowly given
adverse conditions. One emerging research area
in automated vehicles is raindrop detection for automated driver
assistance systems, in order to help such systems compensate for
reduced visibility, or alter their behavior to be more conservative in
these situations. However, ``developing algorithms that work perfectly
under all weather conditions appears to be unrealistic,'' so such a
modular and cautious approach may well be necessary in order to build
robust systems \cite[p. 50]{rainADAS}. Though commercial use of
vision-guided self-driving vehicles on 
sunny days may indeed be possible in the near future, use in
less-than-ideal conditions is likely to be limited for some time,
with human oversight compensating for algorithmic deficiencies. 

Vision-guided systems, now using digital video cameras and
off-the-shelf consumer hardware, do have the benefit of being inexpensive
and insensitive to interference from other nearly devices (unlike
sonar, for example, which becomes problematic in crowded
situations). Some contemporary commercial systems, such as that
developed for Mercedes-Benz's 
self-driving S-class, which is slowly finding its way into consumer
vehicles, are primarily guided based on such visual sensors \cite{makingBertha}. To
these sensors, recent research has added roof-mounted LIDAR arrays.
LIDAR, short for Light Detection and Ranging, is a distance
sensor, which is applied in vehicles to scan the environment with a
rotating array of laser beams to create a detailed 360-degree
representation of objects and their distances. This technology
solves some of the difficulties of image interpretation by default, as
it can provide highly-sensitive information about free space and
obstacles. Shape-detection algorithms can then be used (in addition to
vision-based data) to classify obstacles as different types of
objects: pedestrians, bicyclists, cars, and trucks.\footnote{This also
  works for other kinds of physical objects. See for example the
  detection work done by \cite{fukuda}.}

%% DONE?
%% --Göde Both and reluctance to use Machine Learning (1 p)
%% ---brittle, unpredictable, difficult to ``tune''
%% --Sensors issues and inability to distinguish crushed newspaper from
%% rock (1 p)
%% ---requires a lot of other information, more than simple physical
%% volume
%% ---object identification work in machine vision is slowly progressing,
%% but then we hit the above problems of VV\&T for such models (2 p)

While much computer vision research uses machine-learning algorithms to detect
objects, engineers in automotive applications are justifiably
reluctant to rely on machine-learning: as G\"{o}de Both has noted in his
ethnographic research on developers of driverless cars in
Europe \cite{bothpt1},
machine-learning techniques are brittle and unpredictable \cite{bothpt2}: neither
characteristic makes them suited for software that must be highly
reliable and on which people's lives literally depend. So mixes of
manual and machine-learning approaches are used to do object
detection. Though machine learning can be a highly effective
technique, it is generally difficult or impossible to know what the
system has actually ``learned'' and therefore how it will react in new
and unknown situations \cite{bothpt2}. Pedestrian detection algorithms search for
personlike shapes, where
``personlike'' is determined by, for example, training a classifier
using thousands of images previously labeled (by people) as being images of humans, so
the system can learn the features that correlate with a person being
in a particular region of an image. In a sense, the computer can be
said to develop
a ``concept'' of a person. So long as the right features appear in each
new situation, this approach works; but what the computer has
``learned'' is essentially black-boxed, and resists introspection.
Even discounting these concerns, shape-detection is not a complete
solution, as the knowledge it provides about objects is only
skin-deep: the sensors cannot differentiate a rock from a crumpled
newspaper, and Google's car will serve to avoid
both \cite{gomesCircles}. Further distinguishing objects requires a lot
of information besides position or simple physical volume,
including fine-grained information about the object's surface
appearance, and interpretation of physical properties from observable
behavior (e.g. bouncing or rolling). None of this is fundamentally
impossible, but it presents necessary areas for research.

Detected categories allow the system to make
statistical predictions about likely types of behavior: according to
one of Google's patent applications, bicyclists are likely to be more
erratic than trucks, and should be treated
accordingly \cite{predictPatent}. These sorts of predictions are
something that human drivers do consistently, and are therefore also
likely important to how autonomous vehicles may drive.\footnote{The
  DARPA Urban Challenge crash, the first crash between two autonomous
  cars, provides an important lesson on the vagaries of object
  detection: the classification threshold between moving and
  stationary, set too high, allowed one vehicle to interpret the other
  as stationary, leaving no room for unexpected behavior \cite{collisionPaper}.}
Categories alone, even if achieved, are also insufficient if detected
objects are treated solely as obstacles to be avoided. John Leonard
has attempted to capture situations he believes will cause
trouble for current vehicles. His collection of photographs and videos
includes many situations with police 
directing traffic, especially when combined with sun glare or
occlusions of the sight lines---such as by a mail relay box---that
would make predictions regarding oncoming 
objects difficult.\footnote{Discussion with the author, December 3,
  2014.} His take-away from these situations is that there 
is a significant body of edge-cases that make full autonomy
impractical for the foreseeable future. Instead, humans will be
required to account for these sorts of difficult perceptual situations. 

%%TODO2
%%<Include quotes from the interview>

However, because autonomous cars see---putatively ``as we see''---their sight can
be leveraged as visual evidence of their operation. Computer vision
systems that identify 
pedestrians can be shown to do so, via the detection boxes that act as
diagnostic tools for researchers and direct representations of
internal system information.\footnote{See for example Volvo's ad for
  the S60's pedestrian detection capabilities \cite{volvovideo}.} The new technologies of vehicle
automation thereby produce through 
their operation new forms of evidence, which can be presented through
electronic information media. We can point to these images and identify
that the vehicle is operating as it should. The box inscribed around
the pedestrian shows this plainly. The 3D environment scan from the
LIDAR system presents the same opportunities for ``transparent'' visual
proof. Three-dimensional shapes, standing in stark relief against
the background, bear witness to the sensory operation of the vehicle.
These shapes too are demarcated by boxes, which represent their
computational transformation from information into an object or
artifact of interest. Because the visual detection technologies used
by autonomous vehicles are compatible with the visual technologies of
media representation, new types of seeing are opened to us.

%%Kelly Joyce on fMRI: Magnetic Appeal: MRI and the Myth of Transparency
A coincidence of sensing and representation in the evocative and
powerfully persuasive medium of visual representation stands to shift
the way we perceive driverless systems in their operation, providing
us a different manner of insight and introspection, but also
potentially a different level of obfuscation, than ever before. When
considering how these vehicles are presented to us---and the fervor with
which some researchers demand that we accept them\footnote{For
  example, technical speakers at the The Road Ahead conference at MIT,
hosted by the SENSEable Cities lab (November 21, 2014) suggested
concerned users ``get out of the way'' of coming fully autonomous
technology.}---it is important to 
remember the many black boxes of information-processing behind the
naturalized image of the sensor readout. The rhetoric of the
self-evident visual image, as it operates in the autonomous vehicle
space, can be made more visible by considering contemporary medical
imaging. As researchers have noted about the prevalence of brain
images from fMRI studies in the popular press, these depictions have
strong rhetorical impact on discourse about neuroscience. Colorful
pictures of the brain, as the inheritors of Enlightenment notions of
visual evidence \emph{par excellence}, have a way of convincing the viewer of
the validity of reseachers' claims about physical brain locations and
their effects \cite{lehrerNeuro}, especially since those pictures are presented as the
direct, transparent products of sensing techniques \cite[p.
  76]{kellyMagnetic}. The conflation of 
human and machine vision which makes these images so confusing is as
misleading in vehicle navigation as it is in neuroscience.\footnote{In
her book \emph{Magnetic Appeal}, Joyce argues that ``seeing does not
equal truth or unmediated access to the human body,'' but that
practices equating these are so common that images are often used to
stand for such, despite the doctors involved being well aware of how
institutions and social practices shape these kinds of evidence
\cite[p. 76]{kellyMagnetic}. But she also notes that
popular narratives are particularly bad at heeding the sorts of local
practices that shape technological and image-based knowledge, falling
prey to the ``myth of photographic truth'' \cite[p.
  75]{kellyMagnetic}. These tendencies seem to be of great relevance
when considering other complex, technological projects dependent on
imaging and which use images rhetorically, to stand for the ``truth''
of their ability to perform a task---such as detect a pedestrian in a
crosswalk.} Significant 
statistical processing is necessary to make sense of data in either
realm, processing that is generally beyond the public's gaze
\cite{koerthFish}, but the 
end result leaves the viewer with a false sense that something real
has been sensed, revealed, and affirmed by the image.


%% 2.2) functionalism over understanding
%% --for better or worse these devices are not humanlike in their
%% understanding
%% ----AI winters & the rise of commercial ML AI with much smaller goals
%% (see Wired articles, etc) (3 p)
%% --the ideology that drives them is an engineering mindset: ``just make
%% it work'' (2 p)
%% --pulling the quote from my USC talk: also interesting to note
%% regarding 1.2 that it seems to expect a humanoid robotic chauffeur in
%% a normal car! (1 p)
%% -and of course interviews with people (Ryan Chin; Walker Smith etc.
%% where they matter to this type of vehicle)
\section{Functionalism, Utilitarianism, Ethics} 
We should likewise not be deceived by an apparent parallelism of human
and machine knowledge, or an elision of the difference between human
and machine ``cognition.'' Self-driving cars will not be humanlike in
understanding, even while they can detect and identify pedestrians as
objects of interest within a particular epistemological frame. Whether
or not their machinic perspective is to be lauded is a deep and
philosophical question, but the robots envisioned by current AI
ventures bear little resemblance to those of Asimov, or the dreams that grew
from the Dartmouth Conference. While a wide range of software companies and
startups have entered the AI 
industry, these companies are not primarily focused on
general-purpose AI.\footnote{The primary divisions responsible for two
representative Google projects, Chauffeur and Deep Mind, are
geographically and organizationally separated. A Google Deep Mind
employee I spoke to at MIT said that they had had only very limited
contact with the Chauffeur team. September 10, 2014.} Though certain
ventures still hold out the dream 
of doing so, several AI winters have shown that the creation of
general intelligence is very difficult, and is by no means around the
corner. Those working in the field are well aware of
this \cite{sofgeAIFears}. So much current work is
fundamentally utilitarian, building systems with clear goals, metrics
for success, and market segments.

However, the utilitarian model of AI makes good sense for a number of reasons.
First, much can be achieved with current technologies. Rather than
focusing on what is potentially an exceedingly long-term project,
which would carry much greater risks and more distant rewards,
achieving short term goals is an 
attractive prospect. It can attract investment because it can be
profitable sooner. Though short-term commercial viability is not
necessarily applicable for this vision of the self-driving vehicle, it
is possible, at least, to have working prototypes on the road, generating
interest and publicity, even in relatively controlled conditions. Taking
journalists for test drives drums up interest, even if the technology
is not ready for full-scale deployment. Make no mistake: current
prototypes are highly capable, regardless of their faults, but any claim that
they ``know'' or ``understand'' is a tenuous misuse of words which
could no longer credibly mean what they generally do. Such use opens,
rather than closes, questions.

Second, humanlike characteristics may not even be helpful, in general, for building
specific applications. One would likely not want one's self-driving car
to be preoccupied or emotional. It may be
that, for utilitarian 
purposes like driving, many characteristics of the human are
detrimental, their elimination helpful and intentional. Much of the
discussion around why autonomous vehicles are necessary centers on
just such qualities: distractability, sleepiness, lapses in
concentration.\footnote{A vast proportion of articles make a claim
  like this. For example: ``They don't get sleepy or
  distracted, they don't have blind spots, and there is nothing on
  their `minds' except getting safely from point A to point B'' \cite{merrill}.} We
would not wish to emulate such 
characteristics in robotic systems. However, though these are human
capabilities, presented in this way they exist largely as caricatures
of the human. People possess a
variety of other capabilities which might be helpful to many AI
applications. As AI researcher Doug Lenat wrote in 1997:
\begin{quote}
Before we let robotic chauffeurs drive around our streets, I'd want the
  automated driver to have a general common sense about the value of a
cat versus a child versus a car bumper, about children chasing balls
into the streets, about young dogs being more likely to dart in front
of cars than old dogs (which, in turn, are more likely to bolt than
elm trees are), about death being a very undesirable thing \cite[p.
  122]{ekbia}.\end{quote} 

This is a difficult knowledge and perception problem. But even more,
it is an issue of selfhood, embodiment, even sentience. While cats,
children and bumpers can be identified as objects, and children
chasing balls into the streets can be identified as patterns, a
computer programmed to respond to these stimuli may respond correctly
without ``knowing'' anything. While a machine can be programmed to avoid
running into people, can it have any understanding of death? Can it be
programmed to ``feel guilt''? Does it need to?

%CAN I FIND RESEARCH ON reactions to humanoid vs. non-human
%AI drivers

Current approaches, however, assume this kind of deep understanding is
unnecessary, both for the technical creation of such vehicles as well
as their public acceptance. If the machine behaves in an appropriate
way, like an ideal human driver, it will be a ``satisfactory social
prosthesis'' in Collins's terms, and ``will not appear alien'' even if
it cannot truly ``understand'' \cite[p. 31]{Collins}. Driving, from
this perspective, involves only behavior-specific acts, which can be
satisfactorily emulated using the behavioral coordinates of
action \cite[p. 33--37]{Collins}. And yet the ideology of artificial
intelligence, the focus of the field itself, is bound up in the idea
of ``intelligent'' machines that can be said to ``know.'' And the
knowing involved in action, as opposed to mere behavior, may be
important for human acceptance of highly automated technologies. It is an
interesting read on the changing 
times to notice that Lenat's statement seems to suggest humanoid
robotic drivers operating regular cars. As well as moving toward
functionalist systems, the industry has moved toward embedded systems
within devices, systems that make no pretenses to be humanoid, but
instead revel in appliance-hood. While a dominant dream might once have been
to build a world full of humanoid robots, the conceit of modern
consumer AI---IBM's Watson cloud API, ``internet of things''
approaches---is that we can make everything smart. Whether this shift
in the form of AI 
systems makes customers more or less nervous about computer-driven
vehicles is an empirical question. But it certainly suggests a desire
to make the systems more 
invisible---and actually has important implications for human-computer
interaction, as we will see in chapter \ref{chap:3}. Though it may seem
obvious, it is critical to remember that the devices we discuss here may
have the properties Lenat mentions only in the way that the University
of T\"{u}bingen's AI Mario may be
``self-aware,'' possessing programmed constraints that cause it to
avoid people, and perhaps expect children to behave erratically.
Nonetheless, despite lacking the facilities for actual ethics, these
devices possess an implicit ethics: the
behaviors of these devices will instantiate the moral and
ethical judgments of their human creators, based on
human-authored heuristics and statistical predictions. They will not
\emph{act} ethically, but must
\emph{behave} according to their programmatic ethics. This will be
fundamental to how we understand our relationships to such
complicated technological systems.

%%ETHICS
Robot ethics is an issue of growing importance to society at
large---given the rapidly-expanding uses of robotics for labor, military,
research, entertainment, and healthcare, among other regimes \cite[p. 5-6]{patrickLin}---but we should be very clear what we mean when we
discuss it. Lin, Abney, and Bekey's collection is wide-ranging,
considering perspectives including safety and errors (how should 
robots be introduced in order to minimize adverse effects?), law and
responsibility (when accidents happen, who is liable?), ethical codes
(how, and with what ethical frameworks, can robots be programmed to
operate ``ethically''?), and social impact
(how do we weigh potential for robots to eliminate
jobs?) \cite{robotEthics}. Here, I consider ethical codes specifically,
since they interact with functionalist approaches. The question of
driverless car ethics has spurred quite a volume of news articles,
with varying command of the questions
involved.\footnote{One article discusses a notable
  researcher who has purportedly ``stressed the need for driverless
  cars to be flexible enough in their 
engineering to be able to make ethical judgements that aren't
necessarily written into their programming,'' a statement whose
meaning is difficult to parse \cite{jessicaDavies}. Such statements
presumably 
refer to judgments that are not explicitly defined by programmers, but
instead reliant on some kind of more general ethical calculus. A
program that does things that are not in its programming at all,
however, is incoherent. Even bugs, though unintentional, are part of
the program. This sort of phrasing is unfortunate, as it
risks being read as a robot somehow operating ``outside of its
program,'' making ad-hoc ethical decisions based on criteria of its
own invention (even an ethical system trained via machine-learning
techniques is fundamentally tied to its programming, though its
behaviors were not strictly ``written'' by its human programmers, but
are instead shaped by their selection of training situations and
responses; and the choice to use these techniques in the first place
comes with a certain risk, that a system
might learn the ``wrong'' ethical behaviors, for example). Such
behavior is impossible for anything but a sentient
AI, and focus on it distracts from the real issues
at hand.} Lin himself has been successful in urging a dialogue
within various autonomous vehicle research groups about the ethics of
their products, as well as getting media recognition of this push for
self-conscious development \cite{timeEthics}. However, many articles
elide fundamental distinctions in the situations being discussed,
distinctions which should be foremost in our minds when we consider
the stakes of developing autonomous machines.

A focus on machines behaving ethically is at risk of making errors
similar to the conflation of machine and human vision, knowing, or 
understanding. Ethical behavior is not necessarily strictly defined by
any one ethical code:  ethics and ``ethics'' are not identical. When we
speak of robotic cars ``making decisions'' of what to do in a crisis
situation, we implicitly accept the idea that such decisions are
really being made by the program. Discussing whether or not those
behaviors are ethical risks suggesting that our robots have all the
capabilities---of sensing, knowing, and processing---necessary to
carry out, \emph{in toto}, what we consider to be ethics \emph{qua} itself. At this
stage in AI development, however, what we are truly discussing is not
the potential ethical robot cars but an \emph{ethical calculus} for
autonomous vehicles. Such a calculus would be a quantification of
ethics according to some particular formalism, so as to allow a
computer program to select a course of action based on a particular
situation. In a very real sense, decisions are not
``being made'' by amoral vehicles.\footnote{Not every device behavior
can be predicted, and it would be foolish to place full responsibility
on the programmers: there is real autonomy in devices, in that they
may do things we do not want. But though all devices have bugs and
will be unpredictable in certain circumstances, the first place to
look for ethics, for an implicit or explicit ethical calculus, is the
human beings 
that do the design.} They are being made by software engineers,
self-consciously or not, even if no
explicit calculus is used and the developers' ethics are only
implicitly present as a consequence of coding decisions. Ultimately,
if we want to care about how ethically systems
operate, we must look at how they are programmed, and what goals that
programming is intended to serve. Only with this focus can we agitate
against systems that are fully black-boxed---closed-source, protected
intellectual property potentially defended by anti-circumvention
laws---and enact an ethics impervious to scrutiny.\footnote{Digital
  rights management and anti-circumvention laws are abused routinely
  to lock out consumers and even muzzle security researchers. For a
  recent example, see \cite{higgins}.
}



%% 2.3) what standard of safety? 
%% --better than an average person? or better than the best people? (1 p)
%% --human problems with using projected statistics to define policy:
%% does it make people feel better to know a system is statistically
%% safer when they are uncomfortable with it? see cars/planes for example
%% (3 p)
%% -->perspective of caution? or  ``losing lives every day'' that
%% could be saved?
%% ----how much risk is legitimate?? (2 p)
%% --airlines as a place where we see the different dimensions of
%% statistical safety vs. passenger's feelings of safety (3 p)
%% ---drawing from PARC/CAST documents
\section{Safety and Statistical Risk}

%%ALSO: consider the risk comments in Cox \& Murray, Mar2 notes;
%%trying to decide how many 9s to have; and what counts as evidence?
%%is it just statistics, testing of components? or can we judge based
%%on designs??

%%TODO (DONE) ``competent'' below, TL writes ``?''
%%TODO (DONE): here is the point of ``personal exceptionalism'' I think; fits
%%with this feeling of competence
%%Personal exceptionalism: ``we think we're good drivers''

The primary driving force in the current self-driving car
narrative is safety; specifically, the poor safety record of human
drivers and the potential for machines to do much better, free from
human frailties of distraction and fatigue. How bad are human
drivers, really? Conventional stories of human drivers paint us as
plainly terrible, prone to road rage and drowsiness, and generally tending
to do anything except pay attention to the road.\footnote{Numerous popular
  articles take this position, some based more in fact, and others
  more in rhetoric. But human fallibility is often mistaken for humans
  being useless in 
  all situations, a misconception we will return to in chapter
  \ref{chap:3}.} However, the statistics tell
a slightly different story. Official statistics show that the number of
motor vehicle accidents remained in the range of 10--11 million
per year for most of the late 2000s. The death rate has
decreased overall in this time, settling somewhere below 1.5 deaths
per 100 million vehicle miles, presumably due to a combination of
better safety features (especially since 1990) and other
factors \cite{censusDeaths}. The total number of vehicle-related deaths is much
lower, at a still significant 35,000 deaths per year. But this alone
does not tell the story. At around 1.5 deaths per 100 million vehicle
miles, or about 1 death per 67 million miles, humans seem relatively
competent in a statistical sense---which, from the point of view of
the individual driver, stands to aid our personal exceptionalism,
thinking we are better drivers than we are. The average American, who might
drive 1 million miles\footnote{This is a conservative estimate,
  extrapolating from an average yearly mileage of between 13,000 and
  15,000 miles \cite{fhwa}.} is unlikely to be
involved in a fatal crash in his or her lifetime. Looking at non-fatal
accidents as well, humans get involved in about one accident per
286,000 vehicle miles, still large on the scale of a driver's overall
life experience. Part of why the autonomous vehicle problem is
such a difficult one is that individual human beings can drive a long
time without having an accident.

Theoretically, computers can do better. But especially careful human
drivers can also clearly beat the human average. How safe do
autonomous vehicles need to be in order to be allowed on our roads?
Safer than the average human? Or safer than the very best
drivers?\footnote{Chris Gerdes has admitted, as recently as September
  2014, that the deep, intuitive experience of race-car drivers to
  handle emergency situations is something they are still working to
  match \cite{8truthsandmyths}.}
Such questions have real impact when it comes to how devices are
designed and when they become commercially viable---and are at least
partly ethical questions. The autonomous
vehicle enterprise seems to call for using such projected statistics
to define policy. One narrative is that undue caution in the
rollout of autonomous vehicles will directly ``cost'' lives, since
people are killed by human drivers every day \cite{driverlessfuture}. However,
people are also accustomed to the current automobile death rate, and
any autonomous vehicle crashes are likely to attract deep
scrutiny\footnote{As Jim Womack pointed out to me, there is no good side
to change as a regulator. Regulators are not congratulated when things
go right, only criticized when things go wrong. So some measure of
tentativeness is almost certainly justified to the regulatory mind.
Discussion with the author, December 3, 2014.} as
to whether a human could have prevented the accident. If,
instead, human oversight to otherwise fully automated systems were
required, the additional risks of supervision would also need to be
accounted for\footnote{This is dealt with more thoroughly in chapter
\ref{chap:3}.}---including the potential for risk homeostasis
\cite{Wilde}, the tendency to
behave less cautiously in situations that appear to be more safe.

These are questions of policy, but also questions of human acceptance.
This ideology posits lowering accident rates above all else, leaving
no space for human squeamishness about technology and responsibility. 
The statistical argument suggests that the death rate is all that
matters, but humans are notoriously bad at understanding and
responding to statistics.\footnote{And any possible framework for
  quantitatively measuring and regulating improvement (e.g. number of deaths,
  monetary cost, etc.) hides all manner of assumptions and comes with
  potentially unforeseen consequences: the best decisions by some
  metrics will be non-optimal in others.}

Just how safe these vehicles are expected to be has become a point of
public contention. A 2015 whitepaper by Sivak and Schoettle of the
University of Michigan Transportation Research Institute attempts to
make the case that ``it is not a foregone conclusion that a
self-driving vehicle would ever perform more safely than an
experienced, middle-aged driver,'' due primarily to issues of sensing and
predictive knowledge \cite[p. 7]{SivakSchoettle}.\footnote{This is worth
recognizing, even if it is also trivially true that nothing about AI
development is \emph{a priori} a foregone conclusion. Even if
``computational speed, constant vigilance, and lack of 
distractibility'' are not alone sufficient to beat out all human
drivers \cite[p. 4]{SivakSchoettle}, I expect AI techniques will
likely approach human 
abilities to use predictive 
knowledge, given sufficient development time. Still, Sivak and
Schoettle provide a valid note of caution here.} Most firmly, they
attempt to impress that no conceivable implementation of self-driving
vehicles will have zero fatalities. One popular response to this type of
argument is the following: 
\begin{quote}Of course, the researchers are trying to correct what they regard as
excessive technological optimism. Still, is it entirely fair of them
to compare robocars only to the best drivers? Most accidents are
caused by the worst ones, and it is beginning to become clear that
those are the people that a robot could outperform with one clanky
arm tied behind its back \cite{rossSafety}.\end{quote} 
Others argue that self-driving vehicles should be considered not ``as
bad as a middling 
driver,'' but ``as \emph{good} as one,'' and place great faith in the
``pinnacle of human mastery of software'' to outperform human drivers \cite{templetonB}.
But these perspectives are far too simplistic, and miss the point in a
significant way. The question of whether an
automated vehicle's fatality rate exceeds some people's, matches the
safest drivers, or bests all human drivers stands to determine which
vehicles are legal \cite[p. 6]{SivakSchoettle}. While projecting vehicle
risk functions is tricky enough when comparing against a known human
standard, it becomes even more difficult when one considers that the
``conventional vehicle'' risk function is itself going to change with
future safety technologies, including automated driver assistance
features. The fully self-driving vehicle must be compared to a moving
target. Meanwhile, academic researchers are often unable to gain access to
these systems to test them against exploits
\cite{madrigalHack}.\footnote{The
implication here is quite disturbing: car companies are so far able to
set the agenda for their own testing (and Google is apparently
lobbying to count simulated virtual tests as road-test miles
\cite{harrisVirtual}),
and third-party oversight is not yet possible.}

%(Phillip E. Ross, ``Has Robocar Safety Been Hyped?'' 20 Jan 2015: http://spectrum.ieee.org/cars-that-think/transportation/safety/has-robocar-safety-been-hyped)

In both cases, measurements and predictions of reliability and risk
are key to the development of autonomous vehicles. The aviation and
defense spaces are in some ways ahead of commercial ground vehicle
research in terms of engineering automated systems, in part because
they have already been forced to confront these issues in earnest:
airplane autopilots have a more developed history than automotive
autodriving, aided by the lower complexity of the air environment,
with simple mechanical autopilots available as early as
1912 \cite[p. 16]{NRCAutonomy}. The \emph{Autonomy Research for
  Civil Aviation} report by the National Research Council pays
significant attention to the fact that ``the lack of generally
accepted design, implementation, and test practices for
adaptive/nondeterministic systems will impede the deployment of some
advanced IA [increasingly autonomous] vehicles and systems'' and that
``existing V\&V [verification and
  validation] approaches and methods are insufficient for advanced IA
systems'' for many of the same reasons \cite[p. 2]{NRCAutonomy}.
Among the high-priority research projects they identify as
most pressing and most difficult, they include both the development of
methodologies to ``characterize and bound the behavior of
adaptive/nondeterministic systems'' and the creation of standards for
the ``verification, validation, and certification of IA
systems'' \cite[p. 4]{NRCAutonomy}. As in the automotive space, the
core reason for increasing autonomy is to increase reliability, but
being assured of this reliability is difficult. As the report notes,
however, software creators ``are generally expected to prove that the
software can deliver the intended capabilities at specified
performance levels and in the relevant environment,'' which involves
extensive examination of the code, and testing every logic path,
according to FAA guidelines \cite[p. 39--40]{NRCAutonomy}. But
these approaches are not scaling to more complex systems, and new
validation approaches are required to account for the impacts of human
operators or supervisors on system behavior \cite[p.
  40]{NRCAutonomy}. While
the NRC panel is concerned about validation of existing aeronautical
control systems, automated vehicles may contain 10 times as much code
as modern, highly-computerized aircraft (100 million lines versus 5 to
10 million for the F-35 Joint Strike Fighter)
\cite{reutersF35}.
The NRC panel is not alone in recognizing these 
issues: the Defense Science Board in their task force report \emph{The
  Role of Autonomy in DoD Systems} attempts to address the importance
of the larger environment in which automated systems operate, and
within which they can produce ``unintended operational
consequences'' \cite[p. 2]{DSB}. They warn of \emph{brittle}
platforms, and emphasize the importance of developing ways to predict
and understand the resilience of systems \cite[p. 7, 11]{DSB}. As cars
become more automated, these difficult-to-predict risks will increase
in importance.

%%TODO (DONE): Amount of code required comparatively with cars and planes, guy says
%%fully automated is possible but for ethical considerations:
%%http://mobile.reuters.com/article/idUSKBN0MN1E820150327?irpc=932
%%And mention further problems with nondeterministic systems!

%Include NITRD report, if available? http://scenic.princeton.edu/NITRD-Workshop/index.html 

However, when it comes to public adoption of automated cars,
risk assessment is only part of the story. Does it make us feel better---more
comfortable, more likely to get into autonomous taxis and spend our
money on autonomous cars---just to know that they are statistically safer
than the average driver? Airplanes are one type of vehicle for which
these different dimensions of 
safety---statistical safety compared with perceptions or feelings of
safety---have already become visible. Despite the comparative
statistical safety of flying,\footnote{This information is relatively
  available and publicly known \cite{airlinereporter}, though that does not
  necessarily change people's minds about travel.}
people tend to be more afraid of getting onto an airliner than getting into their
cars. While this may have to do with a number of factors,
including that aircraft do not remain on the ground during operation,
it also represents a situation in which passengers give over their
agency to pilots performing a job they do not understand and could not
take over in an emergency.

%SIMILAR ISSUES WITH TRAINS?\cite{??}

The CAST working group has had to account for these issues in their
recommendations about airplane safety since 1998. Part of their charter---to
reduce the accident rate by 80 percent \cite[p. 28]{PARCCAST}---is
motivated not so much by the rate itself but predictions about its
most visible effects:  if air traffic were to continue to increase
with industry predictions but to maintain the 1997 rate of 1.5 major
crashes per 1 million departures, we would see one fatal crash per
week by 2005, and one per day by 2025.\footnote{See
  \cite{predictmorecrashes},
  for example. This point was made to me by David Mindell, discussion
  with the author, November 12, 2014.}
This frequency of accidents was judged to be intolerable to
airline customers, even though the overall accident rate would be no
larger than its ``very safe'' \cite[p. 129]{PARCCAST} starting
value.\footnote{By comparison, the CAST website describes: ``The
  enhancements made over the past decade have made history in
  commercial aviation - by making it the safest it has ever been.
  Today, fatal accidents are reduced to only one in 22.8 million
  flights, an absolutely remarkable achievement'' \cite{cast-safety.org}.} Though
aircraft would be as safe as they 
had ever been, people, on seeing the sheer number and scale of
accidents, would be unlikely to choose to fly. While the issues facing
aircraft and trust are not precisely the same as those for cars, this
example shows that actual safety rates are only
one component of interest when considering how consumers react to
modes of transportation. Perceived safety---due to accident scale,
publicity, trust, or other factors---may be very different. But
while research into how to get human beings to trust robotic drivers
is being done \cite{rossTrust}, the voices pushing for autonomous cars sooner
rather than later would suggest that the statistics are all that
matters (and indeed, would tend to fabricate those statistics out of
mere predictions). 


\section{Planning, Policy, Cities}

Besides being presented as steps toward road safety, driverless cars
also become part of the rhetoric of personal mobility, ``public''
transportation, and urban planning. Potential or imagined
benefits of these vehicles include empowerment for the blind or
elderly (as discussed in chapter \ref{chap:1}), reduction in
traffic congestion, elimination of parking problems, and
persistent access to a dense network of hireable point-to-point
transport vehicles (essentially, driverless taxis). There is
  also an environmental case being made for these vehicles, as they
  will drive predictably and can therefore be tuned to be highly
  fuel-efficient. The social, environmental, and urban planning
implications are at least worth considering in passing before moving
on to consider alternate models for how these technologies could be
developed.\footnote{Critique of automated vehicle systems on these
  grounds is not new, and Marcia D. Lowe of the Worldwatch
    Institute wrote in 1993, in response to
the US push for automated highways, that ``Even more astonishing [than
  the level of spending] is
the total lack of organized opposition to the idea, despite evidence
that smart cars and highways may well exacerbate the very problems
they are supposed to solve'' \cite{novakNationalAutomated}.}

In my discussions with a fellow researcher who has been studying the
development of self-driving vehicles, it became clear that the
developers see themselves as people trying to change
society.\footnote{This self-image is not limited to those working on
  mobility systems. John Naughton writes---and my contact with
  programmers would lead me to agree---that many ``are fired with a zealous
  conviction that they are doing great stuff for the
  world'' \cite{naughtonTech}.}
They
tend to buy into utopian visions of big cities freed of cars, with no
parked cars on the streets and only driverless cabs. This desire to
change the way cities operate is admirable: our cities have a
difficult hundred-year history of changing to cope with the
automobile, and many of these changes have not been for the better
when examined through a broad lens. The prevalence of automobiles has
literally re-shaped city centers \cite[p. 7]{burdenCities}, widened
streets \cite[p. 10]{burdenCities} \cite[p. 65]{fernandezBoulder}, and generally made many cities less safe for
pedestrians and bicyclists. Many of the streets in the most
pedestrian-friendly zones of older cities would no longer be legal to
build under 
current traffic codes, and the default lane width for arteries has increased
from 10 to 12 feet over the past fifty years, with disastrous results \cite{citylab}. In some of the
United States' most pedestrian-unfriendly cities, major intersections
in the downtown area are more than nine 12-foot-wide lanes across. These changes in 
regulations and practices have contributed to more sterile urban
centers with reduced potential for vibrant civic life \cite[p. 60]{miaraBoston}. Urban planners have
in recent years begun to take the automotive threat seriously, pushing
for reduced lane-width requirements and designing urban areas that
limit vehicle traffic on purpose \cite[p. 67]{fernandezBoulder}. If self-driving
vehicles have a high probability of effecting a positive change on the
cityscape, as some proponents claim, they could be an important
addition to these developments.

Congestion and other environmental factors, such as air pollution, are
other social reasons used to support automated vehicle development.
With carefully controlled, automated driving, the fuel consumption of each
individual car could be reduced, just as so-called ``hypermilers''
today use altered---and very conservative---driving techniques in
order to increase their fuel efficiency \cite{orourkeHypermile}. As discussed in
chapter \ref{chap:1}, researchers envision using V2V communications to
share road information between vehicles, reduce gaps between cars, and
thereby increase throughput and efficiency of the road system. But
these proposals involve numerous potential pitfalls. While coordinated
fleets of vehicles could increase the throughput of roads by 2 to 20
times \cite[p. 229]{adamsBrewerRoadways},\footnote{Adams and Brewer
  cite numbers of 10 solo drivers per square foot of right-of-way per
  hour for conventional highways, and between 100 and 200 for their
  modified approaches that use normal subcompact vehicles and special
  bubble-like micro-cars, respectively.} significantly greater increases would be necessary to match the
throughputs of buses or subway trains, which can be 100 times
greater \cite[p. 222]{loweCars}.\footnote{Specifically, Lowe describes that
  per-hour capacity for 
  a rail line can be as high as 70,000 people, compared with 30,000 for buses
and just 8,000 for private cars.} Much depends on actual usage.\footnote{Subject to
homeostasis of inconvenience and the Jevons paradox, as mentioned in
chapter \ref{chap:1}.} If a factor of 10
increase in highway throughput were possible, but caused a factor of
10 increase in overall miles driven, congestion would not be
ameliorated, and significantly more environmental impact might result. Any
large-scale impacts of automated vehicles are dependent on the
architecture of the overall transportation system.\footnote{As Adams
  and Brewer write, critiquing Intelligent Vehicle-Highways Systems as
  an insufficient solution: ``since the halcyon days of its introduction, the
  automobile has produced half the world's carbon dioxide and hollowed
out cities in the United States, and it is now doing the same in
cities all over the developing world as their inhabitants rush to
embrace industrialization and its mistakes'' \cite[p.
  227]{adamsBrewerRoadways}.} And therefore it is
the transportation system, not automated vehicles alone, that must be remodeled.

Fleets of fully automated vehicles could reasonably be expected to increase
efficiency and decrease reluctance to take trips---making them
less inconvenient by allowing the driver to perform other tasks---in
which case vehicle usage would be expected to increase. This result is actually
implicit in the visions of those who would support fleets of
driverless taxis to move people around cities:  these concepts are
explicitly oriented toward providing for needs that are presently
underserved, and so presenting automated vehicles as a solution to
city transit problems implies greater usage of such vehicles to meet
those needs. Just as increases in the use of cars to commute have
decreased the quality of urban spaces, what is to stop cities of
automated vehicles from
being even more unfriendly to
people? This question was in fact asked at the MIT SENSEable Cities
conference in 2014.\footnote{It received little in the way of reply from the
panel. Nhai Cao suggested that cities will not be built around new
vehicle infrastructure, and that his focus is on building new
capacities into vehicles, which avoid the question. Paolo Santi was
unconcerned with vehicle-to-pedestrian sensing and interaction, but
instead with the very limited domain of mathematical optimization,
a viewpoint that seeks to ignore the potential consequences of such
optimization.} Positive changes to city infrastructure will not happen
automatically. Complex
changes to complex systems require significant modeling work,
experiments, or even trial and error to get right. Old models and
habits need to be broken---like the rules of the AASHTO ``Green
Book,'' which despite criticism are often taken almost as natural 
law \cite[p. 181, 183]{swopeTrenton}---and new ones built to replace them. 

However, we should be aware that these utopian visions of driverless
cities, however contingent, 
are not universal. Looking more closely, one informant described to me, the visions
of driverless car developers
often have a strong male bias. They see businessmen
taking driverless cabs to work, and students getting drunk and, unable
to drive themselves home, using automated vehicles to ferry themselves
back to their apartments. These visions may be greatly influenced by
the developers themselves---primarily male engineers with masculinist
preconceptions---either thoughtlessly or as a marketing strategy. But
regardless of the source, they suggest designed solutions may tend to
be predisposed to certain types of uses, and less amenable to others
that fall too far outside this masculine vision. Other uses are still
possible, of course---Uber is already being used by parents to ferry
their kids to school \cite{hoderParents} \cite{shapiroVan}, even
though it was almost certainly not designed 
for that
purpose---but inherent gender bias in design is not a problem that should be ignored
when making large-scale changes to infrastructure, as it can deeply
affect the ways and frequencies with which people choose to use new
technologies. Indeed, a recent study by the University of Michigan
Transportation Research Institute suggests men may be more
amenable to autonomous vehicle technologies, and it is worth
considering the role that gender-biased technological visions may have
in that
effect \cite{miglioreWomen}.
Thinking about the variety in types, uses, and users of road vehicles---from
subcompacts to vans to pickup trucks; from solo commuting to carrying kids to
hauling construction equipment or moving furniture; from the wealthy
to the poor, the urban to the rural---makes clear that cars are
multipurpose vehicles, with culturally specific uses.
Acknowledging this
specificity is vital for developing fair and just technologies,
particularly
considering the potential for automated taxi fleets to 
be used to justify decreased investment in---or even, in some
simulations, wholesale replacement of
\cite{frazzoliSingapore}---public transportation infrastructure. We must
improve, rather than erode, equitable access to transportation.

%%Perhaps add in a quick aside about a map-only interface that
%%requires a destination for operation: this assumes your driving has
%%a clear and defined destination that you know; cruising around may
%%be possible but is not the intended mode of operation

When the
expected user is the commuting, upper-middle-class, working father or
the privileged college boy (and the opposers of the technology are
branded by some, as we saw, as ``soccer moms''), we risk developing
vehicles that preferentially serve certain dominant uses and not
others. When I asked one informant whether the developers have an idea
of what driverless cars would mean for people who have children, he
replied:  ``They don't think about it.'' While certainly not shared by
all developers in this space, it is not a vision we can afford.

\section{Conclusion}

This chapter has investigated six different perspectives on the
social, cultural, and informational contexts of the autonomous
vehicle:  the technologies of vision and of mapping involved
in its operation, the 
data it collects, the statistics with which it is motivated, the
ethics and epistemologies with which it is designed, and the civic
ends to which it might be mobilized. While in chapter \ref{chap:1} I
illuminated the historical background behind the dominant narrative of
the self-driving vehicle, here I sought to illuminate the
consequences of this vision. If we accept this narrative, these design
fictions, for the automotive future, what comes along for the ride? We
have seen that the automated vehicle so envisioned exists at the
nexus of a number of deep, important questions about our relationships
to technology: What level of privacy is appropriate? What counts as machine
knowledge? How do we design machines that can be genuinely ethical,
and do we need to? We have also seen that these vehicles present a
number of thorny problems not often recognized: What level of
representational work is required to make them function? How do we
predict the relative safeties of devices that do not exist? Can we
achieve better urban design simply by adding autonomy to vehicles?
All of these questions are united by being largely unanswerable---and
the problems, unsolvable---by a focus only on the device itself,
narrowly defined. Instead, these are system-level problems, questions
that must be asked of large networks of human and machine actors with
subtle and shifting interrelationships. Some are likely to be issues
for any approach to vehicle automation; however, alternative models of
automation present the possibility to soften some of the more
troubling questions of machine capabilities. Let us turn to a fresh look at
vehicle automation, drawing from sources outside the dominant
narrative, and explore what effect an alternative model has on the
stakes of automotive automation.

