\chapter{Technological Realities}

Technological Realities (27 p)

1) bring in the telerobotics/supervisory control literature
-why supervisory control? and where is it used? (5 p)
-examine the issue of deskilling vs. the ``irony of automation'' well
known in Human Factors, in which increased automation actually
increases human load (2 p)
-but SC allows for particular combinations of human skills and machine
competencies (1 p)
--used all over industry, aviation, undersea (2 p)

2)Other problems:
--Göde Both and reluctance to use Machine Learning (1 p)
---brittle, unpredictable, difficult to ``tune''
---airlines/Mil worried about the same thing: need better VV\&T, models
for testing (3 p)
--Mapping and route issues
---Google car needs detailed 3D maps in order to operate (1 p)
---also rain, snow, weather (1 p)
--Sensors issues and inability to distinguish crushed newspaper from
rock (1 p)
---requires a lot of other information, more than simple physical
volume
---object identification work in machine vision is slowly progressing,
but then we hit the above problems of VV\&T for such models (2 p)

2.5??) Worth bringing in questions of terminology, ``intelligence''
etc. and the necessary caution in ascribing them to tech; we can look
and see that the ``intelligence'' in action here is really very
different than what we usually mean when we use that term
(human-centric)
--so does knowing this reality change what we want to see? (couple of p)

3) situate the moonshot approach (which may actually be easier in some
senses but less socially acceptable) and the mixed approach in a way
that does not support a teleology of autonomous vehicle development
(think Sheridan's graph of telerobotics, not the SAE's 5 stages) (4 p)
--in historical context of previous systems that are wholly or
partially autonomous (4 p)


