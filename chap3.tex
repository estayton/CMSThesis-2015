\chapter{New Epistemologies}

Impacts / New Epistemologies (21 p)

>>>This is where I weave in automotive history 
>>>Return to streetcar and the re-shaping of public space: 

1) process by which new technologies become entwined with new legal
principles; new epistemologies?
-historical example (3 p)
-what is the new epistemology needed to make sense of devices like
this in the everyday (because we haven't seen them outside of extreme
environments, except in very weak forms)
-the street stands to be re-shaped again, subtly, by the capabilities
of our vehicles (1 p)
-this is the place to bring together the AI, Human Factors,
Law/Policy, and STS/Anthro background (specifically on technology \&
people, edges of HCI, Andy Clark on natural-born cyborgs) to paint the
whole picture of the forces in play (2 p, v. philosophical)

2) VERY different potential impacts if we are looking at fully
autonomous vs. partially autonomous 
--in terms of human role and our understanding of our own role
----referring back to HF/SC (3 p)
--in terms of how the road itself may have to change (not going to
spend a lot of time on this, but it is worth at least a couple
paragraphs) (2 p)
--in terms of legal framework
----it is important to note that ``responsibility'' is not binary
(which is made clear in these examples), and torts may be distributed
across human and nonhuman actors (operator, owner, seller,
manufacturer, distributors) but distribution of torts has an ethical
and ideological stance (2 p)
>>>>BUT: problems with the liability framework: expensive for the
litigant to get expert witnesses; harder to determine whose fault it
is and the human tends to get scapegoated (see industrial accidents,
TMI, or oil rig disasters) (3 p)
>>>>>What can we actually ask the human to do and hold them to task
for? (3 p)
---important to note that issues of litigation for full autonomy are
not even part of the debate yet, we just aren't at the stage where it
is worth discussing (2 p)

