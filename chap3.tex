\chapter{Hybrid Controls, Hybrid Possibilities}
\label{chap:4}

Technological Realities (27 p)

%% 4. What is the alternative to the teleological progression implicit
%% in the for- and against- stories?
%% (start with story of planes [the backing off from full autonomy] or Mars rovers or something
%% -HSC
%% -Autonomy research from other areas
%% -envision some alternatives
%% -some stakes (data) are still applicable, but others may be
%% ameliorated

%% if this will still be long enough, consider eliminating automation
%% and AI from this section and moving them back to ch 1. then these
%% alternative narratives are really not all the alts, but they are
%% those alts that are pretty much never referenced . . . 

\subsection{Human Supervisory Control}

—while Hutchins' cockpit
remembers its speeds through a combination of human activity and
physical cognitive aids,\cite{???} the NHTSA's “vehicle” may sometimes “not
perform a control function.”\cite{???}

ref back to NHTSA/SAE; starting from P. 9 of the DM paper: The documentation of the Google car's Nevada driving test
exposes fractures in the traditional perspectives on vehicle
automation.\cite{???} 

As a way to INTRODUCE HSC, actually

NRC report on aviation (see e.g. 14-15)

\subsection{Lessons from Autonomy Research}

Apollo/Mars rover

Underwater exploration

Aircraft (see Digital Apollo ch 2)

\subsection{Whither Alternate Narratives?}

note especially the contradiction within X's characterization of human
drivers and their future: ``humans are bad drivers and we should all
have our licenses taken away'' compared with ``car nuts should welcome
this because it will free the roads up for them'' which doesn't
actually make sense!
--but there's perhaps a middle ground here in the hybrid narrative

-question of whether people will be allowed to drive
--self-contradictory comments: Daniela Rus saying
-that people will still be able to drive normally
-these questions are deeply embedded in the idea of progress: does
-greater safety imply taking licenses away? a question that really
-only makes sense with a particular vision of autonomy (perhaps a
-better question is what sort of licensing/training is necessary)


1) bring in the telerobotics/supervisory control literature
-why supervisory control? and where is it used? (5 p)
-examine the issue of deskilling vs. the ``irony of automation'' well
known in Human Factors, in which increased automation actually
increases human load (2 p)
-but SC allows for particular combinations of human skills and machine
competencies (1 p)
--used all over industry, aviation, undersea (2 p)


3) situate the moonshot approach (which may actually be easier in some
senses but less socially acceptable) and the mixed approach in a way
that does not support a teleology of autonomous vehicle development
(think Sheridan's graph of telerobotics, not the SAE's 5 stages) (4 p)
--in historical context of previous systems that are wholly or
partially autonomous (4 p)





