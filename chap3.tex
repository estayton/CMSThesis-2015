\chapter{Hybrid Controls, Hybrid Possibilities}
\label{chap:3}

%% 4. What is the alternative to the teleological progression implicit
%% in the for- and against- stories?
%% (start with story of planes [the backing off from full autonomy] or Mars rovers or something
%% -HSC
%% -Autonomy research from other areas
%% -envision some alternatives
%% -some stakes (data) are still applicable, but others may be
%% ameliorated

%%EDITS
%%SUCHMAN: plans and situated actions

%%93: show your hand as to why this was downplayed
%% is it ever stated why? Stated in USC, but may not be in that
%% chapter!!
%% WHY? because it doesn't sound as cool; marketing; 
%% and what about business investment
%--drum up excitement/investment and walk it back 
%--connects w/ engin companies and marginalization of HF
%--interface first approach of Google, w/ similicity, presupposes a
%certain kind of operation

%%TODO (DONE): p 110: high performance racing pushing things? put in a little bit
%%about F1 etc in the conclusion


Most automated vehicle narratives---and all those we have investigated
so far---rest on two primary, interlocking assumptions. First, the nature of the
ideal human-machine interaction for vehicle control is assumed to be
known. Second, an inevitable progression toward not only greater
autonomy but complete autonomy is assumed as the
starting point of these arguments. Rarely if ever does the question of
how much autonomy or supervision to provide to the automated system
enter the discussion as an engineering parameter over which designers
have control, and which should be responsive to larger goals. Instead,
advanced driver-assistance technologies and fully-self-driving
operation with no need for human supervision are recognized as
technologically connected, but perceived as fundamentally dichotomous
approaches. Self-driving operation is considered---often via a
speculative or ideological basis---either good or not-good compared to
current vehicles with driver-assistance technologies; it is not generally
perceived as the upper bound on a spectrum of automation
approaches.\footnote{And as we will see in this chapter, even
  so-called ``full'' autonomy would not, in practice, be so simply
  disconnected from human oversight.} And through all this,
fully-automated operation is assumed to be the ultimate end goal, in a
realm of complete technological possibility: the ideal human-machine
interaction is, in a sense, no human-machine interaction. As I have
outlined, this perspective comes from particular roots (automation
history, artificial intelligence, science fiction), but is not the
only way to envision automated systems. Let us look at
other examples of automated technologies (in space, underwater, and in
the air) to see how previous sets of designers and users have
negotiated compromises of automation, autonomy, and human-control.

%%CUT Given these underlying
%%similarities in terms of human-machine interaction,


%% As a way to INTRODUCE HSC, actually

%% 1) bring in the telerobotics/supervisory control literature
%% -why supervisory control? and where is it used? (5 p)
%% -examine the issue of deskilling vs. the ``irony of automation'' well
%% known in Human Factors, in which increased automation actually
%% increases human load (2 p)
%% -but SC allows for particular combinations of human skills and machine
%% competencies (1 p)
%% --used all over industry, aviation, undersea (2 p)


\section{Human Supervisory Control}

No current ``self-driving'' vehicles designed to operate on public
roadways (as opposed to in controlled conditions) operate in a fully
autonomous mode. The cars operate---and legally, can only
operate---in
a supervised mode, wherein a human driver is responsible for
overseeing the automated systems. Even if the vehicle is capable of
performing a maneuver ``on its own,'' that operation is monitored, at
least intermittently, by a person who can theoretically correct errors
made by the automation.\footnote{Of course there are serious
  limits to this capability, as HSC research shows, and as just discussed
  in the context of airline cockpit automation.} This point is made not
only by numerous articles describing vehicle operations in general terms,
but by Google's
own job postings, searching for ``vehicle safety specialists'' to join
the self-driving car team in Mountain View. The ideal operator will
develop ``a unique set of operational skills'' using the vehicles, and
operate ``comfortably in a fast-paced environment, sometimes managing
up to four communication channels simultaneously via various high- and
low-tech mediums'' \cite{googleJobPosting}. This person's primary
duties include filing daily reports and monitoring operations of the
software with ``constant focus.''\footnote{While this constant focus
  is likely to be solely an artifact of the car as a device still in
  testing phases, the development of a ``unique set of operational
  skills'' is just the sort of learning one might expect to perform in
order to operate a car with a novel human-machine interface. While it
may be Google's intention that only test drivers need develop this
kind of expertise, requiring it of prospective users is not
necessarily a bad idea. While there is a reluctance to require drivers
to acquire new skills, people are already licensed before they are
allowed to drive, and the terms of that licensing may well need to
change with new technologies and new relationships to automotive
technology.} This role is clearly not one for a passive participant,
an observer who sits in the seat, lets the software run (and perhaps
self-diagnose failures), and only presses an abort button in an
emergency. The test driver is instead an active participant in the
complex process of 
vehicle operation. Further details of these drives, however,
are difficult to come by. Test drivers are, expectedly, required to
``keep all project details confidential'' \cite{googleJobPosting}.

%%Use the Hireart job posting; looking for people who can monitor
%%comms and vehicle ops; a complex task; but the actual operations
%%shrouded in secrecy

However, in May of 2012,
Google actually tested one of these vehicles in Nevada, on public
roads, in the only government test yet conducted in the United States.
The documentation of this test---which occurred with engineers Chris
Urmson (the project lead) and Anthony Levandowski in the front
seats---exposes fractures in the traditional perspectives on vehicle
automation \cite{harrisNevada}. Even though the structure of the
checklist only breaks down operation into ``Autonomous,'' ``Driver
Assist,'' and ``Driver Only'' modes, it shows that the human driver
was required to assist, or to take control, at multiple points during
the test drive. The test records a mix of autonomous and
driver-assisted operation when the vehicle faced road construction,
switching into manual mode and requiring human assistance to continue.
These hand-overs were not limited to construction, however: 
``Wojcik [the examiner] also recorded that the car needed driver
assistance with some turns, although she did not note the
circumstances'' \cite{harrisNevada}. This should not be taken
simplistically, as evidence that the system is not sufficiently
advanced. Instead, it is evidence that real operations are more
nuanced than narratives of them tend to allow for, involving mixes of
attention and control that change over time and varying road situations.

This operational mode, then, is more properly a human supervisory mode
than an autonomous one. The study of HSC is by no means new, but
perhaps because it does not interact with human fears of obsolescense,
loss of agency, or robot apocalypse the way AI does---it doesn't stand
to damage our egos in the same way, in part because it just
sounds rather staid and boring---it has not been as commonly
recognized or discussed in popular narratives. And yet, supervisory
control is implied any time an article mentions a human driver or
co-driver monitoring a system, or taking control at a critical moment.
However, these moments are generally implied weaknesses to the device,
as the self-driving vehicle narrative is organized around the ultimate
goal of fully autonomous robot cars. Supervisory control, however,
admits different goals and possibilities.

The core text of the supervisory control literature is Thomas
Sheridan's \emph{Telerobotics, Automation, and Human Supervisory
  Control} from 1992. Though work on such systems had been occurring
at the MIT Man-Machine Systems Laboratory---among other places---for
the previous 30 years, this book represents the first time many of the
concepts that were ``maturing'' over those years of dissertations were
brought together into one source \cite[p. xix]{sheridan}. Despite its
age, the book remains a great introduction to the field\footnote{And in fact, Sheridan is almost prescient in his
  identification of the vehicle automation technologies (Advanced
  Vehicle Control Systems, or AVCS) that would
  make it to market: all of the ones he lists we see today, and the
  last (like automatic lane keeping) are really only just now becoming
  available. Many of these technologies were in development at the
  time, as Sheridan's book came out during the middle of the Eureka
  PROMETHEUS project.} as well as a
much-needed counterpoint to other tendencies in the field of
automation. Human supervisory control history in fact
shares touch-points with the history of automation in its popular
form, but comes more out of theories of management and human factors
engineering. Frederick Winslow Taylor is identified as a key player in
the history, less for his ``dehumanizing'' approach to the worker than
as
his intent to generate ``a new interest in the sensorimotor aspects of
human performance''---in other words, the way that human capabilities
interact with the tools they use to accomplish the tasks they are
set \cite[p. 7]{sheridan}. Later human factors or \emph{ergonomics}
work continued to probe such questions, though many supervisory
control systems had already entered relatively common use, including
aircraft autopilots, automatic elevators, and even, perhaps arguably,
washers and dryers \cite[p. 8]{sheridan}. Though control researchers
interested in the behaviors of humans in interactions with highly
automated systems were already pursuing similar work, supervisory
control, according to Sheridan, truly came into its own as part of
research on the teleoperation\footnote{\emph{Teleoperation} means the
  extension of an operator's sensing and control capacity to a remote
  location, via an artificial assemblage \cite[p. 4]{sheridan}.} of vehicles under time delay,
specifically on the moon. The time delay enforced a fundamental
constraint on direct operation, as the results of any action require
three seconds to be reported back to Earth, and therefore made
apparent the great benefit of having the remotely-operated system
include its own internal control loop to allow it to perform simple
delegated tasks \cite[p. 9]{sheridan}.

As Sheridan describes, the concept of
supervisory control comes from
the idea of human supervision within management structures:  in an idealized
case, a human
supervisor instructs her subordinates, who carry out tasks, summarize
results, and report them back to the supervisor who decides what
further actions are required; the supervisor may
exercise various amounts of monitoring or direct control over the
actions of her subordinates, based on their skill and her trust in
their abilities. Replacing the human subordinates with computerized
components completes the basic analogy to supervisory control as it is
used here. Sheridan's definition of supervisory control in its
strictest sense is that
\begin{quote}one or more human operators are intermittently
  programming and continually receiving information from a computer
  that itself closes an autonomous control loop through artificial
  effectors and sensors to the controlled process or task
  environment \cite[p. 1]{sheridan}.\end{quote}
The less-strict definition loosens the requirement that the device
close a control loop of its own, simply requiring it to interconnect
``through artificial effectors and sensors to the controlled process
or task environment'':  only in the strict case can the computer
operate without the human as an autonomous system ``for some
variables at least some of the time'' \cite[p. 1]{sheridan}. This
emphasis on partial, gradiated control\footnote{The computer system
  may function primarily on the ``efferent or motor side'' to actually
implement directives from the supervisor, subject to its own
sensors \cite[p. 3]{sheridan}. Or it may act principally ``on the
display side,'' processing incoming sensory information into a form
digestible for the supervisor, or, as is usual, it may do some
both \cite[p. 3]{sheridan}. As Sheridan notes, the computer acts
independently ``at least for short periods of time,'' and the human
may assume direct control of the entire system or certain variables
within the system at various points \cite[p. 3]{sheridan}.} is
emblematic of the entire HSC 
project, and represents its fundamental ideological difference from
the AI-focused perspective on automated systems.\footnote{Why is this
  AI-focused view so prominent? I would argue it is because it
  captures the public imagination, via science-fiction, and is
  rhetorically more impressive. As a way to market and drum-up
  excitement and investment, autonomy simply sounds more interesting
  than ``supervision,'' while companies can walk-back their actual
  products toward more comfortable, HSC applications. Additionally, I
  would speculate that the relatively low, marginal position of human
  factors in engineering culture is related.} This is neither a
weakness nor an unwillingness to be sufficiently bold, but a
well-considered engineering strategy. Sheridan identifes seven
motivations to develop supervisory control, of which
six\footnote{Number 5 is not relevant as stated, but is very relevant
  in principle to a distracted driver.} are
eminently relevant to self-driving cars, and so I will include those
here in their entirety:
\begin{quote} (1) to achieve the accuracy and reliability of the
  machine without sacrificing the cognitive capability and
  adaptability of the human,
(2) to make control faster and unconstrained by the limited pace of
  the continuous human sensorimotor capability,
(3) to make control easier by letting the operator give instructions
  in terms of objects to be moved and goals to be met, rather than
  instructions to be used and control signals to be sent,
(4) to eliminate the demand for continuous human attention and reduce
  the operator's workload,
(5) to make control possible even where there are time delays in
  communication between human and teleoperator,
(6) to provide a ``fail-soft'' capability when failure in the
  operator's direct control would prove catastrophic \cite[p. 12]{sheridan}.
\end{quote}

Though supervisory control, as well as human factors engineering by proxy, is
interested in mathematically modeling the human operator in her
engagement with the control system---itself a fraught project in
several ways\footnote{People rarely behave as ideal mathematical
  functions. The black-boxing of the operator into a stimulus-response
system is something HF and HSC continue to struggle with, as it is a
tricky problem to get right. Human operators respond differently under
laboratory test conditions than they do under the stress of actual
operations, a fact that troubled X-15 designers attempting to tune
their fly-by-wire controls \cite[p. 54]{DM}. Though
improvements have been made in this area, and modeling operator
responses is a topic of great interest especially in aeronautical and
astronautical design, human operators continue to be
unpredictable.}---the specifics of such modeling are not necessary to
understand the supervisory control concept or the doors it opens in
understanding the operations of ``self-driving'' vehicles. To provide
a specific example of supervisory control, consider the following
situation: a highly-automated vehicle is set up to operate in a
supervised manner. The vehicle is capable of navigating traffic on its
own, but includes a map interface, a digital display, a shifter that includes an
autonomous mode ``gear'' selection, a standard set of wheel and
pedals, a turn signal control,
and a cruise-control-like control stalk. The user may set a
destination on the map interface and engage the automation from a stop, or engage
the automation while the vehicle is in motion. The user is expected to
be available to assist the vehicle with maneuvers, and oversee its
behavior:  she may take the wheel at any time to direct the vehicle,
or use the pedals to force its speed to alter; she may request lane
changes using the lane signal stalk; she may use the cruise-control
stalk to subtly alter the vehicle's speed to suit traffic and
conditions. At any time when the automation is engaged, she may alter
the destination on the map. When the automation is engaged, the
vehicle will warn if it is encountering a situation it cannot handle,
and will revert to a minimal-risk condition if the operator does not
intervene (e.g. pull to the shoulder and slowly come to a stop).
Whether or not the automation is engaged,
certain ADAS or AVCS are operating in closed-loop mode, including
pedestrian collision detection. And vehicle data including detected
objects and planned paths through the environment are always being
presented on the digital display to assist the user in evaluating the
environment and determining vehicle intent.

This hypothetical vehicle functions in a clearly supervisory mode,
since high-level commands can be provided by the human operator, to be
carried out by the automation in accordance with its sensors.
Information from the environment is processed by the vehicle and
returned to the user via the display, providing the user with more
cues as to the environment and indications as to the status of the
automated system. Such a vehicle bears little resemblance to the
self-driving vehicle envisioned by Google, but looks quite like a
current vehicle might after a decade of evolutionary development of
driver assistance systems. Furthermore, it manages to address all of
Sheridan's supervisory control motivations. It combines the constant
vigilance of automated sensing with the judgment and perception of the
human operator (1 and 2); it allows the operator to provide high level
controls in autonomous mode, instructing the vehicle which lane to be
in or which turns to make without having to maneuver manually (3); it
allows the operator to pay intermittent attention to the vehicle by
including fail-safe modes and the ability to handle most driving
situations (4, 5); in the case of a time delay in response (e.g. due
to drowsiness) or an
operator's failure to control, the vehicle will attempt to behave
safely (5 and 6). While actually designing and engineering such a
system is by no means as simple as sketching it briefly as I have, it
should be clear that a vehicle of this description is not only
potentially very useful, but is at least as plausible as a fully
autonomous robotic car. It presents unique problems of human
interaction and attention which cannot go unremarked upon, but it also
presents unique opportunities for blended 
capabilities that may not only compensate for deficiencies in computer
vision, mapping, or automated sensing, but may do much to address
human discomfort with automation systems and concern with being
outside of the control loop of their automated vehicle.


Issues with attention have always been implicit within the discussion,
since inattention is precisely the human quality that makes
self-driving vehicles most necessary, at least rhetorically. %% But human
%% supervisory control, in its practical applications, has much to teach
%% us about the limits of human attention, and indeed such human factors
%% research is the source of the NHTSA's claim that users of
%% higher-levels of automation will not, and will not be expected to,
%% monitor its operation. An often-used justification
%% for entirely eliminating the human from the vehicle control loop is
%% related to this tendency not to monitor the automation system:
Experience shows that human operators who become reliant on the
automation to perform a task are ill-equipped to take that task back
in a crisis situation, and task hand-over is likely to be
catastrophic \cite{AF447} \cite{Chowpaper}. %% A transportation systems
%% researcher I spoke to 
%% even cited this as the key reason he felt that highway-only
%% self-driving systems would not work, and that companies would have to
%% design for full automation everywhere.
\footnote{Some researchers,
  however, do suggest the opposite: Larsson suggests a less perfect
  system might be better 
  by some metrics, as experience with it is empirically related to
  greater awareness of its weaknesses, and an expectation of having to
take control intermittently \cite{larsson}. Notably, this paper and
others like it are representative of human factors or ergonomics research.}

%% The AF447 crash brings a number of human supervisory control questions to
%% the fore. What is the proper role of pilots? And how quickly can they
%% be expected to take back control authority over particular aspects of
%% aircraft operation that have been previously delegated to an automated
%% system? Are human pilots so unreliable that we should immediately push
%% toward automating all aircraft operations? Or are autopilots so
%% finicky that we should emphasize manual flying skills and reduce or
%% eliminate the use of autopilots. Most likely, the correct answer is
%% somewhere in between.


 While supervisory control is
implicit in much popular writing about autonomous systems,
it is not generally acknowledged as an important and
developed field. Human supervisory control is not even that well
considered in many engineering fields; even today, the human factors
engineer is often low in status within the engineering hierarchy, and
is too often brought in at the last minute to design an interface for
an already specified device,\footnote{This subtlety of the politics
  within engineering communities themselves was not something I was
  previously aware of, but was made clear to me by David Mindell.
  Discussion with the author, September 8, 2014.} rather than actually
being involved in 
device design from the beginning, as effective joint-cognitive systems
engineering
requires. While the related discipline of interaction design or
user-experience design is gaining in importance in the information
technology sector,\footnote{This I can speak to personally, through my
  involvement with programming communities building web and mobile
  applications. Demand for UI/UX designers has increased dramatically,
following the success of design-first product development by IT
companies, most notably Apple.} this too often (though not always)
addresses the interface at a surface-level, without consideration of deeper
dynamics of human and joint cognitive processes. It may be
that this kind of engineering work gains greater recognition with the
entry of Google, Apple, Tesla, and other high-technology companies
into the autonomous vehicle space, but so far Google seems to dream
of minimizing the interface as much as possible, rather than more
broadly considering what sort of interface makes sense \cite{markoffNext}.
While the interface/UX-first approach of tech companies seems as if it
would put interaction at the center, the focus on simplicity
presupposes a certain kind of highly automated operation.
%%Wrap up: Actual use all over industry (Sheridan 10-11), and implicit
%in much popular writing, 
%just not generally acknowledged as an important field in all its complexity
%HSC not even that well considered/thought out in many fields; HF
%engineers low on status and too often just brought in at the last
%minute to design an interface rather than actually being involved in
%device design from the beginning.

\section{Cognitive Networks, Networked Subjectivities}

With our attention on supervisory control, new questions and processes
come into focus. If the machine is clearly no
longer the sole component of analysis, what formerly neglected pieces
must be considered? How do we assess system design or performance, or
think about the ways these newly expanded devices operate? In this
pursuit it is worthwhile to expand our
view to encompass other related work. As Sheridan notes, much study of
supervisory control has gone on elsewhere under different names, but
with similar guiding concepts. Coming instead from cognitive
science,\footnote{And, notably, he responds to the canonical focus of
  cognitive science on the ``mental processes that organize the
  behavior'' of an individual, a position Hutchins cites as having its
standard statement in Newell and Simon's 1972 \emph{Human Problem
  Solving} \cite[p. 265-266]{hutchinsCockpit}. These two early AI pioneers show
up again here, and their individualist focus perhaps sheds some 
light on why the narratives of automated vehicle technology that are
inflected by AI are so different than they would be if told instead
through the lens of supervisory control.}
Edwin Hutchins and his anthropological work on group cognition
processes expands from the traditional focus on the individual agent
to systems of interacting agents and technologies. As he identifies,
the outcomes of tasks are not determined by individual components of a
system but by the overall dynamics of the system, and the behaviors of
a complex thinking system cannot be ``inferred from the properties of
individual agents, alone,'' irrespective of the detail to which those
individuals may be studied or modeled \cite[p. 265]{hutchinsCockpit}.
Hutchins's ``How a Cockpit Remembers Its Speeds'' lays out a framework
for understanding the cognitive activities of a human-machine system,
and uses the example of an airline cockpit to show how the cognitive
properties of a system as a whole may be very different than those of
individual human actors within that system.\footnote{This is also true
for systems that do not actually contain ``automation'' technologies
as part of their make-up. The speed bugs discussed in ``How a Cockpit
Remembers Its Speeds''---which note on the airspeed indicator
important speeds at which the 
flight characteristics of the aircraft need to be altered to maintain
safe operations---are rudimentary in terms of other forms of
cockpit automation, but the issues they present do not disappear with
increased automation. Hutchins and Klausen co-wrote another
article in 1995 titled ``Distributed Cognition in an Airline
Cockpit,'' \cite{hutchinsKlausen} which focuses on the
interactions of the Captain, first 
officer, second officer, and air-traffic control. This exchange,
though it involves control yokes, altitude alerts, and other
technological actants, is primarily focused on the joint capabilities
of the human crew, and presents fundamentally related cognitive
systems issues.} Studying the humans alone,
or the automation alone, does not suffice to explain the overall
behavior of a joint human-machine cognitive system.\footnote{It is
  worth noting here that these are not the remarks of a maverick or a
  ``mere'' anthropologist working on his own. The research involved
  was supported by a grant from the NASA Ames Research Center, as part
  of the Aviation Safety/Automation Program.}

Hutchins is very particular about the point that the speed bugs and
other cockpit devices function as more than just memory aids for the
pilots. They do not na\"{\i}vely increase pilot memory. Instead, they
are information processing tools, responsible for changing the form of
information, giving it a new representation and altering the
interaction with that information: specifically, changing a task that
requires memory and comparison to a readout to a simpler task of
``judgments of spatial proximity'' of the needle to the speed
bugs \cite[p. 282]{hutchinsCockpit}. The whole system can be said to
have a ``memory'' that is distinct from the pilot's memory: the
``\emph{cockpit system} remembers speeds'' by virtue of the pilots
within that system judging a needle on the airspeed indicator against
the position of the bug \cite[p.
  283]{hutchinsCockpit}. I introduce this material here because
Hutchins, by looking at cognitive behaviors within this particular
setting, identifies a new unit of analysis: the cockpit as a whole.
Current vehicle automation systems are more complex than the cockpit's
visual-mechanical aides, but are appropriate subjects of the same kind
of analysis.\footnote{As an aside, I would argue that if a number of
  modern information technologies, like smartphones, seem to have done
relatively little to alter human beings and their capabilities, it is
because the individual 
is still the unit of analysis. When the unit of analysis is instead
set at the individual plus his or her immediate technological or biological
surroundings, much greater differences in capability become apparent.}
Automation likewise takes complicated jobs involving spatial
processing and reasoning (driving between the lines and avoiding other
cars), and transforms them through technological interfaces into other
tasks. It is precisely these changes in task to which we should attend
if we would like to properly understand (and design) the role of the
operator within an automated vehicle. But this perspective is not
represented in the prevailing guides for automated system development.
While Hutchins's cockpit remembers its speeds through a combination of
human activity and physical cognitive aids, the NHTSA's ``vehicle''
may sometimes ``not perform a control function'' \cite[p. 3]{NHTSA} and
nevertheless remain on the road. The NHTSA's documentation instead
focuses simplistically on whether the human or the ``vehicle''
controls a function, rather than addressing what control involves or how it may
be shared between human and automated systems.

A more nuanced view of the cognitive tasks involved in interacting
with automated systems also goes quite far toward clarifying some
of the more difficult parts of system automation. Specifically,
significant research interest in human factors engineering focuses on
the problem of so-called ironies of automation, whereby increasing
automation actually 
increases the load on the operator, rather than decreasing it, and
leads to increased failure rather than increased
reliability \cite{parasuramanW}. It might
not be clear at first glance how this would be, and it is in part an
interface problem---with how information is presented to the operator,
not with the concept of automation itself. But a distributed,
cognitive attention to automation highlights the problem:  an
automation system may transform a task from a complex visual task for the
operator (e.g. driving) to another complex visual task (e.g. monitoring a control
panel with many displays and lights, and switches to press) while the
overall processes within and results of the system remain
fundamentally congruent---the system still ``knows'' how to turn, for
example, but instead of the driver turning the wheel, she interacts
with the automation systems in cognitively intense ways. Attending to
the complex internal dynamics of
the joint system---looking at which tasks are
allocated where, and what the demand on human perceptual and
decision-making systems is, at any particular time---improves the
analysis. As one should expect, 
people drive largely based on experience and
instinct, \cite{knaptonDriverless} not logical 
thought, and similarly system performance should be expected to shift
over time as interactions with automation are internalized. As
Hutchins and Klausen describe, ``It is possible to design computer systems with open
interfaces (Hutchins, 1990) that support learning in joint action but
this can only be done when the designer goes beyond the conception of
the isolated individual user'' \cite[p. 13]{hutchinsKlausen}.
Appropriately recognizing interactions is critical to successful design.

Not to belabor the point, however, the perspective I am describing is
not fundamentally new---simply, its importance is not often
recognized outside of particular disciplinary
boundaries.\footnote{This is probably because the issues are
  somewhat complex, and do not fall into neat dichotomies of
  human/machine. Again, the relatively marginalized position of
  human-factors engineering may be part of this.} Supervisory control and 
joint-cognitive systems approaches are actually in use all over
industry, even in cases where they are not necessarily cast in
those terms: from thermostats and elevators to existing autopilot systems.

%%TODO2 <??? Describe more from Sheridan 10-11 or run paragraphs together>

Hutchins and Klausen make it a particular point to highlight that
human-machine dynamics of information transfer are at the center of
the appropriate and safe operation of aircraft: ``if we step back and
look at the entire aviation system and ask how
it is that aircraft are kept separated from each other, we see that it
is through the propagation of representational state of descriptions
of flight paths into the state of the aircraft controls
themselves'' \cite[p. 14]{hutchinsKlausen}. This situation is not quite
the same for cars, which are generally not orchestrated at a
system-level in the same fashion (though many current and historical
models of vehicle automation would require this). But in principle,
the roadway presents a related 
information processing situation. The propagation of a ``plan'' of the
vehicle's path on the 
road into the state of the vehicle's controls keeps vehicles
physically apart.
And even when that vehicle system involves computerized
displays, sensors, and people, the same point holds;  at a high-level of oversight, as
one would find represented in an employee monitoring an automated
vehicle fleet remotely 
from an office, the parallelism with aircraft operations returns in
full force. There is certainly a difference in the
professionalization of pilots compared with the ``ordinary'' status of an
everyday driver, and this difference in status and roles
changes the amount of training we expect operators to have and the
sorts of interactions that may appeal to them, but there is no
fundamental reason why supervisory or joint-cognitive
systems approaches and
experience are not applicable to automated cars, if appropriately
moderated for the role and status of the driver. 

\section{Cyborgs}

How we ask and answer such questions depends on our understanding (or
not) of our own hybridity with technology. This hybridity deserves
to be more broadly appreciated within the public sphere. 
In fact, I hold that fostering its appreciation is of paramount
importance, one of the great challenges for the public understanding
of technology in the near future. Without an understanding of our
currently hybrid nature, we risk having to choose between blind ludditism
and equally blind technophilia. We come to face questions of our own
obsolescence, and are left with no way out because our way of thinking
about the world no longer corresponds to the world we see and live in
on a day-to-day basis. Either humans should drive, or machines should
drive. The idea that humans drive through machines, or machines drive
via humans, or that human and machine drive in combination, are
practically incomprehensible. They seem like failures---and they are
failures for an ideology of complete technological dominance---when
indeed they may be the roots of our greatest successes. But once we
recognize that we \emph{already} drive through machines, and that they
even already may drive through us---consider blind-spot
warning lights, which are an automated system designed to affect the
human and produce a response, but have no individual capacity to drive
the vehicle---multiple, multifaceted futures of automated vehicle
development open to us. We are not building robotic
chauffeurs, but rather designing ourselves, in some fashion, into more
hybrid, more cyborg, entities.\footnote{To return to science fiction, Phillip K. Dick said in
a 1972 speech: ``Someday a human being, named perhaps Fred White, may shoot a robot
named Pete Something-or-Other, which has come out of a General
Electrics factory, and to his surprise see it weep and bleed. And the
dying robot may shoot back and, to its surprise, see a wisp of gray
smoke arise from the electric pump that it supposed was Mr. White's
beating heart. It would be rather a great moment of truth for both of
them'' \cite{androidHuman}.
While ``Pete Something-or-Other'' does not yet exist, ``Fred White''
is already here. Fully comprehending this is a critical task for the
maintenance of cogent, productive public conversations---and
policymaking---about automated vehicles and automated technologies in general.}

%%TODO (DONE) insert cyborg ref briefly
%While they may permeate Hollywood movies, most people would probably
%say that they are not yet here. 

The cyborg is publicly understood as an artifact of science
fiction. But I would argue that we are
all cyborgs, to a greater or lesser extent, in our interactions with
everyday technologies and devices---not only those people augmenting
their bodies with implanted electrodes, or making use of prosthetics
to replace lost limbs, but everyone who interacts with a computer,
carries a cellular phone, or drives a car. This idea has received
serious attention and consideration in philosophical circles, and
deserves to be revisited here as we consider \emph{why} the hybridity
of vehicle operation is not widely recognized, and what can be done to
increase popular appreciation of this operation and the important
questions it reveals about how we design the future. Our world is
becoming a ``cyborg planet,'' to an extent many have not yet
realized \cite[p. 64]{ekbia}.

The idea of the ``cyborg'' was introduced by Manfred Clynes and Nathan
Kline in 1960 \cite{clyneskline}, as a descriptor for the alteration
of human beings to 
cope with the conditions of outer space: they would be self-regulating
\emph{cybernetic organisms} \cite[p. 66]{ekbia}. While the creation of
such cyborg astronauts was never attempted outside of the imagination
of science fiction writers, astronauts were indeed fashioned into
cyborgs by their existence within suits and spacecraft with which they
had to interface for their survival, and the cyborg idea has become a
powerful cultural force.\footnote{As Donna Haraway, one of the most recognized
philosophers of cyborg culture (and cyborg feminism), describes:
``Contemporary science 
fiction is full of cyborgs---creatures simultaneously
animal and machine, who populate worlds ambiguously natural and
crafted'' \cite[p. 117]{haraway}.} Meanwhile, cyborg technologies
abound in real life, from the extreme to the mundane: implants to
achieve increased sensory range and experience, or to trigger orgasm
\cite[p. 64]{ekbia}; implants to monitor vital signs; telerobotic
arms; bots and chatbots; implants to provide neural control of
vehicles \cite[p. 65]{ekbia}. 

%% ``Modern medicine is also full of
%% cyborgs, of couplings between organism
%% and machine''\cite[p. 117]{???-haraway}. Meanwhile, the cyborg has
%% infiltrated our methods of production and destruction: ``Modern
%% production seems like a dream of cyborg colonization work, a dream
%% that makes the nightmare of Taylorism seem idyllic. And modern war is
%% a cyborg orgy, coded by C3I,
%% command-control-communication-intelligence, an \$84 billion item in 1984's US
%% defence budget''\cite[p. 118]{???-haraway}. 
%% Haraway's view is not altogether pessimistic, as she sees in the cyborg
%% the potential site of a new epistemology that does not recognize or
%% repeat Western binaries and cultural subjugation\cite[p.
%%   118-121]{???-haraway}. But while one can hardly mention cyborgs without
%% mentioning Haraway, hers is not the vision I find most relevant for
%% understanding automated vehicles. Instead, I would like to consider

More fundamentally, philosopher Andy
Clark contends that humans are ``\emph{natural-born}
cyborgs'' \cite[p. 6]{clarkNBC}. This idea is particularly compelling
because it need make no distinction between the analog and the
digital, the material and the virtual. A cyborg nature is not an artifact of the
present moment, but a fundamental component of human experience. To
Clark, our ``ability to enter into deep and complex relationships with
nonbiological constructs, props, and aids'' is what best explains our
distinctive intelligence \cite[p. 5]{clarkNBC}.\footnote{While Ed
  Hutchins looks at these tools and stresses the importance of
  analyzing at a system level, Clark suggests these tools ``are best
  conceived as proper parts of the computational apparatus that
  constitutes our minds'' \cite[p. 6]{clarkNBC}. These conflicting views are
  united in their careful consideration of the properties of nonhuman artifacts,
  and both have something to add to nuanced examination of automation
  systems.} This therefore holds 
not just for cars, or information and computer technologies, but for
all kinds of technology. Our existence as a technological species is
not new, and human technological augmentation goes back into
prehistory. Our engagement with technological tools and
artifacts---whether the tool is a smartphone, a loom, a wheel, or
fire---is what makes us, fundamentally, cyborg beings.\footnote{Though Haraway holds that ``By the late 20th
century, our time, a mythic time, we are all chimeras,
theorized, and fabricated hybrids of machine and organism; in short,
we are cyborgs'' \cite[p. 118]{haraway}, I stand with Clark that we
have long been so. It is only recently however, in part due to the pressure
that AI places on human-machine dualisms, that this state of being has
come to be recognized.}

This concept is particularly powerful and valuable because it
challenges hundreds of years of assumptions about human beings and
human achievement---even some that are deeply ingrained in the human
factors engineering work I lean on in chapter \ref{chap:3}. We are
possessed by the idea that our technologies change, but that the human
being remains fundamentally the same.\footnote{As Richard Horner put
  it to the test pilots assembled at the SETP banquet in 1957 (see
  chapter \ref{chap:3}): ``the
  one link in the manned system which we have that improves the least
  in successive generations, is the man himself'' \cite[p.
    19]{DM}.} However, this assumption rests on a long held 
dichotomy of human and machine, and the idea that the object of
interest is the purely biological human. If prosthetics and artificial
organs teach us anything, however, it should be that these boundaries
are by no means clear. AI is however heir to a world of such sharp
boundaries: mind and body, subject
and object, nature and culture, science and politics. Science, engineering, and broader culture ``embody'' and
``regenerate'' these divisions \cite[p. 327]{ekbia}, but they are
fundamentally only true or useful analytical frameworks, so we must
ask whether they continue to serve us well, or whether they should,
despite their attractive simplicity, be retired.\footnote{Ekbia,
  considering the source of ideologies of AI, sees its roots in 
the teachings of Democritus---carried forward by Hobbes
and Descartes---and the separation of the atomist world and our
representations of it \cite[p. 331]{ekbia}. These dualisms are deeply
engrained in our 
thought patterns, but fail us when thinking about advanced
technologies and their impacts because they favor unrealistic---and
indeed, unreal---totalisms: all human or all machine, fully manual driving or fully
autonomous driving.} Cognitive
anthropology starts to break down these 
divides, with the inspection of a human-machine cognitive system as a
whole, but there is more to be done. Ekbia asks, what happens if we
abandon the outmoded question of whether machines can reach our level
of intelligence or capability, and instead ask how  ``humans and machines
[are] mutually constituted through discursive practices'' \cite[p.
  328]{ekbia}?

%% TODO: FIX THIS FOOTNOTE, make it clearer how it fits

The problem facing AI, according to Lucy Suchman,\footnote{Note that
  Ekbia mobilizes this statement as well in his arguments
  \cite[p. 331-332]{ekbia}.} ``is less that we attribute agency to
computational artifacts than that our language for talking about
agency, whether for persons or artifacts, presupposes a field of
discrete, self-standing entities'' \cite[p. 263]{SuchmanPlans}.
Instead, we should be asking how intelligent behavior emerges in
the interactions of
humans and machines---involving networked, rather than individual,
subjectivities---refusing to fix \emph{a priori} the category of 
the human. Automated vehicles are a product of AI, and are vulnerable
to the same critique. 




\section{Whither Alternate Narratives?}

While there is a perhaps pleasing sort of narrative arc to the idea
that complete and total automation is the ultimate end product of
technological development---restrained only when nasty luddites, backwards lawyers, and
risk-averse policy makers get in the way of natural progress---this
simplistic story of automation
teleology is, as we have seen, not supported by history. While what
has happened in the 
past is not destined to repeat, and the regime of ground vehicle
automation could come to be very different from that of aerospace,
this history at least needs to be taken seriously as evidence for what
can and may occur. There are lessons here we cannot afford to forget.
While it may be just as difficult, or even more difficult, to engineer
an appropriate joint human-machine system that takes the capabilities
and desires of people seriously, rather than trying to minimize human
involvement to the greatest extent possible, that approach might
result in vehicles that 
come to be more socially acceptable in certain ways---notably, in
terms of autonomy and ethics. Furthermore, as I have stated
repeatedly, complete autonomy is fundamentally a chimera.

%%TODO (DONE) add this paragraph
%% ``Urmson said Google believes that while technology that assists drivers
%% can help reduce some accidents, only completely self-driving cars will
%% fully address safety concerns. One reason, he said, is that the better
%% the assistive technology gets, the more risks human drivers will
%% take.''
%% ``Nor, he said, will today's driver-assist technologies just evolve to
%% become fully self-driving cars on their own.

%% `That’s like me saying if I work really hard at jumping, one day I
%% will be able to fly,' he said.''
%% --> he is right that driver assistance tech won't just become
%% driverless, but wrongly assumes driverless is the only reasonable end goal.

Narratives that base themselves on firm distinctions between
``levels'' of automation are partially to blame here. Google's Chris
Urmson wants to separate incremental or assistive 
approaches from ``completely'' self-driving cars, in part perhaps to
hype Google's current work. Google's claim is that only complete
vehicle automation can entirely address road safety issues, citing
increases in risk-taking behavior likely to accrue in incrementally safer systems.
But Urmson goes further to say that the incremental approach will
never achieve the necessary full automation: ``That's like me saying
if I work really hard at jumping, one day I will be able to fly''
\cite{friedGoogle}.
But while layers of assistive technologies cannot be expected to simply become
driverless without high-level control and coordination, this perspective wrongly
assumes that driverless vehicles, ``completely'' automated, are the
only reasonable end goal. 

%% 3) situate the moonshot approach (which may actually be easier in some
%% senses but less socially acceptable) and the mixed approach in a way
%% that does not support a teleology of autonomous vehicle development
%% (think Sheridan's graph of telerobotics, not the SAE's 5 stages) (4 p)
%% --in historical context of previous systems that are wholly or
%% partially autonomous (4 p)

%%Defense Science board report (DSB)
These levels formulations, and their assumptions, are not without
broader criticism. The Defense Science Board, in their report
\emph{The Role of 
Autonomy in DoD Systems}, describes the proliferation of attempts at
automation taxonomies as developing out of Sheridan's work, the very
book cited previously as the bible for human supervisory control \cite[p. 23]{DSB}.
However, the DSB takes a far more nuanced position than many do,
perhaps due to many years worth of accumulated experience within the
military in terms of how to understand and apply autonomy in practical
systems. The DSB
identifies that these levels formulations are ``often incorrectly
interpreted as implying that autonomy is simply a delegation of a
complete task to a computer, that a vehicle operates at a single level
of autonomy and that these levels are discrete and represent scaffolds
of increasing difficulty'' \cite[p. 23-24]{DSB}, all interpretations
made by both the 
NHTSA and SAE taxonomies that may be expected to guide self-driving
vehicle research and policy, and the very interpretations I have been
attempting to critique. Instead, autonomy in practice generally
involves subtasks, and novel mixes of human and computational effort,
not the wholesale transfer of a complete task or process. Levels
formulations lose sight of the fact that autonomy and control are
already fractured and contingent:  what ``level'' is my vehicle
operating in when I drive an automatic transmission in semi-manual
mode, or engage cruise control, or allow a turn signal stalk to
automatically return to neutral after a turn of the wheel is detected,
or stand on the brake pedal to engage ABS in an attempt to bring the
vehicle to a stop as quickly as I have requested. These currently
everyday situations already defy levels, and represent overlapping, shifting
delegations of authority that may change on a second-by-second basis.
I also claim, to add to their critique, that the
levels seem to suggest scaffolds of increasing \emph{desirability},
which, as we have seen in air and space flight, is a matter of perspective.

In addition, the DSB highlights two main negative consequences of the popularity of
levels of autonomy: that it ``deflects focus from the fact that all
autonomous systems are joint human-machine cognitive systems'' and
that it ``reinforces fears about unbounded autonomy'' while obscuring
the fact that no systems are fully autonomous \cite[p. 24]{DSB}. Humans are always
involved somewhere along the line, in the programming, production,
and use of automated systems, though it makes sense that the military,
focused as it is on 
human command hierarchies, would find it especially important to make
this apparent. Even so-called ``full autonomy'' is not likely to be so
simply \emph{full}. At the very least, a human-rated\footnote{This is
  a term of art in engineering: a human-rated (or man-rated) system
  must meet stricter safety standards than one that is unmanned.}
system will need to have 
a stop or abort button. And having such an option implies some human
oversight, which means a number of additional questions soon arise:
How much monitoring? When does it happen? And how is it
regulated or supported by the device? We should recognize that this
monitoring is a valid and valuable component of vehicle operations,
rather than discount it as a stopgap measure. New tools and interfaces
may be necessary to facilitate this process, to allow for the human to
take on new roles in the human-machine relationship: currently, engineers
co-piloting Google's driverless vehicles sit with laptops to monitor
the vehicle telemetry \cite{bilgerAuto}. Data displays
like this, or like the pretty LIDAR visualizations that get used to
explain and dramatize sensor data, could  form the core of the
instrumentation for the automated vehicle. More discussion should be
happening around the dashboard of the future; in the dominant
narrative of the driverless vehicle, the dashboard is vestigial, but
it could be a vibrant space presenting unparalleled new possibilities
for human-machine interaction.

%%TODO (DONE) insert below And panic issues: find the official report, but
%%connect ethics and emergency
%%response; scaffolding a system that helps people is key
%http://m.nautil.us/issue/23/dominoes/fear-in-the-cockpit

Attention, and how to preserve, regulate, monitor, and transact it,
must be at the center of vehicle development.\footnote{I must add here
that attention alone is also not na\"{\i}vely a good thing:
human attention and supervision can be creative (the manager) or
menial (the machine-tender). As one would expect from the analogy with
factory organization, this issue is deeply political. Who gets what
kind of agency, and exerts what kind of attention, is at the center of
the human-machine dynamic in automation technology. Not focusing on
the role of human attention risks that these decisions get made
by fiat, on the basis of other factors.} And the parameters of
these interactions are in part defined by the end goal. Much popular
discussion centers around texting as a key motivator of the need for
automated cars. But as 
David Mindell describes, ``wanting to text is different
from needing a fully automated vehicle,''\footnote{David Mindell,
  discussion with the author, September 8, 2014} and it seems quite possible
to create a vehicle that will allow distractions on the order of the
time span of writing and sending a text even if it is not feasible, or
not culturally acceptable, to build one that is fully
automated.\footnote{Recent AAA research on teen accidents shows
  distraction to be a factor in 58\% of crashes, and that drivers were
not paying attention to the road for an average of 4.1 out of the
final 6 seconds before the crash \cite{greenDistraction}. This
suggests that a system able to 
maintain control of the vehicle for times on the order of 10 seconds,
and stop it safely if necessary, could do a lot to address this kind
of distraction---that is assuming, of course, that risk homeostasis
can be mitigated.} While
the latter means that long periods of time might go by without human
monitoring (perhaps minutes or even hours, depending on system
design), the former merely requires an operator to be able to be 
distracted on the order of 10 seconds without causing a crash. These
are, at their core, very different engineering situations. And though
people do panic in crisis situations such as the TransAsia Airways
Flight 235 crash in which panicked pilots throttled back the wrong
engine \cite{wiseFear},
it may be more appropriate and more useful to consider systems 
that scaffold and support human capacity and attention, rather than
attempting to eliminate it altogether. In
all this, we should avoid the clich\'{e}d question ``People: sinners
or saints?'' which seems so popular in media discussions: it represents
a ``false position'' that does not adequately reflect the complex
nature of operating automated
systems \cite[p. 1]{woodshollnagel}. As operators, we are
sources of both success and failure. But ultimately, as Woods and
Hollnagel point out, all cognitive systems are finite and will
therefore also have errors, lapses, and
failures \cite[p. 1--2]{woodshollnagel}. Success in automation is not
a given that the human presence degrades,\footnote{And it is not fair
  to say that increasing automation necessarily decreases the
  importance of the human. The NRC (National
  Research Council) report on aviation, claims the following: 
``For example, antilock braking systems and airbag systems on cars are
fully automatic, deciding on their own when to act. However, fully
automatic systems as well as fully
autonomous systems depend on humans to define and limit the scope of
their authority and the range of possible actions. Movement along the
continuum typically does not eliminate or diminish the importance
of the humans to the operations of the system, but it does change
their role. In fact, the human's role
becomes more rather than less important when moving toward the
autonomous end of the spectrum
because it is so important to assure that the systems are
properly designed, tested, deployed, and monitored to ensure the
aircraft's continued airworthiness'' \cite[p. 14-15]{NRCAutonomy}}
but something to be  
achieved through careful engineering of automated components and their
human interactions. The PARC/CAST Flight Deck Automation Working Group
found that human pilots are actively involved in mitigating risk from
other system components, and states that ``an exclusive focus on pilot
errors will not take into account the positive actions and decisions
that pilots do on a frequent basis'' \cite[p. 30]{PARCCAST}. 

We should return here to the myth of the ``personless'' factory or the
self-teaching computer program: the ``autonomous'' vehicle is a
similar sort of myth; always a 
product of people, responsible to people, and involving people,
however far in the margins of our vision. Writing this off as
anomaly or failure, and focusing on ``autonomy'' or ``self-driving,''
obscures real complexities of operation that encounters with the world
will inevitably involve. The important questions are when and how 
people are
involved, and this should be empirically not ideologically
driven.\footnote{Of course, beliefs and ideologies are always involved
  in decision-making, but we should at least air them for all to see,
  and weigh them carefully, rather than blindly follow a dominant
  idea of the uses of technology.}
Perhaps there is hope in this altered narrative to ameliorate some of
the issues facing automated vehicles today. 

It seems highly unlikely that a supervisory control perspective will
fix the potential contribution of automated vehicles to the
surveillance state:  data gathering technologies will still be
available, and comprehensive mapping may still be necessary for
operations, though human supervision might reduce
reliance on remote processing of data in the short term, and thereby
improve privacy. In 
any case, the potential still exists to use hardware 
and software to try to restrain and control people's actions and
movements. Supervisory control will also certainly not fix the
reliance on projections of risk for decision-making about vehicles:
while reducing certain risks, this approach may increase others, and
any estimates of such risk will still be estimates only, derived from
models that may be incomplete.

However, human supervisory control stands to alter the dynamics of use
and therefore change the sort of urban planning necessary to account
for these vehicles, as well as the environmental effects of their use.
Systems designed with supervision in mind may operate in a different
way than those designed primarily with autonomy\footnote{This is a
  tricky distinction. In this case
  I mean autonomy over long timescales, and, as major companies seem
  to be working on, autonomy from the human in the vehicle. Even an
  ``autonomous'' car will be supervised at some level, but the
  question here is on what timescale the human is involved in
  monitoring vehicle actions. The ``autonomous'' model really means very
  long timescale supervision, such as from a remote data center. In
  the supervisory control case I am considering here, though such
  supervision could come remotely, it would be more frequent and more
  deeply involved in the vehicle's actions---a military UAV operator
  would be one example of this.} in mind. Operational
differences will change the behaviors favored by these systems:  Can
they operate on their own and drive to remote parking garages? Or be
sent to pick up children from school? How much attention is required
to operate the vehicle, and if that attention comes from the human
within the vehicle, how will that affect how likely people are to
drive rather than take other forms of transport that could reduce,
rather than increase, road congestion? 

Where supervisory control considerations come to totally change
the landscape is in the field of ethics. The ethical operation of
vehicles involving computer code will always depend to an extent on
that code, and the previous discussion about programmatic ethics still
holds. But what changes here is that the vehicle is no longer free to
operate based only on its programmed ethics for long periods of time.
Instead, the human operator is involved in decision-making processes
at a much more granular level. Will some emergency situations likely
be handled in a largely automated fashion, just as today's pedestrian
detection systems can perform an emergency stop to avoid pedestrians a
driver does not see? Almost certainly. But these issues, though
perhaps ethically tricky, are much closer to situations we are already
working through. Ironically, one positive effect of the autonomous vehicle
might be to estrange us from vehicle operations, and thereby draw us
to consider more seriously the ethics programmed into software
systems, but the effect in the dominant narrative seems to be drawing a
clear distinction between current and future automation systems:  in
the NHTSA's policy document, current automation systems like ABS,
traction control, and stability control can override the human, but
``driverless'' systems must have overrides that can be used at any
time \cite[p. 13]{NHTSA}. This distinction is artificial. However,
consistent human involvement in vehicle operations seems to open new
doors for human acceptance, by putting these systems closer to what we
already know. What we should do is consider carefully why we feel that
ABS and traction control can be allowed to override human inputs, and
what separates them from systems that should not be allowed to do so, and
keep this in mind as we continue to develop more complex automation systems.

Additionally, supervisory control presents the possibility for a
renegotiation of car 
culture and its troubled relationship with the automated vehicle. In
my interviews and research work at conferences, I noted especially the
contradictions within common characterizations of human
drivers and their future. One interviewee basically asserted that
humans are bad drivers and that perhaps we should ``all have our
licenses taken away,'' but later stated that ``car nuts'' should
welcome self-driving cars because they will free up the roads for
enthusiast use. This combination does not make sense. The
question of whether people will be allowed to drive also loomed large
in industry conferences, with further comments that do not hold
together: computer scientist Daniela Rus stated at an MIT
conference, in response to
concerns about being 
forced to use automated vehicles, that people will still be able to
drive normally, sharing the roads with automated vehicles in the near
term. She is most likely correct, but this 
does little to allay fears in the face of the statistic that prompted
the question: that 90 percent of accidents are caused by human
error.\footnote{This statistic seems warranted given recent NHTSA crash
  data \cite{NHTSAcrash}. Around 93 percent of crashes with ``critical
  reasons'' involve driver error as one critical reason for the crash.
  But 30 percent involve a roadway factor, and 12 percent involve some
vehicle-related reason. There may be many reasons for a crash, and
estimating when the human is solely or primarily responsible for a
crash is not trivial.} While
human driving is not going away in the near future, the idea that
autonomous vehicles will free up the roads for people who want to
drive manually seems more like a platitude than a thought-through
design strategy. 

%% TODO2 \footnote{FIND ESTIMATES OF rate AND COMMENT ON ACCURACY}

These questions of use and licensing are deeply embedded in the idea
of progress. ``Does the potential for greater safety imply we should be
taking human licenses away?'' is a question that only makes
sense with a particular vision of autonomy, one that does not involve
human supervision. Perhaps a better question is what sort of licensing
and training is necessary, or should be necessary, to operate new
kinds of vehicles, which must be defined by the roles that the human
operator must perform. There is certainly a difference in the intended
operators of vehicles between this situation and historical models,
since airliners are piloted by professionals
and cars are generally considered to be driven by non-professionals.
Nevertheless, people are licensed to drive, and there is no reason to
say that new types of training may not be necessary or good;
or even that some continuing education would not be a positive
influence on traffic safety, on balance, even with present
technologies. But with further human involvement in driving, there is
the possibility of a
middle-ground in the hybrid narrative:  future cars, using increasing
amounts of automated safety technologies,
may be vulnerable to the same critiques as the Nissan GT-R. But these
vehicles could still provide a role for the human and a place for
enthusiasts who want to be involved in vehicle operations.

\section{Conclusion}

In the end, whether an operator is in a vehicle trying to monitor that it
is on the right path and not about to hit an undetected
obstacle; or is waiting
for a car to arrive, wanting to ensure that it is en route
and not in an accident; or is working as a system-technician at a
multinational corporation administering a fleet of automated vehicles,
attempting to ensure they are not being 
hijacked or stolen, the operator functions as a supervisory
controller, enmeshed in a complex network of human and machine
systems. Not recognizing this as such does not make it not so:  it 
only means we are 
likely to neglect the most important parts of the system design---in
terms of its adoption and long-term use---by blindly following a
narrative based on a warped view of history. And we are likely to
ignore new skills and competencies that may be important for
operation (or even vital to operation), or which may be valuable in other ways, for
human reasons. Cars---and transportation systems as a whole---are
designed to serve human needs, and must ultimately answer to society's
ideas about those
needs, even those that might seem to some to be mere ethical squeamishness.







\section{Engineering Standards and Policy Documents}

%% I may be able to do this drawing on the other subsections
%%as a CONCLUSION, wrapping up the other three things
%%and what is LOST/OVERLOOKED; the ability to deal with nuance and
%%REAL ACTIVITY
%%other part of an analysis of this kind of stuff: did it for a
%%reason; what is the work that it does (reactionary thing in some
%%cases; pushes for a particular image of what this can be)
%%--visions of human; SAE history of thinking as engineers; people as
%%predictable black-boxed inputs and outputs

%%maybe split into different sub-headers with context like NHTSA:
%%Agency Trying to Catch Up or w/e

While some of the conventional driverless car narrative comes from
poor understanding of the technology and history, some also comes
straight from contemporary sources, the engineering standards and
policy documents set up to
guide the industry's development. A number of levels-of-autonomy or
levels-of-automation taxonomies have been developed to guide
researchers and public agencies toward an appropriate understanding of
how automation research will progress. While they vary in terms of how
prescriptive or descriptive they are, these documents are easily taken
as evidence for how automated systems must or will develop in
practice.\footnote{The Society of Automotive Engineers in particular
  made an attempt to avoid prescriptive definitions of autonomy in
  their work. But even a descriptive set of levels contains an
  explicit hierarchy, which can be read as an implicit narrative of
  progress or roadmap for development.} And, when closely
interrogated, they present gaps in vision that echo the popular
narratives of automation and AI, and do not sufficiently heed the
historical complexities we examined above.

The primary formulations of levels of autonomy for self-driving cars
have been published by the National Highway Traffic Safety
Administration (NHTSA) and the Society of Automotive Engineers
(SAE).\footnote{The German Federal Highway Research Institute, BASt,
 has also produced a taxonomy that predates the SAE's work, and
 influenced it in key ways.}
These reports exhibit some resonances and contradictions in how these
organizations represent autonomous vehicles and their human drivers.
Viewed in the larger context of the discourse on levels of automation
in general, these speak to different ideas of who the ``driver'' will be
in automated systems. However, both make particular assumptions about
the way autonomy is or will be implemented that may not ultimately be
valid and certainly foreclose on alternative ways in
which systems could be designed. These documents represent an important
vector for the traditional autonomous vehicle narrative, and are
therefore worthy of deeper scrutiny.

\subsection{NHTSA: Reacting to the Industry Narrative}

The NHTSA's levels of automation focus largely on the human driver and
human costs, and seem to represent an inability or unwillingness to
think beyond human-machine oppositions which 
parallels the general narrative tendencies we have already examined in
automation and AI history. 
The agency identifies three ``distinct but related streams of
technological change'': 

\begin{quote}
(1) in vehicle crash avoidance systems that provide warnings and/or
limited automated control of safety functions; (2) V2V communications
that support various crash avoidance applications; and (3)
self-driving vehicles \cite[p. 3]{NHTSA}.
\end{quote}

Though the agency positions these technologies ``as part of a
continuum of vehicle control automation'' \cite[p. 3]{NHTSA}, aligning
themselves with Thomas Sheridan, whose
work forms a fundamental part of chapter \ref{chap:3}, this belies
their three-part dissection of the technological landscape and their
following 5-level taxonomy. They correctly recognize that today's
``crash avoidance and mitigation technologies'' are the ``building blocks
for what may one day lead to a driverless vehicle,'' but incorrectly
assume that an easy line can be drawn between driven and
driverless \cite[p. 3]{NHTSA}. In truth, vehicle automation
technologies present a spectrum of possibilities that are not clearly
delineated, but stretch from the ``fully manual'' control of early
automobiles to current amounts of automation, to potential future
technologies that are remotely monitored. Like the mythical personless
factory, the driverless vehicle can only exist by dismissing its new forms
of labor. 

According to the NHTSA definition, automated
vehicles are those in which ``aspects of a safety-critical control
function (e.g., steering, throttle, or braking) occur without direct
driver input'' \cite[p. 3]{NHTSA}. Therefore, this definition excludes warning
systems that nevertheless still use ``automation'' in varying degrees,
drawing a suspect line between monitoring and action. Though there is
a difference between systems that only provide information and systems
that directly affect the mechanical state of the vehicle, making that
distinction part of the definition of ``automation'' perpetuates an
inaccurate view that automation is only meaningful or important to
consider when it affects a physical mechanism. Information automation
(road condition warnings, traffic notifications, etc.), which has
potentially great implications for safety and the driver's awareness,
is written out of the NHTSA's definition. But information automation
is one of the historically dominant modes of automotive automation,
even in the eyes of the NHTSA, and has been on the research agenda for
decades \cite[p. 11]{wetmore}.

Their definition also excludes non safety-critical ``control
functions.'' The agency is likely attempting to make this distinction
so that cars with currently banal automatic technologies like
automatic transmissions do 
not count as ``automated vehicles,'' but this is the wrong footing on
which to rest a definition of automation, as it neglects both the
vital roles automation plays both in tertiary components as well as
the delicate dance of human and automated control that happens even in
current vehicles. Automated wipers and hazard
lights may well be ``safety critical'' in at least some situations.
The state of the transmission 
and drivetrain, controlled via automatic (read: automated) or manual
shifting, is fundamentally important to throttle control and the
ability to safely control the vehicle. And shifting represents one of
several areas
in which complex relations of human-requested and
automatically provided operating criteria are already visible: both
the driver and automation may be concurrently monitoring the engine
RPMs, and the driver may request shifts that reinforce or override the
preferences of the automation (or which, in the case of a shift that would
over-rev the engine for example, are rejected by the automation). A guide for the
continued automation of automobiles could (and, I would argue, should)
make an honest attempt to reason about such preexisting examples and
build from them. But instead, they are ignored, and thus the
terms of the agency's definition of ``automation'' itself are incoherent
and self-contradictory. By writing out current systems from the field
of ``automated vehicles,'' the NHTSA has hobbled their terminology,
and limited its usefulness to make sense of the broad spectrum of automation
approaches that are actually being used in existing vehicles.

The agency does not view the human-machine system as a whole, focusing
on, by labeling as ``automation,'' only those processes that are
mechanical in result and are arbitrarily considered to be
``safety-critical'' as opposed to ``secondary.'' Even if one does not
wish to include 
the driver as part of the ``vehicle,'' real-world vehicle tasks, or
control functions, are more complicated than simply turning the
steering wheel; but the NHTSA's levels of autonomy only compound these
issues with simplistic distinctions that do not account for the
majority of ways these systems could be engineered. Level 0, or
``no-automation,'' represents 
precisely the illustrated issues inherent in the NHTSA's definition of
\emph{automation}. Level 
1 is the only automation level beyond 0 that does not have the
ability to apply all types of mechanical input to the system: it is
limited to ``one or more specific control functions'' that operate
``independently,'' and only for certain periods of time \cite[p. 4]{NHTSA}. The
necessity of independence between automated ``control functions'' is
especially hard to fathom: in a modern automobile, few if any systems
are truly independent, instead coordinated via electronic management
systems.\footnote{For example, see systems like Bosch's comprehensive
  electronic energy management, EEM.} 

The report is even confused on the number of control functions, though
it ultimately means that both steering and brakes/throttle cannot be
automated at the same time in a level 1 system: 

\begin{quote}
there is no combination of vehicle control systems working in unison
that enables the driver to be disengaged from physically operating the
vehicle by having his or her hands off the steering wheel AND feet off
the pedals at the same time \cite[p. 4]{NHTSA}.
\end{quote}

This continues the agency's curious focus on the physical state of the
driver's body, repeated in level 2 automation to make the distinction
that now the operator can be ``disengaged from physically operating the
vehicle by having his or her hands off the steering wheel AND foot off
pedal at the same time'' [sic] \cite[p. 5]{NHTSA}. This description of bodily state is
the most coherent part of the levels definitions, but makes little
practical sense for defining automation. One can drive hands-and-feet
free in a non-automated vehicle, for short periods of time, in the
right conditions. Does the car become a level 2 during this time?

%%EXPAND on the definitional issue

While this question seems fatuous, it becomes important if the
physical state of the driver's body will define system-level
automation---and, given the NHTSA's power as a regulatory body, come
to define legal aspects of automated vehicle policy, such as how
vehicles are registered. (It is also worth
noting that this clarification is
unnecessarily normative, and does not transfer to other types of
bodily input---like GM's abandoned ``Unicontrol'' \cite[p. 8]{wetmore}---or controls designed
for amputees.) Though this stipulation attempts to address what the
human driver is actually doing, it is insufficiently granular to
account for the various types of mental and physical effort exerted:
from using control stalks to monitoring for vehicles to consulting a
GPS or mentally planning a route to monitoring and adjusting an
automation system via steering-wheel based controls. But the
preoccupation with the body results from the NHTSA's charter: the
human is a safety liability, something to be protected. The operations
of hands and feet represent readily visible and na\"{\i}ve distinctions
between human and automated, even if the distinctions do not turn out
to be particularly useful when faced with further technical scrutiny.
For their stated audience, primarily state lawmakers who are not
likely to be human factors engineers or well-versed in human-machine
interaction, these markers of human action are persuasive, but likely
to contribute to bad policies.

Further levels extend the functions which are automated. Levels 2-4
have automated systems capable of making all the electromechanical
inputs necessary to drive; they differ only in the extent to which
human stewardship is necessary \cite[p. 4--5]{NHTSA}. This exposes na\"{\i}ve assumptions about
how vehicles will be automated:  one whole system at a time---and
these assumptions appear implicitly, again and again, in press
accounts. How much
control is handed back when the human has to take over more vehicle
operation tasks is not clear: One primary control function? All
primary control functions? The NHTSA is trying to define an overall
automation level for a vehicle, but their definition masks system
specificities. The report is not clear on the level of automation for
a vehicle in which some control systems require continuous monitoring
and others can transition to manual control on an appropriate
timescale. And the NHTSA's taxonomy rests on the assumption that the
work of driving a vehicle will remain basically the same, some tasks
simply shuffled to the computer---system control first, monitoring of
control second---with no new cognitive loads placed on the human as a
result. This too is a na\"{\i}ve position that is not supported by current
examples of automation technology, which often result in the
generation of new types of human work (such as monitoring the
automation system itself). This labor is implicit in the NHTSA's
taxonomy, but not examined in great detail. The NHTSA suggests special
training to ``authorize the operation of self-driving
vehicles,'' \cite[p. 11]{NHTSA} and
positions the operator as a subject of education and regulation while
calling upon him or her to be the ultimate decision
authority:\footnote{They mention ``Several State automated vehicle laws
consider the person who activates the automated vehicle system to be
the 'driver' of the vehicle even if that person is not physically
present in the vehicle'': their only commentary on this point is to say
that they know of no current systems (level 4) capable of this
operation \cite[p. 5]{NHTSA}.} the
automated functions should always defer to driver input to the wheel
and pedals (with the sole exception of already-proven technologies
such as traction and electronic stability control) \cite[p. 13]{NHTSA}.

%%--NHTSA as reactionary, as trying to cope with what Industry says;
%%and as history of dealing with people and responsibility in
%%particularly binary ways; REAL PRESSURE to get it OUT THERE to guide
%%development
The NHTSA framework overlooks a large proportion of the actual work of
driving, and is therefore a poor model for actually evaluating and
regulating automated vehicle systems. Its bias toward
full-automation, structuring the hierarchy of levels around full
self-driving vehicles as the technological peak, slants the
development narrative---in popular attention as well as in
potential regulation and legislation---toward one particular approach
to vehicle automation while ignoring or discounting alternatives. I do
not mean to imply that the NHTSA are myopic, but their framework is
deeply influenced by their institutional culture as well as the
circumstances of its creation. Their preliminary policy document was
reactionary, an attempt to regain ground and provide some sort of guidance to an
industry already testing on public roads. The automation-first
approach, which wants to jump directly to an NHTSA level 4 system,
must be tempered for liability reasons since, as the 
NHTSA is very aware (given their research interest in level 2 and 3
systems), the technology is not yet ready.\footnote{They continually
  restate the point that only few level 3, and no level 4, systems
  currently exist, and that neither of these technology levels is yet
  ready for unfettered operation \cite[p. 10, 14]{NHTSA}.} When trying to make space for
technologies that are not yet mature, keeping a human driver in a
position to recover when things go wrong, as the NHTSA does, makes sense as a normative
strategy. However, the NHTSA implicitly buys in to the prevailing
public-facing narrative in industry, that fully-autonomous vehicles
are the obvious, natural end product of the evolution of automated
systems. Their document, therefore, reinforces that narrative.

\subsection{SAE: The Human as Engineered Component}

Contrasting the NHTSA's model with the SAE's provides a deep look into
the priorities of the organizations. The SAE document provides a full
taxonomy of levels of automation for ``on-road motor vehicles,'' and the
presumed audience consists of engineers not policymakers \cite{SAE}. The report
focuses on the three ``highest'' automation levels (``conditional, high
and full automation''), implicitly because these are the newest areas
of research and therefore the most important. The report is careful to
distance itself from the terms ``autonomous'' or ``self-driving'' as used
in the media, preferring instead its own carefully-defined
terminology \cite[p. 5-6]{SAE}. 

The SAE report does try to square itself with NHTSA recommendations,
however approximately, and represents the hidden difference between
NHTSA's levels 2 and 3 as the monitoring of the environment by the
vehicle (in level 2 the vehicle does not monitor, and in level 3 it
does). In contrast, the NHTSA report does not make this plain, and
indeed a level 2 system can be said to ``relinquish control with no
advance warning'' which should be impossible without either knowledge of the
environment or the occurrence of a bug---the system needs something on
which to base its decision to ``relinquish control'' if such a decision
can be said to be made \cite[p. 5]{NHTSA}. Rather than overloading the word ``control,''
the SAE considers the ``dynamic driving task'' to have a number of
components, including the detection and classification of objects and
events, the response to such events, planning of maneuvers, steering
and turning (including lane-holding and changing), acceleration and
deceleration, and ``enhancing conspicuity'' (referring to lighting,
signaling, gesturing, etc.) \cite[p. 6]{SAE}. Similar to the NHTSA's
restricted focus, the SAE's taxonomy 
focuses primarily on four major aspects of this task:
steering, acceleration/deceleration, monitoring of the environment,
and fallback performance, but is somewhat clearer about how tasks are
defined---in particular, it is clearer about the separation of the
driving task into longitudinal and lateral components. Also like the
NHTSA, the SAE's definitions show all systems above level 1 having the
full electromechanical capability to drive the vehicle (to execute
both longitudinal and lateral driving tasks) which makes it difficult
to account for complex hybrid systems in which some tasks may be
highly automated and others not.

The primary distinction for the SAE report is whether the ``human driver'' or ``automated
driving system'' monitors the driving environment \cite[p. 5]{SAE}. The report deals
appropriately with human psychology, stating specifically that higher
levels of automation are based on the expectation that the human need
not ``and therefore will not''\footnote{Attention and task hand-over are
  known issues in aircraft automation, as prolonged inattention leaves
  pilots ill-positioned to regain control of systems when necessary
  (see for example the events leading up to the Air France 447
  crash). This will be a particular focus in chapter \ref{chap:3}} continuously
monitor the environment \cite[p. 9]{SAE}. But their 
picture is still binary, and organized around monolithic tasks, rather
than considering the moment-to-moment 
distribution of human 
cognitive effort. This is implicitly connected to their inclusion of
the lower levels of automation only ``as points of reference to help
bound the full range of vehicle automation'' \cite[p. 2]{SAE}: they are not within
the document's focus, which favors an implicit---to the NHTSA's
explicit\footnote{The NHTSA explains this by deferring to existing
  laws: ``Generally, these laws seem to contemplate vehicle 
automation at Levels 3 and 4, as discussed above, i.e., some form of
self-driving operation. 
Accordingly, these recommendations are tailored to Levels 3 and 4
automation'' \cite[p. 10]{NHTSA}.}---rhetoric of progress toward the
higher levels of autonomy. 
Not only are hybrid task monitoring systems classified as level 2
regardless of their complexity or capabilities, the SAE's taxonomy by
definition leaves out whole classes of systems where the default
execution of the dynamic driving task is up to the human but
monitoring/fallback performance are computerized, or where the dynamic
driving task is shared in a complex way, as is monitoring and fallback
performance. It considers only systems in which longitudinal and
lateral control is handed over to the system earlier in the hierarchy
than monitoring or fallback performance.

The SAE is better about
identifying new types of human work, such as the monitoring of the
automated system, generated by partial automation
strategies.\footnote{For instance, the human driver ``constantly
  supervises dynamic driving task executed by driver assistance
  system'' \cite[p. 3]{SAE}.} They
are also more conscious of issues of the handover of control, and
specifically discuss delayed-release of particular tasks ``when
immediate human takeover could compromise safety,'' slowly
transitioning control of a system from fully autonomous to fully
manual though never lingering in between \cite[p. 4]{SAE}. It should be
noted that this directly contrasts with the NHTSA's driver-focused
approach, as it places fundamental decision authority in the hands of
the computer system. (When development is not necessarily oriented
toward achieving fully autonomous operation quickly---or when a
taxonomy is not designed around making space for Google's self-driving
car project and its ideological orientation---and is instead
focused on incremental improvements, such electronic overrides seem more acceptable.)

The SAE mentions the true complexities of driving at the report's end,
describing how the dynamic driving task is distinct from ``driving'':
``Driving entails a variety of decisions and actions, which may or may
not involve the vehicle being in motion or even being in an active
lane of traffic'' \cite[p. 12]{SAE}. Driving is split into Strategical, Tactical, and
Operational components, where strategical includes trip planning and
route selection, tactical includes maneuvering, and operational
involves ``split-second reactions that can be considered pre-cognitive
or innate'' \cite{Michon}. The SAE is explicit about their exclusion of Strategical
effort, presumably the human's task, from the definition of the
dynamic driving task. However, this admission does expose an
inconsistency within the SAE's taxonomy: level 5 automation (and some
types of level 4 automation) definitionally require Strategical effort
(route selection is implicit within GPS navigation), which is not
separately mentioned as a system capacity in the document.

Despite their differences, these taxonomies are united in lending
credence to the teleological narrative of automation:  further
technological development implies ``higher levels'' of automation, which
imply a decreasing role for human beings. (How different these
taxonomies might appear if structured or numbered differently!)
The tendency, endemic to these taxonomies, to predicate work on
jumping to the flashy upper tiers of automation reinforces the
blindness to complex hybrid human-machine systems, and results from
preconceptions about the value of level 5 autonomy and the
intermediate stages by which it will be achieved. The SAE's taxonomy
is part of a long tradition of levels of automation formulations in
engineering that make these kinds of implicit assumptions in various
ways \cite{parasuramanW} \cite{PSWickens} \cite{ALFUS} \cite{SMART}. Their
greater facility with the technologies involved seems likely to have
contributed to their ability to achieve a somewhat more nuanced view
of vehicle automation than the NHTSA. The
technology-first perspective that shapes these sorts of formulations
likewise comes from a deep history of engineering practice, and a
tendency to treat the user as a black-boxed entity, another signal
processor with defined inputs and outputs, subject to mathematical
modeling---a set of qualities that is highly useful for engineering
work, as it is amenable to descriptive standards and interface
specifications. Such a perspective tends to diminish the attention
paid to
the subtleties of human action and interaction, which engineering,
though it may try to minimize, cannot afford to ignore completely.

While supporting the dominant, teleological narrative of automation's
progress, both
taxonomies ignore the external labor involved in that automation:
human action displaced in space and time. Where do remotely monitored
systems fit? What about human preparatory work that allows for
automated systems to operate? These are questions with serious
implications, to which we will
return in chapters \ref{chap:3} and the conclusion. They are not
answered by either of these autonomy formulations, but should be
addressed by any document that attempts to describe, normalize, or
regulate automated vehicle operations. 

\section{Conclusion}

In a sense, these autonomy
formulations pull together the three previous strands of this
chapter: (1) they are influenced by science and design fictions, and seek
to respond to the narratives of technological revolution as
transmitted through the popular marketing of what nevertheless remain laboratory
objects; (2) they repeat many of the mistakes of the dominant narrative of
automation history, and neglect some of the complexities of
human-machine labor relationships evidenced by their historical
antecedents; and (3) they represent models for the
continued development of cutting-edge AI technologies, which perpetuate
the vision of an unbroken chain of development toward
ever-more-intelligent artifices. 

