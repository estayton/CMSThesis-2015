\chapter{Cyborgs and Chauffeurs}
\label{app:a}

%% Somewhere in here:
%% bring in the qualitative observations of actually driving, the
%% components we interact with on a daily basis
As we have already established, complete vehicle autonomy is not a
realistic technological goal:  human supervisors will remain involved,
somewhere, somehow, on some timescale. We are also not presently lone
human operators exerting full manual control 
of our vehicles. Our cars are already complex human/machine systems,
and it is time that our viewpoint shifts to recognize this fact. We
are not embarking on a move from total human operation to total
machine operation, and therefore from total human agency to total
machine agency, but a change in the blend of capacities and the
technologies involved in vehicle operations. In other words, we are already cyborgs,
and this view has a number of implications for our public and
political engagement with technology.

The cyborg is primarily, publicly, understood as an artifact of science
fiction. While they may permeate Hollywood movies, most people would
probably say that they are not yet here. But I would argue that we are
all cyborgs, to a greater or lesser extent, in our interactions with
everyday technologies and devices---not only those people augmenting
their bodies with implanted electrodes, or making use of prosthetics
to replace lost limbs, but everyone who interacts with a computer,
carries a cellular phone, or drives a car. This idea has received
serious attention and consideration in philosophical circles, and
deserves to be revisited here as we consider \emph{why} the hybridity
of vehicle operation is not widely recognized, and what can be done to
increase popular appreciation of this operation and the important
questions it reveals about how we design the future. Our world is
becoming a ``cyborg planet,'' to an extent many have not yet
realized\cite[p. 64]{ekbia}.

The idea of the ``cyborg'' was introduced by Manfred Clynes and Nathan
Kline in 1960, as a descriptor for the alteration of human beings to
cope with the conditions of outer space: they would be self-regulating
\emph{cybernetic organisms}\cite[p. 66]{ekbia}. While the creation of
such cyborg astronauts was never attempted outside of the imagination
of science fiction writers, astronauts were indeed fashioned into
cyborgs by their existence within suits and spacecraft with which they
had to interface for their survival, and the cyborg idea has become a
powerful cultural force. As Donna Haraway, one of the most recognized
philosophers of cyborg culture (and cyborg feminism), describes:
``Contemporary science 
fiction is full of cyborgs---creatures simultaneously
animal and machine, who populate worlds ambiguously natural and
crafted''\cite[p. 117]{???-haraway}. Meanwhile, cyborg technologies
abound in real life, from the extreme to the mundane: implants to
achieve increased sensory range and experience, or to trigger orgasm
\cite[p. 64]{ekbia}; implants to monitor vital signs; telerobotic
arms; bots and chatbots; implants to provide neural control of
vehicles\cite[p. 65]{ekbia}. ``Modern medicine is also full of
cyborgs, of couplings between organism
and machine''\cite[p. 117]{???-haraway}. Meanwhile, the cyborg has
infiltrated our methods of production and destruction: ``Modern
production seems like a dream of cyborg colonization work, a dream
that makes the nightmare of Taylorism seem idyllic. And modern war is
a cyborg orgy, coded by C3I,
command-control-communication-intelligence, an \$84 billion item in 1984's US
defence budget''\cite[p. 118]{???-haraway}. 

Haraway's view is not altogether pessimistic, as she sees in the cyborg
the potential site of a new epistemology that does not recognize or
repeat Western binaries and cultural subjugation\cite[p.
  118-121]{???-haraway}. But while one can hardly mention cyborgs without
mentioning Haraway, hers is not the vision I find most relevant for
understanding automated vehicles. Instead, I would like to consider philosopher Andy
Clark's contention that humans are ``(disguised) \emph{natural-born}
cyborgs''\cite[p. 66]{ekbia}. This idea is particularly compelling
because it need make no distinction between the analog and the
digital, the material and the virtual. It is not an artifact of the
present moment, but a fundamental component of human experience. To
Clark, our ``ability to enter into deep and complex relationships with
nonbiological constructs, props, and aids'' is what best explains our
distinctive intelligence\cite[p. 66-67]{ekbia}. This therefore holds
not just for cars, or information and computer technologies, but for
all kinds of technology. Our existence as a technological species is
not new, and human technological augmentation goes back into
prehistory. Our engagement with technological tools and
artifacts---whether the tool is a smartphone, a loom, a wheel, or
fire---is what makes us, fundamentally, cyborg beings. 

This concept is particularly powerful and valuable because it
challenges hundreds of years of assumptions about human beings and
human achievement---even some that are deeply ingrained in the human
factors engineering work I lean on in chapter \ref{chap:3}. We are
possessed by the idea that our technologies change, but that the human
being remains fundamentally the same.\footnote{As Richard Horner put
  it to the test pilots assembled at the SETP banquet in 1957: ``the
  one link in the manned system which we have that improves the least
  in successive generations, is the man himself''\cite[p.
    19]{???-mindell}.} However, this assumption rests on a long held 
dichotomy of human and machine, and the idea that the object of
interest is the purely biological human. If prosthetics and artificial
organs teach us anything, however, it should be that these boundaries
are by no means clear. AI is however heir to a world of such sharp
boundaries, as is human factors engineering: mind and body, subject
and object, nature and culture, science and politics\cite[p.
  327]{ekbia}. Science, engineering, and broader culture ``embody'' and
``regenerate'' these divisions\cite[p. 327]{ekbia}, but they are
fundamentally only true or useful analytical frameworks, so we must
ask whether they continue to serve us well, or whether they should,
despite their attractive simplicity, be retired. Cognitive
anthropology starts to break down these 
divides, with the inspection of a human-machine cognitive system as a
whole, but there is more to be done. Ekbia asks, what happens if we
abandon the outmoded question of whether machines can reach our level
of intelligence or capability, and instead ask how  ``humans and machines
[are] mutually constituted through discursive practices?'' \cite[p.
  328]{ekbia}. 

Ekbia, considering the source of ideologies of AI, sees its roots in
the teachings of Democritus---carried forward by Hobbes
and Descartes---and the separation of the atomist world and our
representations of it.\cite[p. ???]{ekbia} These dualisms are deeply engrained in our
thought patterns, but fail us when thinking about advanced
technologies and their impacts because they favor unrealistic---and
indeed, unreal---totalisms: all human or all machine, fully manual driving or fully
autonomous driving. 
The problem facing AI, according to Lucy Suchman\footnote{Note this is cited in
  \cite[p. 331-332]{ekbia}.}, ``is less that we attribute agency to
computational artifacts than that our language for talking about
agency, whether for persons or artifacts, presupposes a field of
discrete, self-standing entities''\cite[p.
  263]{???-suchman-humanmachinereconfigurations-plansandsituatedactions-2e}.
Instead, we should be asking how does intelligent behavior emerge in the interactions of
humans and machines, refusing to fix \emph{a priori} the category of
the human. Automated vehicles are a product of AI, and are vulnerable
to the same critique. Though Haraway holds that ``By the late 20th
century, our time, a mythic time, we are all chimeras,
theorized, and fabricated hybrids of machine and organism; in short,
we are cyborgs''\cite[p. 118]{???-haraway}, I stand with Clark that we
have long been so. It is only recently however, in part due to the pressure
that AI places on human-machine dualisms, that this state of being has
come to be recognized.

This deserves to be more broadly appreciated within the public sphere.
In fact, I hold that fostering its appreciation is of paramount
importance, one of the great challenges for the public understanding
of technology in the near future. Without an understanding of our
currently hybrid nature, we risk having to choose between blind ludditism
and equally blind technophilia. We come to face questions of our own
obsolescence, and are left with no way out because our way of thinking
about the world no longer corresponds to the world we see and live in
on a day-to-day basis. Either humans should drive, or machines should
drive. The idea that humans drive through machines, or machines drive
via humans, or that human and machine drive in combination, are
practically incomprehensible. They seem like failures---and they are
failures for an ideology of complete technological dominance---when
indeed they may be the roots of our greatest successes. But once we
recognize that we \emph{already} drive through machines, and that they
even already may drive through us---consider blind-spot
warning lights, which are an automated system designed to affect the
human and produce a response, but have no individual capacity to drive
the vehicle---multiple, multifaceted futures of automated vehicle
development open to us.

To return to science fiction, where we began, Phillip K. Dick said in
a 1972 speech: 
\begin{quote}
Someday a human being, named perhaps Fred White, may shoot a robot
named Pete Something-or-Other, which has come out of a General
Electrics factory, and to his surprise see it weep and bleed. And the
dying robot may shoot back and, to its surprise, see a wisp of gray
smoke arise from the electric pump that it supposed was Mr. White's
beating heart. It would be rather a great moment of truth for both of
them.\cite{???-http://boingboing.net/2015/03/10/philip-k-dicks-androids-blu.html}
\end{quote}
While ``Pete Something-or-Other'' does not yet exist, ``Fred White''
is already here. Fully comprehending this is a critical task for the
maintenance of cogent, productive public conversations---and
policymaking---about automated vehicles. We are not building robotic
chauffeurs, but rather designing ourselves, in some fashion, into more
hybrid, more cyborg, entities.

%% Need broader recognition of our hybrid human/machine interactions; 
%% --in fact true for many devices, besides just cars
%% -chauffeurs are already cyborgs, the question is how our hybridity
%% must change


