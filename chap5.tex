\chapter{New Epistemologies and Public Space}

Though the histories of the road and its relationship to the
automobile are very much part of the background of this thesis, little
has so far been said about them. We have treated the current status
quo as a foregone conclusion: that cars would have separate spaces on
the roadway, and generally interact with pedestrians in a controlled
fashion.\footnote{This is not, however, to minimize the human
  interaction component between drivers and pedestrians at crosswalks
  and lights. Such interactions are a critical part of navigating city
  streets, and some interesting research work has been done building
  machines that take them seriously. Of particular note is Nicholas
  Pennycooke's thesis on AEVITA\cite{???-aevita}. Negotiating street
  space with unpredictable, and difficult-to-read, human beings a
  major stumbling block for automated vehicles.} However, there was a
time when city streets were not primarily the venue of the automobile,
but were mixed-use areas where children played, vendors sold goods,
and adults walked, talked, biked, and gathered socially. The street as
a social space predates the concept of the street as a throughway for
motorized transportation specifically, and its changes over the past
hundred or so years are greatly responsible for the inhumanity of the
modern urban landscape, an inhumanity that current urban designers are
interested in reversing. 

Early automobiles and streetcars shared the roads with other social
uses. And this space sharing caused problems for the drivers and
manufacturers of fast moving, dangerous vehicles, which all too
frequently inflicted bodily harm on those people with which they
shared the environment. The regimentation of public space into
crosswalks, where pedestrians are legally protected, and other parts
of the roadway from which pedestrians are supposed to be excluded, is
a direct outcome of an early 20th century campaign to reduce
pedestrian deaths. ``In the early days of the automobile, it was
drivers' job to avoid you, not your job to avoid them,'' describes
Peter Norton, the author of \emph{Fighting Traffic}
\cite{???-http://www.vox.com/2015/1/15/7551873/jaywalking-history}.
But accidents nevertheless occurred, and the victims were primarily
children and the elderly\cite{???-vox}; and the deaths of children,
specifically, came to have a new social meaning that made them
particularly abhorrent\cite{???-zelizer}/ 
After Mary Miner's death in 1903, the driver was almost beaten by a
mob\cite[p. 22]{???-zelizer}. Accidental deaths of children were an
alarming problem, with a significant public response:  mobs attacked
the killers, acts of public mourning memorialized the lost, and a
national safety campaign began to attempt to reduce these
deaths\cite[p. 23]{???-zelizer}. Public outrage cast automobiles as
``frivolous playthings'' or ``pleasure cars''\cite{???-vox}, magnified
by the result in a transformation of the sentimental worth of
children\cite[p. 23]{???-zelizer}. 

This context for public space involved a potential radical change in
the uses and meanings of the street. As Zelizer notes, the city street
was formerly a playground for children, in part out of the necessity
to find some space for play amid crowded tenements\cite[p.
  33]{???-zelizer}. But children did not surrender their street games
easily, and their ``eviction'' from the streets came about via safety
campaigns and ``Americanization programs''---which encouraged
immigrant children to use playgrounds, instead of streets, as their
play spaces---catalyzed by the automotive threat\cite[p.
  35]{???-zelizer}. 

37 deaths emerged as fundamental national problem in 1920s
--early press ``pinned most of the blame on parents''
--modern life ``cannot be retarded to enable heedless children to get
out of the way''
38 street games turned to criminal offenses (1914 ish)
--police arrested many children, but fatalities kept increasing

lochlann-jain
13 Mothers had been charged to keep their children out of ``what had
been their play areas'' to keep the roads clear from automobiles
--safety crusaders blamed MOTHERS
--criminalization of children's street games
--homogenize the road for ``transit''

Cars were not even fully responsible for the creation of paved roads;
bicycles start road paving, create the very roads they must then
compete with cars for space on
http://www.vox.com/2015/3/19/8253035/roads-cyclists-cars-history


1) process by which new technologies become entwined with new legal
principles; new epistemologies?
-historical example (3 p)
-what is the new epistemology needed to make sense of devices like
this in the everyday (because we haven't seen them outside of extreme
environments, except in very weak forms)

10 Dangerous Instrumentalities
---court states that ``Until human agency intervenes, they are usually
harmless''
>>> explicitly human agency here: what do we do about MACHINE AGENCY
>>>DOES THE CAR BECOME A DANGEROUS INSTRUMENTALITY WHEN IT CAN OPERATE ITSELF??!


-the street stands to be re-shaped again, subtly, by the capabilities
of our vehicles (1 p)
--and an understanding (or not) of our own hybridity

new epistemologies may be necessary at many levels:
--personal: in terms of human role and our understanding of our own role
----referring back to HF/SC
--system: in terms of how the road itself may have to change
-----if we want to make more human(e) cities
--legal: in terms of legal framework
----it is important to note that ``responsibility'' is not binary
(which is made clear in these examples), and torts may be distributed
across human and nonhuman actors (operator, owner, seller,
manufacturer, distributors) but distribution of torts has an ethical
and ideological stance (2 p)
>>>>sidenote: problems with the liability framework: expensive for the
litigant to get expert witnesses; harder to determine whose fault it
is and the human tends to get scapegoated (see industrial accidents,
TMI, or oil rig disasters)
>>>>>What can we actually ask the human to do and hold them to task
for? (3 p)
Sidenote: important to note that issues of litigation for full autonomy are
not even part of the debate yet, we just aren't at the stage where it
is worth discussing

``Anyone who focuses solely on the technology has not yet grasped how
autonomous driving will change our society,'' says Dr Dieter Zetsche,
chairman of Daimler AG
\cite{???-http://www.slashgear.com/how-mercedes-f-015-self-driving-car-is-shaping-smart-cities-20374602/}
As I have described, these vehicles sit in a much broader network of
social relationships. The city and the street have changed before, and
will change again, to accomodate new technologies, assuming sufficient
social and economic pressure to catalyze that change. Technologies do
not alone bring about those changes, but their presence, availability,
and market viability provide incentives for groups to encourage broad
social and infrastructural changes. The automated vehicle represents a
another possible nexus for change, but how the city will be re-shaped
is an open question. As I held in chapter \ref{chap:2}, however, that
change should not be arbitrary, at the whims of a certain set of
producers and organizations, but part of a large-scale system design
strategy to address the needs of all transportation users.
