\chapter{New Epistemologies and Public Space}
\label{app:b}

Though the histories of the road and its relationship to the
automobile are very much part of the background of this thesis, little
has so far been said about them. We have treated the current status
quo as a foregone conclusion: that cars would have separate spaces on
the roadway, and generally interact with pedestrians in a controlled
fashion.\footnote{This is not, however, to minimize the human
  interaction component between drivers and pedestrians at crosswalks
  and lights. Such interactions are a critical part of navigating city
  streets, and some interesting research work has been done building
  machines that take them seriously. Of particular note is Nicholas
  Pennycooke's thesis on AEVITA\cite{???-aevita}. Negotiating street
  space with unpredictable and difficult-to-read human beings is a
  major stumbling block for automated vehicles.} However, there was a
time when city streets were not primarily the venue of the automobile,
but were mixed-use areas where children played, vendors sold goods,
and adults walked, talked, biked, and gathered socially. The street as
a social space predates the concept of the street as a throughway for
motorized transportation specifically, and its changes over the past
hundred or so years are greatly responsible for the inhumanity of the
modern urban landscape, an inhumanity that current urban designers are
interested in reversing. 

Early automobiles and streetcars shared the roads with other social
uses. And this space sharing caused problems for the drivers and
manufacturers of fast moving, dangerous vehicles, which all too
frequently inflicted bodily harm on those people with which they
shared the environment. The regimentation of public space into
crosswalks, where pedestrians are legally protected, and other parts
of the roadway from which pedestrians are supposed to be excluded, is
a direct outcome of an early 20th century campaign to reduce
pedestrian deaths. ``In the early days of the automobile, it was
drivers' job to avoid you, not your job to avoid them,'' describes
Peter Norton, the author of \emph{Fighting Traffic}
\cite{???-http://www.vox.com/2015/1/15/7551873/jaywalking-history}.
But accidents nevertheless occurred, and the victims were primarily
children and the elderly\cite{???-vox}; and the deaths of children,
specifically, came to have a new social meaning that made them
particularly abhorrent\cite{???-zelizer}/ 
After Mary Miner's death in 1903, the driver was almost beaten by a
mob\cite[p. 22]{???-zelizer}. Accidental deaths of children were an
alarming problem, with a significant public response:  mobs attacked
the killers, acts of public mourning memorialized the lost, and a
national safety campaign began to attempt to reduce these
deaths\cite[p. 23]{???-zelizer}. Public outrage cast automobiles as
``frivolous playthings'' or ``pleasure cars''\cite{???-vox}, magnified
by a transformation in the sentimental worth of
children\cite[p. 23]{???-zelizer}. 

This contest for public space involved a potentially radical change in
the uses and meanings of the street. As Zelizer notes, the city street
was formerly a playground for children, in part out of the necessity
to find some space for play amid crowded tenements\cite[p.
  33]{???-zelizer}. But children did not surrender their street games
easily, and their ``eviction'' from the streets came about via safety
campaigns and ``Americanization programs''---which encouraged
immigrant children to use playgrounds, instead of streets, as their
play spaces---catalyzed by the automotive threat\cite[p.
  35]{???-zelizer}. But the deaths of children at the hands of
automobiles were not solely placed on the shoulders of drivers. As the
death rate became a national crisis, the
early press ``pinned most of the blame on parents'': modern life, it
was said, ``cannot be retarded to enable heedless children to get out
of the way'' \cite[p. 37]{???-zelizer}. Street games were turned into
criminal offenses by around 1914, and the police arrested many
children, but fatalities kept increasing nonetheless\cite[p.
  38]{???-zelizer}. Families were called upon to take a greater role
in protecting children from the perils of the street; mothers were
charged with keeping their children out of ``what had been their play
areas'' in order to keep the roads clear for automobiles, and safety
crusaders tended to blame mothers for the deaths of their own
children\cite[p. 13]{???-lochlannjain}. The homogenization of the road
for ``transit'' \cite[p. 13]{???_lochlannjain} involved not only
family pressures but a concerted legal and public relations campaign. Automakers
and dealers, concerned by public pressure for speed governors on
automobiles, pushed for stricter pedestrian controls across the country, and
auto industry groups exerted significant pressure on the creation of
the 1928 Model Municipal Traffic Ordinance in order to build a law
that was friendly to automobiles\cite{???-vox}. To attempt to
compensate for laws that were rarely followed and enforced,
auto-friendly groups worked to change the public dialogue. Articles
ghostwritten by the National Automobile Chamber of Commerce shifted
the blame for traffic accidents to pedestrians; the AAA sponsored
safety campaigns in schools; and police and citizens were called upon
to shame transgressors in order to set new public standards---even the
name for the infraction, ``jay-walking,'' was intended to create
public opprobrium for the supposed ``hicks'' who did not know how to
behave in cities, and to shift blame to them \cite{???-vox}. This
change in the way streets were utilized entailed a large societal
shift motivated by a new technology, but it would be wrong to say it
was caused by the automobile. Instead, automobility was an enabling
force, which provided auto companies and their supporters the impetus
to shift public standards in a particular direction, and to shape the
street to their own advantage.

Cars were not, however, the only technological device responsible for
the beginning of the street as we know it.
Automobiles were not fundamentally a prerequisite for the creation of the paved
roads that later helped them flourish. Instead, the bicycle marks the
beginning of the ``good roads'' movement, and the push for ubiquitous
road
paving\cite{???-http://www.vox.com/2015/3/19/8253035/roads-cyclists-cars-history}.
Bicycles and the social groups that championed their use--aided by
asphalt manufacturers and other special interests who saw money to be made---created the
very roads that bicycles then had to compete with cars for space on
\cite{???-vox2}. Many of the primary riders who had succeeded in lobbying for
the 1916 Federal Aid Road Act, however, switched over to pioneer the
automobile as its popularity rose in the early 20th century. Between
the technologies of bicycle and automobile, the American roadway
underwent a period of radical change that crystallized the system into
something relatively close to its current form. These historical
examples provide some insight into the process by which new
transportation technologies become entwined with new social standards
and legal principles. Technological epistemologies are deeply involved
in this process: What is a road for? What is a vehicle's
proper role? We know that a road is intended for driving precisely
because we have been taught according to a social code that was
designed to foster automobility.

What is the new epistemology needed to make sense of devices like
automated vehicles in their every-day existence? We are so far not
used to encountering such devices
except in weak forms such as cruise control and automated
transmissions. How should we regard them?
Automobiles were long subject to disputes over their nature: were they
fundamentally safe vehicles misused by people, or fundamentally
dangerous technologies that required careful licensing and use to
make safe? In one such battle over cars as ``dangerous
instrumentalities,'' a court held that ``Until human agency intervenes, they are usually
harmless''\cite[p. 10]{???-lochlannjain}. Such legal statements
explicitly involve human agency, but machine agency presents the
possibility that such claims could be obsolesced. Are our intuitions
affected when and if automated vehicles can be said to ``operate
themselves'' for some periods of time, nonwithstanding the centrality
of human supervision to operations over the long term?
And what new re-shapings of public space will happen as part of the 
popularization of these technologies? Will these vehicles increase the
segregation of the road space, and require new lanes that are even
more insulated from pedestrians? Or will faster reaction times and
always-attentive automated safety features foster tighter, mixed-use
environments? (And additionally, as the concept of replacing traffic
lights with a slots system, as described by Paolo Santi, should have
us asking, which types and classes of users will the new environments
and standards benefit? Will environments improve for those with access
to technology, while being degraded for those without?) 

How we ask and answer such questions depends on our understanding (or
not) of our own hybridity with technology. New epistemologies may be
necessary at multiple levels. At the level of the individual, what is
the role or status of the human, and that of the machine? How do we
classify, and make sense of, users and devices, and their
relationships of supervision and co-operation? At the level of the
system, what new understandings of the role and purpose of the road
itself are needed if we want to make use of automated technologies to
make cities more humane? At the legal level, what is the status of a
vehicle acting autonomously on a certain time scale? And how can
responsibility be apportioned between supervisors and the supervised?
It is important to note that ``responsibility'' is not binary, nor
should it be an all-or-nothing prospect. Legal scholars note that
torts may already be distributed across human and nonhuman actors
(operator, owner, seller, manufacturer, distributors) in legal
rulings, but distribution of torts has an ethical
and ideological stance\cite{???-suemycar,etc}. The distribution of
torts may vary from situation to situation, given the capacities of
the vehicle and the driver\cite{???-disableddriverexample}, but we
should not presume that all the financial responsibility must rest on
one actor alone.\footnote{There are, however, problems with the using
  the current liability framework to deal with blame and
  responsibility in highly-automated systems. It is expensive for the
litigant to get expert witnesses to testify as to the apportionment of
responsibility between human and machine, and the human tends to be
scapegoated all too often in industrial accidents involving human
supervisors interacting with automation.} 
The question of what we can actually ask the human to do, and hold them to task
for, is a critical question both for automation system design and the
legal handling of cases involving such automation.
%% Sidenote: important to note that issues of litigation for full autonomy are
%% not even part of the debate yet, we just aren't at the stage where it
%% is worth discussing

And questions of the appropriate role of the human being involve not
only the supervisor, within the vehicle or in a remote data center,
but other users in the environment. If streets must be re-made in
order to make certain technological configurations viable, is that
re-making something the public is willing to accept? And who will
benefit from it? As Dieter Zetsche, chairman of Daimler AG describes,
``anyone who focuses solely on the technology has not yet grasped how 
autonomous driving will change our
society''\cite{???-http://www.slashgear.com/how-mercedes-f-015-self-driving-car-is-shaping-smart-cities-20374602/}.
But as we have found in the past, changes that favor the users or
manufacturers of automated vehicles may not favor other users of city
spaces, who may find their freedoms foreclosed upon. This history of
the re-making of the city and street, for the bicycle and the
automobile, is part of why I hold that vehicle automation is not an
independent factor to be maximized, but a variable that is firmly
intertwined with the design of the whole transport system.
As I have described, these vehicles sit in a much broader network of
social relationships. The city and the street have changed before, and
will change again, to accomodate new technologies, assuming sufficient
social and economic pressure to catalyze that change. Technologies do
not alone bring about those changes, but their presence, availability,
and market viability provide incentives for groups to encourage broad
social and infrastructural changes. The automated vehicle represents
another possible nexus for change, but how the city will be re-shaped
is an open question. Problematically, the automated car---like the
``smart city'' which has already been critiqued in this
way\cite{???}---presents the opportunity for successful technology
companies to wrest greater control over everyday life, to worm their
way more deeply into our existence and thereby make themselves
indispensable. Automation can be a tool to enable one group of
people---technologists running multinational organizations---at the
expense of the rest of us.\footnote{This is a relatively conventional and
predictable corporate strategy. Opposing it will not be easy, of
course, but it
is at least helpful to be clear that these are the stakes involved.}
But it also does not have to be employed toward those ideological
ends; given sufficient willpower, it can be made to serve others.
As I held in chapter \ref{chap:2}, changes to the city
should not be arbitrary, at the whims of a certain set of
producers and organizations, but part of a large-scale system design
strategy to address the needs of all transportation users. To the list
of major questions facing us as a result of this technology, we should
add: how do we make transportation technology and urban planning sensitive to social
justice and responsive to a broad base of citizens, and inoculate it
against takeover by corporate entities that have only their own best interests
in mind?
