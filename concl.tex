\chapter{Conclusion: Driving the Future}
\label{chap:4}

%%Resisting Teleology / Conclusion (14 p)

%% A forward march of
%% technology does not alone make ``full'' autonomy acceptable or viable,

Our journey through ideologies of automation and driverless vehicle
development has taken us from an explication of the dominant narrative
of the ``self-driving'' vehicle and its histories, through an examination of what it
might mean for our society, to an investigation of an alternative
paradigm for thinking about automation. What
should be clear from this thesis, but which must be made clearer in the
public discourse about increasingly automated technologies, is that a
movement from human systems, through hybrid systems, to fully
autonomous systems is not inevitable. It is not a
requirement of technological progress, but one narrative among many
which depends on technological, and often 
infrastructural, progress as well as a mischaracterization of what
``full'' autonomy is. However, this narrative
presents a convenient cover story for the economic objectives of
organizations involved in automating the car.

\section{Reshaping the Road: A Last Lesson in History}


%% ``In the early days of the automobile, it was
%% drivers' job to avoid you, not your job to avoid them,'' describes
%% Peter Norton, the author of \emph{Fighting Traffic}
%% \cite{voxNorton}.

We often take for granted that cars have separate spaces on
the roadway and interact with pedestrians in a controlled
fashion.\footnote{This is not, however, to minimize the human
  interaction component between drivers and pedestrians at crosswalks
  and lights. Such interactions are a critical part of navigating city
  streets, and some interesting research work has been done building
  machines that take them seriously \cite{aevita}.} However, there was a
time when city streets were not primarily the venue of the automobile,
but were mixed-use areas where children played, vendors sold goods,
and adults walked, talked, biked, and gathered socially. 
This caused problems for the drivers and
manufacturers of fast moving, dangerous vehicles, which all too
frequently inflicted bodily harm on those people with whom they
shared the environment. The regimentation of public space into
crosswalks, where pedestrians are legally protected, and other parts
of the roadway from which pedestrians are supposed to be excluded, is
a direct outcome of an early 20th century campaign to reduce
pedestrian deaths. But accidents nevertheless occurred, and the victims were primarily
children and youths \cite[p. 11]{nortonFighting}; the deaths of
children came to have a new social meaning that made them 
particularly abhorrent.
After Mary Miner's death in 1903, the streetcar driver ``had a narrow escape
from violence at the hands of a mob estimated by the police . . . to
have been 3,000 strong'' \cite[p. 22]{zelizer}. Accidental deaths of children were an
alarming problem, with a significant public response:  mobs attacked
the killers, acts of public mourning memorialized the lost, and a
national safety campaign began to attempt to reduce these
deaths \cite[p. 23]{zelizer}. Public outrage cast automobiles as
``frivolous playthings'' or ``pleasure cars'' \cite[p. 12]{nortonFighting}, magnified
by a transformation in the sentimental worth of
children. 

%% This contest for public space involved a potentially radical change in
%% the uses and meanings of the street. As Zelizer notes, the city street
%% was formerly a playground for children, in part out of the necessity
%% to find some space for play amid crowded tenements \cite[p.
%%   33]{zelizer}. But children did not surrender their street games
%% easily, and their ``eviction'' from the streets came about via safety
%% campaigns and ``Americanization programs''---which encouraged
%% immigrant children to use playgrounds, instead of streets, as their
%% play spaces---catalyzed by the automotive threat \cite[p.
%%   35]{zelizer}.

%% Families were called upon to take a greater role
%% in protecting children from the perils of the street; mothers were
%% charged with keeping their children out of ``what had been their play
%% areas'' in order to keep the roads clear for automobiles, and safety
%% crusaders tended to blame mothers for the deaths of their own
%% children \cite[p. 73]{lochlannjain}. 

%% . Automakers 
%% and dealers, concerned by public pressure supporting speed governors on
%% automobiles, pushed for stricter pedestrian controls across the country, and
%% auto industry groups exerted significant pressure on the creation of
%% the 1928 Model Municipal Traffic Ordinance in order to build a law
%% that was friendly to automobiles \cite{voxNorton}.

%% This
%% change in the way streets were utilized entailed a large societal
%% shift motivated by a new technology, but it would be wrong to say it
%% was caused by the automobile. Instead, a

In an attempt to change this, street games were turned into
criminal offenses by around 1914, but fatalities kept increasing nonetheless \cite[p.
  38]{zelizer}. The deaths of children at the hands of
automobiles were not solely placed on the shoulders of drivers. As the
death rate became a national crisis, the
press ``pinned most of the blame on parents'': modern life, it
was said, ``cannot be retarded to enable heedless children to get out
of the way'' \cite[p. 37]{zelizer}. This was not by accident.\footnote{Articles
ghostwritten by the National Automobile Chamber of Commerce shifted
the blame for traffic accidents to pedestrians; the AAA sponsored
safety campaigns in schools; and police and citizens were called upon
to shame transgressors in order to set new public standards---even the
name for the infraction, ``jay-walking,'' was intended to create
public opprobrium for the supposed ``hicks'' who did not know how to
behave in cities, and to shift blame to them \cite{voxNorton}.} The homogenization of the road
for ``transit'' \cite[p. 73]{lochlannjain} involved not only
pressures on families to keep their children out of the street, but a
concerted legal and public relations campaign for automobile-friendly
traffic laws organized by automakers concerned about automotive speed
governors \cite{voxNorton}. To attempt to
compensate for laws that were rarely followed and enforced,
auto-friendly groups worked to change the public dialogue.
Automobility was an enabling 
force, which provided auto companies and their supporters the impetus
to shift public standards in a particular direction, and to shape the
street to their own advantage.
%% \footnote{Cars were not, however, the
%%   only technological device responsible for 
%% the beginning of the street as we know it. Instead, the bicycle marks the
%% beginning of the ``good roads'' movement, and the push for ubiquitous
%% road
%% paving \cite{voxCycle}.
%% Bicycles and the social groups that championed their use--aided by
%% asphalt manufacturers and other special interests who saw money to be
%% made---created the 
%% very roads that bicycles then had to compete with cars for space on
%% \cite{voxCycle}. Many of the primary riders who had succeeded in lobbying for
%% the 1916 Federal Aid Road Act, however, switched over to pioneer the
%% automobile as its popularity rose in the early 20th century.} 

%% The street as
%% a social space predates the concept of the street as a throughway for
%% motorized transportation specifically, and its changes over the past
%% hundred or so years are greatly responsible for the inhumanity of the
%% modern urban landscape, an inhumanity that current urban designers are
%% interested in reversing. 

%% . What is the new epistemology needed
%% to make sense of devices like 
%% automated vehicles in their everyday existence? We are so far not
%% used to encountering such devices
%% except in weak forms such as cruise control and automated
%% transmissions.


%% affected when and if automated vehicles can be said to

New technologies become entwined with new social standards
and legal principles, and technological narratives and epistemologies are deeply involved
in this process: What is a road for? What is a vehicle's
proper role? We know that a road is intended for driving precisely
because we have been taught according to a social code that was
designed to foster automobility---greatly responsible for the
inhumanity of the modern city that urban designers are interested
in reversing. How should we regard automated vehicles in our environment?
Automobiles were long subject to disputes over their nature: were they
fundamentally safe vehicles misused by people, or fundamentally
dangerous technologies that required careful licensing and use to
make safe?\footnote{In one such battle over cars as ``dangerous
instrumentalities,'' a court held that ``Until human agency
intervenes, they are usually 
harmless'' \cite[p. 70]{lochlannjain}. This raises questions about vehicles that can
be said to ``operate
themselves'' for some periods of time.} Machine agency, even under
supervision, presents the 
possibility that our laws and intuitions must change.
And what new re-shapings of public space will happen as part of the 
popularization of these technologies? Will these vehicles increase the
segregation of the road space, and require new lanes that are even
more insulated from pedestrians? Or will faster reaction times and
always-attentive automated safety features foster tighter, mixed-use
environments? Which types and classes of users will the new environments
and standards benefit? Will environments improve for those with access
to technology, while being degraded for those without?

%% It is important to note that ``responsibility''
%%   is not binary, nor 
%% should it be an all-or-nothing prospect. 
%% but distribution of torts has an ethical
%% and ideological stance 

New epistemologies may be
necessary at multiple levels. At the level of the individual, what is
the role or status of the human and machine? How do we
classify and make sense of users and devices, and their
relationships of supervision and co-operation? At the level of the
system, what new understandings of the role and purpose of the road
itself are needed if we want to make use of automated technologies to
make cities more humane? At the legal level, what is the status of a
vehicle acting autonomously on a certain time scale? And how can
responsibility be apportioned between supervisors and the
supervised?\footnote{Legal scholars note that
torts may already be distributed across human and nonhuman actors
(operator, owner, seller, manufacturer, distributors) in legal
rulings \cite{proximityLiability}. The distribution of
torts to different actors in the system may vary from situation to
situation, given the capacities of 
the vehicle and the driver \cite[p. 267]{suemycar}. But
the current liability framework has problems, including the
potential to scapegoat the human being, as well as the expense and difficulty
of pursuing litigation.}
The question of what we can actually ask the human to do, and hold them to task
for, is critical both for automation system design and the
legal handling of cases involving such automation. And it requires a
broader recognition of our networked, 
cyborg nature.
%% Sidenote: important to note that issues of litigation for full autonomy are
%% not even part of the debate yet, we just aren't at the stage where it
%% is worth discussing

%% There are, however, problems with the using
%%   the current liability framework to deal with blame and
%%   responsibility in highly-automated systems. It is expensive for the
%% litigant to get expert witnesses to testify as to the apportionment of
%% responsibility between human and machine, and the human tends to be
%% scapegoated all too often in industrial accidents involving human
%% supervisors interacting with automation.

\section{Cyborgs In Traffic}

%% How we ask and answer questions about automation depends on our understanding (or
%% not) of our own hybridity with technology. T
%% Our ways of thinking
%% about the world no longer correspond to the world we see and live in
%% on a day-to-day basis
%% Our technological hybridity deserves
%% to be more broadly appreciated within the public sphere.

%% Without an understanding of our
%% currently hybrid nature, we risk having to choose between blind ludditism
%% and equally blind technophilia.

%% \footnote{They seem like failures---and they are
%% failures for an ideology of complete technological dominance---when
%% indeed they may be the roots of our greatest successes.}

Without a recognition of our technological
hybridity, we are caught in a false choice:
either humans should drive (blind ludditism), or machines should 
drive (equally blind technophilia). The idea that humans drive through
machines, or machines drive 
via humans, or that human and machine drive in combination, are
practically incomprehensible. But once we
recognize that we \emph{already} drive through machines, and that they may
already drive through us,\footnote{Consider blind-spot
warning lights, which are an automated system designed to affect the
human and produce a response, but have no independent capacity to drive
the vehicle.} multiple, multifaceted futures of automated vehicle
development open to us. We are not building robotic
chauffeurs, but rather designing ourselves, in some fashion, into more
hybrid, more cyborg, entities. Fostering an appreciation of this is paramount, one of the
  great challenges for the public understanding 
of technology in the near future.\footnote{To return to science fiction, Phillip K. Dick said in
a 1972 speech: ``Someday a human being, named perhaps Fred White, may shoot a robot
named Pete Something-or-Other, which has come out of a General
Electrics factory, and to his surprise see it weep and bleed. And the
dying robot may shoot back and, to its surprise, see a wisp of gray
smoke arise from the electric pump that it supposed was Mr. White's
beating heart. It would be rather a great moment of truth for both of
them'' \cite{androidHuman}.
While ``Pete Something-or-Other'' does not yet exist, ``Fred White''
is already here. Fully comprehending this is a critical task for the
maintenance of cogent, productive public conversations---and
policymaking---about automated vehicles and automated technologies in general.}

%%TODO (DONE) insert cyborg ref briefly
%While they may permeate Hollywood movies, most people would probably
%say that they are not yet here. 

%% The cyborg is publicly understood as an artifact of science
%% fiction. But I would argue that we are
%% all cyborgs, to a greater or lesser extent, in our interactions with
%% everyday technologies and devices---not only those people augmenting
%% their bodies with implanted electrodes, or making use of prosthetics
%% to replace lost limbs, but everyone who interacts with a computer,
%% carries a cellular phone, or drives a car. This idea has received
%% serious attention and consideration in philosophical circles, and
%% deserves to be revisited here as we consider \emph{why} the hybridity
%% of vehicle operation is not widely recognized, and what can be done to
%% increase popular appreciation of this operation and the important
%% questions it reveals about how we design the future. Our world is
%% becoming a ``cyborg planet,'' to an extent many have not yet
%% realized \cite[p. 64]{ekbia}.

%% The idea of the ``cyborg'' was introduced by Manfred Clynes and Nathan
%% Kline in 1960 \cite{clyneskline}, as a descriptor for the alteration
%% of human beings to 
%% cope with the conditions of outer space: they would be self-regulating
%% \emph{cybernetic organisms} \cite[p. 66]{ekbia}. While the creation of
%% such cyborg astronauts was never attempted outside of the imagination
%% of science fiction writers, astronauts were indeed fashioned into
%% cyborgs by their existence within suits and spacecraft with which they
%% had to interface for their survival, and the cyborg idea has become a
%% powerful cultural force. Meanwhile, cyborg technologies
%% abound in real life, from the extreme to the mundane: implants to
%% achieve increased sensory range and experience, or to trigger orgasm
%% \cite[p. 64]{ekbia}; implants to monitor vital signs; telerobotic
%% arms; bots and chatbots; implants to provide neural control of
%% vehicles \cite[p. 65]{ekbia}. 

%% ``Modern medicine is also full of
%% cyborgs, of couplings between organism
%% and machine''\cite[p. 117]{???-haraway}. Meanwhile, the cyborg has
%% infiltrated our methods of production and destruction: ``Modern
%% production seems like a dream of cyborg colonization work, a dream
%% that makes the nightmare of Taylorism seem idyllic. And modern war is
%% a cyborg orgy, coded by C3I,
%% command-control-communication-intelligence, an \$84 billion item in 1984's US
%% defence budget''\cite[p. 118]{???-haraway}. 
%% Haraway's view is not altogether pessimistic, as she sees in the cyborg
%% the potential site of a new epistemology that does not recognize or
%% repeat Western binaries and cultural subjugation\cite[p.
%%   118-121]{???-haraway}. But while one can hardly mention cyborgs without
%% mentioning Haraway, hers is not the vision I find most relevant for
%% understanding automated vehicles. Instead, I would like to consider

%% Our existence as a technological species is
%% not new, and h

Philosopher Andy
Clark---drawing from a deep intellectual history that began in 1960 \cite{clyneskline}---contends that humans are ``\emph{natural-born}
cyborgs'' \cite[p. 6]{clarkNBC}. Cyborgs are a powerful cultural
force,\footnote{As Donna Haraway, one of the most recognized
philosophers of cyborg culture, describes:
``Contemporary science 
fiction is full of cyborgs---creatures simultaneously
animal and machine, who populate worlds ambiguously natural and
crafted'' \cite[p. 117]{haraway}.} but rarely taken seriously as a state
of existence despite the proliferation of cyborg technologies \cite[p.
  65]{ekbia}. Clark's idea is particularly compelling
because it need make no distinction between the analog and the
digital, the material and the virtual. A cyborg nature is not an artifact of the
present moment, but a fundamental component of human experience. To
Clark, our ``ability to enter into deep and complex relationships with
nonbiological constructs, props, and aids'' is what best explains our
distinctive intelligence \cite[p. 5]{clarkNBC}.\footnote{While Ed
  Hutchins looks at these tools and stresses the importance of
  analyzing at a system level, Clark suggests these tools ``are best
  conceived as proper parts of the computational apparatus that
  constitutes our minds'' \cite[p. 6]{clarkNBC}. These conflicting views are
  united in their careful consideration of the properties of nonhuman artifacts,
  and both have something to add to nuanced examination of automation
  systems.} This therefore holds 
not just for cars, or information and computer technologies, but for
all kinds of technology. Human technological augmentation goes back into
prehistory: our engagement with technological tools and
artifacts---whether the tool is a smartphone, a loom, a wheel, or
fire---is what makes us, fundamentally, cyborg beings.
%% \footnote{Though Haraway holds that ``By the late 20th
%% century, our time, a mythic time, we are all chimeras,
%% theorized, and fabricated hybrids of machine and organism; in short,
%% we are cyborgs'' \cite[p. 118]{haraway}, I stand with Clark that we
%% have long been so. It is only recently however, in part due to the pressure
%% that AI places on human-machine dualisms, that this state of being has
%% come to be recognized.}

%% \footnote{As Richard Horner put
%%   it to the test pilots assembled at the SETP banquet in 1957 (see
%%   chapter \ref{chap:3}): ``the
%%   one link in the manned system which we have that improves the least
%%   in successive generations, is the man himself'' \cite[p.
%%     19]{DM}.}

%% AI is however heir to a world of such sharp
%% boundaries: mind and body, subject
%% and object, nature and culture, science and politics. Science, engineering, and broader culture ``embody'' and
%% ``regenerate'' these divisions , but they are
%% fundamentally only true or useful analytical frameworks, so we must
%% ask whether they continue to serve us well, or whether they should,
%% despite their attractive simplicity, be retired.\footnote{Ekbia,
%%   considering the source of ideologies of AI, sees its roots in 
%% the teachings of Democritus---carried forward by Hobbes
%% and Descartes---and the separation of the atomist world and our
%% representations of it \cite[p. 331]{ekbia}. These dualisms are deeply
%% engrained in our 
%% thought patterns, but fail us when thinking about advanced
%% technologies and their impacts because they favor unrealistic---and
%% indeed, unreal---totalisms: all human or all machine, fully manual driving or fully
%% autonomous driving.}
%% If prosthetics and artificial
%% organs teach us anything, however, it should be that these boundaries
%% are by no means clear.

This concept is particularly powerful and valuable because it
challenges hundreds of years of assumptions about human beings and
human achievement---even some that are deeply ingrained in supervisory
control. We are
possessed by the idea that technologies change, but that the human
remains fundamentally the same. However, this assumption rests on a long held 
dichotomy of human and machine, and the idea that the object of
interest is the purely biological human \cite[p. 327, 331]{ekbia}. Cognitive
anthropology starts to break down these 
divides, with the inspection of a human-machine cognitive system as a
whole, but there is more to be done. Ekbia asks, what happens if we
abandon the outmoded question of whether machines can reach our level
of intelligence or capability, and instead ask how  ``humans and machines
[are] mutually constituted through discursive practices'' \cite[p.
  328]{ekbia}?

%% TODO: FIX THIS FOOTNOTE, make it clearer how it fits

%% \footnote{Note that
%%   Ekbia mobilizes this statement as well in his arguments
%%   \cite[p. 331-332]{ekbia}.}

The problem facing AI, according to Lucy Suchman, ``is less that we attribute agency to
computational artifacts than that our language for talking about
agency, whether for persons or artifacts, presupposes a field of
discrete, self-standing entities'' \cite[p. 263]{SuchmanPlans}.
Instead, we should be asking how intelligent behavior emerges in
the interactions of
humans and machines---involving networked, rather than individual,
subjectivities---refusing to fix \emph{a priori} the category of 
the human.

%%\section{The Shape of Automation}

Questions of the appropriate role of the human being involve not
only the supervisor, within the vehicle or in a remote data center,
but other users in the environment. If streets must be remade in
order to make certain technological configurations viable, is that
remaking something the public is willing to accept? And who will
benefit from it? As Dieter Zetsche, chairman of Daimler AG describes,
``anyone who focuses solely on the technology has not yet grasped how 
autonomous driving will change our
society'' \cite{slashgearDavies}. Changes that favor the users or
manufacturers of automated vehicles may not favor other users of city
spaces, who may find their freedoms foreclosed upon. The history of
the remaking of the city and street is part of why I hold that
vehicle automation is not an 
independent factor to be maximized, but a variable that is firmly
intertwined with the design of the whole transport system.
As I have described, these vehicles sit in a much broader network of
social relationships: the city and the street have changed before, and
will change again, to accomodate new technologies, assuming sufficient
social and economic pressure to catalyze that change.\footnote{Technologies do
not alone bring about those changes, but their presence, availability,
and market viability provide incentives for groups to encourage broad
social and infrastructural changes.} The automated vehicle represents
another possible nexus for change, but how the city will be reshaped
is an open question. 

%% And as I find myself repeating continually, because it is such a vital
%% recognition, complete autonomy is ultimately a foreign concept,
%% something no current automated vehicle plans point toward. 

%% Vehicles serve human needs, 
%% are constructed and operated by humans or organizations made of
%% humans, and are ultimately paid for by humans.
Until
vehicles construct themselves, monitor themselves, pay for themselves,
and are intended only to drive themselves around---a strange world
indeed---human oversight is inescapable, and complete autonomy
impossible. While certain types of
labor may be eliminated, and certain laborers marginalized, a
system-level view of automation continues to uncover vast amounts of
human labor---perhaps at the periphery of the car as an individual
object, but unavoidably implicated in the day-to-day reality of
vehicle operations. The method and timescale of automated
operation may greatly impact 
public acceptance, to the point that certain types of human
involvement may be required to produce marketable
technologies.\footnote{Travel sickness is another potential barrier to
  public acceptance. A
  recent University of Michigan study suggests car sickness could be a
  problem for automated vehicle systems,  significantly reducing their
  convenience factor
  \cite{SivakSchoettleSick}. The authors suggest fixes, including
  maximizing passengers' view with large windows and forcing them to
  face forward---but a
  PRT researcher I spoke to described that his group
  had found minimizing or eliminating forward view made riders in
  automated PRT systems
  most comfortable.
  Certain levels of human involvement could be an alternative way to ameliorate travel sickness
  issues.} Though
this may be distasteful to some technologists invested in furthering
technological solutions to social problems, it is not fundamentally
a failure. It is a societal choice to which engineering solutions
should be pliable.

%% ---we are not necessarily making
%% things easier on ourselves by choosing such a solution compared to a
%% more fully automated one, something even the Department of Defense,
%% hardly an organization of technophobes, finds it important to recognize--

Many examples of hybrid systems exist, and joint-cognitive systems
design presents an alternative way of looking at problems, and valuing
human and machine contributions to their solution, that opens the way
for other futures. As we have seen, these approaches and
perspectives are not those of ``mere'' luddites or humanists on the
margins of engineering practice, but of an important segment of
engineers themselves who are vital to the design and construction of
successful real-world systems. The cyborg human-machine system is a
very difficult engineering problem, but even the Department of Defense
recognizes the value of serious attention to hybrid control for 
systems that are tasked with keeping us safe, whether in space, in the
air, or on the ground. In applying this approach, however, it is
important to keep in mind the extent to which we are re-making
ourselves into managers or into machine tenders:  Are we maintaining a sense of
creative agency? Or becoming mere tools of larger organizations? This
is especially important as we will use our  
vehicles every day, and cannot afford to alienate ourselves from them.
A future in which our transportation system is indifferent to us is
not a productive future to be lauded, but a destructive one to be
avoided. 

%% This discussion has tried to frame the overlooked histories involved in
%% notions of autonomy and automation, the ideologies flowing into the
%% dreams of driverless cars and attendant approaches to system
%% design.

 %% There is no reason that commonly accepted ideas
%% about complete vehicle automation are necessarily right---and that
%% partial, or function-by-function automation, for example, is simply a
%% stop-gap measure.
When we use factory automation, artificial intelligence, ``NASA
engineering,'' or autopilot to frame ideas about self-driving
vehicles, we carry forward certain ideas about these technologies that
narrow and constrain our vision, irrespective of actual historical
developments. We risk 
thinking about autopilots as a small step away from being full vehicle
operators, rather than as tools used by vehicle operators to support
their needs. We risk thinking about factory automation as a
teleological process toward the elimination of labor, rather than a
deeply contingent process that changes the forms of certain kinds of
labor into other forms, or shifts that labor in time and space. We
risk thinking about NASA as an organization responsible for
engineering highly automated, redundant systems, rather than an
organization full of people whose jobs include constant monitoring,
supervision, and interactions with the same automated systems. We may
consider new automation through the lens of the artificial, of
building systems to replace people, or through a hybrid lens, building
systems to support humans in their tasks. Both ways of considering
automated systems can often be applied to the same technological
artifact, but result in very different ways of thinking.\footnote{A
speech-to-text system can be seen as either a technology to transcribe 
the human voice into text, instead of using a person, or a tool to assist a
human with transcription. But the choice of perspective will likely do
much to change how such a technology is designed. I would hypothesize that
the focus on what computerized personal assistants can do, for
example, as opposed to the process of human and machine jointly
carrying out a task and accomplishing something together, does much to
explain the repeated failure of personal assistant technology to gain
users.}  Throughout, I have emphasized the contingency of current
automated vehicle plans on specific ideas about the appropriate and
necessary role of technology, and stressed the presence of other ways
to regard vehicle automation technology that could bring about
different futures. In contrast to the simplistic depictions which form
the visible picture of this technology today 
for those not involved in its engineering---of pro-and-con, for-and-against, but all
focused around a particular assumed object---I have described a
range of alternative, cyborg, narratives that complicate our ideas of what
automated vehicles can and will be.

%%  view of the object of study, not asking
%% what automated vehicles will truly be, but only what will this assumed
%% object do

\section{Automation and Social Goals}

Ultimately, from a design and policy perspective, my point is that we
cannot achieve positive social effects by na\"{\i}vely adding autonomy to
existing vehicles. Autonomy and automation are not natural goods to be
fostered wherever they can be, but technological tools and strategies
for achieving particular goals. The manner and extent to which we
build our vehicles to be autonomous stands to produce very different
social impacts, to the point where going from the idea of the
``self-driving vehicle'' to social and cultural impacts is
fundamentally backward, since the vehicle itself is not fixed and
small details may matter a great deal in real-world use. Instead,
social changes have to be at the center of design and implementation.
What are the goals a vehicle is designed to achieve, and which goals
are more important than others? Increasing
statistical safety by a factor of 100? Providing mobility for the
elderly or disabled? Reducing the environmental impact of vehicles and
their emissions? Increasing throughput of the roadway? Reducing
traffic delays for commuters? Equalizing the 
disparities in mobility between the wealthy and the homeless? It is
fantastical thinking to believe that the just-add-autonomy approach
will automatically achieve all these goals. Each presents a
system-wide design problem, involving cars, people, and
infrastructures, that puts different demands on possible
solutions. Some goals may be directly conflicting, such as increasing
mobility while reducing 
environmental impacts.
Others contradict the motives of organizations responsible for
providing the solutions:  while automobile companies can promulgate
ideas about how automated vehicles can provide greater mobility, help
the environment, and reduce traffic, they are unlikely to economically
back concepts that are against their interests---anything that reduces
the number of vehicles on the road,\footnote{Thanks to Joe Dumit for eloquently
  making this observation during my recent visit to UC Davis.}
or reduces the overall cost of those 
vehicles, without providing a new revenue stream to compensate,
represents a deeply suspect investment. 

Problematically, the automated car---like the
``smart city'' which has already been critiqued in this
way \cite{greenfieldSmart}---presents the opportunity for successful technology
companies to wrest greater control over everyday life, to worm their
way more deeply into our existence and thereby make themselves
indispensable. Automation can be a tool to enable one group of
people---technologists running multinational organizations---at the
expense of the rest of society.
But it also does not have to be employed toward those ideological
ends; given sufficient willpower, it can be made to serve others.
Changes to the city
should not be arbitrary, at the whims of a certain set of
producers and organizations, but part of a large-scale system design
strategy to address the needs of all transportation users. Complex
changes to complex systems require significant modeling work,
experiments, or even trial and error to get right. Old models and
habits, like the rules of the AASHTO\footnote{American Association of
  State Highway and Transportation Officials} ``Green
Book'' \cite[p. 181, 183]{swopeTrenton}, need to be broken, and new
ones built to replace them. 

%% The pro-side: millions of deaths if we don't do something now, so we
%% have to accept these
%% The con-side: (regulator) current deaths not on my hands, but if one
%% little girl is killed by a robot I've lost my job

%% But public rhetorics of driverless vehicles have an unfortunate
%% tendency to avoid these complex questions. One popular and troubling
%% argument revolves around responsibility for the rollout of
%% self-driving systems. On the pro-side, 

%% This ideology posits lowering accident rates above all else, leaving
%% no space for human squeamishness about technology and responsibility. 
%% The statistical argument suggests that the death rate is all that
%% matters, but humans are notoriously bad at understanding and
%% responding to statistics.

%% does have a media public face,

%% While we can point to the 90\% of accidents that involve human error,
%% computers will fix some of these types of error but may also cause
%% other problems, and even increase failure rates unrelated to human error.
%% The safety record of ``current'' vehicles is also not fixed; even if
%% we were to accept the popularly figured distinction between
%% incremental autonomy and ``complete'' autonomy approaches, the safety
%% of increasingly autonomous vehicles is a moving target against which
%% the achievements of more radical approaches would need to be
%% compared.

%% But this cartoonlike
%% view of reality it does not do justice
%% to the complexity of the situation. T

%% Complex changes to complex, high-stakes systems produce unpredictable
%% results.

Narratives that claim undue caution in the
rollout of autonomous vehicles will directly ``cost'' lives---since
people are killed by human drivers every day \cite{driverlessfuture}
\cite{baileyReason} \cite{howardRobots}---may be used as a lever to
argue for the moral imperative of 
speedy legalization and acceptance, pitted against risk-averse regulators who face
public censure if but one child is killed by a
``robot car.'' But the statistics about driverless
cars and risk, though partly a function of mathematical modeling, are
also, fundamentally, matters of faith. There is no precise answer, as
humans and machines both err alone or in combination, and
the safety of increasingly autonomous vehicles is a moving target
against which more radical approaches must be compared.
Results of our changes to complex, high-stakes systems may be
unpredictable. So how we implement 
automated vehicle technology is 
about narratives, the stories we believe and tell about technology, as
much as it is about mathematics and modeling the world. When those
stories come to define major policy documents, as they have the
NHTSA's preliminary levels of automation, we are in a dangerous
position moving forward into battles over regulations and
infrastructural investments that will come to define the next era of
the roadway.

%% This is not a question
%% of whether fully automated cars can come to exist, even as they are
%% now envisioned.

%abusive

%% Our choice should not be to either accept this driverless vision or take
%% responsibility for the deaths of thousands. That viewpoint is cartoonlike, coercive,
%% and deeply wrong, and the future is too important to be left up to
%% knee-jerk techno-utopianism and blind ideology.

Heterogeneous
approaches are valuable, since there is no one-size-fits-all
automation solution, no magic technological fix to social and transit
problems. But despite a hesitance to prescribe particular
technologies, current taxonomies and policy documents normalize
certain ideas about automation, while ignoring others. I am not saying that
the science-fiction dreams of 
driverless vehicles are possible or impossible; moreover, I believe in the
potential of computer technologies to achieve groundbreaking results.
But our choices should be between options, between
visions---how we integrate humans and computers to make our
driving safer, our cities more livable, our lives better. The Google model which has
captured so much press attention is one among innumerable imagined or
not-yet-imagined possibilities. Automation is not a
take-it-or-leave-it phenomenon, and the choice to accept one
particular model of vehicle automation or accept tens-of-thousands of
prospective road fatalities is a false choice. There is more on heaven
and earth than is dreamt of in this philosophy; and it is time we ask
of autonomy not the na\"{\i}ve questions of whether it will keep us
safe and improve our cities, but the deep, engaged, and difficult
questions of how it can be leveraged to achieve such aims. The real
question we need to ask is what form technologies of vehicle automation 
should take, and what benefits and acceptable sacrifices adhere to
that choice. Answering this question is a truly multidisciplinary venture,
involving engineering, law, policy, design, and ethics. When we
automate, do we choose to do so with the philosophy of ``do no harm''
or ``save more lives''?\footnote{These are deep and subtle questions,
  but proclaiming them meaningless from an engineering perspective is
  to miss the importance of ideas to guide technological development,
  and the social importance of developed technologies. I am grateful
  to Evan Donahue in particular for a discussion of how the
  implications of the Hippocratic oath differ from those of a
  philosophy specifically oriented around saving the greatest number.
  These two basic premises, extrapolated out to their logical
  conclusions, may diverge rather than converge, in ways that remind
  us of the dystopic potential of quantification and optimization to
  lead to solutions that are fundamentally inhumane. It
  would be foolish 
  to assume that medical systems, themselves a collection of
  technologies and people, are not affected by these kinds of
  distinctions. As these logics are applied elsewhere, their uncertain
  outcomes
  will remain important.} There may be hidden consequences to these 
philosophies---Who is saved? How? Who bears the cost, or is
counted out in order to save the greater number?---that come to deeply
affect how systems engineered from them alter the world.


%% (Evan, on Hippocratic oath: why is it ``do no harm'' not ``save more
%% lives''?
%% there may be hidden consequences to a philosophy of saving more lives)

\section{Lingering Questions}

The purpose of this thesis has been to sweep away some lazy thinking
about automated systems, and to try to reorient a more nuanced dialog
around different sets of issues. As we consider the future of
vehicle automation---whether as
designers, lawmakers, interested technologists, or concerned
citizens---what are the questions we should be asking? What should we
grapple with instead of accepting common framings of the issues
involved? To conclude, let me summarize the main open questions around
which a more productive dialogue about automation could be based.

First, what are the social goals around which automated technology
should be designed and implemented? The relative importance of safety,
convenience, mobility, social equality, and other factors must be
matters of public debate, as they are absolutely central to the
project of automation.\footnote{To the list
of major questions facing us as a result of this technology, given the
interest in full-system design, we might
wish to
add: how do we make transportation technology and urban planning sensitive to social
justice and responsive to a broad base of citizens, and inoculate it
against takeover by corporate entities that have only their own best interests
in mind? But this question is larger than the autonomous vehicle arena.}

Second, for any particular design, what are the appropriate roles of
human and machine in the human-machine system that automated vehicles
will necessarily be? These roles will not be a pure function of
technological capabilities, but will involve numerous other factors,
and will alter the skills that human operators will be expected to
acquire, and the types of licensing and control that are therefore
necessary. 

Third, as part of this question, how do humans
respond to AI systems that cannot simply be said to
``think,'' ``know,'' or ``understand'' in the same ways that we do (at
least not yet) but nevertheless hold our lives in their hands?
And how can human decision-making in the design of such systems be
reasonably and successfully audited?
These concerns may be ameliorated by human involvement and
supervision in operations, which means we must consider what types and
time-scales of supervision are socially or legally valuable, and
therefore should be built into that technology. 

Fourth, how should we value different automation approaches, or
estimate their risks, when the technologies involved require complex
changes to complex systems?  This
involves balancing the statistical benefits of different automation
approaches, and coming to conclusions about how to reasonably
estimate risks, test and certify systems, and set engineering
priorities based on this information.

Fifth, relatedly, what is the appropriate role of statistics and risk
projections in governing policy on automated technologies? How safe is
safe enough, for whom, and how is this
regulated? Assuming we can come to reasonable predictions about costs
and benefits, we must close the loop between this information and our
broader social goals. We need to decide how to weigh uncertainties in
our evaluation of which goals matter and which are achievable in the
near term or long term.

%% 2) wrap up with the big questions we are faced with, that we have to
%% grapple with instead of accepting teleology/common media framings: (1
%% p)
%% --1) what is the appropriate role for the human in the human/machine
%% system? (1 p) (and what skills do they need to acquire, what licensing
%% is necessary?)
%% --2) what is the appropriate role of statistics and risk in governing
%% policy on the issue? can you opt-out? how do we balance projected
%% statistical benefits of disparate approaches? (0.5 p)
%% --3) how safe is safe enough? for whom? and how is this regulated? (1
%% p)
%% -----safe for drivers? for pedestrians?
%% -----what types of pedestrians?
%% --4) how do we feel about AI systems that cannot really be said to
%% ``think'', ``know'', or ``understand'' but which nevertheless ``hold
%% our lives in their hands'' (0.5 p) 
%% ----and how is this changed by the promise of continued human
%% supervision?
%% ----what types of supervision, on what time scales, are
%% socially/legally valuable, valuable enough to be build into our technology?
%% ----This may well imply we need the human involved, for that very
%% reason
%% ----and we should KEEP the human involved as a matter of design
%% principle (2 p)

%% These considerations may well imply the need to maintain
%% significant
%% human involvement in order to mitigate risk, achieve public
%% acceptance, and address the human goals that drive technological
%% development.

%%  not ideologically
%% driven.
%% the future is too important to be left up to
%% knee-jerk techno-utopianism and blind ideology: 


The ``autonomous'' vehicle is a
similar sort of myth to that of the ``personless'' factory or the
self-teaching computer program: always a 
product of people, responsible to people, and involving people,
however far in the margins of our vision. Writing this off as
anomaly or failure, and focusing on ``autonomy'' or ``self-driving,''
obscures real complexities of operation that encounters with the world
will inevitably involve. The important considerations of when and how 
people are involved should be addressed carefully and
empirically, not by knee-jerk techno-utopianism and blind ideology.
There is hope in this altered narrative to ameliorate some of
the issues facing automated vehicles today. 
Rather than ``to automate or not to automate,'' we should ask
``how and why do we automate?'' Even in the most automated
technologies, their 
autonomy is largely an artifact of the lens through which we engage
with them---these systems involve delicate dances of human and machine
components, dispersed through time and space, and only pushed to the
margins of a point of view that takes the technological object itself,
not its sociopolitical and cognitive contexts, as the object of study.
Recognizing that continued human involvement will occur as a matter of
principle is 
the most important first step we can take toward rebalancing the
narrative of automation toward something more productive, which takes
seriously the technological hybridity of our cyborg past, present, and future.
In addition, this recognition paves the way for a political critique
of automated vehicles 
that asks, knowing 
that humans remain involved, who they are, where they are, and how
the system is being designed to serve their needs or impede their
agency.

%% \footnote{Of course, beliefs and ideologies are always involved
%%   in decision-making, but we should at least air them for all to see,
%%   and weigh them carefully as I have tried to do here, rather than blindly follow a dominant
%%   idea of the uses of technology.}

%% TODO make sure there isn't a hanging word!
