% -*-latex-*-
% 
% For questions, comments, concerns or complaints:
% thesis@mit.edu
% 
%
% $Log: cover.tex,v $
% Revision 1.8  2008/05/13 15:02:15  jdreed
% Degree month is June, not May.  Added note about prevdegrees.
% Arthur Smith's title updated
%
% Revision 1.7  2001/02/08 18:53:16  boojum
% changed some \newpages to \cleardoublepages
%
% Revision 1.6  1999/10/21 14:49:31  boojum
% changed comment referring to documentstyle
%
% Revision 1.5  1999/10/21 14:39:04  boojum
% *** empty log message ***
%
% Revision 1.4  1997/04/18  17:54:10  othomas
% added page numbers on abstract and cover, and made 1 abstract
% page the default rather than 2.  (anne hunter tells me this
% is the new institute standard.)
%
% Revision 1.4  1997/04/18  17:54:10  othomas
% added page numbers on abstract and cover, and made 1 abstract
% page the default rather than 2.  (anne hunter tells me this
% is the new institute standard.)
%
% Revision 1.3  93/05/17  17:06:29  starflt
% Added acknowledgements section (suggested by tompalka)
% 
% Revision 1.2  92/04/22  13:13:13  epeisach
% Fixes for 1991 course 6 requirements
% Phrase "and to grant others the right to do so" has been added to 
% permission clause
% Second copy of abstract is not counted as separate pages so numbering works
% out
% 
% Revision 1.1  92/04/22  13:08:20  epeisach

% NOTE:
% These templates make an effort to conform to the MIT Thesis specifications,
% however the specifications can change.  We recommend that you verify the
% layout of your title page with your thesis advisor and/or the MIT 
% Libraries before printing your final copy.
\title{Driverless Dreams: Narratives, Ideologies, and the Shape of the
Automated Car}
%% Replacing or Augmenting the Human
%% Chauffeurs or Cyborgs: Ideologies and the Shape of the Self-Driving Car
%TODO change the title of my other piece if I go with this thesis
%title
%Self-Driving Stories: Automated Vehicles and the Human Being
%The Shape of the Self-Driving Car: Ideologies, Strategies, and the
%Human Being
%Vehicles of Ideology: Narratives, Impacts, and the Shape of the
%  Self-Driving Car
%Self-Driving Dreams: Narratives, Ideologies, and the Shape of the
%Automated Car



\author{Erik Stayton}
% If you wish to list your previous degrees on the cover page, use the 
% previous degrees command:
       \prevdegrees{Sc.B., Brown University (2007)}
% You can use the \\ command to list multiple previous degrees
%       \prevdegrees{B.S., University of California (1978) \\
%                    S.M., Massachusetts Institute of Technology (1981)}
\department{Comparative Media Studies}

% If the thesis is for two degrees simultaneously, list them both
% separated by \and like this:
% \degree{Doctor of Philosophy \and Master of Science}
\degree{Master of Science in Comparative Media Studies}

% As of the 2007-08 academic year, valid degree months are September, 
% February, or June.  The default is June.
\degreemonth{June}
\degreeyear{2015}
\thesisdate{May 18, 2015}

%% By default, the thesis will be copyrighted to MIT.  If you need to copyright
%% the thesis to yourself, just specify the `vi' documentclass option.  If for
%% some reason you want to exactly specify the copyright notice text, you can
%% use the \copyrightnoticetext command.  
%%\copyrightnoticetext{\copyright Erik Stayton, 2015. All rights reserved.}

% If there is more than one supervisor, use the \supervisor command
% once for each.
\supervisor{T. L. Taylor}{Associate Professor}
\supervisor{David Mindell}{Professor}

% This is the department committee chairman, not the thesis committee
% chairman.  You should replace this with your Department's Committee
% Chairman.
\chairman{T. L. Taylor}{Department Committee Chair, Director of Graduate Studies}

% Make the titlepage based on the above information.  If you need
% something special and can't use the standard form, you can specify
% the exact text of the titlepage yourself.  Put it in a titlepage
% environment and leave blank lines where you want vertical space.
% The spaces will be adjusted to fill the entire page.  The dotted
% lines for the signatures are made with the \signature command.
\maketitle

% The abstractpage environment sets up everything on the page except
% the text itself.  The title and other header material are put at the
% top of the page, and the supervisors are listed at the bottom.  A
% new page is begun both before and after.  Of course, an abstract may
% be more than one page itself.  If you need more control over the
% format of the page, you can use the abstract environment, which puts
% the word "Abstract" at the beginning and single spaces its text.

%% You can either \input (*not* \include) your abstract file, or you can put
%% the text of the abstract directly between the \begin{abstractpage} and
%% \end{abstractpage} commands.

% First copy: start a new page, and save the page number.
\cleardoublepage
% Uncomment the next line if you do NOT want a page number on your
% abstract and acknowledgments pages.
% \pagestyle{empty}
\setcounter{savepage}{\thepage}
\begin{abstractpage}
% $Log: abstract.tex,v $
% Revision 1.1  93/05/14  14:56:25  starflt
% Initial revision
% 
% Revision 1.1  90/05/04  10:41:01  lwvanels
% Initial revision
% 
%
%% The text of your abstract and nothing else (other than comments) goes here.
%% It will be single-spaced and the rest of the text that is supposed to go on
%% the abstract page will be generated by the abstractpage environment.  This
%% file should be \input (not \include 'd) from cover.tex.

% 327 words, we are OK ( < 350)
In this work Erik Stayton examines dominant and alternative ideas about ground
vehicle automation, and concludes that current and imagined automation
technology is far more hybrid than 
is often recognized, presenting different questions about necessary or
appropriate roles for human beings.

Automated cars, popularly rendered as ``driverless'' or
``self-driving'' cars, are a major sector of technological
development in artificial intelligence and present a variety of
questions for design, policy, and the culture at large. This work addresses the
dominant narratives and ideologies around self-driving 
vehicles and their historical antecedents, examining both the media's representation
of self-driving vehicles and 
the sources of the idea, common both among the media and many
self-driving vehicle researchers, that complete vehicle autonomy is
the most valuable future vision, or even the only one  worth
discussing and investigating. 
This popular story has important social stakes (including
surveillance, responsibility, and access), embedded in the
technologies and fields involved in visions of full
automation (machine vision, mapping, algorithmic ethics), which bear investigating
for the possible futures of 
automation that they present.
However, other potential narratives for looking at automation exist,
representing lenses from
literature in the fields of human supervisory control and
joint-cognitive systems design. These fields---compared with
that of AI---provide a very
different read on what automation means and where it is headed in the
future, which leads to the possibility of different futures, with
different stakes and trade-offs. Finally, this work examines what cultural
understandings need to change to 
make this (cyborg) picture more broadly comprehensible, and suggest potential
impacts for policy and future technological development. It argues that
a broader appreciation for 
our hybrid engagements with machines, and recognition that automation
alone does not solve any social problems, can alter public
opinion and policy in productive ways, away from focus on
``autonomous'' robots divorced from human agency, and toward
system-level joint human-machine designs that address social needs. 

\end{abstractpage}

% Additional copy: start a new page, and reset the page number.  This way,
% the second copy of the abstract is not counted as separate pages.
% Uncomment the next 6 lines if you need two copies of the abstract
% page.
% \setcounter{page}{\thesavepage}
% \begin{abstractpage}
% % $Log: abstract.tex,v $
% Revision 1.1  93/05/14  14:56:25  starflt
% Initial revision
% 
% Revision 1.1  90/05/04  10:41:01  lwvanels
% Initial revision
% 
%
%% The text of your abstract and nothing else (other than comments) goes here.
%% It will be single-spaced and the rest of the text that is supposed to go on
%% the abstract page will be generated by the abstractpage environment.  This
%% file should be \input (not \include 'd) from cover.tex.

% 327 words, we are OK ( < 350)
In this work Erik Stayton examines dominant and alternative ideas about ground
vehicle automation, and concludes that current and imagined automation
technology is far more hybrid than 
is often recognized, presenting different questions about necessary or
appropriate roles for human beings.

Automated cars, popularly rendered as ``driverless'' or
``self-driving'' cars, are a major sector of technological
development in artificial intelligence and present a variety of
questions for design, policy, and the culture at large. This work addresses the
dominant narratives and ideologies around self-driving 
vehicles and their historical antecedents, examining both the media's representation
of self-driving vehicles and 
the sources of the idea, common both among the media and many
self-driving vehicle researchers, that complete vehicle autonomy is
the most valuable future vision, or even the only one  worth
discussing and investigating. 
This popular story has important social stakes (including
surveillance, responsibility, and access), embedded in the
technologies and fields involved in visions of full
automation (machine vision, mapping, algorithmic ethics), which bear investigating
for the possible futures of 
automation that they present.
However, other potential narratives for looking at automation exist,
representing lenses from
literature in the fields of human supervisory control and
joint-cognitive systems design. These fields---compared with
that of AI---provide a very
different read on what automation means and where it is headed in the
future, which leads to the possibility of different futures, with
different stakes and trade-offs. Finally, this work examines what cultural
understandings need to change to 
make this (cyborg) picture more broadly comprehensible, and suggest potential
impacts for policy and future technological development. It argues that
a broader appreciation for 
our hybrid engagements with machines, and recognition that automation
alone does not solve any social problems, can alter public
opinion and policy in productive ways, away from focus on
``autonomous'' robots divorced from human agency, and toward
system-level joint human-machine designs that address social needs. 

% \end{abstractpage}

\cleardoublepage

\section*{Acknowledgments}

This is the acknowledgements section.  You should replace this with your
own acknowledgements.

%%%%%%%%%%%%%%%%%%%%%%%%%%%%%%%%%%%%%%%%%%%%%%%%%%%%%%%%%%%%%%%%%%%%%%
% -*-latex-*-
