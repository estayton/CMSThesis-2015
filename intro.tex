\chapter{Introduction}

Intro / Visions / Public Faces (14 p)

>>>Story hook: currently, this calls out to be a brief historical
sweep through self-driving car dreams from the 1920s/30s to the
present; but it could also begin with the reveal of the Google car; or
the Youtube video of the guy in the Mercedes tricking the lane assist
function to drive for him on the highway (2 p)

1) establish what the technology is: in its varied types and
dimensions: this is not ONE THING or ONE IDEA (4 p)
---INCLUDING historical models: see Erkki Huhtamo, media archaeology
(2 p)
>>>>pull from my essay!

General Motors's 1950s Firebird concept
vehicles sold themselves on this research, despite the fact that
Firebird II “had no automated capabilities whatsoever.”ii GM's
promotional films suggested the vehicle could be controlled
electronically from traffic control towers—like those in use in
aviation—placed along major highways: the car was “under the direction
of an 'electronic brain' on a dream highway of the future.”iii This
would have been the realization of the New York World's Fair exhibit
from two decades earlier: GM's Futurama from the 1939 World's Fair
depicted cars that maintained distance from each other via a
“sophisticated system of radio control.”iv Also in the 1950s, RCA's
Vladimir Zworykin, a lead inventor of television technology, was
working on an intelligent road system of his own. His concept,
inspired by “railroad block signals,” used circuits embedded in the
road to magnetically sense vehicle speed and location, placing sensing
and coordination capabilities outside of the vehicle, in road-side
systems.v Zworykin's centralized planning model would send
instructions to individual cars, and a 1/40th scale demonstration
system was built for the 1960 Highway Research Board meeting in
Washington D.C.vi


2) deal in general with some of the popular portrayals of the
technology: the nature of the self-driving car as it will appear in
the near future is not self-evident, no matter how loud Google's voice
or how large their investment (4 p)
---you would be excused for thinking XYZ, based on what you read in
the popular press
---but this is a flawed picture of the future, and an incomplete
picture of the tech

>One vision is that these are coming in 2-5 years; law just needs to
>get out of the way
>Another is that there are big problems to be solved, the technology
>isn't there
-both suggest moral /ethical/legal quandaries
HOWEVER nobody (popularly) seems to dispute what the object is that is
>being discussed, or how else it might look. both these narratives
>tend to take a particular technological picture and either support it
>or illustrate its problems
>idea that ''if you are asked to take control, we've failed'' as a culture of developers

The popular visions of this technology focus on the future: the
idea that in just two decades the majority of cars on the road will be
fully autonomous. Even respected business information and consulting
bodies have bought into this dream.\footnote{For example, the IHS
  predicts 54 million such vehicles by 2035, which is not as extreme,
  but still a sizable fraction of road vehicles \cite{IHSstudy}} In these vehicles, the users would
step in, select a destination, and would then be free to read, sleep,
watch a movie, answer emails, or otherwise occupy themselves without
needing to pay any attention to the operation of the vehicle. While this
vision has its benefits, it makes many people nervous about
ceding their driving agency to a computer system.\cite{clytton}



3) and shift to the lead-in to the main thesis sections: 
KEY QUESTIONS: What are the stories we are hearing vs. Other ones out
there, why? and what are the stakes for these stories?

these technologies or visions of them enact certain ideologically charged
activities (through their epistemologies; or even to some extent
ontological position of a self-driving car that ``knows'' nothing);
and that these can be traced through the histories and current
discourses of the fields involved, and might lead us to expect very
different social/cultural impacts (2 p)

**What can I disprove?
I can disprove that this conventional story is right, true, and the
only story to tell.**

Note: I need more voices from actual modern driverless car research
papers, though looking at the Dickmanns ones more is a good idea! need
a section specifically on how the ideologies come through!

Note: I set out to plumb the depths of the cultural and social
implications of self-driving vehicle technology. In my na\"{\i}vet\'{e}, I
conceived of driverless cars as just-another-AI-technology, something
understood and on which a meaningful ideological critique could be
based. But as I researched, I discovered that not only did
self-driving cars lack a fixed social meaning---their uses and
contexts still up for grabs---but the technology itself is unspecified
in profound and important ways. An apt lesson for an early researcher,
here was a thing which had not reached the stage of becoming, in
Latour's parlance, a ``black box,'' despite the media's treatment of
it as such. Below the surface, controversy bubbles. The deeper one
goes, the more ambivalent the mess becomes. Certain claims seem
constant across projects, but others differ wildly. Their stated goals
often seem subtly off-set from their assumptions and end-products. And
even more fundamentally, the idea of the ``self-driving'' vehicle,
where the human does nothing at all, is hardly a self-evident one.
Turning up other research paths---and indeed, numerous historical
counterexamples---made it painfully clear that the story I wanted was
not the story I could tell. The time had not yet come when the social
and cultural impacts could reasonably be predicted in a cohesive way.
But, as I think such things are absolutely fundamental to the way we
should think about new technologies, I am unwilling to drop them
entirely. So what follows is a patchwork, though, I hope, an artful
one. Several technological possibilities produce several visions of
what society might be, but each will ultimately have to come into
contest not only with physical reality, but the social realms of law,
technology, and public acceptance. Rather than taking the visions as
my starting point, I must dig deeper: putting them in the context of
historical fields and precedents, and shining light through their gaps
for all to see. And that is what I hope this project has become, a
deep look into the ideologies and rhetorics of the self-driving car,
one which hints at what society might be like once these
vehicles are common, but moreso investigates what different models for these vehicles
exist, and what their public face might be if different people wrote,
and different ideas guided, their stories. Stories have power, and it
beehoves me, on finding the stories we tell to be so deeply one-sided,
to try to rebalance the narrative.
