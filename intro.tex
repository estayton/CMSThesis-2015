\chapter{Introduction}
\label{chap:1}

Intro / Visions / Public Faces (14 p)

It is early August, 1925, and a strange spectacle greets viewers on
Broadway in New York City. An empty car lurches uncertainly down the
street,\footnote{This description is adapted from Time
  Magazine\cite{???-Science-radio-auto}.} followed closely behind by a
chase car, filled with radio equipment and several men, one driving,
and one furiously operating a set of controls. The lead car features a
diamond-shaped antenna sprouting from where the back seat should be,
which is instead filled with a motley collection of dials and tubes,
wires and batteries. A man stands on the running board of the
jittering contraption, but keeps his
hands well away from the controls, as if to tempt fate, as the car,
unsteady in its movements, turns onto Fifth Avenue and comes,
ungracefully, to a halt.

While present work on automated vehicles by Google, Mercedes, Tesla,
and others might lead one to think that self-driving vehicles are a
new idea, specifically enabled by recent advances in artificial
intelligence, they have not merely been dreamed about, but actually
built in prototype form, as far back as 1925, when Houdina Radio Control
built the ``Linrrican Wonder'' from a 1926 Chandler, and demonstrated
it live, on the streets of New York City. But it would be wrong to
characterize a 90-year engineering journey as the continual
progression toward a ``modern,'' developed device. There is nothing
fundamentally natural about the way automated--or, often interchangeably
``autonomous,'' ``driverless,'' or ``self-driving''\footnote{I indeed use these terms
  interchangeably at times even within this text, but I should mention
the specifics of their use here. Autonomous, driverless, and self-driving
vehicles are generally one and the same, though the terms
``driverless'' and
``self-driving'' are used particularly in the press when talking about
these systems. Both ideas presuppose largely self-contained operation
on the part of the automated system. An ``automated'' vehicle however
does not necessarily presuppose self-contained operation, and cars
have become increasingly automated over the past 50 years (through
ABS, traction control, cruise control, automatic transmissions, etc.) without, so
far at least, becoming autonomous. This point, simple though it
sounds, is at the center of this thesis and is specifically the focus
of chapter \ref{chap3}.}---vehicles are being
envisioned, designed, and talked about. Instead, automated vehicle
development has sustained several successive, overlapping paradigms,
in part, though not completely driven by the technical capabilities
of, and excitement about, contemporary technologies. However, equally
important in this story are the ideologies of control, of human and
machine interaction, that shape engineering practice and vehicle
development, and alter how developments are represented and discussed
in popular media. What we believe self-driving cars to be---or what we
believe they soon will be---is a product of the popular, teleological understanding
of technology development, which draws from existing technocratic
impulses and narratives\footnote{These narratives are deeply, though
  not exclusively, American, and are intertwined with stories of
  American technological progress over the last century, from the
  Manhattan Project to the Apollo Program to ARPAnet to IBM's
  Watson.}. However, alternative narratives exist, and 
are very influential among certain groups of automation researchers,
but deserve much broader recognition outside of these circles. In
order to understand the way that automated vehicles have been figured
popularly, the implications of that vision, and the alternatives to
it, we should start by understanding the varied types and dimensions
of historical automotive automation research.

 Self-driving cars are born from the
same history as interchangeable parts, assembly lines, machine tools,
and scientific management, as electrification, computerization, and
networked interconnection. When asked how to make something better,
safer, faster, easier, more productive, we involve machines or at
least what H. M. Collins describes as ``machine-like actions''\cite{???}. It is
vital that we recognize this is not a new predilection, but is
something that is has been with us since the Industrial Revolution.
Furthermore, technological visions and concepts inscribe excitement about the
technologies of the present moment, and this tendency has deeply
affected the history of automated vehicle research. From the railroad to the
Internet (passing through the telegraph, the airship, electrification,
television, the automobile, the rocket), each new mode of
communication (or transportation\cite{???}) ushers in new visions of our
collective future. We are famously bad at these predictions\cite{???} but they
nevertheless drive the popular imaginary, which has included automated
vehicles since the 1920s, and other types of automation for hundreds
of years previous\footnote{For more on this, see chapter
  \ref{chap1}.}. Both visions of and research on self-driving
technology have taken on aspects of the central information
technologies of their day, and re-inscribed them as control
technologies for physical devices.

Houdina's early experiments in radio-controlled vehicles, which began
this chapter, put us into a time period in which radio was an
exciting, emerging technology, which was envisioned as a way to take
the driver out of the vehicle, even if a human's direct commands were
still necessary for vehicle operations. Aachen Motors's ``Phantom
Auto,'' operating on the same principles, toured Milwaukee a year
later,\cite{???} and returned as an attraction in Fredericksburg in 1932\cite{???}.
Radio control had at this point been public knowledge for at least two
decades, since Nikola Tesla's 1898 demonstration of a radio-controlled
boat,\cite{???} but though it was not fundamentally new the use of the
technology on public roads nevertheless shocked onlookers.
Contemporary news articles describe the machine as if it drove itself,
as if the physical position of the person was what mattered, not the
fact that a human was always using the radio transmitter. ``Driverless,
it will start its own motor, throw in its clutch, twist its steering
wheel, toot its horn, and it may even 'sass' the policeman at the
corner''\cite{???}. In ``one of the most amazing products of modern science'' the
car proceeds ``as though there were an invisible driver at the wheel''
in an effect both ``uncanny and mystifying''\cite{???}. Such work,
though it seems relatively primitive in retrospect (hardly more than a
glorified RC-car, though it was a compelling demonstration of the
power of radio control to remove the driver from a full-size vehicle), may well have
helped capture the popular imagination. This basic idea would reemerge
a few years later.

The 1939 New York World's Fair opened to great acclaim. Covering over
a thousand acres and attracting a total of over 44 million visitors,
the fair was a massive public spectacle demonstrating the technical
prowess of American industry and providing a grand vision for the
future of the nation. The structure of the exhibition—exemplified in
its grand (and oddly-named) buildings, the Perisphere and
Trylon—manifested the modernist techno-utopian ideology\footnote{As a
  side note, this has been described not only as techno-utopian but as
an outgrown of commodity scientism, a triumph of a sort of
marketing-driven approach to technology that is more rhetoric than
substance.\cite{???-DMsclass} It seems likely that commodity
scientism, or a similar phenomenon, is deeply involved in the ways
modern technological narratives are generated and consumed.} that drove the
dramatic technological displays\cite{???}. The great success of the fair was
GM's Futurama exhibit: the line for the exhibit routinely stretched to
two miles in length, and 28,000 people visited the exhibit each day.
The Futurama exhibit presented an idea of what the world would look
like in 1960, one which included trains and aircraft, a farm, and an
amusement park.\cite{???} But though GM's displays went far beyond automobiles,
they were still the focal point: the bird's-eye view presented a
future city with multi-lane elevated expressways filled with largely
automated vehicles. For their concept cars, GM separated the control
functionality for steering from that for the maintenance of speed and
distance between cars. The car's lateral position was maintained via
curved barriers that would restore the position of the car via
gravity, a low-tech solution to the more complicated problem of
steering. However, the Futurama exhibit described that distance
between cars would be maintained via a ``sophisticated system of radio
control''\cite{???}. Prior to electronic circuits, this would have been
accomplished via vacuum tubes, a technology well known at the time,
and certainly subject to significant hurdles in terms of mechanical
reliability. It was not until digital electronics were invented that
hardware became the relatively stable component of driverless systems,
with software reliability becoming the key issue\cite{???}. But despite the
fragility of glass tubes traveling at high speed, and despite the fact
that the technology could work well only for sufficiently
well-designed roads (high-speed highways, built specifically for these
vehicles, with attendant infrastructural costs), such a system was
reasonably coherent as a vision of what 1960 might be, from the
perspective of attendees. This aspirational future was clean,
modernist, and efficient; the rationalization of the factory, already
achieved and itself a subject of a great deal of public
interest---through factory tours of major Detroit automotive plants, and
working GM and Ford assembly lines at the earlier Century of Progress
Exhibition in Chicago (1933-1934)\cite{???}---had spread to the city and
countryside, where working hours were reduced through electrification,
and even driving could become leisure, the mere supervision of a
device operating via radio control.

But radio was not the only revolutionary medium to be included in
 portrayals of the self-driving car of the future. The cover of the
 April 1936\cite{???} edition of the magazine Modern Mechanix betrays a
 different vision of the autonomous vehicle: the so-called ``electric
 eye automobile'' which would steer itself via a control loop, using an
 array of ``electric eyes,'' or photocells, on the car to track a light
 beam projected from the car and reflected by mirrors in the road
 surface. As the article, titled ``Light Beams Steer Super Racing
 Cars,'' describes:
\begin{quote}
With speeds, such as recently attained by the famous Sir Malcom
Campbell, already approaching the point where human reflexes are too
slow to insure safe control of the car, science has turned to the
photo electric cell for a possible solution. A proposed driverless car
involves the use of multiple electric eyes as the heart of its
steering mechanism. A powerful beam of light directed at a large lens
on the front of the car is concentrated on steel mirrors set at an
angle in the trackbed. The reflections are “caught” by the electric
eyes which convey the electrical impulses to a mechanical-electrical
brain which keeps the speeding car on its course.\cite{???}
\end{quote}

The cover pictures an intrepid ``driver'' using a camera to take a
``movie recording'' of the car's performance, while he leaves the
``driving responsibilities to the mechanical and electrical brain''\cite{???}.
Responsible for initially putting the car into motion, the driver can
then step back from the task of driving, allowing the electrical
controls of mechanical linkages within the car---meticulously diagrammed
within the magazine article---to convey him safely and speedily down
the track. The photocells within the vehicle's sensing mechanism would
operate relays controlling the steering linkage, closing the loop
between sensing and acting in the manner of a ``teleological,''
self-governing mechanism with corrective feedback.\cite{???} But how
the car's governor could be modified to allow it to race with or
beside other cars is not mentioned, which is telling since it is the
complexities of the road environment that make automated driving so difficult.

By 1960, driverless vehicles had not been brought to market, but
interest in the technology continued. Ford's vice president of
engineering and research, Andrew Kucher, was referenced in the \emph{Chicago
Daily Tribune}, April 25, 1959, in a speech he gave at Northwestern
University, taking seriously the idea of autonomous cars. The article,
titled ``In 50 Years: Cars Flying Like Missiles!'' asks readers: ``can
you imagine flying automobiles directed by automatic guidance
systems?''\cite{???} Matt Novak suggests that ``Arthur Radebaugh's syndicated
Sunday comic 'Closer Than We Think' was also a likely inspiration''
for contemporary images of advanced car technologies---in this case
the vision of flying cars shown in \emph{The Jetsons} in 1962, a
dream related to, but still distinct from, the driverless dream---as in
1958 it depicted hovering 
cars floating on an air cushion in an ``already proved'' concept
publicized by the same Andrew Kucher\cite{???}. Newspapers from April of that
year also describe a 3-foot-long model of Kucher's ``Glideair'' that was
demonstrated to reporters in Detroit\cite{???}. The car may not be driverless,
but the people depicted riding in the ``flying carpet car'' in
Radebaugh's illustration don't seem to be paying much attention to
where it is going, and the road surface is bordered with ridges that
visually suggest the idea that the car will keep itself within the
demarcated space.

One particularly salient public image from the 1950s comes from the
Saturday Evening Post, which ran an advertisement for ``America's
Electric Light and Power Companies'' that depicted self-driving cars
coasting through an idyllic landscape, guided by electricity.\cite{???} The roadways
are clean and clear, stretching off confidently into the distance. The
landscape, tamed and controlled, yet retains its natural beauty:
nowhere to be seen are smokestacks and the power plants generating the
guiding force of these vehicles. Instead we see the rural, even
pastoral, landscape largely unbesmirched, only cut across by the
rationalization of a futuristic interstate highway system\footnote{Recall that
the Federal Aid Highway Act of 1956 began the creation of the
interstate system as we have it today}. Within the finned, bubble
canopied auto sits a nuclear family, father,
mother, daughter, and son, enjoying family time, a game of dominoes.
Perfectly coiffed, snappily dressed, and totally at ease, the mother
and daughter play, while the son admires a model of a futuristic
delta-wing jet aircraft and the father looks on. Their faces glow with
expressions of domestic contentment. The father, in the ``driver's
seat,'' is turned away from the wheel and the thoroughly contemporary
dashboard and instrument cluster, facing backward. The caption reads: 
\begin{quote}
One day your car may speed along an electric super-highway, its speed
and steering automatically controlled by electronic devices embedded
in the road. Travel will be more enjoyable. Highways will be made
safe—by electricity! No traffic jams . . . no collisions . . . no
driver fatigue.\cite{???}
\end{quote}
The limitless possibilities of electricity, measured and controlled by
America's Electric Light and Power Companies, reproduce a conservative
present into an equally conservative future. 

If electricity was still the force that would be harnessed to make the
roadways safe, what were the ``mechanical equivalents of sense
organs''\cite{???-wiener}for the driverless car? New
wartime and postwar developments in electronics, the same ones that
inspired new science fiction visions of the future, drove actual
research in automated vehicle systems. In the early 1950s, General
Motors and RCA developed a scale model of an automated highway system,
completed in 1953, that would allow the two companies to experiment
with developing real steering and distance control systems.\cite{???}
General Motors's 1950s Firebird concept
vehicles sold themselves on this research, despite the fact that
Firebird II ``had no automated capabilities whatsoever.''\cite{???} GM's
promotional films suggested the vehicle could be controlled
electronically from traffic control towers---like those in use in
aviation---placed along major highways: the car was ``under the direction
of an 'electronic brain' on a dream highway of the future.''\cite{???} This
would have been the realization of the New York World's Fair exhibit
from two decades earlier. Building to make
this optimistic vision of GM's engineering ability a reality,
engineers lead by Joseph Bidwell and Lawrence Hafstad within GM
Research installed ``pick-up coils'' on a 1958 Chevrolet: through a
feedback system like that of the ``electric eye automobile,'' these
coils would adjust the vehicle's steering.\cite{???} But unlike the
light-guided vehicle from the 1930s, this car had a new guidance
mechanism, which is entirely consistent with the image presented in
the Saturday Evening Post: a wire carrying alternating current would
be embedded in the road and would be tracked by the pick-up coils on
the car. A GM press release announced on February 14, 1958:
\begin{quote}
An automatically guided automobile cruised along a one-mile check road
at General Motors Technical Center today, steered by an electric cable
beneath the concrete surface. It was the first demonstration of its
kind with a full-size passenger car, indicating the possibility of a
built-in guidance system for tomorrow’s highways.... The car rolled
along the two-lane check road and negotiated the banked turn-around
loops at either end without the driver’s hands on the steering wheel.\cite{???}
\end{quote}

Instead of the mechanical ``half-pipes'' of the 1939 Futurama, advances in
technology would allow a future of automated steering powered
by alternating current. Electromagnetism supplemented electricity as the
guidance mechanism of choice. Also in the 1950s, RCA's
Vladimir Zworykin, a lead inventor of television technology, was
working on an intelligent road system of his own. His concept,
inspired by ``railroad block signals,'' used circuits embedded in the
road to magnetically sense vehicle speed and location, placing sensing
and coordination capabilities outside of the vehicle, in road-side
systems.\cite{???} Zworykin's centralized planning model would send
instructions to individual cars, and a 1/40th scale demonstration
system was built for the 1960 Highway Research Board meeting in
Washington D.C.\cite{???}

The 1964 New York World's Fair hosted a second Futurama exhibit, in
which GM presented an automated highway system much like Zworykin's
system next to multitudinous nuclear-powered concepts. The exhibit was
thus described: 
\begin{quote}
A revolutionary 'Autoline' expands the capacity of a three-lane
expressway: Electronically, a control-tower operator steers, brakes
and sets the speed of each car in an automatic lane; groups of cars
move at equal intervals as a group.\cite{???}
\end{quote}
 Advances in
electronics, and the expansion of early control theory into
cybernetics through the work of Norbert Wiener (who gave numerous
public lectures through the 1950s) and his contemporaries seemed
poised to make revolutionary applications of electrical control
possible. Wiener himself weighed in, in a lecture given in 1960, to
make the case for automated control systems: ``by the time we are able
to react to our senses and stop the car which we are driving, it may
already have run head on into a wall,'' he wrote, and the answer to
this problem was the lightning speed of cybernetic control.\cite{???} But real
developments in automated driving did not materialize, and research
languished throughout the late 1960s and 1970s.\cite{???}

In contrast to earlier radio or track-based electrical control, the
``modern'' era of self-driving vehicle research begins with
computer-vision based approaches. While computer vision techniques had
been experimented with for vehicle guidance since 1969, it was not
until the 1980s that microelectronics became powerful enough to
process images in near real time and compact enough to be placed on
the vehicle itself\cite{???}. In the 1980s, Ernst Dickmanns's lab at the
University Bundswerhr, in Munich, Germany was active in some of the
earliest self-driving car research and development\cite{???}. Their early
vehicle, a specially equipped 5-ton Mercedes-Benz van---computerized
controls could be used to perform all necessary driver inputs---was
fitted with cameras, other sensors, and an image-processing system to
close the control loop and drive the vehicle based on visual
information.\cite{???} This early experiment, VaMoRs, paralleled in the United
States with a Carnegie Mellon-developed vehicle based on a Chevrolet
panel van, was successfully tested on roads without traffic at up to
60 miles per hour,\cite{???} and was soon followed by new projects produced as
part of the EUREKA PROMETHEUS project\cite{???}. The largest driverless vehicle
research project of the time, funded by EU member states to the tune
of almost 750 million Euros, PROMETHEUS ran from 1987 to 1995 and
involved the VaMP vehicle from Dickmanns's research lab along with its
Daimler-Benz twin, VITA-II. Typical of vehicles of the time, VITA used
a CCIR analog video-camera signal processed through a framegrabber
which would feed the visual information into the transputer-based
parallel-processing system\cite{???}. The VITA-II system, scaled down to fit in
a passenger car (an S-class SEL 500), again used digitized analog
video signals to detect lanes and other vehicles\cite{???}. Additional sensors
detected brake pressure, temperature, steering angle, acceleration in
lateral and longitudinal directions, and yaw\cite{???}. The camera systems used
applied two-camera setups with stabilized two-axis rotation to allow
them to follow objects of interest in the scene\cite{???}. And the
technology largely worked, with the PROMETHEUS project coming to a
successful conclusion in 1994 and 1995 with 1000 kilometers of largely
autonomous operation in normal traffic conditions on Paris motorways,
as well as a finale drive from Munich to Copenhagen\cite{???}.

Further autonomous vehicle research has progressed incrementally from
these vision-based beginnings. Significant interest was spurred by the
DARPA Grand Challenge in 2004 and 2005,\cite{???} and the DARPA Urban Challenge
in 2007, which brought together industry groups and universities to
try to extend the capabilities of fully autonomous vehicles. While the
Grand Challenge technologies bear little resemblance to current
navigation approaches designed to deal with the complexities of
real-world road environments,\footnote{These vehicles basically
  followed a predetermined GPS path, with gross navigation skills to
  avoid obstacles at smaller scales. However, the desert environment,
  though hostile in many ways, is not nearly so complex as an urban or
suburban road. The navigational approaches applied there are
fundamentally different than those necessary for an every-day
self-driving vehicle.} but the Urban Challenge
vehicles, both via their sensor setups and algorithmic approaches to
navigation, can be interpreted as another step along the path to the
modern self-driving car, as instantiated within Google's design approach. 

But approaches today have a variety of forms. While Google is making
headlines with its self-driving vehicle research, and is defining the
look of the autonomous car with its roof mounted LIDAR scanner, other
companies are developing vehicles with slightly different sensors and
techniques. Google's approach uses cameras and the rotating laser
scanner, along with detailed pre-mapping and highly-accurate
differential GPS\cite{???}. Mercedes, so far, focuses more on cameras and radar,
without using expensive LIDAR systems or requiring exquisitely
detailed pre-mapping of streets---and some of these features are
already seen in the new S-class luxury sedan, though these vehicles
have not yet reached the level of capability of Mercedes's ``Bertha''
test vehicle\cite{???}. But these sensing choices come with their own
trade-offs in terms of operations, safety, and how safety can be
determined and proven. Even Apple claims, as of the time of writing,
to be working on an autonomous vehicle, though its approach has not
yet been made public. Tesla Motors is also working on an iterated
approach to vehicle autonomy, focusing publicly on software updates to
existing cars that add new automated features. Volvo is both involved
in a public test of vehicle autonomy in Sweden, and is moving forward
with automated systems in its production cars, building from years of
research in Advanced Driver Assistance Systems (ADAS) and Advanced
Vehicle Control Systems (AVCS). It is important to recognize that
there is a spectrum of approaches being developed today, from iterated
improvements on current designs to radical approaches that attempt to
achieve full self-driving at once. Automated\footnote{I use ``automated''
  precisely here. It is one of the main failures of the
  ``self-driving'' term that it appears to make a clear distinction: 
  either something can drive itself or it cannot. But there are even
  levels here: Under what circumstances can it be self-driving? Is
  complete self-driving capability necessary before the term applies?
  These questions will return again and again through this thesis,
  especially in chapters \ref{chap1} and \ref{chap3}. }---or even
``self-driving''---vehicles are not just one, uniform type of
thing. And, critically, there is no clear
line to be drawn between their extremes, as we will see in chapter
\ref{chap3}. 


%% >>>Story hook: currently, this calls out to be a brief historical
%% sweep through self-driving car dreams from the 1920s/30s to the
%% present; but it could also begin with the reveal of the Google car; or
%% the Youtube video of the guy in the Mercedes tricking the lane assist
%% function to drive for him on the highway (2 p)
%% 1) establish what the technology is: in its varied types and
%% dimensions: this is not ONE THING or ONE IDEA (4 p)
%% ---INCLUDING historical models: see Erkki Huhtamo, media archaeology
%% (2 p)
%% >>>>pull from my essay!
%%THIS STUFF DUPLICATES CH1; sort it out!


%% DO include the CES depictions; and contrast with the 1950s ad, at
%% least in a footnote!
One would be excused, however, for thinking differently. The popular
portrayal of self-driving technology is not often very nuanced. While
there is a fair amount of popular argument about whether these
vehicles will be beneficial (or how beneficial they will be) along
various axes of interest, there is relatively little discussion of
what these vehicles will actually be---outside of discussions of
Google's desire to eliminate steering wheels facing off against
regulators, or questions of whether the automakers' gradual approach
or Google's ``moonshot'' approach will triumph: the implicit
assumption is that everyone is headed for the same technological
goals, the same ultimate type of device.

At the Consumer Electronics Show in Las Vegas, in January
2015, visions of the driverless car stood front and center. Prototypes
and concepts at CES worked glitz and glamour into the dream of a
driverless future. Mercedes-Benz's F015 autonomous car
concept forgoes the traditional seating arrangement in favor of
rotating chairs that turn the center of the car into a lounge or
meeting space.\cite{???} The car's side windows are relatively minimal, its
doors taken up primarily by touch screens. In fact, the interior is
covered in touchable displays, set up for an immersive digital
experience. But beneath this shiny media exterior sits a relatively
unglamorous purpose:  its publicity photos show serious-looking, young white
businesspeople in uniform grey work clothing.\cite{???} The environment is
high tech and sharply clinical. Far from an exuberant depiction of the
promise of media technology in the automobile, this future is so
serious as to be dull: a homogeneous work space bleeding out into
other parts of life. Compared to America's Electric Light and Power
Companies' ad from the 1950s, where the car was a space for idyllic,
family life, what we think we will use these prospective vehicles for
has changed to the point where productivity is the stated or unstated
goal. Our assumptions have changed with relatively little
investigation into why, but continue to be the background of both
optimistic and pessimistic viewpoints.

One popular vision of this technology is fundamentally optimistic: it
puts forward the
idea that these vehicles will be commercially available in as little
as five years, and that in just two decades the majority of cars on the road will be
fully autonomous. Even respected business information and consulting
bodies have bought into this dream.\footnote{For example, the IHS
  predicts 54 million such vehicles by 2035, which is not as extreme,
  but still a sizable fraction of road vehicles \cite{IHSstudy}} In
these vehicles, the users would 
step in, select a destination, and would then be free to read, sleep,
watch a movie, answer emails, or otherwise occupy themselves without
needing to pay any attention to the operation of the vehicle. While this
vision has its benefits, it makes many people nervous about
ceding their driving agency to a computer system, especially on such a
potentially short timescale.\cite{clytton} Google's Chris Urmson is
still claiming that self-driving cars will be available in 5 years (a
claim first made in August, 2014, a personal goal based on when his
eldest son can get his license\cite{???-GomesHidden}), and expects
these vehicles not to have steering wheels or the possibility of human
control. That these vehicles would operate on public roadways is not
stated explicitly---and seems optimistic given the speed of regulation
alone---but is implied in these kinds of statements. Elon Musk expects
that once sufficiently capable vehicles 
come to market, human drivers may be
outlawed.\cite{???-http://www.forbes.com/sites/leoking/2015/03/19/googles-gargantuan-push-for-cars-with-no-steering-wheel-by-2020/}.
In this vision, the technology is ready, and law and policy need to
adapt. ``Robots can already outdrive humans,'' this view says, ``now
everyone needs to get out of the way''
\cite{???-http://www.popsci.com/cars/article/2013-09/google-self-driving-car}. 

%% 2) deal in general with some of the popular portrayals of the
%% technology:  (4 p)
%% ---you would be excused for thinking XYZ, based on what you read in
%% the popular press
%% ---but this is a flawed picture of the future, and an incomplete
%% picture of the tech

However, another common viewpoint is that there are big problems to be
solved, and the technology isn't ready yet, whatever Urmson and Google
say. Questions of safety, legality, and insurance cast doubts on this
sort of optimistic timescale. Moral and ethical
quandaries---driverless car ``trolley problems''---question how
autonomous machines will respond to unusual situations and whether
people will accept those responses. Would american freedom rhetoric
allow the outlawing of human drivers \cite{???-http://www.washingtonpost.com/blogs/wonkblog/wp/2015/03/18/should-we-outlaw-human-drivers-in-a-world-of-driverless-cars/}? 
Perhaps not. Would we accept a computerized device that kills ``3,300
Americans'' per year, but replaces the ``roughly 33,000 lives a year
that perish on U.S. roads as a result of human
error''\cite{???-http://www.washingtonpost.com/blogs/innovations/wp/2015/03/16/driverless-cars-a-tremendous-innovation-with-a-glaring-achilles-heel/}?
This is a difficult empirical question about the propensity of people
to want to have a recognizable site for blame, and the tolerance of
human beings for computer error. Among others, Bill Gurley, an Uber
investor interviewed because of his connection to companies engaged in
driverless vehicle research, believes that the reliability requirement
for these systems---''four nines,'' as he describes it---given the
catastrophic potential for errors renders the technology ``a long way
off''\cite{???--http://www.businessinsider.com/bill-gurley-is-skeptical-of-driverless-cars-2015-3}.
Even statements from very respected autonomy researchers are used to
support this view: John Leonard, notable as one of the researchers who
worked on MIT's entrant to the DARPA Urban Challenge in 2007, has gone
on the record multiple times about his doubts, and has become a very
public face for driverless car skepticism\cite{???-gomes}.

From all this, one would be excused for thinking this was just an
argument between the technophiles and the naysayers, the daring dreamers versus the
conservative pragmatists. But there is much more to the story than is
apparent in this simplistic narrative. Both popular visions of this
technology focus on the future, and do relatively little to
investigate the past or present, how we got to 
where we are now, or even what level of autonomy we currently see in
our vehicles. And few people seem to dispute what the object is that is
being discussed, or how else it might look.\footnote{Though it should
  be said that John Leonard, for example, has much more nuanced views
  on the subject than get represented in these narratives.} Both these narratives 
tend to take a particular technological picture and either support it
or illustrate its problems, without challenging that fundamental
picture. The idea that ''if you are asked to take control, we've
failed'' has become deeply ingrained in our development
culture, more than just as a risk-mitigation strategy but as a guiding
ideology But as we will see, this view doesn't necessarily hold together given
historical examples.

%---The future doesn't look like we think it does, pull material from
%CMS801 final paper! 
%3) and shift to the lead-in to the main thesis sections: 

%%TODO consider: ; or even to some extent ontological position of a
%%self-driving car that ``knows'' nothing 

However, the nature of the self-driving car as it will appear in
the near future is not self-evident, no matter how loud Google's voice
or how large their investment. These technologies and visions of them
enact certain ideologically charged activities---such as supporting
particularly research agendas---and these can be traced through the
histories and current discourses of the fields involved.
The particular epistemologies of the research fields that contribute
to this vision greatly affect what is considered, and valued, and
pushed forward. And recognizing the ideological charges behind this
work, and examining alternatives, might lead us to expect a very
different technology with very
different social and cultural impacts. 

%% 1) What are the stories we are hearing? 
%% 2) Where do they come from?
%% 3) What are the stakes for these stories?
%% 4) What are other ones out there, why aren't they so heard? and 
%% 5) Where do we go from here? What understandings need to change to
%% make the broader picture comprehensible?

%TODO alter the end of this if the conclusion becomes two chapters

This thesis is organized around several key questions. First, what are
the dominant narratives around self-driving vehicles and where do they
come from? Between the introduction, and chapter \ref{chap1}, I
examine both the media's representation of self-driving vehicles and
the sources of the idea, common both among the media and among
self-driving vehicle researchers, that complete vehicle autonomy is
the only future vision worth discussing and investigating. Second,
what are the stakes of telling this kind of story? In chapter
\ref{chap2} I describe some of the technologies involved in
automation, and the fields with which visions of full automation are
intertwined: computer vision, urban planning, machine ethics, and
others. Taking the popular narrative at face value presents the
possibility of certain kinds of futures---with important
disadvantages---while foreclosing on others in ways that are important
to examine. Third, what other potential narratives for automation, and
particularly vehicle automation, exist? In chapter \ref{chap3} I
examine an alternative paradigm for thinking about automation, looking
both at some of the literature in the field of human supervisory
control and historical examples from other regimes of automation that
make the case for why supervisory control is an acceptable---and in
many ways, more apt---alternative for considering what automation is
and does. But, as I describe, supervisory control presents a very
different read on what automation means and where it is headed in the
future, which leads to the possibility of different futures, with
different stakes and trade-offs. Finally, where can we go, as a
culture, from this realization? What understandings need to change to
make this picture more broadly comprehensible? And what should it mean
for policy? While this is a broad topic, itself worthy of a thesis, I
argue in chapter \ref{concl}, the conclusion, that the contrast
between the rhetorics of artificial intelligence and supervisory
control, which really describe many of the same artifacts, points the
way. A broader appreciation for our hybrid lives, our hybrid
engagements with machines, can alter public opinion and policy in
productive ways. This appreciation allows us to shift the fundamental
questions we ask about technology in a way that is, hopefully, as
productive for others as it was for this thesis.

Fundamentally, I am a scientist at heart,\footnote{I mean this not necessarily in
  the methods of what is considered modern, experimental science,
  which does not fit easily within the realms of historical critique,
  but in terms of ethos, and some notion of falsifiability.} and I seek to approach my
work in that light. What idea, or hypothesis, does this thesis attempt
to disprove? While I say many things that have significant backing,
though they may not themselves be provable\footnote{I hope they are,
  convincing, or at least suggestive.} in any real sense, my core case
is that there is more than one story to tell about road vehicle
automation.\footnotemark One can argue about approaches to automation and autonomy,
based on historical examples, empirical data, or ethnographic
examples, but the idea that the conventional story about automation
(that it is clearly and naturally headed toward completeness and
totality) is right, true, and the only story to tell is clearly
without basis. Ultimately, I argue that the narrative of self-driving cars, as
generally conceived, is far narrower than history suggests it should
be. A broader view of this narrative shows a different side of
automated vehicles that is more hybrid, more complex, and presents
very different questions for policymakers, engineers, designers, and
the public. 

%% Note: I need more voices from actual modern driverless car research
%% papers, though looking at the Dickmanns ones more is a good idea! need
%% a section specifically on how the ideologies come through!

\footnotetext{ I set out to plumb the depths of the cultural and social
implications of self-driving vehicle technology. In my na\"{\i}vet\'{e}, I
conceived of driverless cars as just-another-AI-technology, something
understood and on which a meaningful ideological critique could be
based. But as I researched, I discovered that not only did
self-driving cars lack a fixed social meaning---their uses and
contexts still up for grabs---but the technology itself is unspecified
in profound and important ways. An apt lesson for an early researcher,
here was a thing which had not reached the stage of becoming, in
Latour's parlance, a ``black box,'' despite the media's treatment of
it as such. Below the surface, controversy bubbles. The deeper one
goes, the more ambivalent the mess becomes. Certain claims seem
constant across projects, but others differ wildly. Their stated goals
often seem subtly off-set from their assumptions and end-products. And
even more fundamentally, the idea of the ``self-driving'' vehicle,
where the human does nothing at all, is hardly a self-evident one.
Turning up other research paths---and indeed, numerous historical
counterexamples---made it painfully clear that the story I wanted was
not the story I could tell. The time had not yet come when the social
and cultural impacts could reasonably be predicted in a cohesive way.
But, as I think such things are absolutely fundamental to the way we
should think about new technologies, I am unwilling to drop them
entirely. So what follows is a patchwork, though, I hope, an artful
one. Several technological possibilities produce several visions of
what society might be, but each will ultimately have to come into
contest not only with physical reality, but the social realms of law,
technology, and public acceptance. Rather than taking the visions as
my starting point, I must dig deeper: putting them in the context of
historical fields and precedents, and shining light through their gaps
for all to see. And that is what I hope this project has become, a
deep look into the ideologies and rhetorics of the self-driving car,
one which hints at what society might be like once these
vehicles are common, but moreso investigates what different models for these vehicles
exist, and what their public face might be if different people wrote,
and different ideas guided, their stories. Stories have power, and it
beehoves me, on finding the stories we tell to be so deeply one-sided,
to try to rebalance the narrative.}
