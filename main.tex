% -*- Mode:TeX -*-

%% IMPORTANT: The official thesis specifications are available at:
%%            http://libraries.mit.edu/archives/thesis-specs/
%%
%%            Please verify your thesis' formatting and copyright
%%            assignment before submission.  If you notice any
%%            discrepancies between these templates and the 
%%            MIT Libraries' specs, please let us know
%%            by e-mailing thesis@mit.edu

%% The documentclass options along with the pagestyle can be used to generate
%% a technical report, a draft copy, or a regular thesis.  You may need to
%% re-specify the pagestyle after you \include  cover.tex.  For more
%% information, see the first few lines of mitthesis.cls. 

%\documentclass[12pt,vi,twoside]{mitthesis}
%%
%%  If you want your thesis copyright to you instead of MIT, use the
%%  ``vi'' option, as above.
%%
%\documentclass[12pt,twoside,leftblank]{mitthesis}
%%
%% If you want blank pages before new chapters to be labelled ``This
%% Page Intentionally Left Blank'', use the ``leftblank'' option, as
%% above. 

\documentclass[12pt,twoside]{mitthesis}
\usepackage{lgrind}
%% These have been added at the request of the MIT Libraries, because
%% some PDF conversions mess up the ligatures.  -LB, 1/22/2014
\usepackage{cmap}
\usepackage[T1]{fontenc}
\pagestyle{plain}

%% This bit allows you to either specify only the files which you wish to
%% process, or `all' to process all files which you \include.
%% Krishna Sethuraman (1990).

\typein [\files]{Enter file names to process, (chap1,chap2 ...), or `all' to
process all files:}
\def\all{all}
\ifx\files\all \typeout{Including all files.} \else \typeout{Including only \files.} \includeonly{\files} \fi

\begin{document}

% -*-latex-*-
% 
% For questions, comments, concerns or complaints:
% thesis@mit.edu
% 
%
% $Log: cover.tex,v $
% Revision 1.8  2008/05/13 15:02:15  jdreed
% Degree month is June, not May.  Added note about prevdegrees.
% Arthur Smith's title updated
%
% Revision 1.7  2001/02/08 18:53:16  boojum
% changed some \newpages to \cleardoublepages
%
% Revision 1.6  1999/10/21 14:49:31  boojum
% changed comment referring to documentstyle
%
% Revision 1.5  1999/10/21 14:39:04  boojum
% *** empty log message ***
%
% Revision 1.4  1997/04/18  17:54:10  othomas
% added page numbers on abstract and cover, and made 1 abstract
% page the default rather than 2.  (anne hunter tells me this
% is the new institute standard.)
%
% Revision 1.4  1997/04/18  17:54:10  othomas
% added page numbers on abstract and cover, and made 1 abstract
% page the default rather than 2.  (anne hunter tells me this
% is the new institute standard.)
%
% Revision 1.3  93/05/17  17:06:29  starflt
% Added acknowledgements section (suggested by tompalka)
% 
% Revision 1.2  92/04/22  13:13:13  epeisach
% Fixes for 1991 course 6 requirements
% Phrase "and to grant others the right to do so" has been added to 
% permission clause
% Second copy of abstract is not counted as separate pages so numbering works
% out
% 
% Revision 1.1  92/04/22  13:08:20  epeisach

% NOTE:
% These templates make an effort to conform to the MIT Thesis specifications,
% however the specifications can change.  We recommend that you verify the
% layout of your title page with your thesis advisor and/or the MIT 
% Libraries before printing your final copy.
\title{Driverless Dreams: Narratives, Ideologies, and the Shape of the
Automated Car}
%% Replacing or Augmenting the Human
%% Chauffeurs or Cyborgs: Ideologies and the Shape of the Self-Driving Car
%TODO change the title of my other piece if I go with this thesis
%title
%Self-Driving Stories: Automated Vehicles and the Human Being
%The Shape of the Self-Driving Car: Ideologies, Strategies, and the
%Human Being
%Vehicles of Ideology: Narratives, Impacts, and the Shape of the
%  Self-Driving Car
%Self-Driving Dreams: Narratives, Ideologies, and the Shape of the
%Automated Car

% Asleep at the Wheel: Dangerous Fantasies of a Driverless Future



\author{Erik Lee Stayton}
% If you wish to list your previous degrees on the cover page, use the 
% previous degrees command:
       \prevdegrees{Sc.B., Brown University (2007)}
% You can use the \\ command to list multiple previous degrees
%       \prevdegrees{B.S., University of California (1978) \\
%                    S.M., Massachusetts Institute of Technology (1981)}
\department{Program in Comparative Media Studies/Writing}

% If the thesis is for two degrees simultaneously, list them both
% separated by \and like this:
% \degree{Doctor of Philosophy \and Master of Science}
\degree{Master of Science in Comparative Media Studies}

% As of the 2007-08 academic year, valid degree months are September, 
% February, or June.  The default is June.
\degreemonth{June}
\degreeyear{2015}
\thesisdate{May 8, 2015}

%% By default, the thesis will be copyrighted to MIT.  If you need to copyright
%% the thesis to yourself, just specify the `vi' documentclass option.  If for
%% some reason you want to exactly specify the copyright notice text, you can
%% use the \copyrightnoticetext command.  
%%\copyrightnoticetext{\copyright Erik Stayton, 2015. All rights reserved.}

% If there is more than one supervisor, use the \supervisor command
% once for each.
\supervisor{T. L. Taylor}{Associate Professor}
\supervisor{David Mindell}{Professor}

% This is the department committee chairman, not the thesis committee
% chairman.  You should replace this with your Department's Committee
% Chairman.
\chairman{T. L. Taylor}{Department Committee Chair, Director of Graduate Studies}

% Make the titlepage based on the above information.  If you need
% something special and can't use the standard form, you can specify
% the exact text of the titlepage yourself.  Put it in a titlepage
% environment and leave blank lines where you want vertical space.
% The spaces will be adjusted to fill the entire page.  The dotted
% lines for the signatures are made with the \signature command.
\maketitle

% The abstractpage environment sets up everything on the page except
% the text itself.  The title and other header material are put at the
% top of the page, and the supervisors are listed at the bottom.  A
% new page is begun both before and after.  Of course, an abstract may
% be more than one page itself.  If you need more control over the
% format of the page, you can use the abstract environment, which puts
% the word "Abstract" at the beginning and single spaces its text.

%% You can either \input (*not* \include) your abstract file, or you can put
%% the text of the abstract directly between the \begin{abstractpage} and
%% \end{abstractpage} commands.

% First copy: start a new page, and save the page number.
\cleardoublepage
% Uncomment the next line if you do NOT want a page number on your
% abstract and acknowledgments pages.
% \pagestyle{empty}
\setcounter{savepage}{\thepage}
\begin{abstractpage}
% $Log: abstract.tex,v $
% Revision 1.1  93/05/14  14:56:25  starflt
% Initial revision
% 
% Revision 1.1  90/05/04  10:41:01  lwvanels
% Initial revision
% 
%
%% The text of your abstract and nothing else (other than comments) goes here.
%% It will be single-spaced and the rest of the text that is supposed to go on
%% the abstract page will be generated by the abstractpage environment.  This
%% file should be \input (not \include 'd) from cover.tex.

% 327 words, we are OK ( < 350)
In this work Erik Stayton examines dominant and alternative ideas about ground
vehicle automation, and concludes that current and imagined automation
technology is far more hybrid than 
is often recognized, presenting different questions about necessary or
appropriate roles for human beings.

Automated cars, popularly rendered as ``driverless'' or
``self-driving'' cars, are a major sector of technological
development in artificial intelligence and present a variety of
questions for design, policy, and the culture at large. This work addresses the
dominant narratives and ideologies around self-driving 
vehicles and their historical antecedents, examining both the media's representation
of self-driving vehicles and 
the sources of the idea, common both among the media and many
self-driving vehicle researchers, that complete vehicle autonomy is
the most valuable future vision, or even the only one  worth
discussing and investigating. 
This popular story has important social stakes (including
surveillance, responsibility, and access), embedded in the
technologies and fields involved in visions of full
automation (machine vision, mapping, algorithmic ethics), which bear investigating
for the possible futures of 
automation that they present.
However, other potential narratives for looking at automation exist,
representing lenses from
literature in the fields of human supervisory control and
joint-cognitive systems design. These fields---compared with
that of AI---provide a very
different read on what automation means and where it is headed in the
future, which leads to the possibility of different futures, with
different stakes and trade-offs. Finally, this work examines what cultural
understandings need to change to 
make this (cyborg) picture more broadly comprehensible, and suggest potential
impacts for policy and future technological development. It argues that
a broader appreciation for 
our hybrid engagements with machines, and recognition that automation
alone does not solve any social problems, can alter public
opinion and policy in productive ways, away from focus on
``autonomous'' robots divorced from human agency, and toward
system-level joint human-machine designs that address social needs. 

\end{abstractpage}

% Additional copy: start a new page, and reset the page number.  This way,
% the second copy of the abstract is not counted as separate pages.
% Uncomment the next 6 lines if you need two copies of the abstract
% page.
% \setcounter{page}{\thesavepage}
% \begin{abstractpage}
% % $Log: abstract.tex,v $
% Revision 1.1  93/05/14  14:56:25  starflt
% Initial revision
% 
% Revision 1.1  90/05/04  10:41:01  lwvanels
% Initial revision
% 
%
%% The text of your abstract and nothing else (other than comments) goes here.
%% It will be single-spaced and the rest of the text that is supposed to go on
%% the abstract page will be generated by the abstractpage environment.  This
%% file should be \input (not \include 'd) from cover.tex.

% 327 words, we are OK ( < 350)
In this work Erik Stayton examines dominant and alternative ideas about ground
vehicle automation, and concludes that current and imagined automation
technology is far more hybrid than 
is often recognized, presenting different questions about necessary or
appropriate roles for human beings.

Automated cars, popularly rendered as ``driverless'' or
``self-driving'' cars, are a major sector of technological
development in artificial intelligence and present a variety of
questions for design, policy, and the culture at large. This work addresses the
dominant narratives and ideologies around self-driving 
vehicles and their historical antecedents, examining both the media's representation
of self-driving vehicles and 
the sources of the idea, common both among the media and many
self-driving vehicle researchers, that complete vehicle autonomy is
the most valuable future vision, or even the only one  worth
discussing and investigating. 
This popular story has important social stakes (including
surveillance, responsibility, and access), embedded in the
technologies and fields involved in visions of full
automation (machine vision, mapping, algorithmic ethics), which bear investigating
for the possible futures of 
automation that they present.
However, other potential narratives for looking at automation exist,
representing lenses from
literature in the fields of human supervisory control and
joint-cognitive systems design. These fields---compared with
that of AI---provide a very
different read on what automation means and where it is headed in the
future, which leads to the possibility of different futures, with
different stakes and trade-offs. Finally, this work examines what cultural
understandings need to change to 
make this (cyborg) picture more broadly comprehensible, and suggest potential
impacts for policy and future technological development. It argues that
a broader appreciation for 
our hybrid engagements with machines, and recognition that automation
alone does not solve any social problems, can alter public
opinion and policy in productive ways, away from focus on
``autonomous'' robots divorced from human agency, and toward
system-level joint human-machine designs that address social needs. 

% \end{abstractpage}

\cleardoublepage

\section*{Acknowledgments}

Any substantial work of writing has a long gestation period, and this
has been no exception. A great number of people come to mind who have
contributed to the existence of this thesis over the last 20 months,
but I fear I must 
inevitably leave some out. To them I can say only \emph{mea maxima culpa}. 

No master's thesis is complete without a
solid advising team, and my most heartfelt thanks go to T. L. Taylor
for her careful reading and editing, and her ceaseless dedication (including periodic
game nights) to ensuring we all make it through the master's process with most of our
sanity intact. David Mindell has also been a continual resource
throughout the writing of this thesis---and indeed is cited frequently
within it---and I look forward to continuing to work with him on
future projects. Nick Montfort (who has exhorted me repeatedly to, as
much as possible, interrogate the technical objects themselves),
William Uricchio (who has challenged my terminological precision),
Heather Hendershot (who has kept my eye on science fiction) have all
been indispensable in their own ways, and have given freely of both
thesis and career advice. Discussions with Kate Crawford, Nick Seaver,
Tarleton Gillespie, Merritt Roe Smith, Stefan Helmreich, and G\"{o}de Both have also
shaped this work and made it what it is; as have innumerable
discussions at conferences. Conversations with a number of informants
(including Ryan Chin, Jim Womack, and John Leonard)
furnished important material, and though I cannot list all of them by
name, I wish to thank them all for giving of their time.

All of my classmates, in the classes of 2014, 2015 and 2016 have been great
friends, supporters, and sounding boards for ideas. But I fear I must
specifically thank Liam Andrew, Chelsea Barabas, Suruchi Dumpawar,
Jesse Sell, Ainsley Sutherland, and Wang Yu for discussing
self-driving vehicles and technological ideology with me 
more times than some of them probably cared to---and additionally Desi
Gonzalez, Kyrie Caldwell, and Lacey Lord for being available for
games and gripe sessions throughout the process. On that note, my old
friend Kristin Bergman, and my new friend Andrea Morales Coto, have
been invaluable companions on this adventure, providing much needed
levity and perspective in trying times. I also thank my forthcoming
classmates at MIT HASTS for their input and encouragement,
particularly Alison Laurence and Clare Kim.

My parents, Barbara Mindell and Lee Stayton, have often been the first
to hear my ideas, and have provided irreplaceable editing and feedback
on this and other projects during my two years at MIT. I am so pleased
to have been able to share this work with you, and both it, and I,
have profited immensely from your input and your presence. Abigail
Strait, whose food, hospitality, and seemingly endless supply of tea I have
partaken of innumerable (at least more than twenty!) times during the
writing of this thesis, you will always have my gratitude.

But most of all, Evan Donahue, whose keen commentary, acerbic wit, and
drive to re-envision the future of humanity has spurred me on: this
project exists in no small part due to his influence, and he most of
all deserves thanks. As we have 
often said, this was supposed to be the Fourth Great and Bountiful
Human Empire. Let us strive make it so.

 

%%%%%%%%%%%%%%%%%%%%%%%%%%%%%%%%%%%%%%%%%%%%%%%%%%%%%%%%%%%%%%%%%%%%%%
% -*-latex-*-

% Some departments (e.g. 5) require an additional signature page.  See
% signature.tex for more information and uncomment the following line if
% applicable.
% \include{signature}
\pagestyle{plain}
  % -*- Mode:TeX -*-
%% This file simply contains the commands that actually generate the table of
%% contents and lists of figures and tables.  You can omit any or all of
%% these files by simply taking out the appropriate command.  For more
%% information on these files, see appendix C.3.3 of the LaTeX manual. 
\tableofcontents
\newpage
\listoffigures
%\newpage
%\listoftables


\chapter{Ideologies and Their Stakes}

Ideologies (and their stakes) (visions/histories) (32 p)

\subsection{Automation of Work}
%% 1.1) deep history is the mechanization/automation of work (4 p)
%% --going back to factories/IRev: replacing human competencies with
%% machine competencies and moving the people into new roles
%% --which engages old debates about the role of the human

Human technological progress since antiquity has
involved continual re-negotiations of human labor and the roles of
animals and mechanisms in the labor process. Due to a confluence
of factors---the continuing miniaturization of computing technology, new
advances in machine learning and artificial intelligence algorithms, a
gradual increase in battery capacities, faster wireless networks---the
horizons for everyday automation are broader than ever. 

The popular visions of this technology focus on the future: the
idea that in just two decades the majority of cars on the road will be
fully autonomous. Even respected business information and consulting
bodies have bought into this dream.\footnote{For example, the IHS
  predicts 54 million such vehicles by 2035, which is not as extreme,
  but still a sizable fraction of road vehicles \cite{IHSstudy}} In these vehicles, the users would
step in, select a destination, and would then be free to read, sleep,
watch a movie, answer emails, or otherwise occupy themselves without
needing to pay any attention to the operation of the vehicle. While this
vision has its benefits, it makes many people nervous about
ceding their driving agency to a computer system.\cite{clytton} News articles fret
about what will happen when no one knows how to drive manually any
more,\cite{pross} a classic fear of ``de-skilling'' that is implicated in so many
other implementations of computers. Furthermore, coexistence with
autonomous or automated systems is presented as a fundamentally new
situation, as if human beings had never before had to work and live
with and next to automated systems, presenting new benefits and
dangers, and requiring new roles for their human tenders.

However, automation already has a deep
history in the industrial sector. Current debates and fears about de-skilling, human jobs,
and the role and value of human labor return us to questions that have
plagued factories, and labor's relationship to machinery, since the
early Industrial Revolution.

Robotic cars are sometimes situated as
the next step for robots after the factory, their final emergence into
the real world having conquered the factory floor.\footnote{See for
  example ``Robot Vehicles'' in RobotWorx \cite{robotworx}, which describes automated cars as having the sensors
  industrial robots have had for many years.} But just what the
processes of standardization, mechanization, and automation (or
``automatization'') have done to the factory, and to laborers in it, is
not clearly understood among many who write about autonomous cars. Within this
forgotten history---which is substituted for by an imagined person-less
factory that does not exist in the real world---there are even lessons to be
learned by the research community. This past is relevant, perhaps
more than ever, to the future of transportation.

A search for the beginnings of industrial automation takes us to the middle of
the 18th century: Vaucanson's mechanical loom dates to 1741, and formed the basis of
 later developments in weaving by Joseph Marie Jacquard.\cite[p. 9]{dieboldImpact} 
But the first true example of industrial automation originating in the
United States does not come until Oliver Evans' work in the 1780s on
automated grist mills.\cite[p. 5]{roesmithYankee} Through a series of elevators and descenders,
horizontal screws, spreaders and rakes, his mill moved grain from raw
agricultural commodity to finished product: sifted flour. And ideally,
all parts of the process would occur without human intervention. 

In reality, the process both increased efficiency and decreased the costs
of production, so much so that the same basic machinery is still in
use today in some smaller milling operations.\cite{wyegrist} None of his individual
inventions---which he lists as the elevator, ``conveyer,'' hopper-boy,
drill, and descender in his 1795 miller's guide---was a particularly
groundbreaking achievement, but what Evans did was place these devices
in succession so as to allow continuous production, and the
elimination of many slow human jobs that degraded the quality of the
product by tracking dirt and contaminants around inside the
mill.\cite[p. 203]{evansMillguide} 

It took some time for the high level of automation found in the Evans
mill to spread across other industries, and the mill may find its
closest cousins in the ``automatic'' factories of the 1950s and 1960s
and the roboticized factories of today, but Evans's contemporaries
were not uninterested in increasing efficiency and output. Paul
Revere, one of America's early industrialists, applied shifts in
manufacturing techniques to transition himself from an
artisan worker to manager and overseer of others over his long
metalworking career.\cite[p. 187]{martello} Like a small number of his postrevolutionary
contemporaries, he improved his circumstances by becoming a manager
and owner rather than a laborer, but manufacturing itself was a site of public
debate, pitted against the ``inherent virtue'' of agricultural pursuits.
Tench Coxe, a political economist, wrote in 1810 that ``new machines
and power sources allowed even greater productivity with less labor,
further underscoring the connection between technology and republican
virtue.''\cite[p. 217]{martello} To Coxe's romantic view, these machines  worked ``as if they
were animated beings, endowed with all the talents of their inventors,
laboring with organs that never tire, and subject to no expense of
food, or bed, or raiment, or dwelling.''\cite[p. xxv]{coxe}

But these romantic words did not represent the whole reality of
industrial machine labor. Human labor of
maintenance and supervision is implicit in these manufacturing
machines, but it is rendered invisible by the rhetoric that the
machines themselves require no bed or board. At the same time, Coxe's
use of the word ``endowed'' should focus our attention on exactly which
of the ``talents'' of the inventors have been automatized, and the human
labor necessarily involved in that conferring of capabilities. 

Arsenal practice was the site of multiple revolutions in U.S.
manufacturing technology through the 1800s and early 1900s, notably
the development of the so-called “American system” of interchangeable
parts, through which a gradual increase in mechanization would seem to
continue, driven by the tight tolerances necessary for this production
method. And yet armorers and managers at Harper's Ferry resisted the
mechanization of their craft,\footnote{Blanchard's many automatic
  machines for making gunstocks were of particular
  importance,\cite[p. 56]{roesmithHarpers} but a wide variety of machinery
  was implemented in the gun-making process at Harper's Ferry. Most of
 these machines still required significant human labor. Nash's
 barrel-turning machine in part mechanized the production of barrels
 of standardized dimensions: it consisted of a lathe on a wooden
 frame, with human-operated props to hold the barrel in
 place.\cite[p. 119]{roesmithHarpers} The worker also had to continually
 measure the barrel with a caliper, and adjust the device's chisel
 appropriately.\cite[p. 121]{roesmithHarpers} This gradual implementation of
 mechanized labor was continued in further machines produced by Hall.
 His straight-cutting machine, an early version of a milling machine,
 had as its distinctive feature the ability to be tended by ``common
 hands'' without a loss of
accuracy.\cite[p. 239]{roesmithHarpers}}, and while certain competencies were
transferred from the skilled worker
to the technical apparatus, human oversight and operation was still
integral to the production of weaponry using the new technology. It was
not clear until after the fact that more mechanization was necessarily
better: Harper's Ferry remained ``competitive'' with Springfield's
costs through the mid 1830s.\cite[p. 324]{roesmithHarpers}

The same exchange of competencies characterized Ford's assembly line
production as well, which is cited by David Hounshell as the rise of
true mass production in America.\cite[p. 217]{hounshell} Ford's factory developed fixtures
and gauges, designed to allow for use by unskilled machine tenders. As Donald
Norman writes in Things that Make us Smart---itself, reminiscent of Ed
Hutchins's cognitive anthropology and work on distributed cognition
systems---``the world remembers things for us, just by being
there.''\cite[p. 147]{normanThings} But while the gauge simplifies the
assurance  of quality, it does not
automate it: it simply changes the effort from a more complex judgment
of quality and measurement to a simpler one. While he instituted the
five-dollar day to attempt to solve labor problems at the factory, and
compensate laborers for becoming part of the ``production machine,''
Ford also attracted a wide variety of well-educated skilled mechanics
to his automobile plants.\cite[p. 223]{hounshell} Like Evans, Blanchard, Hall, and others
before them, these mechanics applied their skills to design machines,
and simplify and standardize work processes. The individual judgment
of the assembly line laborer was displaced into standardized tools and
fixtures, built into these technologies by the labor of skilled
machinists and designers.

Meanwhile, Taylorism in factories created ``new managerial
functions'' performed by ``new classes of people with new titles and
more clearly specified responsibilities.''\cite[p. 120]{aitken} A focus on the
people---who are they? where are they? what are they doing?---shows that
one of the fundamental and enduring characteristics of Taylor's
system, the expansion of management roles and the further division of
labor, is not about mechanical automation but about new and altered
types of human work. Industrial processes in the early 1900s continued
the removal of management and strategic decision making from the
workers most physically engaged in product production, installing it
instead within formal organizational structures and the employees that
constituted them. 

This pattern of delayed recognition and contingent change repeats for
numerical control in assembly line production. Numerical control (NC),
developed in the 1940s and 1950s as an outgrowth of World War II
research into feedback systems, slowly began to produce industrial
robots that could perform factory tasks without direct human
intervention. Robots slowly began to replace assembly line jobs such
as spray painting and welding, but adoption was gradual, with only
about 6,000 robots in use in American factories by the mid
1970s.\cite[p. 159]{nyeAmericas} Industrial robots, while automating tasks,
had a way of generating large contingents of skilled human laborers
who still needed to be paid for their services. Industrial robots were
complicated, and needed a variety of skilled workers to tend them, and
to repair them when the broke down. These early experiments did not
increase profits because of the volume of highly skilled labor needed
to keep the machines operating.\cite[p. 162]{nyeAmericas} The development of NC machines
proceeded with a specific interest in eliminating skilled workers, but
the jobs that disappeared were largely unskilled or semi-skilled
laborers.\cite[p. 164]{nyeAmericas} And while Norbert Weiner, in his 1950 book \emph{The Human Use of
Human Beings}, prophesied the end of ``deadly uninteresting'' jobs, which
would be mechanized within 20 years, such changes have still not
totally come to pass.\cite[p. 161]{nyeAmericas} To compound the problem, new industries of
skilled workers---record-and-playback machine designers, and NC machine
programmers---sprang up to furnish factories with their tools.

The process of standardization,
mechanization, and automation has been a process of attempting to
wrest control of the production of components from those closest to
handling them and concentrating it in the hands of management. The Air Force's Integrated Computer-Aided Manufacturing program (ICAM)
brings to light further complexities in the story of the automated
factory: ICAM attempted to aid shop floor automation by automating certain
management functions, ``to try to reduce the enormous indirect costs
that have resulted from the effort to reduce labor costs and remove
power and judgment from the shop floor,'' costs that have continued to
dog new rationalization strategies.\cite[p. 330]{nobleForces} ICAM, like the mythical
ouroboros, sought to offer automation as the ``solution to the problems
generated by automation,'' providing automated scheduling functions,
inventory control, and design tools to ``provide better management
control'' and ``free management from excessive routine duties to do
creative work''---the creative work that the management had attempted to
place in their own hands, in the first place, through earlier
processes of rationalization.\cite[p. 330]{nobleForces}  Automation, Noble's ICAM example shows,
can be used both to routinize work---for the manual laborer---and to
eliminate the routine in favor of the creative---for managers and newly
generated classes of creative workers. 

%% Noble cites 330, 330
Automation may look very different depending on where in the hierarchy
a person happens to fall, but the historical lesson is that human
involvement remains, though altered in space, time, and kind. As John
Diebold pointed out in 1953, there will be ``no worker-less
factories as a result of automation.''\cite[p. 63-64]{dieboldNew}

So we come to our first of many contradictions at the heart of
automated vehicle research. Their promise:
to provide to us a greater measure of creativity in the act of
driving, to remove some of the ``menial'' and routine tasks of manual
control in favor of strategic decision-making. In this analogy, the
driver goes from being the manual laborer to the creative manager. But
despite a focus on relieving tedium, this is not the way these systems
have primarily been envisioned. Instead, in the process of following
the dream of fully automated operation--where the human labor has been
entirely removed from the shoulders of the person in the vehicle, and
displaced to the invisible labors of mapping, programming, and
monitoring---engineers are designing systems where the ``driver''
seems present largely to ensure the operation of capital-intensive
machinery, burdened with new but perhaps even more menial tasks of
machine tending.\footnote{See for example Tom Simonite, “Lazy Humans
Shaped Google's New Autonomous Car,”\cite{simonite}  which discusses
the human role within Google's test vehicle,
and the company's response. This is purely speculation, but due to
the way the system operated previously, it is possible at least one Google
employee fell asleep at the wheel, which was the catalyst for their
concern and change in focus.} 

Why might this be? To understand it, we must understand something
about AI history, and the ideologies intertwined with artificial
intelligence research.

%% This approach
%% presents serious risks. It may be some time before the human inside
%% the car can be entirely disengaged with the driving task,2 if that is
%% even something we want as a culture, which means an interim period of
%% operation potentially characterized by “hours of boredom punctuated by
%% moments of terror.”3 The danger of human inattention,4 which has shown
%% up in aircraft automation with sometimes disastrous results, is
%% actually pushing aircraft manufacturers to more fully involve the
%% human in the process of flying, even as cockpits become more and more
%% computerized.5 And by focusing on ever more automation, rather than
%% appropriate automation, we may also be removing some of the parts of
%% driving that are most enjoyable: by replacing the skilled craftsman
%% with the automaton and the machine tender, we risk making driving
%% sterile and dull. 

\subsection{A Brief History of Artificial Intelligence}
%% 1.2) place to introduce the brief overview of AI history, of what goes
%% into the different approaches
%% --starting from historical visions of automata (1 p)
%% --classical symbolic approaches / physical symbol system (3 p)
%% ----which in a way we have returned to today w/ explicit mapping
%% --Rodney Brooks subsumption architecture and ``world is its own best
%% model'' (3 p)
%% --Leaning heavily on H. R. Ekiba and his approach to illuminating the
%% unstated assumptions of AI research areas (2 p)
Intelligent machines are not a new idea. The myths about Hephaestus and
his creations, notably Talos, a golden female automaton, come to us
from antiquity,\cite{mccorduck}  but continue to be cited as historical antecedents in
literature on autonomous robots and their ethical issues.\cite[p. 3]{patricklin} Automata,
or rather semblances of automata, appear in Hellenic Egypt, with
priests as puppeteers pulling their strings.\cite[Ch. 1]{mccorduck} The history of
artificial life is intertwined with that of autonomous machines: the
creation of Pygmalion's Galatea echoes the same practices and
concerns, as does the story of Mary Shelley's “modern Prometheus” (of
particular note for the purposes of this argument, Shelley's vision of
artificial life is inspired and physically mobilized by static
electricity, the infusing of a ``spark of being'' into the creature; and
electrostatics were a highly public research topic and indeed public
spectacle, complete with live demonstrations, in the 1700s and early
1800s).\cite[p. 44]{shelley} Judah Loew ben Bezalel, a Talmudic scholar, is in legend the
creator of the golem, a being animated from clay who functioned as a
spy against the Gentiles.\cite[Ch. 1]{mccorduck} Though Loew's was not the only golem
recorded in myth, the rabbi occupies a special position among the most
prominent AI researchers of the 20th century: Pamela McCorduck records
that Marvin Minsky and Joel Moses grew up with a ``family tradition
that they are descendants of Rabbi Loew,'' and Moses claims a number of
other American scientists, including John von Neumann and Norbert
Wiener, also consider themselves among the descendants.\cite[Ch. 1]{mccorduck} This is all
to say that while the technological drivers of conceptual visions are
more contemporary, ancient myth and legend continue to subtly underpin
research in autonomy and artificial intelligence.

%% Drosz and Vaucanson
As Jessica Riskin chronicles in her studies of eighteenth and
nineteenth century automata, clever inventors, interested in going
beyond mere representation, created a variety of impressive
simulations of life---that is simulation in its modern sense, meaning
experimental models that can elucidate the natural, rather than its
contemporary sense which would have meant artifice.\cite[p. ??]{riskinDuck}
Makers of automata strove to imitate the very materials of life,
hoping to ``make the parts of their machine work as much as possible
like the parts of living things and thereby to test the limits of
resemblance between synthetic and natural life.''\cite[p. ??]{riskinDuck}
Automata of this era ``bled,'' ``defecated,'' and ``breathed,'' though some
of these functions were themselves faked, such as in the case of
Vaucanson's Duck. Nevertheless, imitation was central to the project,
and in this way these early automata prefigured at least some of the
developments in AI and Alife.

%% Steam governors
But many of these automata, despite being surprisingly accurate
mimeses of life, did little in terms of interaction with the
environment. Meaningful interaction requires closing the loop between
sensing and acting in the manner of a
 ``teleological,'' self-governing mechanism with corrective feedback.
 Indicative of Norbert Wiener's later research into cybernetics, such
 corrective feedback mechanisms had been studied since at least the
 late 18th century, when James Watt incorporated a governor into his
 steam engine. Watt himself had pulled from earlier applications of
 governors in windmills, which had been used since at least the 17th
 century.\cite{richardhills} James Clerk Maxwell, most famous for his equations of
 electricity and magnetism published in the 1865 paper ``A Dynamical
 Theory of the Electromagnetic Field'' (among the most important
 equations in physics), published a paper in 1868 on centrifugal
 governors in steam engines in the Proceedings of Royal Society. This
 paper, ``On governors,'' became one of the central papers in early
 control theory.\cite{ottomayr} Bringing together a number of existing
 areas including control theory,
 cybernetics---from the Greek \emph{cybernetes} meaning
 ``steersman''\cite[p. 6]{wienerMainIdeas}---extended their reach to more complex
 systems: ``control and communication in the animal and machine''. As
 Norbert Wiener puts it, control, or the
 feedback mechanism, is necessary for the extension of information
 theory into communication theory. Cybernetics envisions the world in
 terms of feedback mechanisms, which can be used to explain a variety
 of phenomena in living organisms: homeostasis, balance, and motion
 disorders like locomotor ataxia and Parkinsons all fall within the
 cybernetic sphere\cite[p. 10-15]{wienerMainIdeas}. And all are envisioned
 as outcomes of systems that pass messages internally. In this way
 cybernetics is a forerunner of the discipline of
 artificial intelligence, which is interested in re-creating many of the same
 self-regulating systems within computer systems.

%% 1950s Dartmouth Conf
%% and subsequent winter (search Winston, Six Ages)
Integral to the history of AI as a field is that it was fundamentally
interdisciplinary from the start. Like its forebear cybernetics, it
brought together researchers from physics, mathematics, biology, and
early cognitive science. The field began in earnest with the Dartmouth
Conference in 1956, which brought together many of the enduring big
names in the field. Hosted by John McCarthy (who originated the name
``Artificial Intelligence''), Marvin Minsky, Nathaniel Rochester, and
Claude Shannon, attendees included Trenchard More, Oliver Selfridge,
Ray Solomonoff, Allen Newell, and Herbert Simon: all were united by
``the idea that there was a rigorous and objective way of explaining
the human intellect.''\cite[Ch. 5]{mccorduck} The research areas of
the ``Dartmouth Summer Research Project'' included language learning
and use, ``neuron'' networks,\footnote{Neural networks are one of the
  intriguing long-term stories of AI research, subject of much
  controversy over the years regarding whether or not they would
  actually work. A couple of theoretical developments altered them
  from a curiosity to one of the main techniques in modern AI. This
  half-century journey presaged by the one sentence: ``Partial
  results have been obtained by the problem needs more theoretical
  work''.\cite{dartmouthconf} CAN I FIND THE PERSON WHO SAID SOMETHING
SIMILAR?} self-improving machines, and computational 
creativity.\cite{dartmouthconf} Early successes spurred romantic
predictions, and by 1960, human-level intelligence was predicted by
some to be only a decade away.\cite[p. 3]{winston} Overexpectation, however,
lead to a first AI ``winter'' from about 1965 to 1970, in which the
grand promises of AI were shown to be much further off: the current
techniques simply did not yield advances at the required rate. As Pat
Winston put it: ``Everyone searched for a kind of philosopher's stone,
a mechanism that when placed in a computer would require only data to
become truly intelligent.''\cite[p. 4]{winston} But by the 1970s
research was improving, and excitement building again.


%% DARPA SCI
The early to mid 1980s were also a time of great developments in
Artificial Intelligence, an era of ``celebrity science,'' high hopes,
big investments, and subsequent great public disappointment with the
coming of another ``AI Winter'' beginning in 1987 and
1988.\footnote{These are roughly the dates Russell and Norvig give in
  \emph{Artificial Intelligence} \cite{russellnorvig}} But despite the
warnings of Roger Schank and Marvin Minsky, that overoptimistic
expectations for AI would result in another winter like the previous
one in the 1970s, overall expectations were high, especially within
the business community, which funded companies and assimilated AI
techniques into real applications.\cite[afterword]{mccorduck} Though the 1980s continued divides
within the field about approaches to artificial intelligence, it
actually resulted in a wide variety of successful projects based on
improvements to expert systems, machine learning, natural language
processing, and computer vision.\cite[afterword]{mccorduck} The 1983 US Strategic Computing
initiative, led by Robert Kahn at DARPA, had AI as its third focus
area, with “image understanding” and interpretation—made possible by
the digitized image—as long range project goals. In its revised
10-year plan, the initiative even included an autonomous land vehicle
alongside a pilot's associate and computerized battle management
software. But the project suffered serious management problems, and
was eventually canceled, precipitating the general crash in AI
funding—through which research quietly continued, waiting for another
up-tick in public interest. But the Strategic Computing project,
whatever its lofty goals, was no failure. McCorduck cites Roland and
Shiman as saying that ``AI now performs miracles unimagined when SC
began, though it can't do what SC promised,'' which speaks to the
important developments that were made in the service of DARPA's vision.\cite[afterword]{mccorduck}


Though there are a number of approaches to AI research, different
ideas of how machine intelligence can be achieved, it is instructive
to look specifically at three how AI paradigms envision their project. 

%% Classical symbol approaches (3)
%%specifically w/ reference to ideologies
The first important AI paradigm is the classical symbolic system
approach. Associated with Allen Newell and Herb Simon, the idea of the
physical-symbol system hypothesis is that ``symbols lie at the root of
intelligent action''\cite[p. 109]{newellsimon}. Therefore not only
does intelligence require symbolic manipulation, it may indeed be
coextensive with physical-symbol systems, in other words a
physical-symbol system has ``necessary and sufficient means'' for
intelligence and intelligent action.\cite[p. 111]{newellsimon} Such
objects are symbol systems in that they contain symbols and processes
that act upon symbols. And they are physical in that they obey
physical laws and are realizable, in reality, through engineering.
These symbol systems would arrive at answers through a technique known as
heuristic search: by looking through a tree of possibilities in an
intelligent way, we arrive at the appropriate answer.\footnote{The key
  point of heuristic search is that such answers are approximate, but
  arrived at quickly, rather than exact, but arrived at slowly or,
  perhaps, never at all.} Intelligence is applied in heuristic search
by the pruning of the tree: rather than having to apply brute force to
search the entire space, an intelligent system applying heuristic
search makes decisions at each node as to which branches are most
likely to produce a good result and searches those.\cite[p.
  124]{newellsimon} As Newell and Simon wrote what makes a problem a
problem is ``not that a large amount of search is required for its
solution, but that a large amount \emph{would} be required if a requisite
level of intelligence were not applied'': the task of intelligence is
to ``avert the ever-present threat of the explosion of
search.''\cite[p. 125]{newellsimon} 

The conceit then, of the physical-symbol system hypothesis is twofold.
First, it assumes that
human beings essentially operate in this manner: that we apply
symbolic logic and heuristic search to provide for our intelligent
actions. Second, it assumes that computers can be true physical-symbol
systems. These two assumptions are not necessary clear. John Searle
essentially rejects the computer as a physical-symbol system in his
Chinese room example. Instead, the computer (room) is seen as a cheap
imitation of such a system: a room into which strange symbols are
passed, the appropriate responses looked up in a book,
and then passed out again, all without anything in the room having
access to their meaning.\cite{chineseSearle} Though meaningless
symbols are being processed by such a contraption, his view is that no
electronic computer ``can really manipulate symbols, nor really
designate or interpret anything at all,''\cite{escapingBoden} This is
a philosophical question of whether or not computers are capable of
true intelligence, not a matter of whether or not they can
convincingly imitate intelligence (and other philosophers of AI, such
as Margaret Boden, do not take Searle's view on the subject). 

The former point proves harder to dodge. While it may be true, as is
fundamental to the field of AI and which even Searle holds, that
machines can think because ``we are precisely such machines,''\cite[p.
83]{chineseSearle} and therefore it should be possible to create
intelligent machines, this does not guarantee that symbol systems are
the way to achieve intelligence. It may well be that introspection on
thought is a large part of why the logical theory seemed so
compelling: we like to think we are logical. Researchers like Newell
and Simon used ``think aloud'' experiments to identify problem solving
techniques,\cite[Ch. 10]{mccorduck} which seems naturally to suggest a logical response: when
asked to describe how we came to some decision, basing it in logic
seems the most acceptable alternative. While we certainly may apply logic
and process symbols, there is good reason to think that isn't how we
spend most of our time. Symbolic logic takes a lot of mental capacity,
so generally, we use other sorts of shortcut processes to come to
decisions. Pollock calls these ``quick and inflexible'' or ``Q\&I''
models\cite[p. 120]{pollock} and Dennett refers to them as ``habitual methods'' or
mechanical routines\cite[p. 157]{dennett} We rely on Q\&I models to do
much of our day-to-day reasoning because it is ``very important for
humans to be able to make rough probability judgments''\cite[p.
  120]{pollock}, and accepting the output of such approximate models
is not at all unreasonable in the absence of evidence for their
inappropriateness to a situation. A logical solution may be difficult
to arrive at in some situations. The physical-symbol system hypothesis
is just that, a hypothesis, and, historically, the classical symbolic
approach did not yield results as quickly as expected. This was part
of the reason for the first AI winter, and researchers moved on to
other approaches. While logic is part of the puzzle, it seems that it
is not the entireity of it. But this was the archetypal dream of early
AI: humanlike intelligence via logic systems.

%% Rodney Brooks (3)
One of those researchers pioneering a new way of doing AI was Rodney
Brooks. Part of a different wave of AI researchers, disillusioned with
the failures of logic-based robotic systems (like Shakey, the SRI
robot named for its tendency to shake when in motion)\cite[Ch. 10 (CHECK)]{mccorduck} to
achieve intelligent results, he posited a new way to build intelligent
robots, defined by the subsumption architecture.\cite[p. 353]{mobilebrooks} Above and
beyond the problems that physical-symbol system based AI had in
relatively controlled domains, robots controlled with these
techniques, overwhelmed by the complexity of the real world, responded
slowly and ineptly. To rectify this, Brooks attempted to cut out
cognition altogether, focusing instead only on sensing and
action.\cite[Afterword]{mccorduck} Rather than attempt to engineer a
human-level intelligence at once, when years of research had failed to
produce anything approaching these results, why not start small?
Insects, after all, lack cognition, but respond more adroitly to the
world than Shakey and its contemporaries. Uncertainty everywhere in
the physical world makes modeling the world extremely difficult: the
model has a tendency to get out of sync, and there needs to be
something to return it to accuracy. By building systems that can
accommodate uncertainty, Brooks believed, more progress could be
made.\cite[p. 347]{mobilebrooks} Instead of modeling the world, treat
the world as ``its own best model'' and focus on embodiment and action
in the world.\cite[p. 256]{ekbia}

Rather than being built into a single integrated system, the
subsumption approach is by definition modular. More complex behaviors
are built out of simper ones: level zero competence might avoid objects,
while the next level would wander aimlessly, and the next would
attempt to wander to places it had not been before.\cite[p.
  351--352]{mobilebrooks} Each layer is separate (and can actually run
on its own processor), and can read data from
and write data into the layer below it (hence the term subsumption,
the higher levels subsume the lower). When a higher level fails or
cannot run, lower levels continue to operate, and basic behavior is
maintained.\cite[p. 355]{mobilebrooks} This approach has tempting biological connections as well:
accurately or not, one can imagine the human being as a robot with a subsumption
architecture, where breathing and heartbeat are lower than balance,
which is lower than voluntary motion, which is lower than logical
thought. And temptingly, it is an approach that can handle
uncertainties, as each layer is built to be robust in the inevitable
event of inaccuracies and lost messages. Theoretically, one could
imagine building systems of great complexity using these approaches,
but this architecture, the radical relativist response to the
structural realism of the physical-symbol system, did not alone make
it past robots of insect intelligence.

Brooks himself abandoned the project of building intelligence from
these humble, subsumptive blocks. Instead, his research on building
the Cog robot shifts focus from ``emergence'' to ``integration,'' and
reversed some of his initial fervor to avoid representation.\cite[p.
  258]{ekbia} He skipped the middle of the evolutionary tree, straight
to humanoid forms, because the evolutionary approach was too slow: 
Brooks reported, it was ``starting to look like, if I was really
lucky, I might be remembered as the guy who built the best artificial
cat,'' a distinction he apparently did not desire.\cite[p.
  65]{brooksflesh}\footnote{As cited in \cite[p. 258]{ekbia}.}
But to do this, bootstrapped knowledge from other sources was
necessary, bringing back some of the old approaches.

%% Deep learning
A number of conceptually different approaches to AI have been tried
over the past 50 years, often with the same sorts of ambivalences as
Brooks's later work.\footnote{Ekbia's book does an excellent job
  exploring and cataloging many of these, more than I have space for.
  But logical and neorobotic approaches appear most relevant as
  broad-strokes historical precursors for inclusion here.} Ideas fade
and resurface as fashions change, and greater computational power
allows ``failed'' techniques to be tried anew. Ours is the age of
statistical AI. What the physical-symbol system and embodied cognition
have in common is their attempt to start from first-principles or zero
knowledge, and to build inexorably toward intelligence. The difficulty
of this road led Brooks to recombine some of the old manner of
knowledge representation into his robotic techniques. The new source
of bootstrapping information is data, ``Big'' and ``small.'' Current
techniques are primarily statistical in nature, leveraging these data
sets by training machine-learning systems; but as we will see
machine-learning systems have properties that make them ill-suited for
safety-critical systems, and autonomous vehicles are designed with a
mix of approaches that allows for more introspection into their
workings. The buzzword of the day seems to be ``deep learning,'' which
appears to yield radical new possibilities everywhere it is
applied---it really just means a return of neural
networks,\footnote{These systems still have little or nothing to do
  with actual neurons; they are brain-like in only the barest
  toy-model sort of way, despite how they are often represented in the
  press} which, armed with some improvements in weight generation, more layers of
nodes, and more data to train with, have been able to eclipse many of
the previous techniques\cite{???}.





%%   $$$$$$$$      NOTE      $$$$$$$$$
%%
%%
%%
%%      More HISTORY, Less CRITIQUE
%%                (here)
%%
%%




%% ``self-aware'' mario, image recognition and other
%% claptrap (1) (incl. John von Neumann & ALife (See CMS790 paper)??)
But as we have seen before, the AI hype-machine is again doing its
work. And popular claims about the utopic promise of deep learning
abound. But it is important to understand why this technique is not a
magic bullet in order to understand the real shape and implications
of the driverless car vision. Google's and Stanford's recent
improvements in image recognition\cite{markoffImage} triggered a wave
of popular speculation about computer vision meeting or surpassing
that of humans. And yet as we will see, the sort of ``understanding''
involved is limited, dealing only with static images and objects
captioned by humans in the training data. The systems would not be
able to answer other questions about the scenes that have to do with
the material properties of the objects, or likely results of various
actions.\cite{gomesJordan} Another recent slew of articles focuses on
the ``self-aware'' Mario created by researchers at the University of
T\"{u}bingen. Any pretense to worry about a ``self-aware AI . . . with
an insatiable desire for material wealth'' that knows ``how to kill''\cite{vincentMario}
is simply journalistic excess\footnote{As the
  researchers well know. This case is just a convenient example of how
such stories spiral out from the lab and acquire new meanings.}, and suggest
significantly more care must be taken in the use of such terms. Mario
is programmed with emotional states like a word-processing program is
programmed with different modes or display parameters. These states
are simply caricatures of emotions, and that plus the program's
natural-language interface makes it appear more eerily sci-fi than it
actually is. But this is the environment that automous cars, as a
research area of artificial intelligence, find themselves in today:
part of a new surge (bubble?) of interest in the field, driven by new
or newly extended techniques. But the ideologies related to artificial
intelligence are not the only ones implicated in self-driving car
research, as these vehicles are not merely programs but mobile robots
involved in gathering, mapping, seeing, ``knowing,'' and keeping-safe.


%% 2.1) data gathering and mapping
%% --this approach presupposes large amounts of data collection (maps),
%% and opens the way to more information about the passengers (2 p)
%% --see Google's patents in particular which talk about sensing things
%% like the number of passengers (4 p)
\subsection{Data Gathering and Monitoring} 

%%   $$$$$$$$      NOTE      $$$$$$$$$
%%
%%
%%
%%      More HISTORY, Less CRITIQUE
%%                (here)
%%
%%
%% ``this assumes xyz'' rather than ``but really abc''


Autonomous vehicles are and will continue to be networked
technologies. This connectedness brings with it great possibilities
for coordinating traffic and improving city planning, as well as great
risks to privacy and security, whether devices are networked with each
other or simply connected to central servers. 

With what networks, and for what reasons, will autonomous vehicles be
connected? The term autonomous night suggest that these vehicles will
operate without network connections, but vehicle autonomy is meant as
non-reliance on human input, rather than as a description of the
vehicle's disconnection from other information systems. While certain
guidance systems have been highly autonomous in the informational
sense—inertial guidance systems for intercontinental ballistic
missiles come to mind in particular,\cite{mackenzie} and Google's vehicles themselves
use inertial navigation aids\cite{knightfurther}—much so-called autonomous navigation
depends on access to global positioning satellites. Google's
driverless car technology in particular depends on highly accurate,
and very expensive,\footnote{An autonomous vehicle researcher at MIT
  quoted prices in the range of $70,000 to $100,000 per device for the
GPS alone, while noting that those would of course come down with
greater production volume.} differential GPS technology.

There is still space for significant debate about the roles of human
beings in so called “self-driving” systems. Despite numerous news
articles suggesting otherwise,i humans will be involved in the driving
process, in at least some ways, for the foreseeable future. Deciding
exactly what ways those are is a matter of both engineering practice
and appealing to the market and its squeamishness about
computer-controlled driving.ii But from another point of view,
``autonomous'' vehicles, regardless of the role of the human, will be
anything but autonomous in practice. They will be networked. GM's
OnStar service already connects vehicles to central servers for
purposes of safety, security, and convenience. It can automatically
alert the authorities in case of an accident or theft, and can in
principle be used to locate the vehicle at any time. The service also
provides vehicle diagnostics and a connection to a tablet or
smartphone that allows the owner to configure settings, lock and
unlock doors, and even operate the lights and horn from any
location.iii While these vehicles are not (yet) autonomous, OnStar's
capabilities suggest features that will become more common in highly
connected and computerized vehicles.

Vehicles that can receive information from each other and from the
roadway are a top priority for the NHTSA, and have been on the
research agenda for decades.i General Motors's 1950s Firebird concept
vehicles sold themselves on this research, despite the fact that
Firebird II “had no automated capabilities whatsoever.”ii GM's
promotional films suggested the vehicle could be controlled
electronically from traffic control towers—like those in use in
aviation—placed along major highways: the car was “under the direction
of an 'electronic brain' on a dream highway of the future.”iii This
would have been the realization of the New York World's Fair exhibit
from two decades earlier: GM's Futurama from the 1939 World's Fair
depicted cars that maintained distance from each other via a
“sophisticated system of radio control.”iv Also in the 1950s, RCA's
Vladimir Zworykin, a lead inventor of television technology, was
working on an intelligent road system of his own. His concept,
inspired by “railroad block signals,” used circuits embedded in the
road to magnetically sense vehicle speed and location, placing sensing
and coordination capabilities outside of the vehicle, in road-side
systems.v Zworykin's centralized planning model would send
instructions to individual cars, and a 1/40th scale demonstration
system was built for the 1960 Highway Research Board meeting in
Washington D.C.vi

But there is potential to do far more with the available information,
and data collected by the vehicle to make possible its own functioning
could be made to serve other purposes. Information about the vehicle
and its surroundings, including the locations of cars and pedestrians,
precise GPS coordinates of the vehicle itself, and the vehicle's speed
and acceleration, not only represent important knowledge for
path-finding by the vehicle itself, but new sources of potential
revenue for the groups in position to collect them. Uber, which
through its GPS-enabled ride-hiring applications still collects only a
fraction of the data that would be available through a self-driving
vehicle, has, as already noted, agreed to share its ride data with
cities. 

Google has envisioned vehicles that can determine their number of
occupants, and use facial-recognition or other biometric systems to
identify them. According to the patent, these vehicles could prevent
unauthorized persons from putting a child in a car, prevent convicted
sex offenders from operating their vehicles within the
legally-required distances of schools and playgrounds, or prevent a
car's doors from being opened (even from the inside) by a child unless
an authorized adult is present.i These are only visions, not yet
realities, but represent a safety culture which has decided
technological surveillance and enforcement are appropriate responses
to the mere potential of criminal behavior. Whether or not protecting
against these threats is an appropriate use of this information is a
matter for societal judgment, but such proposals, if enacted, would
require these vehicles to have unprecedented levels of very sensitive
knowledge about people and their lives: biometrics, criminal
histories, family and trust networks. 

Data ideology tells us that we can understand ourselves better through
data, that we can use it to determine patterns we never knew were
there. <MORE>

\subsection{Maps and Mapping}

In order to drive with us, autonomous systems will have to understand,
for at least a practical sense of “understanding,” traffic rules and
their accompanying signs, signals, lanes, and customs. This is, at its
core, a highly complex problem, because human understanding is built
through years of experience. It is through existing as a human being
in a particular cultural context that we know to drive on roads but
not on sidewalks, and how to tell the difference. 

Autonomy research seeks to create robotic systems that can function
“on their own,” but as we have already seen these systems are very
often connected to other information networks. The vehicles in the
DARPA Grand Challenge did not navigate “on their own”: they used GPS
to follow a path laid out for them in advance, using their autonomy
only to avoid obstacles like rocks and ruts.i And though successful
road tests have been accomplished without navigational assistance,
using only visual stimuli (such as the EUREKA PROMETHEUS project
previously mentioned), modern systems are tending to use more external
stimuli, rather than less, in an attempt to increase safety. Even as
advanced as it is, Google's autonomous vehicle technology requires
hyper-detailed 3D maps in order to operate properly on public
roadways.ii These maps are generated by vehicles outfitted with
special sensor arrays, like the LIDAR Google uses for their autonomous
vehicles, which drive a route and collect data which can be used to
reconstruct the model.iii 

Pre-made maps are used so the vehicle knows where stoplights, signs,
and curbs are, reducing the computational load on the machine in the
crowded visual landscape of driving, and allowing it to focus on
elements of the environment that are changing rather than those likely
to be static.i But Mapping requires the world to largely remain as it
is, and claims a unique capability to represent the real, objectively
and diagrammatically. But the required maps expose certain frailties
in these devices: they must know about speed limits, about traffic
lights, about rules of the road that were never designed for
autonomous systems. These devices must be molded to us, both to our
caprices and to our longstanding, ingrained laws and habits. They must
include historical knowledge, rooted in the legal and social histories
of roadways, which may differ between cities and states, and certainly
between countries across the world.

the necessary level of continual mapping is a massive task if the
vehicles must be usable everywhere. More likely, certain areas will be
mapped and restricted, or separate divided public rapid transit
systems will operate on roads that can be carefully monitored.i While
Mountain View, California may be mapped early, rural West Virginia or
Northern Maine may not be mapped as soon or as frequently.
Inequalities may be increased if routes frequented by upper-middle
class professional commuters, most likely to own new autonomous cars,
are mapped first, while roads around low-income communities are
ignored. Local customs and behaviors differ, and even if maps are
available, the same vehicle programming may not work for Los Angeles,
Boston, and the rural Midwest, let alone Singapore, Mumbai or Cairo.
The map, for all of its objective standardization, still represents
real places subject to cultural histories and vulnerable to
socio-economic dynamics. 



\subsection{Machine Vision}
%% 2.2) machine vision
%% --pull from the Spectator paper (4 p)
%% INCL. machine learning
Interpretation of the world around us is a task that seems
particularly easy for human beings, but particularly difficult for
machines. The invention of the photocell, early a tool for workplace
monitoring and surveillance, provided a simple channel through which
electrical systems could respond to the amount of light reaching
them.i
Though the photocell provides the computer with access to brightness
information over time, it falls far short of the ability of human eyes
to perceive detail and depth, identify shapes, and interpret
expression and motion. Emulating these characteristics represents a
key goal of artificial intelligence research. DARPA's 1983 Strategic
Computing Initiative included image interpretation as one of its main
focus areas.i But it is only relatively recently that real-time video
processing became possible for camera-based navigation on computer
systems small enough to fit in a standard automobile.ii And machine
vision problems, including object recognition and scene
interpretation, continue to be difficult, even with increased
processing power and new algorithms. 
As an engineering discipline, computer vision takes a decidedly
practical and reductionist view of what it means to see. The goal is
generally not to achieve creative interpretation or aesthetic
valuation, but to differentiate free space from things a robot should
not run into.i But this so-called objective focus still encodes
certain subjective judgments about objects (including people),
behaviors, and intent. And while computer vision is having success
with object detection, there is a wide variety of human knowledge
about objects and scenes that is missing in current computer models.ii

Though vision has not always been the sensory mode that dominated
autonomous vehicle research—earlier research focused on radio control
or electromagnetic tracks—vision is a particularly tempting sense to
use, as it is integral to how humans drive. In an attempt to build
autonomous vehicles that can operate without infrastructural changes,
research has moved away from tracks and cables toward vision-guided
systems. New approaches were pioneered by Ernst Dickmanns at
University Bundswerhr in Munich, with the vision-guided VaMoRs van,
and continued via the EUREKA PROMETHEUS project in 1987, in which
Dickmanns and Daimler-Benz built cars guided by analog video
cameras.iii Like the earlier VITA project by Daimler that used an
analog video-camera signal processed through a framegrabber, these
cars digitized analog video at relatively low resolutions. The
features the systems searched for, including lane markings and other
cars, are geometrically distinct and visible even in small images.iv

Vision-guided systems, now using digital video cameras and
off-the-shelf consumer hardware, have the benefit of being inexpensive
and insensitive to interference from other nearly devices (unlike
sonar, for example, which becomes problematic in crowded situationsi).
Some commercial systems, such as that developed for Mercedes-Benz's
self-driving S-class, which is slowly finding its way into consumer
vehicles, are primarily guided based on such visual sensors.ii To
these sensors, recent research has added roof-mounted LIDAR arrays.
LIDAR, short for Light Detection and Ranging, is effectively a depth
sensor, which is applied in vehicles to scan the environment with a
rotating array of laser beams to create a detailed 360-degree
representation of objects and their distances.iii This technology
solves some of the difficulties of image interpretation by default, as
it can provide highly-sensitive information about free space and
obstacles. Shape-detection algorithms can then be used (in addition to
vision-based data) to classify obstacles as different types of
objects.iv 

Pedestrian detection algorithms search for person-like shapes, where
“person-like” is determined by, for example, processing thousands of
images previously classified (by people) as being images of humans, so
the system can learn the features that correlate with a person being
in a particular region of an image.

These sensors and techniques impact how we are known by machine-vision
systems. Based on the views of these sensors, objects in the
environment are placed into categories by the vehicle's computers:
pedestrian, bicyclist, car, truck. Categories allow the system to make
statistical predictions about likely types of behavior: according to
one of Google's patent applications, bicyclists are likely to be more
erratic than trucks, and should be treated accordingly.i An obstacle
detected as a person can be expected to move, perhaps erratically,
while an object that is considered inanimate will not.\footnote{The
  DARPA Urban Challenge crash, the first crash between two autonomous
  cars, provides an important lesson on the vagaries of object
  detection: the classification threshold between moving and
  stationary, set too high, allowed one vehicle to interpret the other
  as stationary, leaving no room for unexpected behavior.i} 

The new technologies of vehicle automation thereby produce through
their operation new forms of evidence, which can be presented through
electronic information media. Because autonomous cars see—putatively
“as we see”—their sight can be leveraged as visual evidence. Computer
vision systems that identify pedestrians can be shown to do so, via
the detection boxes that act as diagnostic tools for researchers and
direct representations of internal system information. 



%% 2.2) what standard of safety? 
%% --better than an average person? or better than the best people? (1 p)
%% --human problems with using projected statistics to define policy:
%% does it make people feel better to know a system is statistically
%% safer when they are uncomfortable with it? see cars/planes for example
%% (3 p)
%% -->perspective of caution? or  ``losing lives every day'' that
%% could be saved?
%% ----how much risk is legitimate?? (2 p)
%% --airlines as a place where we see the different dimensions of
%% statistical safety vs. passenger's feelings of safety (3 p)
%% ---drawing from PARC/CAST documents


%% 2.3) functionalism over understanding
%% --for better or worse these devices are not humanlike in their
%% understanding
%% ----AI winters & the rise of commercial ML AI with much smaller goals
%% (see Wired articles, etc) (3 p)
%% --the ideology that drives them is an engineering mindset: ``just make
%% it work'' (2 p)
%% --pulling the quote from my USC talk: also interesting to note
%% regarding 1.2 that it seems to expect a humanoid robotic chauffeur in
%% a normal car! (1 p)
%% -and of course interviews with people (Ryan Chin; Walker Smith etc.
%% where they matter to this type of vehicle)
>>pull from my paper? but really, this needs to be informative not critical


\chapter{The Stakes of our Stories}
\label{chap:3}

3. Stakes of stories/tech (swap 4?)
-data collection
-mapping
-ML/machine vision
-functionalism vs. Understanding
-statistics and risk
-planning, policy, cities



>traffic-light detection and problems of infrastructure??


2.5??) Worth bringing in questions of terminology, ``intelligence''
etc. and the necessary caution in ascribing them to tech; we can look
and see that the ``intelligence'' in action here is really very
different than what we usually mean when we use that term
(human-centric)
--so does knowing this reality change what we want to see? (couple of
p)

We find ourselves in an era of seemingly continual technological
change that is strangely most noticeable in the most mundane parts of
our lives---how we shop, how we communicate, how we find
partners---almost as if ``designed by a bored researcher who kept one
thumb permanently on the fast-forward button''\cite[p.
  7]{???-Neuromancer1984}. No longer does the future seem to be
defined by flying cars and jetpacks. Instead, it is defined by
information, and its collection and use. The stakes of autonomous
vehicles are thereby deeply intertwined with the
stakes of other networked information technologies. And it is in these
current models that we should start the search for why competing
narratives of self-driving car development matter.

 In May of 2014, the European
Union's Court of Justice 
ruled in the landmark Costeja decision that since Google is processing
``personal data,'' and acting as a ``data controller,'' it may be
compelled to remove links to pages containing personal information
from its search results.\cite{???} The EU ruling, in deference to European
tradition and contrary to that of the United States, places an
individual's rights to privacy above the ability of users to access
information online.\cite{???} There is growing recognition that publicly
available data can be highly sensitive and that it may be beneficial
to allow individuals certain legal rights to control their own
electronic reputations, at least in particular circumstances. Costeja
opens the floodgates---from previously limited, targeted removals
through court cases and copyright law---to the possibility of free and
open public removal of ``public'' information.\footnote{This removal,
  however, is controlled and curated by Google itself, which has not
  waited for guidance on how to proceed but has forged forward on its
  own, as a way to set the standards by which information removal can
  be justified.\cite{???-http://www.theguardian.com/technology/2015/feb/18/the-right-be-forgotten-google-search}} But much of our
information is even harder to control.

Many current data-driven business models\footnote{See for example
  Google, which has as its fundamental revenue stream proceeds from
  advertising, which is sold by virtue of it being more accurate or
  targeted than other channels can provide. Facebook's new ad network
  competes in a similar space. Many of these companies provide
  services for ``free,'' where in reality the services are paid for by
the user in the data they generate and/or the advertisements they
view.} are fundamentally united in that increases in
functionality are predicated on invasions of or encroachments on what
we used to think was private, and represent increasingly invasive data
collection and sharing at a massive scale---often this information is
used internally to improve services, but it may also be aggregated and
sold to third parties, and in either event may be stolen or leaked by
disgruntled employees or thieves. This
is particularly important as a precursor of things to
come in the autonomous vehicle space, as such vehicles will allow
large amounts of data to be collected and shared with other entities. 


\subsection{Data Gathering and Monitoring} 

Clearly, new technologies will require new information, new types of
sensing, in order to operate; but it may be that not all the
information that is collected should be shared outside the immediate
context of its use. What is sensed, and what can be appropriately
transmitted back to servers for processing and storage, is of
paramount importance: this is a question of privacy in context.\cite{???}
Following Nissenbaum, privacy must always be seen in the context of
particular users and a particular use.\cite{???} It is not that data about our
commuting routes, for example, should never be collected, but that collected data
should not be sent uncritically to any third party without our
knowledge or consent, a violation of our information norms.\cite{???} 
Automated vehicles are and will continue to be networked
technologies. This connectedness brings with it great possibilities
for coordinating traffic and improving city planning, as well as great
risks to privacy and security, whether devices are networked with each
other or simply connected to central servers. There
may be legitimate uses for certain types of sensitive information, but
while providing it to municipal governments specifically to assist in
city planning may be legitimate, selling it to advertisers to help
them design more effective billboards may not be. Privacy issues
involving motor vehicles are likely to become much more complicated as
vehicles become able to record more, and potentially “know” more,
about their passengers.


%% 2.1) data gathering and mapping
%% --this approach presupposes large amounts of data collection (maps),
%% and opens the way to more information about the passengers (2 p)
%% --see Google's patents in particular which talk about sensing things
%% like the number of passengers (4 p)

With what networks, and for what reasons, will autonomous vehicles be
connected? Current driverless car concepts depend on networked
information for vehicle guidance. While, historically, certain
guidance systems have been insulated from communications---inertial
guidance systems for intercontinental ballistic
missiles are a particular example of this,\cite{mackenzie}---and
Google's vehicles use inertial navigation devices\cite{knightfurther}
alongside other sources of position data, current navigation systems
depend on global positioning satellites. Google's approach to
driverless car development currently necessitates accurate
and expensive\footnote{An autonomous vehicle researcher at MIT
  quoted prices in the range of $70,000 to $100,000 per device for the
GPS alone, while noting that those would of course come down with
greater production volume.} differential GPS receivers.

But a number of other developments already suggest that autonomous
vehicles will be connected to more than just global positioning
networks. GM's OnStar service already connects equipped vehicles to central servers for
purposes of safety, security, and convenience. The system has the
ability to automatically alert the authorities in case of an accident
or theft. It also provides vehicle diagnostics to the owner's tablet
or smartphone, and the OnStar app also allows the owner to configure settings, lock and
unlock doors, and remotely operate the lights and horn.\cite{onstar}
While these vehicles are not (yet) autonomous, OnStar's present
capabilities are representative of features that will become more common in highly
connected and computerized vehicles, including autonomous ones.

Additionally, vehicles that can receive information from each other and from the
roadway are a top priority for the NHTSA, and have been on the
research agenda for decades.\cite[p. 11]{wetmore} These concepts would
collect and use data to streamline traffic flow and provide
information to reduce delays and accidents. But there is potential to
do far more with the available information,
and data collected by the vehicle to make possible its own functioning
could be made to serve other purposes. Information about the vehicle
and its surroundings, including the locations of cars and pedestrians,
precise GPS coordinates of the vehicle itself, and the vehicle's speed
and acceleration, not only represent important knowledge for
path-finding by the vehicle itself, but new sources of potential
revenue for the groups in position to collect them. Uber, which
through its GPS-enabled ride-hiring applications still collects only a
fraction of the data that would be available through a self-driving
vehicle, has agreed to share its ride data with municipalities for purposes of
city planning.\cite{uberJardin} Though this data is ostesibly being
shared for the public good, it also serves private ends: to curry favor with
authorities that might otherwise attempt to shut the service down. The
ride data can also be used to provide internal predictions to help
Uber increase revenue by directing drivers to the right locations at
the right times. And it would not be far-fetched, in the current
information landscape, to see such information sold to third-party advertisers.

% consider 1-2 sentences on risks of connected devices; http://blog.kaspersky.com/internet-of-crappy-things/

Google has envisioned vehicles that can determine their number of
occupants, and use facial-recognition or other biometric systems to
identify them. According to one patent,\cite{predictPatent} these vehicles could prevent
unauthorized persons from putting a child in a car, prevent convicted
sex offenders from operating their vehicles within the
legally-required distances of schools and playgrounds, or prevent a
car's doors from being opened (even from the inside) by a child unless
an authorized adult is present. These are only visions, and patents
are notorious for trying to cover as many possible angles of a
technology even if they are not intended to be applied in practice.
But these suggestions represent a perspective on safety and societal
order that posits technological surveillance and enforcement as an
appropriate preventative measure against criminal behavior. Whether or not protecting
against these threats is an appropriate use of this information is a
matter for societal judgment, but such proposals, if enacted, would
require these vehicles to have unprecedented levels of very sensitive
knowledge about people and their lives: biometrics, criminal
histories, family and trust networks. 

Prevailing data ideology tells us that we can understand ourselves better through
these data, that we can use them to determine patterns we never knew were
there. Out of this ideology come publications like Uber's blog\footnote{???}, which
describes customer ``insights'' gained through their ride data, which
are ostensibly interesting to the public. ``Look here,'' they seem to
say, ``we can tell you about \emph{you}.'' But it important to
recognize that the ideologies of data collection---for corporate
profit and public use---are deeply intertwined, and strongly
influenced by the possibility of using machine learning techniques and
statistical analyses to find and exploit patterns. Automated vehicles
are part of this general culture, and without significant effort to
resist erosions of privacy and data protection stand to open our lives
to greater scrutiny in terms of the means of and reasons for our
personal mobility.

%%More on the problems of data ideology? or not the place for it?


\subsection{Maps and Mapping}

But the information that may be collected and processed by automated
vehicles is not only about human inhabitants of the environment: other
types of data collection are also implicated in current
visions of the driverless car project. In order to drive with us,
autonomous systems will have to understand,
for at least a practical sense of ``understanding,'' traffic rules and
their accompanying signs, signals, lanes, and customs. This is, at its
core, a highly complex problem of interpretation and representation
for machines. Human understanding is built
through years of experience: it is through existing as a human being
in a particular cultural context that we know to drive on roads but
not on sidewalks, and how to tell the difference. But we do not have
this luxury in training machines.

Localization has been a fundamental issue for robotics since the
beginning of the field. In order to figure out how to act, a robot
needs to know where it is; and in general, robotic environments can be
expected to change rapidly with motion, and to vary significantly from
place-to-place\cite[p. 4]{???-SLAMbook}. Motions cannot, for the most part,
be pre-planned:  they must continually be reevaluated in new and
emerging situations, as the robot moves through the environment. As a
key part of this puzzle the robot must be able to determine its own
relationship to some target location it needs to get to: this
``target-robot relation'' is an ``unavoidable''\footnote{It is
  possible to create robots that navigate without any modeling of the
environment, as Fern\'{a}ndez-Madrigal et al. point out. However, this
severely limits the range of behavior that is possible. Without
reference to the environment, errors slowly accrue, and a system that
navigates in a known space without localization will find itself
getting further and further off-track.} component of successful
navigation.\cite[p. 5]{???-SLAMbook} This relation can be expressed
in different forms---either quantitative representations such as maps,
or logical, prepositional statements---and these representations carry
their own techniques of interpretation. 

But there is more to navigation than pure localization. While
localization is the usage of known elements in the environment to
estimate a robot's position, those elements themselves must be known
in order to perform this process. Mapping is the complementary
process, the estimation of ``\emph{unknown} spatial relations that
exist between environment elements'' in order to allow for subsequent
navigation\cite[p. 5]{???-SLAMbook}. There is a somewhat tricky
precedence problem we face here:  in order to build maps with
autonomous systems, they must know where they are; in order to know
where they are, they must have maps with which to measure
against.\cite[p. 6]{???-SLAMbook}

Returning to the history of AI,
localization was part of the cause of the poor performance of the
robot Shakey: world modeling through a symbolic approach was slow,
which was rendered even more problematic by the relative lack of
computing power.\cite{???-mccorduck?}
Though one way to determine robot position in the world is through
simultaneous localization and mapping (SLAM)---a serious research area in
robotics, as it is ideal for areas that are impractical to map
beforehand, such as ones that are always changing---it is easier to localize by comparing
measurements to a known map. However, that implies that the map is
pre-made, and therefore sets limits on the rate at which the
environment can change and still allow the robot to operate. By
contrast, the subsumption architecture involved an
attempt to avoid the localization problem by pure sensory response to
the environment, but building a robot that exhibits complex and
predictable behaviors in complex environments, through this approach,
is exceedingly difficult, as evidenced by Brooks's shift to hybrid
modeling approaches. But new AI techniques do allow researchers to
take advantage of new kinds of data.\cite{???} Today, localization and
mapping are both considered ``satisfactorily solved in practical
situations''---though SLAM techniques
are still somewhat less reliable or developed---given sufficient
computational resources and environmental data.\cite[p.
  5-6]{???-SLAMbook}

%%Fernández-Madrigal, Juan-Antonio, and José Luis Blanco Claraco.
%%2013. Simultaneous Localization and Mapping for Mobile Robots :
%%Introduction and Methods. Hershey, Pa: IGI Global, 2013. eBook
%%Collection (EBSCOhost), EBSCOhost (accessed February 21, 2015).

Here, the connections of autonomous systems to information networks
again become important. The vehicles in the
DARPA Grand Challenge did not navigate ``on their own'': they used GPS
to follow a path laid out for them in advance, using their autonomy
only to avoid obstacles like rocks and ruts.\footnote{John Leonard,
  discussion with the author, December 3, 2014} And though successful
road tests have been accomplished without navigational assistance,
using only visual stimuli (such as the EUREKA PROMETHEUS project
in the 1990s, which navigated only with cameras and therefore could
only follow the road in an automated fashion, rather than navigate to
given GPS coordinates)\cite{???}, modern systems are tending to use more external
stimuli, rather than less, in an attempt to increase safety and
vehicle capability. Even as
advanced as it is, Google's autonomous vehicle technology requires
hyper-detailed 3D maps in order to operate properly on public
roadways.\cite{???} These maps are generated by human-piloted vehicles outfitted with
special sensor arrays, like the LIDAR Google uses for their autonomous
vehicles, which drive a route and collect data which can be used to
reconstruct the model used for future drives.\cite{???} 

The vehicle does not operate in a SLAM mode, at least not completely.
Pre-made maps are used so the vehicle knows where stoplights, signs,
and curbs are, reducing the computational load on the machine in the
crowded visual landscape of driving, and allowing it to focus on
elements of the environment that are changing rather than those likely
to be static.\cite{???} Prior knowledge of speed limits should make the car's
behavior more reliable and predictable in all conditions, even if
speed limit signs are missing or obscured---consider how often human
drivers, when faced with an absence of signs, base their behaviors on
supposition or prior knowledge. When Google's car was certified for
testing in Nevada, Google was allowed to pre-select the route the car
would take, so that they could build the comprehensive model the
system requires beforehand.\cite{???} The system would likely not have been
capable of passing a test in which the examiner could have added
detours on the fly. And though Google claims to have driven more than
700,000 miles with their cars, those are not 700,000 unique miles. A
limited, thoroughly pre-mapped route has been driven many times to
achieve those numbers.\cite{???}

Mapping claims a unique capability to
represent the real, objectively
and diagrammatically, but also requires that world to remain largely
static, at least on the order of how long it takes to update the map
for a particular region. The necessary level of continual mapping is a massive task if the
vehicles must be usable everywhere. The United States alone contains almost
8.5 million road miles\footnote{Data as of 2008,
  http://blog.cubitplanning.com/2010/02/road-miles-by-state/\cite{???}},
and it took years for Google Streetview to acquire the level of
coverage it currently has. The maps required for driverless car
localization are a significantly more difficult project in terms of
amount of data, reliability of data, and therefore frequency of
updates. The utopian discourse of driverless cars
implicitly suggests that such vehicles will be available everywhere,
and are the solution to nationwide transit problems. But widespread
egalitarian access to maps-based devices depends upon a rapid,
widespread mapping initiative. The success of such an initiative is
dependent not only on a large amount of human effort and capital
investment, but a series of decisions about what regions should take priority.

TomTom\cite{???} and Nokia\cite{???} both claim to be attempting such a mapping project,
but it will not happen immediately or all-at-once. More likely,
certain areas will be 
mapped and restricted, or separate divided public rapid transit
systems will operate on roads that can be carefully
monitored.\cite{???} While
Mountain View, California may be mapped early, rural West Virginia or
Northern Maine may not be mapped as soon or as frequently.
Inequalities may be increased if routes frequented by upper-middle
class professional commuters, most likely to own new autonomous cars,
are mapped first, while roads around low-income communities are
ignored. Such decisions are easy to imagine being justified by
commercial exigencies---limited funds, potential markets---but would
cut directly against the utopian narratives of driverless cars.

Furthermore, the seemingly universalizing forces of maps and computer programming
have a tendency to hide issues of geographical and cultural
specificity, which are rendered invisible in these utopian narrative.
But though engineering practice holds that these issues are
conquerable, they should be anything but invisible. More than
abstract problems of localization and mapping, automated vehicles
present the problem of having to exist in an environment that is
highly complicated, varied, and cultured. Maps alone are insufficient
for anything but the most simplistic view of vehicle operations.
Programmed devices
must must know about speed limits, about traffic
lights, about rules of the road that were never designed for
autonomous systems. These devices must respond to human caprices and
be adapted to longstanding, ingrained laws and habits. They must
include historical knowledge, rooted in the legal and social histories
of roadways, which may differ between cities and states, and certainly
between countries across the world. Local customs and behaviors differ, and even if maps are
available, the same vehicle programming may not work for Los Angeles,
Boston, and the rural Midwest, let alone Singapore, Mumbai or Cairo.
The map, for all of its objective standardization, still represents
real places subject to cultural histories and vulnerable to
socio-economic dynamics. These social and regional issues are often
ignored in the driverless vehicle narrative, but nevertheless stand to
be critical to the manner in which these technologies could come to
enter everyday life.



\subsection{Machine Vision}
%% 2.2) machine vision
%% --pull from the Spectator paper (4 p)
%% INCL. machine learning
For all this, however, vehicle localization and sensing depends upon
visual interpretation of the vehicle's surroundings. Interpretation of
the world around us is a task that seems
particularly easy for human beings, but particularly difficult for
machines. The invention of the photocell, early a tool for workplace
monitoring and surveillance, provided a simple channel through which
electrical systems could respond to the amount of light reaching
them.\cite{???} Though the photocell can easily provide a computer system
with access to brightness information over time, perceiving detail and
depth, identifying shapes, and interpreting expression and motion are
all capabilities of human vision that require more sophisticated
technologies to reproduce. DARPA's 1983 Strategic Computing Initiative
included image interpretation as one of its main
focus areas.\cite{???} But it is only relatively recently that real-time video
processing, needed for camera-based navigation, became feasible for computer
systems small enough to fit in a standard automobile.\cite{???} And machine
vision problems, including object recognition and scene
interpretation, continue to be difficult, even with increased
processing power and new algorithms. 

As an engineering discipline, computer vision takes a decidedly
practical and reductionist view of what it means to see. The goal is
generally not to achieve creative interpretation or aesthetic
valuation, but to identify particular objects in a scene, to recognize
faces, or to differentiate free space from things a robot should
not run into.\cite{???} But this so-called objective focus still encodes
certain subjective judgments about objects (including people),
behaviors, and intent. Consider, for instance, the issue of object
detection. While it may be a objective question whether or not a
particular object (e.g. a wrench) is physically present and visible in a scene, it is not
necessarily so self-evident which objects are noteworthy or important
to detect (e.g. does the worm gear on an adjustable wrench in addition
to the wrench itself)---these choices depend on applications, and the judgments of
designers as to what is worth measuring. And while computer vision is having success
with object detection, there is a wide variety of human knowledge
about objects and scenes that is missing in current computer
models, including propositional understandings (``what would happen if
. . ?''), projections about occlusions (``what is behind that?''), and
connections to other sensory modes (``what would that feel like if I
touched it?'').\cite{???} 

Regardless, though vision has not always been the sensory mode that dominated
autonomous vehicle research, it is a particularly attractive sense to
use as it is integral to how humans drive. In an attempt to build
autonomous vehicles that can operate without infrastructural changes,
research has moved away from automated systems controlled and
constrained via tracks and cables toward vision-guided
systems. New approaches were pioneered by Ernst Dickmanns at
University Bundswerhr in Munich, with the vision-guided VaMoRs van,
and continued via the EUREKA PROMETHEUS project in 1987, in which
Dickmanns and Daimler-Benz built cars guided by analog video
cameras.\cite{???} Like the earlier VITA project by Daimler that used an
analog video-camera signal processed through a framegrabber, these
cars (VITA-II and VaMoRs-P) digitized analog video at relatively low resolutions. The
features the systems searched for, including lane markings and other
cars, are geometrically distinct and visible even in small
images.\cite{???} That these vehicles were designed primarily for
highway operation under constant human supervision made this strategy
acceptable, since pedestrian interactions were likely to be
rare.\footnote{The documentation of the VITA-II mentions pedestrian
  interaction management as future work necessary for changing the
  working domain from the highway to urban and country environments.\cite{???}}

%%Can cite SivakSchoettle p. 3 on weather; and they cite Lavrinc 2014 there

The apparent primacy of computer vision, however, only holds for fair
weather. Rain, sleet, and 
snow interfere with vision-guided systems, and currently prevent them
from operating safely, as numerous more skeptical news articles are
happy to note.\cite{knightFurther}\cite{gomesObstacles} While the
human eye is generally very good at
pattern recognition and filtering out noise, getting computers to
appropriately recognize all 
the necessary road objects already tests
cutting-edge techniques. Precipitation not only makes roads slippery
and less safe for all drivers, it reduces visibility---which decreases
the distances at which objects can be detected, and therefore also the
time the system has to react---as well as potentially decreasing image
contrast and presenting interference to the image---through drops or
streaks on the glass through which images are taken.\cite{???-rainACAS} These issues also
affect people (though our brains can filter out raindrops on the
windshield rather than interpreting them as obstacles), but human
drivers would, at least ideally, choose to drive more slowly given
adverse conditions. One emerging research area
in automated vehicles is raindrop detection for automated driver
assistance systems, in order to help such systems compensate for
reduced visibility, or alter their behavior to be more conservative in
these situations. However, ``developing algorithms that work perfectly
under all weather conditions appears to be unrealistic,'' so such a
modular and cautious approach may well be necessary in order to build
robust systems.\cite{???-rainACAS} While commercial use of
vision-guided self-driving vehicles on 
sunny days may indeed be possible in the near future, use in
less-than-ideal conditions is likely to be limited for some time,
while human oversight compensates for algorithmic deficiencies. 

Vision-guided systems, now using digital video cameras and
off-the-shelf consumer hardware, do have the benefit of being inexpensive
and insensitive to interference from other nearly devices (unlike
sonar, for example, which becomes problematic in crowded
situations\footnote{John Leonard, discussion with the author, December
3, 2014.}). Some contemporary commercial systems, such as that developed for Mercedes-Benz's
self-driving S-class, which is slowly finding its way into consumer
vehicles, are primarily guided based on such visual sensors.\cite{???} To
these sensors, recent research has added roof-mounted LIDAR arrays.
LIDAR, short for Light Detection and Ranging, is a distance
sensor, which is applied in vehicles to scan the environment with a
rotating array of laser beams to create a detailed 360-degree
representation of objects and their distances. This technology
solves some of the difficulties of image interpretation by default, as
it can provide highly-sensitive information about free space and
obstacles. Shape-detection algorithms can then be used (in addition to
vision-based data) to classify obstacles as different types of
objects: pedestrians, bicyclists, cars, and trucks.\cite{???}

%% DONE?
%% --Göde Both and reluctance to use Machine Learning (1 p)
%% ---brittle, unpredictable, difficult to ``tune''
%% --Sensors issues and inability to distinguish crushed newspaper from
%% rock (1 p)
%% ---requires a lot of other information, more than simple physical
%% volume
%% ---object identification work in machine vision is slowly progressing,
%% but then we hit the above problems of VV\&T for such models (2 p)

While much computer vision research uses machine-learning algorithms to detect
objects, engineers in automotive applications are justifiably
reluctant to rely on machine-learning: as Göde Both has noted in his
ethnographic research on developers of driverless cars in Europe,\cite{bothpt1}
machine learning techniques are brittle and unpredictable:\cite{bothpt2} neither
characteristic makes them suited for software that must be highly
reliable and on which people's lives literally depend. So mixes of
manual and machine-learning approaches are used to do object
detection. Though machine-learning can be a highly-effective
technique, it is generally difficult or impossible to know what the
system has actually ``learned'' and therefore how it will react in new
and unknown situations.\cite{bothpt2} Pedestrian detection algorithms search for
person-like shapes, where
``person-like'' is determined by, for example, training a classifier
using thousands of images previously labeled (by people) as being images of humans, so
the system can learn the features that correlate with a person being
in a particular region of an image. In a sense, the computer can be
said to develop
a ``concept'' of a person. So long as the right features appear in each
new situation, this approach works; but what the computer has
``learned'' is essentially black-boxed, and resists introspection.
Even discounting these concerns, shape-detection is not a complete
solution, as the knowledge it provides about objects is only
skin-deep: the sensors cannot differentiate a rock from a crumpled
newspaper, and Google's car will serve to avoid
both.\cite{gomesCircles} Further distinguishing objects requires a lot
of information besides position or simple physical volume,
including fine-grained information about the object's surface
appearance, and interpretation of physical properties from observable
behavior (e.g. bouncing or rolling). None of this is fundamentally
impossible, but it presents necessary areas for research.

Detected categories allow the system to make
statistical predictions about likely types of behavior: according to
one of Google's patent applications, bicyclists are likely to be more
erratic than trucks, and should be treated
accordingly.\cite{predictPatent} These sorts of predictions are
something that human drivers do consistently, and are therefore also
likely important to how autonomous vehicles may drive.\footnote{The
  DARPA Urban Challenge crash, the first crash between two autonomous
  cars, provides an important lesson on the vagaries of object
  detection: the classification threshold between moving and
  stationary, set too high, allowed one vehicle to interpret the other
  as stationary, leaving no room for unexpected behavior.\cite{???}}
Categories alone, even if achieved, are also insufficient if detected
objects are treated solely as obstacles to be avoided. One notable
researcher who I spoke to drives around searching 
for, and attempting to photograph, situations he believes will cause
trouble for current vehicles. His collection includes many situations with police
directing traffic, especially when combined with sun glare or
occlusions of the sight lines---such as by a mail relay box---that
would make predictions regarding oncoming 
objects difficult. His take-away from these situations is that there
is a significant body of edge-cases that make full autonomy
impractical for the forseeable future. Instead, humans will be
required to account for these sorts of difficult perceptual situations. 

%%TODO
%%<Include quotes from the interview>

However, because autonomous cars see---putatively ``as we see''---their sight can
be leveraged as visual evidence of their operation. Computer vision systems that identify
pedestrians can be shown to do so, via the detection boxes that act as
diagnostic tools for researchers and direct representations of
internal system information. The new technologies of vehicle
automation thereby produce through 
their operation new forms of evidence, which can be presented through
electronic information media. We can point to these images and identify
that the vehicle is operating as it should. The box inscribed around
the pedestrian shows this plainly. The 3D environment scan from the
LIDAR system presents the same opportunities for ``transparent'' visual
proof. Three-dimensional shapes, in standing in stark relief against
the background, bear witness to the sensory operation of the vehicle.
These shapes too are demarcated by boxes, which represent their
computational transformation from information into an object or
artifact of interest. Because the visual detection technologies used
by autonomous vehicles are compatible with the visual technologies of
media representation, new types of seeing are opened to us.

A coincidence of sensing and representation in the evocative and
powerfully persuasive medium of visual representation stands to shift
the way we perceive driverless systems in their operation, providing
us a different manner of insight and introspection, but also
potentially a different level of obfuscation, than ever before. When
considering how these vehicles are presented to us---and the fervor with
which some researchers demand that we accept them\cite{???}---it is important to
remember the many black boxes of information processing behind the
naturalized image of the sensor readout. The rhetoric of the
self-evident visual image, as it operates in the autonomous vehicle
space, can be made more visible by considering contemporary medical
imaging. As researchers have noted about the prevalence of brain
images from fMRI studies in the popular press, these depictions have
strong rhetorical impact on discourse about neuroscience. Colorful
pictures of the brain, as the inheritors of Enlightenment notions of
visual evidence par excellence, have a way of convincing the viewer of
the validity of reseachers' claims about physical brain locations and
their effects,\cite{???} especially since those pictures are presented as the
direct, transparent products of sensing techniques. The conflation of
human and machine vision which makes these images so confusing is as
misleading in vehicle navigation as it is in neuroscience. Significant
statistical processing is necessary to make sense of data in either
realm, processing that is generally beyond the public's gaze\cite{???} but the
end result leaves the viewer with a false sense that something real
has been sensed, revealed, and affirmed by the image.


%% 2.2) functionalism over understanding
%% --for better or worse these devices are not humanlike in their
%% understanding
%% ----AI winters & the rise of commercial ML AI with much smaller goals
%% (see Wired articles, etc) (3 p)
%% --the ideology that drives them is an engineering mindset: ``just make
%% it work'' (2 p)
%% --pulling the quote from my USC talk: also interesting to note
%% regarding 1.2 that it seems to expect a humanoid robotic chauffeur in
%% a normal car! (1 p)
%% -and of course interviews with people (Ryan Chin; Walker Smith etc.
%% where they matter to this type of vehicle)
\subsection{Functionalism, Utilitarianism, Ethics} 
We should likewise not be deceived by an apparent parallelism of human
and machine knowledge, or an elision of the difference between human
and machine ``cognition.'' Self-driving cars will not be human-like in
understanding, even while they can detect and identify pedestrians as
objects of interest within a particular epistemological frame. Whether
or not their machinic perspective is to be lauded is a deep and
philosophical question, but the robots envisioned by current AI
ventures bear little resemblance to those of Asimov, or the dreams that grew
from the Dartmouth Conference. While a wide range of software companies and
startups have entered the AI 
industry, these companies are not primarily focused on
general-purpose AI.\footnote{The primary divisions responsible for two
representative Google projects, Chauffeur and Deep Mind, are
geographically and organizationally separated. A Google Deep Mind
employee I spoke to at MIT said that they had had only very limited
contact with the Chauffeur team. September 10, 2014.} Though certain
ventures still hold out the dream 
of doing so, several AI winters have shown that the creation of
general intelligence is very difficult, and is by no means around the
corner. Those working in the field are well aware of
this.\cite{???-articlerebuttingmusk-http://www.popsci.com/bill-gates-fears-ai-ai-researchers-know-better?cmpid=currents} So much current work is
fundamentally utilitarian, building systems with clear goals, metrics
for success, and market segments.

Though computer science and philosophies of AI have been using
``intelligence,'' ``knowledge,'' and ``understanding,'' among other
words, to talk about computers since the beginning of the field, these
uses should not be taken at face value. ``Intelligence'' is slippery,
and its definition is not constant over time. It is difficult to
define intelligence in ourselves\cite{???} and yet another thing to define it
in relation to other entities. Weaving was once considered to be a
peculiarly human capability, a sign of an advanced, intelligent
mind.\cite{???} But after Vaucanson's loom allowed mechanical devices to weave
seemingly on their own, this capacity was no longer seen as uniquely
human, and was no longer a marker of intelligence. The same process
occurred with chess in the 1990s. When IBM's Deep Blue beat Gary
Kasparov, chess ceased to be the standard by which intelligence could
be judged, precisely because it had been achieved. Real intelligence
had to lie elsewhere: for example, in the game Go, mastery of which
has continued to elude machines.\cite{???}

Nevertheless, machines manage to do things that seem ``intelligent.'' So
though these terms are heuristics for understanding the observed
behaviors of machines, they slip slowly over time from self-conscious
scare-quoted use into casually accepted statements. While automatic
translation may seem ``intelligent,'' or a system that can define \emph{\'{e}toile}
as ``star'' may seem to possess ``knowledge,'' this intelligence or
knowledge is perhaps very different than our own. A deep
epistemological question presents itself: how do we know, and how do
machines ``know''? Many AI systems operate via statistical pattern
recognition, so we may ask whether we believe human intelligence is
also merely pattern recognition: does a system that can associate star
with its definition really know what a star is? Is mere linguistic
association sufficient for knowledge?

H. R. Ekbia\cite{???} and others remind us that we should be skeptical of the
applications of these terms to computational processes. As we have
seen, Searle's thought experiment of the ``Chinese room''
argues that symbol processing and pattern recognition alone is not
intelligence, though from the outside the results may appear to be
intelligent. Despite the faults of his argument, his caution
about ascribing overly ambitious human ideas to computational
processes is warranted. Even Pamela McCorduck, a colleague of several
notable AI researchers and a believer in the field in general, hedges
on how intelligent some of the programs she discusses in \emph{Machines Who
Think} actually are. Ornstein, Smith, and Suchman, in their 1984
article ``Strategic Computing,'' warn of the difference between domain
capabilities and ``common sense,'' and suggest that ``unwarranted
optimism'' and a particular funding climate (issues also present today)
push researchers to mask the shortcomings of AI with ``semantic
shifts.''\cite{???} We alter the definitions of ``knowledge'' and ``understanding''
to fit what our machines can do, and these claims, taken literally,
``give rise to unrealistic confidence in the power of the technology.''\cite{???}
The two-way process of linguistic and technological change---that
intelligence gets applied to describe whatever researchers manage to
achieve, while real ``intelligence'' retreats away from each
computational advance---leaves these terms poorly defined.

However, the utilitarian model of AI makes good sense for a number of reasons.
First, much can be achieved with current technologies. Rather than
focusing on what is potentially an exceedingly long-term project,
which would carry much greater risks and more distant rewards,
achieving short term goals is an 
attractive prospect. It can attract investment because it can be
profitable sooner. Though short-term commercial viability is not
necessarily applicable for this vision of the self-driving vehicle, it
is possible, at least, to have working prototypes on the road, generating
interest and publicity, even in relatively controlled conditions. Taking
journalists for test drives drums up interest, even if the technology
is not ready for full-scale deployment. Make no mistake: current
prototypes are highly capable, regardless of their faults, but any claim that
they ``know'' or ``understand'' is a tenuous misuse of words which
could no longer credibly mean what they generally do. Such use opens,
rather than closes, questions.

Second, humanlike characteristics may not even be helpful, in general, for building
specific applications. One would likely not want one's self-driving car
to be preoccupied or emotional.\cite{???-wiredFutureofAI} It may be that, for utilitarian
purposes like driving, many of the characteristics of the human are
detrimental, their elimination helpful and intentional. Much of the
discussion around why autonomous vehicles are necessary centers on
just such qualities: distractability, sleepiness, lapses in
concentration.\cite{???} We would not wish to emulate such
characteristics in robotic systems. However, though these are human
capabilities, presented in this way they exist largely as caricatures
of the human. People possess a
variety of other capabilities which might be helpful to many AI
applications. As AI researcher Doug Lenat wrote in 1997:
\begin{quote}
Before we let robotic chauffeurs drive around our streets, I'd want the
  automated driver to have a general common sense about the value of a
cat versus a child versus a car bumper, about children chasing balls
into the streets, about young dogs being more likely to dart in front
of cars than old dogs (which, in turn, are more likely to bolt than
elm trees are), about death being a very undesirable thing.\cite[p.
  122]{ekbia}\end{quote} 

This is a difficult knowledge and perception problem. But even more,
it is an issue of selfhood, embodiment, even sentience. While cats,
children and bumpers can be identified as objects, and children
chasing balls into the streets can be identified as patterns, a
computer programmed to respond to these stimuli may respond correctly
without ``knowing'' anything. While a machine can be programmed to avoid
running into people, can it have any understanding of death? Can it be
programmed to ``feel guilt''? Does it need to?

%CAN I FIND RESEARCH ON reactions to humanoid vs. non-human
%AI drivers

Current approaches, however, assume this kind of deep understanding is
unnecessary, both for the technical creation of such vehicles as well
as their public acceptance. And yet the ideology of artificial
intelligence, the focus of the field itself, is bound up in the idea
of ``intelligent'' machines that can be said to ``know.'' It is an
interesting read on the changing 
times to notice that Lenat's statement seems to suggest humanoid
robotic drivers operating regular cars. As well as moving toward
functionalist systems, the industry has moved toward embedded systems
within devices, systems that make no pretenses to be humanoid, but
instead revel in appliance-hood. While once, a dominant dream might have been
to build a world full of humanoid robots, the conceit of modern
consumer AI---IBM's Watson cloud API, ``internet of things''
approaches---is that we can make everything smart. Whether this shift in the form of AI
systems makes customers more or less nervous about computer-driven
vehicles is an empirical question. But it certainly suggests a desire
to make the systems more 
invisible---and actually has important implications for human-computer
interaction, as we will see in chapter \ref{chap:3}. Though it may seem
obvious, it is critical to remember that the devices we discuss here may
have the properties Lenat mentions only in the way that the University
of T\"{u}bingen's AI Mario may be
``self-aware,'' possessing programmed constraints that cause it to
avoid people, and perhaps expect children to behave erratically.
Nonetheless, despite lacking the facilities for actual ethics, these
devices possess an implicit ethics: the
behaviors of these devices will the instantiate moral and
ethical judgments of their human creators, based on
human-authored heuristics and statistical predictions. This will be
fundamental to how we understand our relationships to such
complicated technological systems.

%%ETHICS
Robot ethics is an issue of growing importance to society at
large---given the rapidly-expanding uses of robotics for labor, military,
research, entertainment, and healthcare, among other regimes\cite[p.
  5-6]{patrickLin}---but we should be very clear what we mean when we
discuss it. Lin, Abney, and Bekey's collection is wide-ranging,
considering perspectives including safety and errors (how should
robots be introduced in order to minimize adverse effects?), law and
responsibility (when accidents happen, who is liable?), ethical codes
(how, and with what ethical frameworks, can robots be programmed to
operate ``ethically''?), and social impact
(how do we weigh potential for robots to eliminate
jobs?).\cite{robotEthics} Here, I consider ethical codes specifically,
since they interact with functionalist approaches. The question of
driverless car ethics has spurred quite a volume of news articles,
with a greater or lesser handle on the questions
involved.\footnote{One article discusses a notable
  researcher who has purportedly ``stressed the need for driverless
  cars to be flexible enough in their 
engineering to be able to make ethical judgements that aren't
necessarily written into their programming,'' a statement whose
meaning is difficult to parse.\cite{jessicaDavies} Such statements presumably
refer to judgments that are not explicitly defined by programmers, but
instead reliant on some kind of more general ethical calculus. A
program that does things that are not in its programming at all,
however, is incoherent. Even bugs, though unintentional, are part of
the program. This sort of phrasing is unfortunate, as it
risks being read as a robot somehow operating ``outside of its
program,'' making ad-hoc ethical decisions based on criteria of its
own invention (even an ethical system trained via machine learning
techniques is fundamentally tied to its programming, though its
behaviors were not strictly ``written'' by its human programmers, but
are instead shaped by their selection of training situations and
responses; and the choice to use these techniques in the first place
comes with a certain risk, that a system
might learn the ``wrong'' ethical behaviors, for example). Such
behavior is impossible for anything but a sentient
AI, as I discuss here; and focus on it distracts from the real issues
at hand.} Lin himself has been successful in urging a dialogue
within various autonomous vehicle research groups about the ethics of
their products, as well as getting media recognition of this push for
self-conscious development.\cite{timeEthics} However, many articles
elide fundamental distinctions in the situations being discussed,
distinctions which should be foremost in our minds when we consider
the stakes of developing autonomous machines.

A focus on machines behaving ethically is at risk of making similar
errors to the conflation of machine and human vision, knowing, or
understanding. Ethical behavior is not necessarily strictly defined by
any one ethical code:  ethics and ``ethics'' are not identical. When we
speak of robotic cars ``making decisions'' of what to do in a crisis
situation, we implicitly accept the idea that such decisions are
really being made by the program. Discussing whether or not those
behaviors are ethical risks suggesting that our robots have all the
capabilities---of sensing, knowing, and processing---necessary to
carry out, \emph{in toto}, what we consider to be ethics \emph{qua} itself. At this
stage in AI development, however, what we are truly discussing is not
the potential ethical robot cars but an \emph{ethical calculus} for
autonomous vehicles. Such a calculus would be a quantification of
ethics according to some particular formalism, so as to allow a
computer program to select a course of action based on a particular
situation. However, in a very real sense, decisions are not
``being made'' by amoral vehicles.\footnote{Not every device behavior
can be predicted, and it would be foolish to place full responsibility
on the programmers: there is real autonomy in devices, in that they
may do things we do not want. But though all devices have bugs and
will be unpredictable in certain circumstances, the first place to
look for ethics, for an implicit or explicit ethical calculus, is the human beings
that do the design.} They are being made by software engineers,
self-consciously or not, even if no
explicit calculus is used and the developers' ethics are only
implicitly present as a consequence of coding decisions. Ultimately,
if we want to care about how ethically systems
operate, we must look at how they are programmed, and what goals that
programming is intended to serve.



%% 2.3) what standard of safety? 
%% --better than an average person? or better than the best people? (1 p)
%% --human problems with using projected statistics to define policy:
%% does it make people feel better to know a system is statistically
%% safer when they are uncomfortable with it? see cars/planes for example
%% (3 p)
%% -->perspective of caution? or  ``losing lives every day'' that
%% could be saved?
%% ----how much risk is legitimate?? (2 p)
%% --airlines as a place where we see the different dimensions of
%% statistical safety vs. passenger's feelings of safety (3 p)
%% ---drawing from PARC/CAST documents
\subsection{Safety and Statistical Risk}

The primary driving force in the current self-driving car
narrative is safety; specifically, the poor safety record of human
drivers and the potential for machines to do much better, free from
human frailties of distraction and fatigue. How bad are human
drivers, really? Conventional stories of human drivers paint us as
plainly terrible, prone to road rage and drowsyness, and generally tending
to do anything except pay attention to the road\footnote{Numerous popular
  articles take this position\cite{???}}. However, the statistics tell
a slightly different story. Official numbers show that the number of
motor vehicle accidents remained in the range of 10--11 million
per year for most of the late 2000s. The death rate seems to have
decreased overall in this time, settling somewhere below 1.5 deaths
per 100 million vehicle miles, presumably due to a combination of
better safety features (especially since 1990) and other
factors.\cite{???} The total number of vehicle-related deaths is much
lower, at a still significant 35,000 deaths per year. But this alone
does not tell the story. At around 1.5 deaths per 100 million vehicle
miles, or about 1 death per 67 million miles, humans seem relatively
competent in a statistical sense. The average American, who might
drive 1 million miles\cite{???} is unlikely to be
involved in a fatal crash in his or her lifetime. Looking at non-fatal
accidents as well, humans get involved in about one accident per
286,000 vehicle miles. Part of why the autonomous vehicle problem is
such a difficult one is that these numbers are relatively high.

Theoretically, computers can do better. But especially careful human
drivers can also clearly beat the human average. But how safe do
autonomous vehicles need to be in order to be allowed on our roads?
Safer than the average human? or safer than the very best drivers?
Such questions have real impact when it comes to how devices are
designed and when they become commercially viable---and are at least
partly ethical questions. The autonomous
vehicle enterprise seems to call for using such projected statistics
to define policy. One common narrative is that undue caution in the
rollout of autonomous vehicles will directly ``cost'' lives, since
people are killed by human drivers every day.\cite{???-blog} However,
people are also accustomed to the current automobile death rate, and
any autonomous vehicle crashes are likely to attract deep
scrutiny\footnote{As Jim Womack has pointed out, there is no good side
to change as a regulator. Regulators are not congratulated when things
go right, only criticized when things go wrong. So some measure of
tentativeness is almost certainly justified to the regulatory mind.
(Discussion with the author, December 3, 2014)} as
to whether a human could have prevented the accident.\cite{???} If,
instead, human oversight to otherwise fully automated systems were
required, the additional risks of supervision would also need to be
accounted for (this is dealt with more thoroughly in chapter
\ref{chap:3})---including risk homeostasis\cite{???}, the tendency to
behave less cautiously in situations that appear to be more safe.

These are questions of policy, but also questions of human acceptance.
This ideology posits lowering accident rates above all else, leaving
no space for human squeamishness about technology and responsibility. 
The statistical argument suggests that the death rate is all that
matters, but humans are notoriously bad at understanding and
responding to statistics.\footnote{And any possible framework for
  quantitatively measuring and regulating improvement (e.g. number of deaths,
  monetary cost, etc.) hides all manner of assumptions and comes with
  potentially unforseen consequences: the best decisions by some
  metrics will be non-optimal in others.}

Just how safe these vehicles are expected to be has become a point of
public contention. A 2015 whitepaper by Sivak and Schoettle of the
University of Michigan Transportation Research Institute attempts to
make the case that ``it is not a foregone conclusion that a
self-driving vehicle would ever perform more safely than an
experienced, middle-aged driver,'' due primarily to issues of sensing and
predictive knowledge \cite[p. i]{???-SivakSchoettle}.\footnote{This is worth
recognizing, even if it is also trivially true that nothing about AI
development is a foregone conclusion \emph{a priori}. I believe that
even if ``computational speed, constant vigilance, and lack of
distractibility'' are not alone sufficient to beat out all human
drivers,\cite[p. 4]{???-SivakSchoettle} AI techniques will approach human abilities to use predictive
knowledge, given sufficient development time.} Most firmly, they
attempt to impress that no conceivable implementation of self-driving
vehicles will have zero fatalities. One popular response to this type of
argument is the following: 
\begin{quote}''Of course, the researchers are trying to correct what they regard as
excessive technological optimism. Still, is it entirely fair of them
to compare robocars only to the best drivers? Most accidents are
caused by the worst ones, and it is beginning to become clear that
those are the people that a robot could outperform with one clanky
arm tied behind its back.''\cite{???-RossSafety}\end{quote} 
But this perspective is far too simplistic. The question of whether an
automated vehicle's fatality rate exceeds some peoples', matches the
safest drivers, or bests all human drivers stands to determine which
vehicles are legal.\cite[p. 6]{???-SivakSchoettle} While projecting vehicle
risk functions is tricky enough when comparing against a known human
standard, it becomes even more difficult when one considers that the
``conventional vehicle'' risk function is itself going to change with
future safety technologies, including automated driver assistance
features. The fully self-driving vehicle must be compared to a moving target.

%(Phillip E. Ross, ``Has Robocar Safety Been Hyped?'' 20 Jan 2015: http://spectrum.ieee.org/cars-that-think/transportation/safety/has-robocar-safety-been-hyped)

In both cases, measurements and predictions of reliability and risk
are key to the development of autonomous vehicles. The aviation and
defense spaces are in some ways ahead of commercial ground vehicle
research in terms of engineering automated systems, in part because
they have already been forced to confront these issues in earnest:
airplane autopilots have a more developed history than automotive
autodriving, aided by the lower complexity of the air environment,
with simple mechanical autopilots available as early as
1912.\cite[p. 16]{???-NRCAutonomy} The \emph{Autonomy Research for
  Civil Aviation} report by the National Research Council pays
significant attention to the fact that ``the lack of generally
accepted design, implementation, and test practices for
adaptive/nondeterministic systems will impede the deployment of some
advanced IA [increasingly autonomous] vehicles and systems'' and that
``existing V&V [verification and
  validation] approaches and methods are insufficient for advanced IA
systems'' for many of the same reasons.\cite[p. 2]{???-NRCAutonomy}
Among the high-priority research projects they identify as being both
most pressing and most difficult, they include both the development of
methodologies to ``characterize and bound the behavior of
adaptive/nondeterministic systems'' and the creation of standards for
the ``verification, validation, and certification of IA
systems''\cite[p. 4]{???-NRCAutonomy}. As in the automotive space, the
core reason for increasing autonomy is to increase reliability, but
being assured of this reliability is difficult. As the report notes,
however, software creators ``are generally expected to prove that the
software can deliver the intended capabilities at specified
performance levels and in the relevant environment,'' which involves
extensive examination of the code, and testing every logic path,
according to FAA guidelines.\cite[p. 39--40]{???-NRCAutonomy} But
these approaches are not scaling to more complex systems, and new
validation approaches are required to account for the impacts of human
operators or supervisors on system behavior.\cite[p.
  40]{???-NRCAutonomy} The NRC panel is not alone in recognizing these
issues: the Defense Science Board in their task force report \emph{The
  Role of Autonomy in DoD Systems} attempts to address the importance
of the larger environment in which automated systems operate, and
within which they can produce ``unintended operational
consequences''\cite[p. 2]{???-DSB}. They warn of \emph{brittle}
platforms, and emphaisze the importance of developing ways to predict
and understand the resilience of systems\cite[p. 7, 11]{???-DSB}

%Include NITRD report, if available? http://scenic.princeton.edu/NITRD-Workshop/index.html 

However, when it comes to public adoption of automated technology,
risk assessment is only part of the story. Does it make us feel better---more
comfortable, more likely to get into autonomous taxis and spend our
money on autonomous cars---just to know that they are statistically safer
than the average driver? Airplanes are one type of vehicle for which
these different dimensions of 
safety---statistical safety compared with perceptions or feelings of
safety---have already become visible. Despite the comparative statistical safety of flying,
people tend to be more afraid of getting on an airliner than getting in their
cars.\cite{???} While this may have to do with a number of factors,
including that aircraft do not remain on the ground during operation,
it also represents a situation in which passengers give over their
agency to pilots performing a job they do not understand and could not
take over in an emergency.\cite{???-canIFindAnythingOnThis}

%SIMILAR ISSUES WITH TRAINS?\cite{??}

The CAST working group has had to account for these issues in their
recommendations about airplane safety since 1998. Part of their charter---to
reduce the accident rate by 80 percent\cite{???-PARCCAST}---is
motivated not so much by the rate itself but predictions about its
most visible effects:  if air traffic were to continue to increase
with industry predictions but to maintain the 1997 rate of 1.5 major
crashes per 1 million departures, we would see one fatal crash per
week by 2005, and one per day by 2025\footnote{See
  \cite{http://articles.orlandosentinel.com/1999-11-07/news/9911070029_1_plane-aviation-safety-type-of-accident},
  for example. This point was made to me by David Mindell, discussion
  with the author, November 12, 2014.}
This frequency of accidents was judged to be intolerable to
airline customers, even though the overall accident rate would be no
larger than its ``very safe''\cite{???-PARCCAST} starting
value.\footnote{By comparison, the CAST website describes: ``The
  enhancements made over the past decade have made history in
  commercial aviation - by making it the safest it has ever been.
  Today, fatal accidents are reduced to only one in 22.8 million
  flights, an absolutely remarkable achievement.''\cite{???-cast-safety.org}} Though
aircraft would be as safe as they 
had ever been, people, on seeing the sheer number and scale of
accidents, would be unlikely to choose to fly. While the issues facing
aircraft and trust are not precisely the same as those for cars, this
example shows that actual safety rates are only
one component of interest when considering how consumers react to
modes of transportation. Perceived safety---due to accident scale,
publicity, trust, or other factors---may be very different. But
while research into how to get human beings to trust robotic drivers
is being done\cite{???}, the voices pushing for autonomous cars sooner
rather than later would suggest that the statistics are all that
matters (and indeed, would tend to fabricate those statistics out of
mere predictions). 


\subsection{Planning, Policy, Cities}

Besides being presented as steps toward road safety, driverless cars
also become part of the rhetoric of personal mobility, ``public''
transportation, and urban planning. Potential or imagined
benefits of these vehicles include empowerment for the blind or
elderly (as we discussed in chapter \ref{chap:1}), a reduction in
traffic congestion, elimination of parking problems, and the
persistent access to a dense network of hireable point-to-point
transport vehicles (essentially, driverless taxis). There is
  also an environmental case being made for these vehicles, as they
  will drive predictably and can therefore be tuned to be very
  fuel-efficient. The social, environmental, and urban planning
implications are at least worth considering in passing before moving
on to consider alternate models for how these technologies could be
developed. 

In my discussions with a fellow researcher who has been studying the
development of self-driving vehicles, it became clear that the
developers see themselves as people trying to change
society.\footnote{This self-image is not limited to those working on
  mobility systems. John Naughton writes---and my contact with
  programmers would lead me to agree---that many ``are fired with a zealous
  conviction that they are doing great stuff for the
  world.''\cite{http://www.theguardian.com/commentisfree/2015/feb/22/google-tech-elite-living-in-a-parallel-universe-john-naughton}}
They
tend to buy into utopian visions of big cities freed of cars, with no
parked cars on the streets and only driverless cabs. This desire to
change the way cities operate is admirable: our cities have a
difficult hundred-year history of changing to cope with the
automobile, and many of these changes have not been for the better
when looked at through a broad lens. The prevalence of automobiles has
literally re-shaped city centers,\cite{???-kemp} widened
streets,\cite{???-kemp} and generally made many cities less safe for
pedestrians and bicyclists. Many of the streets in the most pedestrian
friendly zones of older cities would no longer be legal to build under
current traffic codes, and the default lane widths for arteries have increased
from 10 to 12 feet over the past fifty years, with disastrous results.\cite{???-http://www.citylab.com/design/2014/10/why-12-foot-traffic-lanes-are-disastrous-for-safety-and-must-be-replaced-now/381117/} In some of the
United States's most pedestrian-unfriendly cities, major intersections
in the downtown area are more than 9 lanes across, each of which is 12
feet wide. These changes in 
regulations and practices have contributed to more sterile urban
centers with reduced potential for vibrant civic life. Urban planners have
in recent years begun to take the automotive threat seriously, pushing
for reduced lane width requirements and designing urban areas that
limit vehicle traffic on purpose.\cite{???-kemp} If self-driving
vehicles have a high probability of effecting a positive change on the
cityscape, as some proponents claim, they could be an important
addition to these developments.

Congestion and other environmental factors, such as air pollution, are
other social reasons used to support automated vehicle development.
With carefully controlled, automated driving, the fuel consumption of each
individual car could be reduced, just as so-called ``hypermilers''
today use altered---and very conservative---driving techniques in
order to increase their fuel efficiency.\cite{???} As discussed in
chapter \ref{chap:1}, researchers envision using V2V communications to
share road information between vehicles, reduce gaps between cars, and
thereby increase throughput and efficiency of the road system. But
these proposals involve numerous potential pitfalls. While coordinated
fleets of vehicles could increase the throughput of roads by 2 to X
times, significantly greater increases would be necessary to match the
throughputs of buses or subway trains, which can be 100 times
greater.\cite{???-kemp} Much depends on actual usage---subject to
homeostasis of inconvenience and the Jevons paradox. If a factor of 10
increase in highway throughput were possible, but caused a factor of
10 increase in overall miles driven, congestion would not be
ameliorated, and significantly more environmental impact might result. Any
large-scale impacts of automated vehicles are dependent on the
architecture of the overall transportation system. And therefore it is
the transportation system, not automated vehicles alone, that must be remodeled.

Fleets of fully automated vehicles could reasonably be expected to increase
efficiency and decrease reluctance to take trips---making them
less-inconvenient by allowing the driver to perform other tasks---in
which case vehicle usage would be expected to increase. This result is actually
implicit in the visions of those who would support fleets of
driverless taxis to move people around cities:  these concepts are
explicitly oriented toward providing for needs that are presently
under-served, and so presenting automated vehicles as a solution to
city transit problems implies greater usage of such vehicles to meet
those needs. So as increases in the use of cars to commute have
decreased the quality of urban spaces, what is to stop cities of
automated vehicles from
being even more unfriendly to
people? This question was in fact asked at the MIT SENSEable Cities
conference in 2014.\footnote{It received little in the way of reply from the
panel. Nhai Cao suggested that cities will not be built around new
vehicle infrastructure, and that his focus is on building new
capacities into vehicles, which avoid the question. Paolo Santi was
unconcerned with vehicle-to-pedestrian sensing and interaction, but
instead with the very limited domain of mathematical optimization,
a viewpoint that seeks to ignore the potential consequences of such
optimization.} Positive changes to city infrastructure will not happen
automatically. Complex
changes to complex systems require significant modeling work,
experiments, or even trial and error to get right. Old models and
habits need to be broken---consider for example the rules (about road
widths, for instance) enshrined
both in the official urban planning regulations as well as their
surrounding culture, rules which are often taken almost as natural
law\cite{???-kemp}---and new ones built to replace them. 

However, we should be aware that these visions, however contingent,
are not universal. Looking more closely, my informant described to me, the visions
of driverless car developers
often have a strong male bias. They see businessmen
taking driverless cabs to work, and students getting drunk and, unable
to drive themselves home, using automated vehicles to ferry themselves
back to their apartments. These visions may be greatly influenced by
the developers themselves---primarily male engineers with masculinist
preconeptions---either thoughtlessly or as a marketing strategy. But
regardless of the source, they suggest designed solutions may tend to
be predisposed to certain types of uses, and less amenable to others
that fall too far outside this masculine vision. Other uses are still
possible, of course---Uber is already being used by parents to ferry
their kids to school, even though it was almost certainly not designed
for that
purpose\cite{???-http://healthland.time.com/2013/11/14/for-more-parents-its-uber-to-the-rescue/--http://www.nytimes.com/2013/09/26/fashion/moms-van-is-called-uber.html?_r=0}---but
inherent gender bias in design is not a problem that should be ignored
when making large-scale changes to infrastructure, as it can deeply
affect the ways and frequencies with which people choose to use new
technologies. Indeed, a recent UK study suggests men may be more
amenable to autonomous vehicle technologies, and it is worth
considering the role that gender-biased technological visions may have
in that
effect\cite{???-http://www.nytimes.com/2013/09/26/fashion/moms-van-is-called-uber.html?_r=0}.
Thinking about the variety in types, uses, and users of road vehicles---from
subcompacts to vans to pickup trucks; from solo commuting to carrying kids to
hauling construction equipment or moving furniture; from the wealthy
to the poor, the urban to the rural---makes clear that cars are
multipurpose vehicles, with culturally-specific uses. When the
expected user is the commuting, upper-middle-class, working father or
the privileged college boy (and the opposers of the technology are
branded by some, as we saw, as ``soccer moms''), we risk developing
vehicles that preferentially serve certain dominant uses and not
others. When I asked one informant whether the developers have an idea
of what driverless cars would mean for people who have children, he
replied:  ``they don't think about it.'' While certainly not shared by
all developers in this space, it is not a vision we can afford.




\appendix
\chapter{Tables}

\begin{table}
\caption{None}
\label{arm:table}
\begin{center}
\begin{tabular}{||l|l||}\hline
Table & if \\\hline
need  & be \\\hline
\end{tabular}
\end{center}
\end{table}

\clearpage
\newpage

\chapter{Figures}

\vspace*{-3in}

\begin{figure}
\vspace{2.4in}
\caption{A figure.}
\label{arm:fig1}
\end{figure}
\clearpage
\newpage

\begin{figure}
\vspace{2.4in}
\caption{Another figure.}
\label{arm:fig2}
\end{figure}
\clearpage
\newpage

%% This defines the bibliography file (main.bib) and the bibliography style.
%% If you want to create a bibliography file by hand, change the contents of
%% this file to a `thebibliography' environment.  For more information 
%% see section 4.3 of the LaTeX manual.
\begin{singlespace}
\bibliography{main}
\bibliographystyle{plain}
\end{singlespace}

\end{document}

